\section{Research Vision}

At a very high level, this research tackles the same problems tackled by many
existing research projects like BOOM Analytics \cite{alvaro2010boom}, Bloom
\cite{alvaro2011consistency}, Bloom$^L$ \cite{conway2012logic}, TARDiS
\cite{crookstardis}, and IPA \cite{holt2016disciplined}:

\begin{quotation}
  \emph{How can we make writing weakly consistent programs more principled?}
\end{quotation}

BOOM Analytics, Bloom, and Bloom$^L$ use data-centric declarative programming
and program analysis based on the CALM theorem. TARDiS exposes the entire
history of divergent computations to users in a git-like system to empower them
to programmatically resolve conflicts. IPA uses a type system to ensure weakly
consistent data doesn't leak into strongly consistent data. \todo{Cite and
summarize \cite{balegas2015putting}, \cite{roy2015homeostasis},
\cite{li2012making}, \cite{li2014automating}, and \cite{gotsman2016cause}.}

This system builds off existing research on CRDTs and \iconfluence{} to provide
a library of CRDTs, a language to express invariants over those CRDTs, the
ability to transactionally update the replicated CRDTs, and a decision
procedure to algorithmically determine whether the transactions are
\iconfluent{} with respect to the invariants, ensuring coordination free
execution if they are and potentially giving illustrative counterexamples when
they are not. A simple example of psuedocode written with this system may look
something like this:

\begin{Python}
# Create an increment/decrement counter CRDT.
x = PNCounter(0)

# Set the invariant that the counter must always be non-negative.
set_invariant(x >= 0)

# This transaction is invariant confluent and would be accepted by the system.
@transaction
def foo():
  x.increment(42)

# This transaction is not invariant confluent and would be rejected by the
# system.
@transaction
def bar():
  x.decrement(1)
\end{Python}

A more complex example of psuedocode may look something like this:

\begin{Python}
# A map from employee id to employee name.
employees = Map[Int, String]()
# The number of employees.
num_employees = PNCounter(0)
# Employee ids partitioned into teams.
teams = Set[Set[Int]]

# The number of employees has to equal the size of employees.
set_invariant(num_employees = employees.size())
# All ids in teams must appear in employees.
set_invariant(forall team in teams. team subset employees.keys())
# All teams must be disjoint.
set_invariant(forall a, b in teams. a != b => a intersect b = {})
# Every employee must be on a team
set_invariant(union(teams) = employees.keys())

# This transaction which adds a new employee would be checked for
# invariant-confluence by the system.
@transaction
def add_employee(name):
  id = employees.unique_id()
  employees.put(name, id)
  num_employees.increment(1)
  teams[0].add(id)
\end{Python}

This pseudocode brings up a lot of \emph{theoretical questions}...
\begin{inparaitem}
  \item What CRDTs and what operations can we support?
  \item What invariant languages can we support?
  \item Given the set of CRDTs and invariants, is determining invariant
    confluence even decidable?
  \item If so, can it be determined efficiently?
  \item If not, how can we limit the CRDTs, operations, or invariants to make
    it decidable?
\end{inparaitem}
...and a lot of \emph{systems questions}:
\begin{inparaitem}
  \item Is the system expressive enough to write real-world programs?
  \item Can the system be implemented as a library in an existing programming
    language?
  \item How do we architect and build an actual system out of this?
\end{inparaitem}

This research aims to answer these questions and develop a first-of-its-kind
system that allows programmers to write coordination free distributed systems
with more discipline than ever before.
