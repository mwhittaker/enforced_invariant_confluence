\section{\iconfluence{}, Formally}\label{sec:formalism}
Our invariant language is defined by the grammar in \figref{invariant-grammar}
which includes linear equations and inequalities over variables in a finite set
$\var$ and integer constants in $\ints$.

\begin{figure}[h]
  \centering

  \newcommand{\atom}{\textsf{atom}}
  \newcommand{\aexp}{\textsf{aexp}}
  \newcommand{\bexp}{\textsf{bexpr}}
  \begin{gather*}
    \begin{array}{rrll}
      \atom & ::= & k & \text{\emph{constants}} \\
            & |   & x & \text{\emph{variables}} \\
      &&& \\
      \aexp  & ::= & \atom         & \text{\emph{atom}} \\
             & |   & -\atom        & \text{\emph{negation}} \\
             & |   & \aexp + \aexp & \text{\emph{addition}} \\
             & |   & \aexp - \aexp & \text{\emph{subtraction}} \\
      &&& \\
      \bexp  & ::= & \aexp \leq \aexp  & \text{\emph{inequality}} \\
             & |   & \lnot \bexp       & \text{\emph{negation}} \\
             & |   & \bexp \land \bexp & \text{\emph{conjunction}} \\
             & |   & \bexp \lor \bexp  & \text{\emph{disjunction}} \\
    \end{array}
  \end{gather*}

  \caption{
    A grammar for our linear equation and linear inequality invariant language.
    Note that the comparison operators $=$, $\neq$, $<$, $>$, and $\geq$ are
    not included because they can be defined in terms of the existing
    operators.
  }
  \label{fig:invariant-grammar}
\end{figure}

A \emph{database state} is a total function $D: \var \to \ints$.  We say a
database state $D$ \emph{satisfies invariant} $I$, denoted $I(D)$, if $I$
evaluates to true after all variables in $I$ have been replaced by their
mapping in $D$.
%
For example, if $\var = \set{x, y}$, then $D_1 = \set{x:1, y:2}$ and $D_2 =
\set{x:42, y:99}$ are both databases. If $I \defeq x \geq 10$, then $D_2$
satisfies $I$ (i.e. $I(D_2)$) but $D_1$ does not (i.e. $\lnot I(D_1)$).

A \emph{transaction} is a partial function $t: \var \rightharpoonup \ints$.
\emph{Applying a transaction} $t$ to database state $D$, denoted by $D \circ
t$, produces a new database state defined as follows:
\[
  (D \circ t)(x) \defeq \begin{cases*}
    D(x) + t(x) & $x \in \dom{}(t)$ \\
    D(x)        & otherwise
  \end{cases*}
\]
For example, $t_1 = \set{x: 42}$ is the transaction which increments $x$ by 42.
$t_2 = \set{x: 1, y:-2}$ is the transaction which increments $x$ by 1 and
decrements $y$ by $-2$. If $D = \set{x: 100, y:100}$, then $D \circ t_1 =
\set{x:142, y:100}$ and $D \circ t_2 = \set{x:101, y:98}$.

Note that transaction application is commutative. That is for all database
states $D$ and transactions $t_1$ and $t_2$, $D \circ t_1 \circ t_2 = D \circ
t_2 \circ t_1$. Also note that transactions are \emph{blind} in that they do
not read from the database. We'll relax this assumption in \secref{imp}, but
for now we're keeping things simple.

A transaction chain $C$ created from a set of transactions $T$ is a sequence of
transactions $t_1, \ldots, t_n$ where $t_i \in T$ for $1 \leq i \leq n$. Let $D
\circ C$ be syntactic sugar for $D \circ t_1 \circ \cdots \circ t_n$. We say
$C$ \emph{satisfies invariant} $I$ starting at $D$, denoted $I(C @ D)$, if for
every prefix $C'$ of $C$, $D \circ C'$ satisfies invariant $I$. Note that $I(C
@ D)$ implies $I(D)$ (the empty prefix of $C$) and $I(D \circ C)$ (the entire
prefix of $C$).

A set of transactions $T$ is \emph{\iconfluent{}}\footnote{An astute reader
will notice that this isn't actually the definition of \iconfluence{} presented
in \cite{bailis2014coordination}. It's a simpler one. Rest assured, this
simpler definition will be hard enough to deal with. We'll explore the
ramifications of this simplification in \secref{alie}.} with respect to
invariant $I$ if for all database states $D$ and chains $C_1$ and $C_2$ created
from $T$, if $D$ satisfies $I$, $C_1$ satisfies $I$ starting at $D$, and $C_2$
satisfies $I$ starting at $D$, then $D \circ C_1 \circ C_2$ satisfies $I$.

For example, let $\var = \set{x}$ and consider a transactions $t_+ = \set{x:
1}$ and $t_- = \set{x: -1}$ which increment and decrement $x$ respectively. Let
$I \defeq x > 0$ be the invariant that the value $x$ in the database is
positive.
%
$\set{t_+}$ is \iconfluent{} with respect to $I$ because given any database $D$
and chains $C_1$ and $C_2$, if $x$ is positive in $D$, then $x$ will be
certainly be positive in $D \circ C_1 \circ C_2$.

On the other hand, $\set{t_-}$ is not \iconfluent{} with respect to $I$. We can
prove this with a simple counterexample. Let
  $D = \set{x:2}$,
  $C_1 = t_-$, and
  $C_2 = t_-$.
Note that $D$ and $D \circ t_1 = \set{x:1}$ satisfy the invariant, so $I(D)$,
$I(C_1 @ D)$, and $I(C_2 @ D)$. However, $D \circ C_1 \circ C_2 = \set{x: 0}$
does not satisfy the invariant. Note that this is just a formalization of the
proof we saw in \figref{decrement}.
