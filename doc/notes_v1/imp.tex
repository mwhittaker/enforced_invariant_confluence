\section{\imp{}}\label{sec:imp}
So far, we've modelled a transaction $t: \var \rightharpoonup \ints$ as a
partial function from variables to integers. Intuitively, a transaction chooses
some \emph{fixed} subset of variables and increments and decrements each by
some \emph{fixed} amount. We'll call these kinds of transactions \emph{fixed
transactions}. While simple, fixed transactions are \emph{inexpressive}. We can
model, for example, a transaction that increments $x$ by 42 and decrements $y$
by 14, but we \emph{can't} model a transaction that increments $y$ by $x$ or a
transaction that increments $y$ by 1 whenever $x$ is positive and increments
$z$ whenever $x$ is negative.

In this subsection, we introduce a simple imperative programming language
\imp{}\footnote{That's IMP \cite{winskel1993formal} with an $\iinvariant{}$.}.
Instead of looking at fixed transactions, we'll look at \imp{}
transactions---transactions written in \imp{}---and ask ourselves the question
of how to determine whether a set of \imp{} transactions is \iconfluent{} with
respect to some invariant.

\begin{figure}[h]
  \centering
  \begin{gather*}
    \begin{array}{rrll}
      \impaexp\;
        a & ::= & k       & \text{\emph{constant}} \\
          & |   & x       & \text{\emph{variable}} \\
          & |   & read(x) & \text{\emph{database read}} \\
          & |   & a + a   & \text{\emph{addition}} \\
          & |   & a - a   & \text{\emph{subtraction}} \\
          & |   & - a     & \text{\emph{negation}} \\
      &&& \\
      \impbexp\;
        b & ::= & \textsf{true}  & \text{\emph{true}} \\
          & |   & \textsf{false} & \text{\emph{false}} \\
          & |   & a \odot a      & \text{\emph{linear constraint}} \\
          & |   & \lnot b        & \text{\emph{negation}} \\
          & |   & b \lor b       & \text{\emph{disjunction}} \\
          & |   & b \land b      & \text{\emph{conjunction}} \\
      &&& \\
      \impcom\;
        c  & ::= & \impskip        & \text{\emph{skip}} \\
           & |   & x := a          & \text{\emph{assignment}} \\
           & |   & add(x, a)       & \text{\emph{database adds}} \\
           & |   & c; c            & \text{\emph{sequence}} \\
           & |   & \impif{b}{c}{c} & \text{\emph{conditional}} \\
      &&& \\
      \odot & ::= & =    \vert
                    \neq \vert
                    <    \vert
                    \leq \vert
                    >    \vert
                    \geq & \text{\emph{comparator}}
    \end{array}
  \end{gather*}

  \caption{
    A grammar for \imp{}: a simple imperative transaction language inspired by
    IMP \cite{winskel1993formal} and $\mathcal{L}$ \cite{roy2015homeostasis}.
    The language consists of arithmetic expressions $\impaexp{}$, boolean
    expressions $\impbexp{}$, and commands $\impcom{}$.
  }
  \label{fig:imp-grammar}
\end{figure}

The syntax of \imp{} is defined by the grammar in \figref{imp-grammar}. The
semantics of \imp{} can be defined denotationally. That is, we can define a
denotational semantics $\denote{\cdot}$ where for an \imp{} program $c$,
$\denote{c}: (\var \to \ints) \to (\var \rightharpoonup \ints)$ is a function
from a database to a fixed transaction. Defining these semantics is left as an
exercise to the reader.  See \cite{cs6110sp2016lecture19} for guidance.
\todo{
  Define, explicitly, the denotational semantics if need be. Intuitively, we'd
  take the vanilla denotational semantics $\benote{\cdot}$ of IMP and then let
  $\denote{c}(D) = \benote{c}(D) - D$ where $D_2 - D_1$ is the fixed
  transaction taking database $D_1$ to $D_2$.
}

\emph{Applying an \imp{} transaction} $c$ to database state $D$, denoted by $D
\circ c$, is defined as $D \circ c \defeq D \circ (\denote{c}(D))$.  Note that
unlike fixed transaction application, \imp{} transaction application is
\emph{not} commutative! For example, consider the database $D=\set{(x, 0)}$ and
\imp{} transactions $c_1 = add(x, 1)$ and $c_2 = add(x, read(x))$.
\[
  D \circ c_1 \circ c_2
    = \set{(x, 1)} \circ c_2
    = \set{(x, 2)}
    \neq \set{(x, 1)}
    = \set{(x, 0)} \circ c_1
    = D \circ c_2 \circ c_1
\]

An \imp{} transaction chain $B$ created from a set of \imp{} transactions $T$
is a sequence of \imp{} transactions $c_1, \ldots, c_n$ where $c_i \in T$ for
$1 \leq i \leq n$. Each \imp{} transaction chain $B$ starting with database
state $D$ has a corresponding fixed transaction chain $C$ denoted
$\denote{B}(D)$. Let $D_i = D \circ c_1 \circ \cdots \circ c_i$. Then, the
corresponding fixed transaction chain is $\denote{c_1}(D_0), \ldots,
\denote{c_n}(D_{n-1})$. We say $B$ \emph{satisfies invariant} $I$ starting at
$D$, denoted $I(B@D)$, if the corresponding fixed transaction chain $C$
satisfies invariant $I$ starting at $D$.
%
For example, if $B = c_1, c_2, c_3$, then $\denote{B}(D) = t_1, t_2, t_3$ where
\begin{align*}
  t_1 &= \denote{c_1}(D), \\
  t_2 &= \denote{c_2}(D \circ t_1), \\
  t_3 &= \denote{c_3}(D \circ t_1 \circ t_2)
\end{align*}

We say a set of \imp{} transactions $T$ is \iconfluent{} with respect to
invariant $I$ if for all database states $D$, all \imp{} chains $B_1$ and $B_2$
created from $T$, and corresponding fixed transaction chains $C_1$ and $C_2$,
if $D$ satisfies $I$, $B_1$ satisfies $I$ starting at $D$, and $B_2$ satisfies
$I$ starting at $D$, then $D \circ C_1 \circ C_2$ satisfies $I$. We define
$k$-\iconfluence{} similar to how we did in \secref{oneisenough}.

Now that we've formalized what it means for a set of \imp{} transactions to be
\iconfluent{} with respect to some invariant, how do we determine
\iconfluence{} algorithmically? One initial hunch would be to apply the same
technique from \secref{solution} with slight modification. Unfortunately,
\thmref{imp-1-iconfluence} presents a challenge.

\begin{theorem}\label{thm:imp-1-iconfluence}
  For \imp{} transactions, $1$-\iconfluence $\centernot \implies$ \iconfluence.
\end{theorem}
\begin{proof}
  Consider the following invariant and \imp{} transactions:
  \begin{align*}
    I &\defeq
      (-2 \leq x \leq 2 \land y = 0) \lor (x = 1 \land 0 \leq y \leq 2) \\
    c_1 &\defeq \impif{x = 0 \land y = 0}{
      add(x, 1)
    }{\impif{x = 1 \land y = 0} {
      add(y, 1)
    }{
      \impskip
    }} \\
    c_2 &\defeq \impif{x = 0 \land y = 0}{
      add(x, -1)
    }{\impif{x = 1 \land y = 0} {
      add(y, 1)
    }{
      \impskip
    }}
  \end{align*}

  $T = \set{c_1, c_2}$ is $1$-\iconfluent{} with respect to $I$, but not
  $2$-\iconfluent{} with respect to $I$. Proving this is easiest if we
  interpret \imp{} transactions geometrically, as we did with fixed
  transactions.

  As before, a database over $n$ variables corresponds to a points in
  $\ints^n$, and an invariant $I$ corresponds to the set of all points (or
  databases) that satisfy it. For example, $I$ is visualized in \figref{iviz}.
  An \imp{} transaction $c$ can be thought of as vector-valued functions $c:
  \ints^n \to \ints^n$ that maps each point $D$ in $\ints^n$ to a
  $n$-dimensional vector $\denote{c}(D)$. For example, $c_1$ and $c_2$ are
  visualized in \figref{c1viz} and \figref{c2viz}.  The visualization of $c_1$
  has two vectors. The $(1, 0)$ vector originating at the origin corresponds to
  $c_1$ adding $1$ to $x$ whenever $x = 0 \land y = 0$. Similarly, the $(0, 1)$
  originating at $(1, 0)$ corresponds to $c_1$ adding $1$ to $y$ whenever $x =
  1 \land y = 0$.

  Using this geometric interpretation, it's easy to perform a brute-force check
  that $T$ is $1$-\iconfluent with respect to $I$. However, consider the
  transaction chains $B_1 = \set{c_1, c_2}$ and $B_2 = \set{c_2, c_1}$ starting
  in database state $D = (0, 0)$. We know that
  \begin{itemize}
    \item $D = (0, 0)$ satisfies $I$,
    \item $D \circ c_1 = (0, 1)$ satisfies $I$,
    \item $D \circ c_1 \circ c_2 = (1, 1)$ satisfies $I$,
    \item $D \circ c_2 = (-1, 0)$ satisfies $I$, and
    \item $D \circ c_2 \circ c_1 = (-1, 0)$ satisfies $I$,
  \end{itemize}
  but $D \circ C_1 \circ C_2 = D \circ (1, 0) \circ (0, 1) \circ (-1, 0) \circ
  (0, 0) = (0, 1)$ does not satisfy $I$. Therefore, $T$ is not 2-\iconfluent
  with respect to $I$ and is therefore not \iconfluent with respect to $I$.
\end{proof}

\todo{
  \thmref{imp-1-iconfluence} puts us in a bit of pickle. Think about how to get
  out of it. Maybe we check for $k$-\iconfluence for some $k$ and then use a
  reservation system to bound the amount of divergence between replicas? That
  doesn't seem to make sense for a system with high transaction throughput.
  Maybe there are some other tricks we can play?
}

\begin{figure}[h]
  \centering

  \newcommand{\gridandpoints}{
    \draw[ultra thick] (-2, 0) -- (2, 0);
    \draw[ultra thick] (0, 0) -- (0, 2);
    \draw (-2, 0) grid (2, 2);
    \foreach \x/\y in {-2/0, -1/0, 0/0, 1/0, 2/0, 1/1, 1/2} {
      \node[point] (\x-\y) at (\x, \y) {};
    }
  }

  \begin{subfigure}[c]{0.3\textwidth}
    \centering
    \begin{tikzpicture}
      \gridandpoints{}
    \end{tikzpicture}
    \caption{Visualization of $I$.}
    \label{fig:iviz}
  \end{subfigure}
  \begin{subfigure}[c]{0.3\textwidth}
    \centering
    \begin{tikzpicture}
      \gridandpoints{}
      \draw[vec] (0, 0) -- (1, 0);
      \draw[vec] (1, 0) -- (1, 1);
    \end{tikzpicture}
    \caption{Visualization of $c_1$.}
    \label{fig:c1viz}
  \end{subfigure}
  \begin{subfigure}[c]{0.3\textwidth}
    \centering
    \begin{tikzpicture}
      \gridandpoints{}
      \draw[vec] (0, 0) -- (-1, 0);
      \draw[vec] (1, 0) -- (1, 1);
    \end{tikzpicture}
    \caption{Visualization of $c_2$.}
    \label{fig:c2viz}
  \end{subfigure}

  \caption{
    A geometric interpretation of $I$, $c_1$, and $c_2$ from
    \thmref{imp-1-iconfluence}. $I$ corresponds to a subset of $\ints^2$. $c_1$
    and $c_2$ correspond to vector valued functions of type $\ints^2 \to
    \ints^2$.
  }
  \label{fig:icviz}
\end{figure}

