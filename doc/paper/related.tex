\section{Related Work}\label{sec:related}
\todo{Cite and summarize \cite{balegas2015putting} and \cite{roy2015homeostasis}}

\subsection{RedBlue Consistency}
In their 2012 OSDI publication, Li \etal{} introduce RedBlue consistency as an
intermediary consistency level between eventual consistency and strict
serializability \cite{li2012making}. Each transaction is labelled either red or
blue. A RedBlue consistent system serializes red transactions while running
blue transactions concurrently.

Li \etal{} also prove that if blue operations commute with all other
operations, then the system is convergent. That is, all replicas will converge
to the same result. To make transactions more commutative, they divide each
transaction into a \emph{generator transaction} and a \emph{shadow operation}.
This is similar to how an \imp{} transaction $c$ produces a fixed transaction
$\denote{c}(D)$. They also claim that 1-\isafety{} implies \iconfluence{}.

\subsection{Automating the Choice of Consistency Levels}
In Li \etal{}'s RedBlue consistency paper, programmers had to manually generate
shadow operations for each transaction and manually label each shadow operation
as either blue or red. In their 2014 USENIX ATC paper, Li \etal{} aim to
automate both these things \cite{li2014automating}.

First, we assume that transactions store all their state in a relational
database. We treat each relation in the database as a set CRDT, and we treat
each field in each relation as some sort of CRDT as well (e.g. LWW register,
G-Counter). Users specify which CRDTs to use by annotating their SQL
\texttt{CREATE} statements. Doing this, each transaction produces one shadow
operation for every possible sequence of database operations it invokes. For
example, assuming \texttt{x} is stored in the database as a CRDT, the following
code produces three shadow operations:
  the first is a no-op,
  the second decrements \texttt{x}, and
  the third increments \texttt{x}.
\begin{Verbatim}
  void generator(string s) {
      if (SHA1(s) == SOME_CONSTANT) {
          if (x >= 10) {
              x -= 10;
          }
      } else {
          x += 10;
      }
  }
\end{Verbatim}

Each shadow operation is then fed through a weakest precondition solver to
determine the weakest precondition over the parameters of the shadow operations
that imply the shadow operation is \isafe{}. These findings are cached. At
runtime, a generator operation is run to produce its corresponding shadow
operation. The shadow operation's weakest precondition is found in the cache,
and the shadow operation is checked against it. If it satisfies the weakest
precondition, then it is labelled blue. Otherwise, it is labelled red.

\subsection{'Cause I'm Strong Enough}
Li \etal{}'s work on automatically labelling transactions as red or blue uses
\isafety{}. As we've seen, this is one of the weaker properties that implies
\iconfluence{}. In their 2016 POPL paper, Gotsman \etal{} propose a much
stronger property that implies \iconfluence{} \cite{gotsman2016cause}.

Gotsman \etal{} first propose a general consistency model which subsumes a
bunch of hybrid consistency models like RedBlue consistency. They then present
versions of \istrength{} and  \istrengthstar{} generalized to their consistency
model. The proof rules are very complex.
