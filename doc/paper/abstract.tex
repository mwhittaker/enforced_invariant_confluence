\section{Abstract}
\emph{Strong consistency} allows programmers to ignore many of the complexities
of a distributed system, treating it as if it were running on a single machine.
However, enforcing strong consistency requires \emph{coordination}, and
coordination leads to unavailability (at worst) or
increased latency (at best)~\cite{gilbert2002brewer, abadi2012consistency}. Is
there some way to globally enforce an application's invariants without needing
to coordinate?

Yes! Well, sometimes.

Bailis et al. developed the notion of invariant confluence (\iconfluence{}) as
a necessary and sufficient condition for coordination
freedom~\cite{bailis2014coordination}. Intuitively, if a set of transactions is
invariant confluent with respect an invariant, then the invariant is guaranteed
to hold even when the transactions are run without coordination.  Technically,
a set of transactions is \iconfluent{} with respect to some invariant $I$ if
merging database states that satisfy $I$ always results in a database state
that also satisfies $I$.  Bailis characterized many common invariants (e.g.
uniqueness constraints, foreign key constraints), showing which could be
maintained without coordination. However, this characterization required
hand-written proofs. This research project aims to expand on Bailis' research
by providing a set of CRDTs~\cite{shapiro2011comprehensive,
shapiro2011conflict}, a language of invariants, and a language of transactions
that facilitates the ability for \iconfluence{} to be determined
algorithmically. An \iconfluence{} decision procedure would allow programmers
to guarantee that their programs can execute correctly without coordination.
