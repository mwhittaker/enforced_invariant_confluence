\section{\iconfluence{} Alternatives}
\begin{todoitemize}
  \item \todo{\istrength $\implies$ \istrengthstar?}
  \item \todo{\iconfluence $\implies$ 1-\iconvergence?}
  \item \todo{Find something between \isafety{} and \iconfluence.}
  \item \todo{1-iconfluence with derived txns to start}
  \item \todo{Explain alternatives.}
\end{todoitemize}

\thmref{imp-1-iconfluence} and \thmref{wimp-1-iconfluence} are depressing
results. Given a set of \imp{} or \wimp{} transactions $T$ and an invariant
$I$, we'd like to determine if $T$ is \iconfluent{} with respect to $I$ in much
the same way we did with fixed transactions in \secref{solution}. We'd like to
generate a formula $F$ (which we give to Z3) that is a tautology if and only if
$T$ is \iconfluent{} with respect to $I$. However, since 1-\iconfluence{} is
not equivalent to \iconfluence{}, it's not clear what a formula $F$ would even
look like.

Clearly, we're stumped. Algorithmically determining \iconfluence{} seems to be
too hard of a problem. However, not all hope is lost. Consider an analogy.
Determining whether a sequence of reads and writes is \emph{view serializable}
is a really hard problem. In fact, it is NP-complete. To circumvent the
difficulty of the problem, we introduce \emph{conflict serializability}.
Determining whether a history is conflict serializable is pretty easy, and
conflict serializability implies (but is not equivalent to) view
serializability.

This means that we can quickly determine whether a history is conflict
serializable. If it is, then it is \emph{definitely} view serializable. If it
is not, well, then it \emph{might be} view serializable or it \emph{might not
be}. Conflict serializability merely approximates view serializability. There
are some view serializable histories which are not conflict serializable, but
the two are close enough that we don't lose any sleep over it.

Like determining view serializability, determining \iconfluence{} is a hard
problem. In this section we present multiple analogues of conflict
serializability: efficiently decidable properties of a set of transactions and
an invariant that imply (but are not equivalent to) \iconfluence{}. We might
not be able to determine \iconfluence{}, but we might be able to find a tight
approximation.

We consider the following list of properties. Some properties are visualized
with state diagrams in \figref{iconfluence-alternatives-diagrams}. The
relationship between the different properties is illustrated in
\figref{iconfluence-alternatives-implications} and proved formally in
\appref{alternativesproofs}.

\begin{itemize}
  \item \textbf{0-\isafety.}
    $\forall D, D', c.\>
      I(D) \land
      I(D') \implies
      I(D \circ \denote{c}(D'))$

  \item \textbf{k-\isafety.}
    $\forall D, D', B, C, c.\>
       C = \denote{B}(D) \land
       |B| \geq k \land
       I(B @ D') \implies
       I(C @ D) \land
       I(D \circ C \circ \denote{c}{D' \circ C})$

  \item \textbf{\isafety.}
    $\forall D, D', B, C, c.\>
       C = \denote{B}(D) \land
       I(B @ D') \implies
       I(D \circ C \circ \denote{c}(D' \circ C))$

  \item \textbf{\ipreservation.} $\forall D, c.\>
       I(D) \implies
       I(D \circ \denote{c}(D))$

  \item \textbf{\iconfluence.}
    $\forall D, B_1, B_2, C_1, C_2. \>
       C_1 = \denote{B_1}(D) \land
       C_2 = \denote{B_2}(D) \land
       I(B_1 @ D) \land
       I(B_2 @ D) \implies
       I(D \circ C_1 \circ C_2)$

  \item \textbf{k-\iconfluence.}
    $\forall D, B_1, B_2, C_1, C_2. \>
       C_1 = \denote{B_1}(D) \land
       C_2 = \denote{B_2}(D) \land
       |B_1|, |B_2| \leq k \land
       I(B_1 @ D) \land
       I(B_2 @ D) \implies
       I(D \circ C_1 \circ C_2)$

  \item \textbf{1-\iconfluence.}
    $\forall D, c_1, c_2.\>
       I(D) \land
       I(D \circ \denote{c_1}(D)) \land
       I(D \circ \denote{c_2}(D)) \implies
       I(D \circ \denote{c_1}(D) \circ \denote{c_2}(D))$

  \item \textbf{\istrengthstar{} \cite{gotsman2016cause}.}
    \leqnomode
    \begin{flalign*}
      & \exists G_0 \subseteq \dbs \times \dbs.& \\
      & G_0(I) \subseteq I \land{} & \tag*{S2.} \\
      & \forall c, D, D'.\>
          D \in I \land
          (D, D') \in G_0^* \implies
          (D', D' \circ \denote{c}(D)) \in G_0 & \tag*{S3.}
    \end{flalign*}
    \reqnomode
    where $I$ is viewed as a set of database states.

  \item \textbf{\istrength{} \cite{gotsman2016cause}.}
    \leqnomode
    \begin{flalign*}
      & \exists G_0 \subseteq \dbs \times \dbs. & \\
      & G_0(I) \subseteq I \land{} & \tag*{T2.} \\
      & \forall c.\>
        \exists P_1, \ldots, P_n \subseteq \dbs.\>
        \exists Q_1, \ldots, Q_n \subseteq \dbs.\> & \tag*{T3.} \\
      & \quad I \subseteq \cup_{i=1}^n P_i \land {} & \tag*{T3b.} \\
      & \quad \forall i \in \set{1, \ldots, n}.\> & \\
      & \quad \quad P_i \subseteq Q_i \land {} & \tag*{T3c.} \\
      & \quad \quad G_0(Q_i) \subseteq Q_i \land {} & \tag*{T3d.} \\
      & \quad \quad Q_i \times \denote{c}(P_i)(Q_i) \subseteq G_0 & \tag*{T3e.}
    \end{flalign*}
    \reqnomode
    where $I$ is viewed as a set of database states.

  \item \textbf{1-\iconvergence.}
    $\forall D, D', D'', c', c''.\>
      I(D) \land I(D') \land I(D'') \land
      I(D \circ \denote{c'}(D')) \land
      I(D \circ \denote{c''}(D'')) \implies
      I(D \circ \denote{c'}(D') \circ \denote{c''}(D''))$
\end{itemize}

\tikzstyle{automatable}=[fill=green!20]
\tikzstyle{intractable}=[fill=red!20]
\begin{figure}[h]
  \centering

  \begin{tikzpicture}
    \node[automatable] (0-isafety) at (0, 0) {0-\isafety};
    \node[automatable, below=of 0-isafety]     (1-isafety)      {1-\isafety};
    \node[automatable, below=of 1-isafety]     (k-isafety)      {$k$-\isafety};
    \node[intractable, right=of 0-isafety]     (istrengthstar)  {\istrengthstar};
    \node[automatable, above=of istrengthstar] (ipreservation)  {\ipreservation};
    \node[intractable, right=of istrengthstar] (iconfluence)    {\iconfluence};
    \node[automatable, below=of iconfluence]   (k-iconfluence)  {$k$-\iconfluence};
    \node[automatable, below=of k-iconfluence] (1-iconfluence)  {1-\iconfluence};
    \node[automatable, below=of istrengthstar] (istrength)      {\istrength};
    \node[automatable, below=of istrength]     (1-iconvergence) {1-\iconvergence};

    \draw[ultra thick, impl] (0-isafety) -> (istrengthstar);
    \draw[ultra thick, impl] (istrengthstar) -> (iconfluence);
    \draw[ultra thick, impl] (0-isafety) -> (istrength);
    \draw[impl] (iconfluence) -> (k-iconfluence);
    \draw[impl] (k-iconfluence) -> (1-iconfluence);
    \draw[impl] (istrengthstar) -> (ipreservation);
    \draw[impl] (0-isafety) -> (1-isafety);
    \draw[impl] (1-isafety) -> (k-isafety);
    \draw[impl] (istrength) -> (istrengthstar);
    \draw[impl] (0-isafety) -> (1-iconvergence);
    \draw[impl] (1-iconvergence) -> (iconfluence);
  \end{tikzpicture}

  \caption{
    An illustration of the relationship between various \iconfluence{}
    alternatives. We have proven bold implications in both directions. For
    example, we have proven both \istrengthstar{} $\implies$ \iconfluence{} and
    \iconfluence{} $\protect \centernot \implies$ \istrengthstar{}. Conditions
    shaded green can be automatically determined using an SMT solver.
    Conditions shaded red are not easy to automatically determine.
  }
  \label{fig:iconfluence-alternatives-implications}
\end{figure}

\begin{figure}[h]
  \centering

  \begin{subfigure}[b]{0.3\textwidth}
    \centering
    \begin{tikzpicture}[scale=2]
      \inode[label=above:$D$]{d1}{0, 0}
      \inode[label=above:$D'$]{d2}{1, 0}
      \dnode{b1}{0, -1}
      \nnode{b2}{1, -1}
      \draw[txn] (d2) -- (b2) node[midway, right]{$\denote{c}(D')$};
      \draw[dtxn] (d1) -- (b1) node[midway, left]{$\denote{c}(D')$};
    \end{tikzpicture}
    \caption{0-\isafety}
    \label{fig:0-isafety}
  \end{subfigure}%
  \begin{subfigure}[b]{0.3\textwidth}
    \centering
    \begin{tikzpicture}[scale=2]
      \inode[label=above:$D$]{a}{0, 0}
      \dnode{b}{0, -1}
      \draw[txn] (a) -- (b) node[midway, left]{$\denote{c}(D)$};
    \end{tikzpicture}
    \caption{\ipreservation}
    \label{fig:ipreservation}
  \end{subfigure}%
  \begin{subfigure}[b]{0.3\textwidth}
    \centering
    \begin{tikzpicture}[scale=2]
      \inode[label=above:$D$]{a}{0, 0}
      \inode[label=left:$D_{c}$]{l1}{$(a) + (240:1)$}
      \inode[label=right:$D_{d}$]{r1}{$(a) + (300:1)$}
      \dnode{b}{$(l1) + (300:1)$}
      \draw[txn] (a) -- (l1) node[midway, left] {$\denote{c}(D)$};
      \draw[txn] (a) -- (r1) node[midway, right] {$\denote{d}(D)$};
      \draw[txn] (l1) -- (b) node[midway, left] {$\denote{d}(D)$};
      \draw[txn] (r1) -- (b) node[midway, right] {$\denote{c}(D)$};
    \end{tikzpicture}
    \caption{1-\iconfluence}
    \label{fig:1-iconfluence}
  \end{subfigure}

  \vspace{1cm}

  \begin{subfigure}[b]{0.5\textwidth}
    \centering

    \begin{tikzpicture}[scale=1.5]
      \inode[label=above:$D_0$]{l1}{0, 4}
      \dnode[label=left:$D_1$]{l2}{0, 3}
      \dnode[label=left:$D_2$]{l3}{0, 2}
      \dnode[label=left:$D_3$]{l4}{0, 1}
      \dnode{l5}{0, 0}

      \inode[label=above:$D_0'$]{r1}{1, 4}
      \inode[label=right:$D_1'$]{r2}{1, 3}
      \inode[label=right:$D_2'$]{r3}{1, 2}
      \inode[label=right:$D_3'$]{r4}{1, 1}
      \nnode{r5}{1, 0}

      \draw[txn] (r1) -- (r2) node[midway, right]{$\denote{c_1}(D_0')$};
      \draw[txn] (r2) -- (r3) node[midway, right]{$\denote{c_2}(D_1')$};
      \draw[txn] (r3) -- (r4) node[midway, right]{$\denote{c_3}(D_2')$};
      \draw[txn] (r4) -- (r5) node[midway, right]{$\denote{c_4}(D_3')$};

      \draw[txn] (l1) -- (l2) node[midway, left]{$\denote{c_1}(D_0')$};
      \draw[txn] (l2) -- (l3) node[midway, left]{$\denote{c_2}(D_1')$};
      \draw[txn] (l3) -- (l4) node[midway, left]{$\denote{c_3}(D_2')$};
      \draw[txn] (l4) -- (l5) node[midway, left]{$\denote{c_4}(D_3')$};
    \end{tikzpicture}

    \caption{$k$-\isafety, $k=3$}
    \label{fig:k-isafety}
  \end{subfigure}%
  \begin{subfigure}[b]{0.5\textwidth}
    \centering

    \begin{tikzpicture}[scale=1.5]
      \inode[label=above:$D$]{a}{0, 0}
      \inode[label=left:$D_{c_1}$]{l1}{$(a) + (240:1)$}
      \inode[label=left:$D_{c_2}$]{l2}{$(l1) + (240:1)$}
      \inode[label=left:$D_{c_3}$]{l3}{$(l2) + (240:1)$}

      \inode[label=right:$D_{d_1}$]{r1}{$(a) + (300:1)$}
      \inode[label=right:$D_{d_2}$]{r2}{$(r1) + (300:1)$}
      \inode[label=right:$D_{d_3}$]{r3}{$(r2) + (300:1)$}

      \dnode{b}{$(l3) + (300:3)$}

      \draw[txn] (a) -- (l1) node[midway, left] {$\denote{c_1}(D)$};
      \draw[txn] (l1) -- (l2) node[midway, left] {$\denote{c_2}(D_{c_1})$};
      \draw[txn] (l2) -- (l3) node[midway, left] {$\denote{c_3}(D_{c_2})$};
      \draw[txn] (a) -- (r1) node[midway, right] {$\denote{d_1}(D)$};
      \draw[txn] (r1) -- (r2) node[midway, right] {$\denote{d_2}(D_{d_1})$};
      \draw[txn] (r2) -- (r3) node[midway, right] {$\denote{d_3}(D_{d_2})$};
      \draw[txn] (l3) -- (b) node[midway, left] {$\denote{c_1, c_2, c_3}(D)$};
      \draw[txn] (r3) -- (b) node[midway, right] {$\denote{d_1, d_2, d_3}(D)$};
    \end{tikzpicture}

    \caption{\iconfluence}
    \label{fig:iconfluence}
  \end{subfigure}

  \begin{subfigure}[b]{0.49\textwidth}
    \centering
    \begin{tikzpicture}[scale=2]
      \inode[label=above:$D$]{s2a}{0, 1}
      \dnode{s2b}{0, 0}
      \draw[gtxn] (s2a) -- (s2b) node[midway, left]{$\denote{c}(D)$};

      \inode[label=above:$D$]{s3a}{1, 1}
      \nnode[]{s3b}{1, 0}
      \nnode[label=above:$D'$]{s3c}{2, 1}
      \nnode[]{s3d}{2, 0}
      \draw[txn] (s3a) -- (s3b) node[midway, left] {$\denote{c}(D)$};
      \draw[dgtxn] (s3c) -- (s3d) node[midway, right] {$\denote{c}(D)$};
      \draw[gtxn] (s3a) -- (s3c) node[midway, above] {$G_0^*$};
    \end{tikzpicture}
    \caption{\istrengthstar}
    \label{fig:istrengthstar}
  \end{subfigure}
  \begin{subfigure}[b]{0.49\textwidth}
    \centering
    \begin{tikzpicture}[scale=2]
      \inode[label=above:$D$]{D}{0, 0}
      \inode[]{l1}{$(D) + (240:1)$}
      \inode[]{r1}{$(D) + (300:1)$}
      \dnode{m}{$(l1) + (300:1)$}

      \inode[label=above:$D'$]{D'}{$(r1) + (1.5, 0.5)$}
      \nnode[]{x}{$(D') + (0, -1)$}

      \inode[label=above:$D''$]{D''}{$(D') + (1, 0)$}
      \nnode[]{y}{$(D'') + (0, -1)$}

      \draw[txn] (D) -- (l1) node[midway, left] {$\denote{c'}(D')$};
      \draw[txn] (D) -- (r1) node[midway, right] {$\denote{c''}(D'')$};
      \draw[txn] (l1) -- (m) node[midway, left] {$\denote{c''}(D'')$};
      \draw[txn] (r1) -- (m) node[midway, right] {$\denote{c'}(D')$};

      \draw[txn] (D') -- (x) node [midway, left] {$\denote{c'}(D')$};

      \draw[txn] (D'') -- (y) node [midway, left] {$\denote{c''}(D'')$};
    \end{tikzpicture}
    \caption{1-\iconvergence}
    \label{fig:1-iconvergence}
  \end{subfigure}

  \caption{
    An illustration of \iconfluence{} alternatives. \todo{Describe drawings.}
  }
  \label{fig:iconfluence-alternatives-diagrams}
\end{figure}
