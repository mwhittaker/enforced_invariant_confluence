\section{\iconfluence{}, Geometrically}\label{sec:geometry}
\newcommand{\inva}{x + y \geq 0}
\newcommand{\invb}{x - y \leq 0}
\newcommand{\invc}{y \geq 1}
\newcommand{\invd}{y \geq 3}
\newcommand{\inv}{(\inva \land \invb \land \invc) \lor (\invd)}

Sometimes---like with the two examples from the previous section---we can
eyeball whether a transaction is \iconfluent{} with respect to an invariant.
Other times, not so much. For example, which transactions are \iconfluent{}
with respect to the invariant $\inv$? It's not immediately clear.
Surprisingly, we can interpret the question ``Is $T$ \iconfluent{} with respect
to $I$?'' geometrically, and doing so will make answering the question much
easier!

Consider databases over $n$ variables $\var = \set{x_1, \ldots, x_n}$. We can
imagine a database $D$ as the $n$-dimensional point $(D(x_i), \ldots, D(x_n))$.
For example, if $\var = \set{x, y}$, then the database $\set{x:1, y:2}$
corresponds to the two-dimensional point $(1, 2)$.

We can imagine $I$ as the set of points (or databases) in $\ints^n$ that
satisfy $I$. For example, the invariant $x + y = 2$ corresponds to the set of
two-dimensional points $(k_x, k_y)$ where $k_x + k_y = 2$. This corresponds to
the set of databases $\set{x:k_x, y:k_y}$ where $k_x + k_y = 2$.

Moreover, if we canonicalize $I$ by eliminating negations and converting to
conjunctive normal form, then the set of points satisfying $I$ is the
intersection (for $\land)$ and union (for $\lor$) of the solutions to a set of
linear equations and inequalities, which is always straightforward to draw. For
example, the set of points satisfying $\inv$ is shown in
\figref{geometric-example}.

\begin{figure}
  \newenvironment{griddedpic}{
    \begin{tikzpicture}[scale=0.4]
      \clip (-4, -1) rectangle (4, 4);
  }{
      \draw (-4, -1) grid (4, 4);
      \draw[ultra thick] (-4, 0) -- (4, 0);
      \draw[ultra thick] (0, -1) -- (0, 4);
    \end{tikzpicture}
  }

  \begin{subfigure}[b]{0.24\textwidth}
    \centering
    \begin{griddedpic}
      \fill[red, opacity=0.4]
        (-10, 10) -- (10, -10) -- (10, 10) -- cycle;
    \end{griddedpic}
    \caption{$\inva$}
  \end{subfigure}
  \begin{subfigure}[b]{0.24\textwidth}
    \centering
    \begin{griddedpic}
      \fill[blue, opacity=0.4]
        (-10, -10) -- (10, 10) -- (-10, 10) -- cycle;
    \end{griddedpic}
    \caption{$\invb$}
  \end{subfigure}
  \begin{subfigure}[b]{0.24\textwidth}
    \centering
    \begin{griddedpic}
      \fill[green, opacity=0.4]
        (-10, 1) -- (10, 1) -- (10, 10) -- (-10, 10) -- cycle;
    \end{griddedpic}
    \caption{$\invc$}
  \end{subfigure}
  \begin{subfigure}[b]{0.24\textwidth}
    \centering
    \begin{griddedpic}
      \fill[orange, opacity=0.4]
        (-10, 3) -- (10, 3) -- (10, 10) -- (-10, 10) -- cycle;
    \end{griddedpic}
  \caption{$\invd$}

  \end{subfigure}
  \begin{subfigure}[b]{\textwidth}
    \centering
    \begin{griddedpic}
      \fill[purple, opacity=0.4]
        (-1, 1) -- (-3, 3) -- (-10, 3) -- (-10, 10) -- (10, 10) -- (10, 3) --
        (3, 3) -- (1, 1) -- cycle;
    \end{griddedpic}
    \caption{$\inv$}
  \end{subfigure}

  \caption{An illustration of the solution to $\inv$.}
  \label{fig:geometric-example}
\end{figure}

Next, we can imagine a transaction $t = \set{x_1:k_{x_1}, \ldots, x_n:k_{x_n}}$
as an $n$-dimensional vector $(k_{x_1}, \ldots, k_{x_n})$. For example, given
the invariant $x + y = 2$, the transaction $\set{x:2, y:4}$ corresponds to the
vector $(2, 4)$. Applying a transaction $t$ to a database $D$ can be understood
as adding the vector representing $t$ to the point representing $D$.  For
example, applying the transaction $\set{x:2, y:4}$ to the database $\set{x:10,
y:20}$ corresponds to adding the vector $(2, 4)$ to the point $(10, 20)$
bringing us to the point $(12, 24)$.

Now, the question of whether a set of transactions $T$ is \iconfluent{} with
respect to invariant $I$ has been reduced to the following question. Starting
at an arbitrary point $D$ in the solution space of $I$, if we can add vectors
$t_1, t_2, \ldots, t_m$ to $D$ and vectors $u_1, u_2, \ldots, u_n$ to $D$ all
while staying in the solution space of $I$, then are we guaranteed that $D +
t_1 + t_2 + \ldots + t_m + u_1 + u_2 + \ldots + u_n$ is in the solution space
of $I$?

Turning again to our example in \figref{geometric-example}, it is now clear
that any transaction $\set{x:k_x, y:k_y}$ where $y \geq 0$ and $|x| \leq y$ is
\iconfluent{} with respect to $\inv$. These correspond to the vectors $(k_x,
k_y)$ that point up and not too far left or right.
