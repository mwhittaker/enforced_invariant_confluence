\section{\wimp{}}\label{sec:wimp}
We began this paper by looking at fixed transactions and concluded in
\thmref{one-is-enough} that if a set of fixed transactions is 1-\iconfluent{}
with respect to an invariant $I$, then it is also \iconfluent{}. Then in
\secref{imp}, we enriched transactions from simple fixed partial functions
to full-blown programs written in \imp{}. Unfortunately, this increase in
expressiveness came at the cost of the nice fact that 1-\iconfluence{} is
equivalent to \iconfluence{}, as we saw in \thmref{imp-1-iconfluence}.

Perhaps \imp{} is too powerful of a transaction language. In particular, it has
two powerful features that fixed transactions do not have. First, \imp{} has
conditionals allowing the control flow of an \imp{} program to depend on the
database against which it is run. Second, even without conditionals, the
behavior of an \imp{} transaction can still depend on the database state
against which it is run. For example, the behavior of the transaction $add(x,
read(x))$ depends on the initial value of $x$, even though the transaction does
not use a conditional.

Perhaps if we strip \imp{} of some its features, we'll regain the equivalence
of 1-\iconfluence{} and \iconfluence{}. That is, perhaps there is an
intermediate language more powerful than fixed transactions but weaker than
\imp{} for which 1-\iconfluence{} is equivalent to \iconfluence{}.

In this subsection, we'll consider one such language, \wimp{}: a weakened
version of \imp{}. Then, we'll conclude that for \wimp{}---as for
\imp{}---1-\iconfluence{} does not imply \iconfluence{}.

Formally, \wimp{} is \imp{} without variables, boolean expressions, skip,
assignment, and conditionals. Moreover, each database variable can be added to
at most once. Essentially, \wimp{} programs get to add a single arithmetic
expression to each database variable. While weaker than \imp{} programs,
\wimp{} programs are more powerful than fixed transactions because arithmetic
expressions can include database reads and therefore \wimp{} programs can
depend on the initial state of the database. In fact, if we remove database
reads from \wimp{}, we get a language of fixed transactions.

Since \wimp{} is a subset of \imp{}, all the definitions from \imp{} carry over
directly for \wimp{}. Also note that like \imp{}, \wimp{} transaction
application is not commutative.

As mentioned above, \wimp{} is not weak enough to enjoy the equivalence of
1-\iconfluence{} and \iconfluence{}: a fact we now prove.

\begin{theorem}\label{thm:wimp-1-iconfluence}
  For \wimp{} transactions, $1$-\iconfluence $\centernot \implies$ \iconfluence.
\end{theorem}
\begin{proof}
  Consider the following invariant and \wimp{} transactions:
  \begin{align*}
    I   &\defeq x \neq 0 \lor y = 0 \\
    c_1 &\defeq y.add(1) \\
    c_2 &\defeq x.add(read(x) + 1) \\
    c_3 &\defeq x.add(read(x) - 1)
  \end{align*}

  As usual, we can visualize invariants and transactions geometrically, as
  shown in \figref{wimp-icviz}. We can perform a brute-force case analysis over
  all pairs of transactions and all points to show $T = \set{c_1, c_2, c_3}$ is
  1-\iconfluent{} with respect to $I$. This is left as exercise to the reader.

  However, consider the \wimp{} chains $B_1 = \set{c_2, c_1}$ and $B_2 =
  \set{c_3, c_1}$ in database $D = (0, 0)$. We know that
  \begin{itemize}
    \item $D = (0, 0)$ satisfies $I$,
    \item $D \circ c_2 = (1, 0)$ satisfies $I$,
    \item $D \circ c_2 \circ c_1 = (1, 1)$ satisfies $I$,
    \item $D \circ c_3 = (-1, 0)$ satisfies $I$, and
    \item $D \circ c_3 \circ c_1 = (-1, 1)$ satisfies $I$,
  \end{itemize}
  but $D \circ C_1 \circ C_2 = D \circ (1, 0) \circ (0, 1) \circ (-1, 0) \circ
  (0, 1) = (0, 2)$ does not satisfy $I$. Therefore, $T$ is not 2-\iconfluent
  with respect to $I$ and is therefore not \iconfluent with respect to $I$.
\end{proof}

\todo{
  We're in a pickle again! I think this result also might suggest that the
  claims in \cite{balegas2015putting} are wrong? More likely, I'm
  misunderstanding the paper.
}

\newcommand{\width}{2}
\newcommand{\smallwidth}{1}
\newcommand{\height}{2}

\newcommand{\gridxy}{
  \draw[ultra thick] (-\width, 0) -- (\width, 0);
  \draw[ultra thick] (0, -\height) -- (0, \height);
  \draw (-\width, -\height) grid (\width, \height);
}

\newcommand{\forxy}[1]{
  \foreach \x in {-\width, ..., -1} {
    \foreach \y in {-\height, ..., \height} { #1 } }
  \foreach \x in {1, ..., \width} {
    \foreach \y in {-\height, ..., \height} { #1 } }
  \foreach \x in {0} {
    \foreach \y in {0} { #1 } }
}

\newcommand{\forxysmall}[1]{
  \foreach \x in {-\smallwidth, ..., -1} {
    \foreach \y in {-\height, ..., \height} { #1 } }
  \foreach \x in {1, ..., \smallwidth} {
    \foreach \y in {-\height, ..., \height} { #1 } }
  \foreach \x in {0} {
    \foreach \y in {0} { #1 } }
}

\begin{figure}[h]
  \centering
  \newcommand{\thescale}{0.7}

  \begin{subfigure}[b]{0.24\textwidth}
    \centering
    \begin{tikzpicture}[scale=\thescale]
      \gridxy{}
      \forxy{\node[point] (\x-\y) at (\x, \y) {};}
    \end{tikzpicture}
    \caption{Visualization of $I$.}
    \label{fig:wimp-iviz}
  \end{subfigure}
  \begin{subfigure}[b]{0.24\textwidth}
    \centering
    \begin{tikzpicture}[scale=\thescale]
      \gridxy{}
      \forxy{\node[point] (\x-\y) at (\x, \y) {};}
      \forxy{\draw[vec] (\x, \y) -- (\x, \y + 1);}
    \end{tikzpicture}
    \caption{Visualization of $c_1$.}
    \label{fig:wimp-c1viz}
  \end{subfigure}
  \begin{subfigure}[b]{0.24\textwidth}
    \centering
    \begin{tikzpicture}[scale=\thescale]
      \gridxy{}
      \forxy{\node[point] (\x-\y) at (\x, \y) {};}
      \forxysmall{\draw[vec] (\x, \y) -- (\x + \x + 1, \y);}
    \end{tikzpicture}
    \caption{Visualization of $c_2$.}
    \label{fig:wimp-c2viz}
  \end{subfigure}
  \begin{subfigure}[b]{0.24\textwidth}
    \centering
    \begin{tikzpicture}[scale=\thescale]
      \gridxy{}
      \forxy{\node[point] (\x-\y) at (\x, \y) {};}
      \forxysmall{\draw[vec] (\x, \y) -- (\x + \x - 1, \y);}
    \end{tikzpicture}
    \caption{Visualization of $c_3$.}
    \label{fig:wimp-c3viz}
  \end{subfigure}

  \caption{
    A geometric interpretation of $I$, $c_1$, $c_2$, and $c_3$ from
    \thmref{wimp-1-iconfluence}. Note that the visualizations include only a
    subset of the points satisfying $I$ and that not all vectors are drawn.
  }
  \label{fig:wimp-icviz}
\end{figure}

