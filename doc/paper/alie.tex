\section{\iconfluence{}, For Real This Time}\label{sec:alie}
In \secref{formalism}, we defined a set of fixed transactions $T$ to be
\emph{\iconfluent{}} with respect to invariant $I$ if for all database states
$D$ and chains $C_1$ and $C_2$ created from $T$, if $D$ satisfies $I$, $C_1$
satisfies $I$ starting at $D$, and $C_2$ satisfies $I$ starting at $D$, then $D
\circ C_1 \circ C_2$ satisfies $I$. In \secref{imp}, we used an almost
identical definition for \imp{} transactions.

Regrettably, I have to admit that I've been lying to you this entire time. This
definition of \iconfluence{} is wrong! Specifically, it is a simplification of
the original definition originally proposed by Bailis
\etal{}~\cite{bailis2014coordination}. In light of our little white lie, we'll
rename \iconfluence{} to \liconfluence{}\footnote{pronounced
``lie-confluence''} and refer to Bailis \etal's properly defined property as
\iconfluence{} from now on.

First, let's review Bailis \etal's definition of \iconfluence{}. We begin with
a set of transactions $T$, an invariant $I$, and an arbitrary database state
$D$ that satisfies the invariant. We then consider two arbitrary database
states $D_1$ and $D_2$ that were derived from $D$ by applying an arbitrary
number of transactions or merges such that the invariant was always maintained.
If the merge of $D_1$ and $D_2$ is guaranteed to satisfy the invariant, then
$T$ is \iconfluent{} with respect to $I$. A state diagram illustrating this
definition is given in \figref{iconfluence-example} where $D_d$ and $D_c$ play
the role of $D_1$ and $D_2$.

\begin{figure}[h]
  \centering
  \begin{tikzpicture}[scale=1]
    \inode[label=above:$D$]{D}{0,0};
    \inode[label=left:$D_a$]{a}{-2,-1};
    \inode[label=right:$D_b$]{b}{0,-1};
    \inode[label=right:$D_c$]{c}{2,-1};
    \inode[label=left:$D_d$]{d}{-1,-2};
    \dnode[label=left:$D_e$]{e}{0,-3};
    \draw[txn] (D) -- (a);
    \draw[txn] (D) -- (b);
    \draw[txn] (D) -- (c);
    \draw[txn] (a) -- (d);
    \draw[txn] (b) -- (d);
    \draw[txn] (d) -- (e);
    \draw[txn] (c) -- (e);
  \end{tikzpicture}
  \caption{A state diagram illustrating \iconfluence{}.}
  \label{fig:iconfluence-example}
\end{figure}

Clearly, \iconfluence{} implies \liconfluence{}; that is, if $T$ is
\iconfluent{} with respect to $I$, then it is also \liconfluent{} with respect
to $I$. However, \liconfluence{} does not imply \iconfluence{}. As a simple
counterexample, let $I \defeq 0 \leq x \leq 2$ and $t_1 =
\impif{x=0}{add(x,1)}{}$. By inspection, $\set{t_1}$ is \liconfluent{} with
respect to $I$. However, letting $D=\set{x:0}$ and each transaction be $t_1$ in
\figref{iconfluence-example}, we have:
\begin{align*}
  D   &= \set{x:0} \\
  D_a =
  D_b =
  D_c &= \set{x:1} \\
  D_d &= \set{x:2} \\
  D_e &= \set{x:3}
\end{align*}
$D$, $D_a$, $D_b$, $D_c$, and $D_d$ all satisfy $I$, but $D_e$ does not. Thus,
$T$ is not \iconfluent{} with respect to $I$.

This is an unfortunate result. It says that for a set of transactions $T$ and
invariant $I$, if \liconfluence{} does not imply \iconfluence{}, then all our
efforts to determine \liconfluence{} are for naught. However, not all hope is
lost. Perhaps for some transactions $T$ and invariants $I$, \liconfluence{} is
equivalent to \iconfluence{}. When this is the case, we can check whether $T$
is \iconfluent{} with respect to $I$ by checking if it is \liconfluent{} with
respect to $I$.

\subsection{\ireplayability{}}
So when is \liconfluence{} equivalent to \iconfluence{}? Let's take a look at a
\emph{wrong} first guess.

Look at $D$, $D_a$, $D_b$, and $D_d$ in \figref{iconfluence-example}. If there
existed some \imp{} transaction chain $B$ such that $D \circ B = D_d$ and
$I(B@D)$, then we could remove the split and merge created by $D_a$ and $D_b$
and replace it with the chain $B$ from $D$ to $D_d$. This is shown in
\figref{mergeremoved}. For simplicity, we've assumed $B = c_1, c_2, c_3$. Now
that \figref{mergeremoved} doesn't contain any intermediate merges,
\liconfluence{} is equivalent to \iconfluence{}.

\begin{figure}[h]
  \centering
  \begin{tikzpicture}[yscale=1.2]
    \inode[label=above:$D$]{D}{0,0};
    \inode[]{a}{-1,-1};
    \inode[]{b}{-1,-2};
    \inode[label=left:$D_d$]{d}{-1,-3};
    \inode[label=right:$D_c$]{c}{1,-3};
    \dnode[label=left:$D_e$]{e}{0,-4};
    \draw[txn] (D) -- node[pos=0.6,above]{$c_1$} (a);
    \draw[txn] (a) -- node[midway,left]{$c_2$} (b);
    \draw[txn] (b) -- node[midway,left]{$c_3$} (d);
    \draw[txn] (d) -- (e);
    \draw[txn] (D) -- (c);
    \draw[txn] (c) -- (e);
  \end{tikzpicture}
  \caption{A state diagram illustrating \iconfluence{}.}
  \label{fig:mergeremoved}
\end{figure}

This suggests the following sufficient condition for \liconfluence{} to be
equivalent to \iconfluence{}. For a set of transactions $T$ and invariant $I$,
if for all databases $D$, chains $B_1$ and $B_2$ where $I(B_1@D)$, $I(B_2@D)$,
and $I(D \circ \denote{B_1}(D) \circ \denote{B_2}(D))$ there exists a chain
$B_3$ such that $D \circ B_3 = D \circ \denote{B_1}(D) \circ \denote{B_2}(D)$
and $I(B_3@D)$, then $T$ is \liconfluent{} with respect to $I$ if and only if
it is \iconfluent{} with respect to $I$. Intuitively, we can repeatedly replace
merges of chains $B_1$ and $B_2$ with a single chain $B_3$. Repeating this
process will leave us with a state diagram without any intermediate merges.
Given such an state diagram, we can check for \liconfluence{}. If a set of
transactions $T$ and an invariant $I$ satisfies this property, we'll say that
$T$ is \ireplayable{} with respect to $I$.

Unfortunately, this line of reasoning is wrong. Consider the state diagram
shown in \figref{ireplayability-notgoodenough}. Here, we can't use
\ireplayability{} to reduce the diagram to a simple diamond.

\begin{figure}[h]
  \centering
  \begin{tikzpicture}[yscale=1.2]
    \inode[]{a}{ 1, 5};
    \inode[]{b}{ 0, 4};
    \inode[]{c}{ 2, 4};
    \inode[]{d}{ 0, 3};
    \inode[]{e}{ 2, 3};
    \inode[]{f}{ 0, 2};
    \dnode[]{g}{ 2, 1};
    \foreach \x/\y in {a/b, a/c, b/d, c/d, c/e, d/f, e/g, f/g} {
      \draw[txn] (\x) -- (\y);
    }
  \end{tikzpicture}
  \caption{A state diagram illustrating the inadequacy of \ireplayability{}.}
  \label{fig:ireplayability-notgoodenough}
\end{figure}

\begin{itemize}
  \item \todo{%
    Figure out if \ireplayability{} can be used to determine \iconfluence{}.
  }
  \item \todo{%
    Figure out how hard it is to determine \ireplayability{}. I think the
    Collatz conjecture can be reduced to it, but I haven't worked out the
    kinks.
  }
\end{itemize}

% Unfortunately, determining \ireplayability{} is hard. Like, really really hard.
% To show just how hard it is, let me introduce the Collatz conjecture. Take any
% positive integer $n$ and repeatedly perform the following: half $n$ if $n$ is
% even and otherwise multiply it by $3$ and add $1$. For example, letting $n =
% 10$, we have $10, 5, 16, 8, 4, 2, 1, 4, 2, 1, \ldots$. The Collatz conjecture
% states that for any $n$, this process will eventually lead to the value $1$. Is
% the Collatz conjecture true? Nobody knows. It's an unsolved problem.
%
% Now, we'll show that if we could determine \ireplayability{}, we could solve
% the Collatz conjecture. Assume for contradiction that we had an algorithm to
% determine whether a set of transactions $T$ was \ireplayable{} with respect to
% an invariant $I$.
