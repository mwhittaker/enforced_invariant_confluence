\section{Introduction}
When an application designer decides to replicate a piece of data, they have to
make a fundamental choice between weak and strong consistency. Replicating the
data with strong consistency using a technique like state machine
replication~\cite{schneider1990implementing, lamport1998part,
liskov2012viewstamped, ongaro2014search, moraru2013there, vincent2015designing,
oki1988viewstamped, lamport2005generalized, lamport2006fast} makes the
application designer's life very easy. To the application developer, a strongly
consistent system behaves exactly like a single-threaded system running on a
single node, so reasoning about the behavior of the system is
simple~\cite{herlihy1990linearizability}.  Unfortunately, strong consistency is
at odds with performance. The CAP theorem and PACELC theorem tell us that
strongly consistent systems suffer from higher latency at best and
unavailability at worst~\cite{gilbert2002brewer, brewer2012cap,
abadi2012consistency}. On the other hand, weak consistency models like eventual
consistency~\cite{vogels2009eventually}, PRAM
consistency~\cite{lipton1988pram}, causal consistency~\cite{ahamad1995causal},
and others~\cite{lloyd2011don, mehdi2017can} allow data to be replicated with
high availability and low latency, but they put a tremendous burden on the
application designer to reason about the complex interleavings of operations
that are allowed by these weak consistency models. In fact, weak consistency
models strip an application developer of one of the earliest and most effective
tools that is used to reason about the execution of programs: application
invariants~\cite{hoare1969axiomatic, balegas2015towards}. Even if every
transaction executing in a weakly consistent system individually maintains an
application invariant, the system as a whole can produce invariant violating
states.

Is it possible for us to have our strongly consistent cake and eat it with high
availability too? Can we replicate a piece of data with weak consistency but
still ensure that its invariants are maintained? Yes, sometimes. Bailis et al.\
introduced the notion of \invariantconfluence{} as a necessary and sufficient
condition for when invariants can be maintained over replicated data without
the need for any coordination~\cite{bailis2014coordination}. If an object is
\invariantconfluent{} with respect to an invariant, we can replicate it with
the performance benefits of weak consistency and (some of) the correctness
benefits of strong consistency. However, the task of identifying whether or not
an object actually is \invariantconfluent{} remains a challenge. Bailis et al.\
manually categorized a set of common objects, transactions, and invariants
(e.g.\ foreign key constraints on relations, linear constraints on integers) as
\invariantconfluent{} or not, but ideally we would have a general purpose
program that can automatically determine \invariantconfluence{} for us.
\textbf{The first main thrust of this paper is to make progress towards a
general-purpose \invariantconfluence{} decision procedure.}

Designing a general-purpose \invariantconfluence{} decision procedure is
difficult\footnote{Actually, it's impossible. Determining whether an object is
\invariantconfluent{} is, in general, undecidable. Still, we can develop a
decision procedure that works well in the common case.}. Many existing
approaches instead develop a decision procedure for \invariantclosure{}, a
sufficient (but not necessary) condition for
\invariantconfluence{}~\cite{li2012making, li2014automating}. The existing
approaches check to see if an object is \invariantclosed{}. If it is, then they
conclude that it is also \invariantconfluent{}. If it's not, then the current
approaches are unable to conclude anything about whether or not the object is
\invariantconfluent{}. In this paper, we take a step back and study the
underlying reason \emph{why} \invariantclosure{} is not necessary for
\invariantconfluence{}. We find that \invariantconfluence{} is fundamentally a
property about reachability, about whether certain values of our replicated
object can actually be obtained in the execution of a system. We discover that
unreachable, yet invariant satisfying, states cause \invariantclosure{} to be
an unnecessary condition. Using this understanding, we construct a set of
modest conditions under which \invariantclosure{} and \invariantconfluence{}
are in fact \emph{equivalent}, allowing us to reduce the problem of determining
\invariantconfluence{} to that of determining \invariantclosure{}.

Then, we use these conditions to design a general-purpose interactive
\invariantconfluence{} decision procedure. Users iteratively interact with the
decision procedure, eliminating unreachable states from the invariant, moving
towards the conditions under which \invariantclosure{} and
\invariantconfluence{} are equivalent. Meanwhile, the decision-procedure
generates example reachable and unreachable states which help the user
recognize patterns that describe reachability. We also leverage our intuitions
about reachability to develop a new sufficient condition for
\invariantconfluence{}, dubbed \mergereducibility. \Mergereducibility{} covers
some cases that are not covered by \invariantclosure{}, and it can be
automatically checked without any user interaction.

\textbf{The second main thrust of this paper is to take a step beyond
\invariantconfluence{} and develop a generalization of \invariantconfluence{}
called segmented \invariantconfluence{}.} While \invariantconfluence{}
characterizes objects that can be replicated \emph{without any} coordination,
segmented \invariantconfluence{} allows us to replicate
non-\invariantconfluent{} objects with \emph{minimal coordination}. The main
idea is that invariant satisfying states are divided into a number of segments,
and the set of allowable transactions within each segment is restricted.
Within a segment, servers act without any coordination, synchronizing only to
transition across segment boundaries. This design highlights the trade-off
between application complexity and coordination-freedom; more complex
applications require more segments which require more coordination, and
vice-versa.

In closing, here is an outline of the paper and of our contributions:
\begin{itemize}
  \item
    We propose a novel expression-oriented definition of \invariantconfluence{}
    that is both formal and simple (\secref{InvariantConfluence}).

  \item
    We develop an understanding of why \invariantclosure{} is not necessary and
    use this understanding to develop conditions under which it is both
    necessary and sufficient (\secref{InvariantClosure}).

  \item
    We exploit these conditions to design a general-purpose interactive
    decision procedure for \invariantconfluence{}
    (\secref{InteractiveDecisionProcedure}).

  \item
    We again exploit these conditions to design a novel non-trivial sufficient
    condition for \invariantconfluence{}, \mergereducibility, that does not
    depend on user interaction (\secref{MergeReduction}).

  \item
    We present segmented \invariantconfluence{}: a generalization of
    \invariantconfluence{} that uses a small amount of coordination to maintain
    invariants for replicated objects that are otherwise not
    \invariantconfluent{} (\secref{SegmentedInvariantConfluence}).

  \item
    We implement our decision procedures and sufficient conditions for both
    \invariantconfluence{} and segmented \invariantconfluence{} in a system
    called Lucy, then use Lucy to evaluate our methods (\secref{Evaluation}).
\end{itemize}
