\section{Introduction}
When an application designer decides to replicate a piece of data, they have to
make a fundamental choice between weak and strong consistency. Replicating the
data with strong consistency using a technique like state machine
replication~\cite{schneider1990implementing, lamport1998part,
liskov2012viewstamped, ongaro2014search} makes the application designer's life
very easy. To the application developer, a strongly consistent system behaves
exactly like a single-threaded system running on a single node, so reasoning
about the behavior of the system is simple. Unfortunately, strong consistency
is at odds with performance. The CAP theorem and PACELC theorem tell us that
strongly consistent systems suffer from higher latency at best and
unavailability at worst~\cite{brewer2012cap, abadi2012consistency}. On the
other hand, weak consistency models like eventual~\cite{vogels2009eventually},
PRAM~\cite{lipton1988pram}, and causal~\cite{ahamad1995causal} consistency
allow data to be replicated with high availability and low latency, but they
put a tremendous burden on the application designer to reason about the complex
interleavings of operations that are allowed by these weak consistency models.
In fact, weak consistency models strip an application developer of one of the
earliest and most effective tools that is used to reason about the execution of
programs: application invariants~\cite{hoare1969axiomatic}. Even if every
transaction executing in a weakly consistent system maintains an application
invariant, the system as a whole can produce invariant-violating states.

% When people want to replicate data, they have to make a choice between strong and weak consistency. When choosing strong consistency, everything makes sense, reasoning is easy. But CAP and the latency one tell us that the avilability and latency of our system will be reduced. If we choose weak consistency, then things run really fast, but we can no longer reason about the application. In fact, we risk the possibility one our greatest tools that we use to reason about software: invariants. With weak consistency models , even if every transaction preserves the invariant, the invariant may still be broken when stuff is interleaved.

Is it possible for us to have our strongly consistent cake and eat it with high
availability too? Can we replicate a piece of data with weak consistency but
still ensure that its invariants are maintained? Yes, sometimes. Bailis et al.\
introduced the notion of invariant-confluence as a necessary and sufficient
condition for when invariants can be maintained over replicated data without
the need for any coordination~\cite{bailis2014coordination}. If an object is
invariant-confluent with respect to an invariant, we can replicate it with the
performance benefits of weak consistency and (some of) the correctness benefits
of strong consistency. However, the task of identifying whether or not an
object actually is invariant-confluent remain a challenge. Bailis et al.\
manually categorized a set of common objects, transactions, and invariants
(e.g.\ foreign key constraints on relations, linear constraints on integers) as
invariant-confluent or not, but ideally we would have a general purpose program
that can automatically determine invariant-confluence for us. \textbf{The main
thrust of this paper is to make progress towards a general-purpose
invariant-confluence decision procedure.}

% Is it possible to have our cake and eat it to? Can we maintain invariants without coordination? For some applications, yes. Invariant confluence (cite bailis) introduced a necessary and sufficient condition for when its possible to have coordination free replication while maintiang invariants. If somethign is iconfluent, it can be replicated wihtout coordination. However, determining whethere something is invariant-confluent is a very challening task. Bailis recognized some patterns of things which are invariant confluent, but ideally we would have a general purpose program to automatically determine it for us. The ultimate goal of this paper is to... In this paper, we make progress towards a more-general purpose iconfluence decision procedure, improving on the current state-of-the-art approaches.

Designing a general-purpose invariant-confluence decision procedure is
difficult\footnote{Actually, it's impossible. Determining whether an object is
invariant-confluent is, in general, undecidable. Still, we can develop a
decision procedure that works well in the common case.}. Many existing
approaches instead develop a decision procedure for invariant-closure, a
sufficient (but not necessary) condition for
invariant-confluence~\cite{li2012making, li2014automating}. The existing
approaches check to see if an object is invariant-closed. If it is, then they
conclude that it is also invariant-confluent. If it's not, then the current
approaches are unable to conclude anything about whether or not the object is
invariant-confluent. In this paper, we take a step back and study the
underlying reason \emph{why} invariant-closure is not necessary for
invariant-confluence. We find that invariant-confluence is fundamentally a
property about reachability, about whether certain values of our replicated
object can actually be obtained in the execution of a system. We discover that
unreachable, yet invariant-satisfying, states cause invariant-closure to be an
unnecessary condition. Using this understanding, we construct a set of modest
conditions under which invariant-closure and invariant-confluence are in fact
\emph{equivalent}, allowing us to reduce the problem of determining
invariant-confluence to that of determining invariant-closure.

% This is hard. Not only undecidable, but also not really an easy thing to do. Existing systems often prove a sufficient condition called invariant-closure for invariant-confluence. So if the object is invariant-closed, they determine its invariant-confluent, but it its not invariant-closed, the system doesn't know anythign. In this paper, we study why the condition is not necessary, finding that the fundamental problem is that of unreachable invariant satisfying states. Using this insight, we develop general conditions under which invariant-closure is both necessary and sufficient.

Then, we use these conditions to design a general-purpose interactive
invariant-confluence decision procedure. Users iteratively interact with the
decision procedure, eliminating unreachable states from the invariant and
moving towards the conditions under which invariant-closure and
invariant-confluence are equivalent. Meanwhile, the decision-procedure
generates example reachable and unreachable states which help the user
recognize patterns that describe reachability. We also leverage our intuitions
about reachability to develop a new sufficient condition for
invariant-confluence, dubbed merge-reducibility. Merge-reducibility covers some
cases that are not covered by invariant-closure, and it can be automatically
checked without any user interaction.

% Then, we use our insight to develop a general purpose interactive invariant-confluence decision procedure. Users continuously interact with the decision procedure to eleiminate unreachabel states from the invariant. Meanwhile, the decision procedure provides examples to the user that they can use to identify which points are reachable and which are not. This decision procedure can be used to show lots of things are invariant-confluent, and is also robust to user error.

% While the general purpose interaactive dp is good, it does rely on user input. We additionally invent a new sufficient condition for invariant-confluence known as merge-reducbility which exploits the insigths from the previous theory. This new sufficient condition can handle cases that invariant-closure cannot and doesn't require user feedback.

Next, we take a step beyond invariant-confluence and develop a generalization
of invariant-confluence called segmented invariant-confluence. While
invariant-confluence characterizes objects which can be replicated
\emph{without any} coordination, segmented invariant-confluence allows us to
replicate non-invariant-confluent objects with \emph{minimal coordination}. The
main idea is that invariant-satisfying states are divided into a number of
segments, and the set of allowable of transactions within each segment is
restricted. Within a segment, servers act without any coordination,
synchronizing only to transition across segment boundaries. This design
highlights the trade-off between application complexity and
coordination-freedom; more complex applications require more segments which
require more coordination, and vice-versa.

% Finally, we address the elephant in the room and ask the question, what if something is not invariant-confluent though? We develop a generalization of invariant-confluence called segmented invariant-confluence. While invariant-confluence only characterizes objects which can be replicated without any coordination, segmented invariant-confluence allows us to replicate objects with minimal coordination. This approach also generalizes existing lock based approaches to handling non-invariant-confluent stuff.

% Finally, we implement all the decision procedures and the system models in a system called Lucy. We perform an evaluation to look at when the decision procedures are effective and measure the performance benefits of our implemetnation. Overall,

In closing, here is an outline of the paper and of our contributions:
\begin{itemize}
  \item
    We propose a novel expression-oriented definition of invariant-confluence
    that is both formal and simple (\secref{InvariantConfluence}).

  \item
    We develop an understanding of why invariant-closure is not necessary and
    use this understanding to develop conditions under which it is both
    necessary and sufficient (\secref{InvariantClosure}).

  \item
    We exploit these conditions to design a general-purpose interactive
    decision procedure for invariant-confluence
    (\secref{InteractiveDecisionProcedure}).

  \item
    We again exploit these conditions to design a novel non-trivial sufficient
    condition for invariant-confluence, merge-reducibility, that does not
    depend on user interaction (\secref{MergeReduction}).

  \item
    We present segmented invariant-confluence: a generalization of
    invariant-confluence that uses a small amount of coordination to maintain
    invariants for replicated objects that are otherwise not
    invariant-confluent (\secref{SegmentedInvariantConfluence}).

  \item
    We implement our decision procedures and sufficient conditions for both
    invariant-confluence and segmented invariant-confluence in a system called
    Lucy, then use Lucy to evaluate the efficacy or our methods
    (\secref{Evaluation}).
\end{itemize}

% Hype up problem that invariant confluence solves
% Introduce iconfluence and hype it up as a really good thing.
% Discuss how it's hard to determine iconfluence and a lot of existing
% approaches fall short.
% Describe how we take the problem one step further
% Overview two new approaches
% Describe how often times we're not iconfluent
% Overivew new segmented iconfluence stuff
% contributions
%   - new expression-based definitions of iconfluence
%   - understanding of iclosure and new suff and necc
%   - interactive decision procedure
%   - new suff conditions
%   - segmented iconfluence
%   - implementation in lucy + evaluation...
