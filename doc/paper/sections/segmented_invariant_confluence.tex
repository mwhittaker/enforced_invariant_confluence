\section{Segmented Invariant Confluence}\seclabel{SegmentedInvariantConfluence}
\newcommand{\IsIclosed}{\textsc{IsIclosed}}

If a distributed object is \invariantconfluent{}, then the object can be
replicated without the need for any form of coordination to maintain the
object's invariant. But what if the object is \emph{not} \invariantconfluent{}?
In this section, we present a generalization of \invariantconfluence{} called
\defword{segmented \invariantconfluence{}} that can be used to maintain the
invariants of non-\invariantconfluent{} objects, requiring only a small amount
of coordination. In \secref{Evaluation}, we see that replicating a
non-\invariantconfluent{} object with segmented \invariantconfluence{} can
achieve between 100x and 10x more throughput than linearizable replication for
certain workloads.

The main idea behind segmented \invariantconfluence{} is to segment the state
space into a number of segments and restrict the set of allowable transactions
within each segment in such a way that the object is \invariantconfluent{}
\emph{within each segment} (even though it may not be globally
\invariantconfluent{}). Then, servers can run coordination-free within a
segment and need only coordinate when transitioning from one segment to
another. We now formalize segmented \invariantconfluence{}, describe the system
model we use to replicate segmented \invariantconfluent{} objects, and
introduce a segmented \invariantconfluence{} decision procedure.

\subsection{Formalism}
Consider a distributed object $O = (S, \join)$, a start state $s_0 \in S$, a
set of transitions $T$, and an invariant $I$. A segmentation $\Sigma = (I_1,
T_1), \ldots, (I_n, T_n)$ is a sequence (not a set) of $n$ segments $(I_i,
T_i)$ where every $T_i$ is a subset of $T$. $O$ is \defword{segmented
\invariantconfluent{}} with respect to $s_0$, $T$, $I$, and $\Sigma$,
abbreviated \defword{\sTISconfluent}, if the following conditions hold:
\begin{itemize}
  \item
    The start state satisfies the invariant (i.e. $I(s_0)$).

  \item
    $I$ is covered by the invariants in $\Sigma$ (i.e. $I = \cup_{i=1}^n I_i$).
    Note that invariants in $\Sigma$ do \emph{not} have to be disjoint. That
    is, they do not have to partition $I$; they just have to cover $I$.

  \item
    $O$ is \invariantconfluent{} within each segment. That is, for every $(I_i,
    T_i) \in \Sigma$ and for every state $s \in I_i$, $O$ is
    \sticonfluent{s}{T_i}{I_i}.
\end{itemize}

{\tikzstyle{point}=[shape=circle, fill=flatgray, inner sep=2pt, draw=black]
\tikzstyle{region}=[draw=none]
\tikzstyle{region1}=[region, fill=flatred!50]
\tikzstyle{region2}=[region, fill=flatgreen!50]
\tikzstyle{region3}=[region, fill=flatblue!50]
\tikzstyle{region4}=[region, fill=flatpurple!50]

\newcommand{\pointgrid}[4]{{
  \newcommand{\argxmin}{#1}
  \newcommand{\argxmax}{#2}
  \newcommand{\argymin}{#3}
  \newcommand{\argymax}{#4}

  \draw[] (\argxmin, 0) to (\argxmax, 0);
  \draw[] (0, \argymin) to (0, \argymax);
  \foreach \x in {\argxmin, ..., \argxmax} {
    \foreach \y in {\argymin, ..., \argymax} {
      \node[point] (\x-\y) at (\x, \y) {};
    }
  }
}}

\begin{figure}[ht]
  \centering

  \newcommand{\subfigwidth}{0.45\columnwidth}
  \newcommand{\subfighspace}{0.3cm}
  \newcommand{\tikzhspace}{0.4cm}
  \newcommand{\tikzscale}{0.5}
  \newcommand{\xmin}{-2}
  \newcommand{\xmax}{2}
  \newcommand{\ymin}{-2}
  \newcommand{\ymax}{2}

  \begin{subfigure}[t]{\subfigwidth}
    \centering
    \begin{tikzpicture}[scale=\tikzscale]
      \draw[region1] (\xmin.5, \ymax.5) rectangle (-0.5, 0.5);
      \draw (-0.5, 0.5) to (\xmin.5, 0.5);
      \draw (-0.5, 0.5) to (-0.5, \ymax.5);
      \pointgrid{\xmin}{\xmax}{\ymin}{\ymax}
    \end{tikzpicture}
    \caption{$(I_1, T_1)$.}
    \figlabel{SegZ21}
  \end{subfigure}
  \begin{subfigure}[t]{\subfigwidth}
    \centering
    \begin{tikzpicture}[scale=\tikzscale]
      \draw[region2] (-0.5, 0.5) rectangle (\xmax.5, \ymin.5);
      \draw (-0.5, 0.5) to (\xmax.5, 0.5);
      \draw (-0.5, 0.5) to (-0.5, \ymin.5);
      \pointgrid{\xmin}{\xmax}{\ymin}{\ymax}
    \end{tikzpicture}
    \caption{$(I_2, T_2)$.}
    \figlabel{SegZ22}
  \end{subfigure}

  \vspace{12pt}

  \begin{subfigure}[t]{\subfigwidth}
    \centering
    \begin{tikzpicture}[scale=\tikzscale]
      \draw[region3] (-0.5, \ymax.5) rectangle (0.5, \ymin.5);
      \draw (-0.5, \ymax.5) to (-0.5, \ymin.5);
      \draw (0.5, \ymax.5) to (0.5, \ymin.5);
      \pointgrid{\xmin}{\xmax}{\ymin}{\ymax}
    \end{tikzpicture}
    \caption{$(I_3, T_3)$.}
    \figlabel{SegZ23}
  \end{subfigure}
  \begin{subfigure}[t]{\subfigwidth}
    \centering
    \begin{tikzpicture}[scale=\tikzscale]
      \draw[region4] (\xmin.5, 0.5) rectangle (\xmax.5, -0.5);
      \draw (\xmax.5, -0.5) to (\xmin.5, -0.5);
      \draw (\xmax.5, 0.5) to (\xmin.5, 0.5);
      \pointgrid{\xmin}{\xmax}{\ymin}{\ymax}
    \end{tikzpicture}
    \caption{$(I_4, T_4)$.}
    \figlabel{SegZ24}
  \end{subfigure}

  \caption{An illustration of \exampleref{SegmentedZ2}}
  \figlabel{SegmentedZ2}
\end{figure}
}

\begin{example}\examplelabel{SegmentedZ2}
  Consider again the object $O = (\ints \times \ints, \join)$, transactions $T
  = \set{t_{x+1}, t_{y-1}}$, and invariant $I = \setst{(x, y)}{xy \leq 0}$ from
  \exampleref{Z2}, but now let the start state $s_0$ be $(-42, 42)$. $O$ is
  \emph{not} \sTIconfluent{} because the points $(0, 42)$ and $(42, 0)$ are
  reachable, and merging these points yields $(42, 42)$ which violates the
  invariant. However, $O$ is \sTISconfluent{} for $\Sigma = (I_1, T_1)$, $(I_2,
  T_2)$, $(I_3, T_3)$, $(I_4, T_4)$ where
  \begin{align*}
    I_1 &= \setst{(x, y)}{x < 0, y > 0}       &T_1 &= \set{t_{x+1}, t_{y-1}} \\
    I_2 &= \setst{(x, y)}{x \geq 0, y \leq 0} &T_2 &= \set{t_{x+1}, t_{y-1}} \\
    I_3 &= \setst{(x, y)}{x = 0}              &T_3 &= \set{t_{y-1}} \\
    I_4 &= \setst{(x, y)}{y = 0}              &T_4 &= \set{t_{x+1}}
  \end{align*}
  $\Sigma$ is illustrated in \figref{SegmentedZ2}. Clearly, $s_0$ satisfies the
  invariant, and $I_1, I_2, I_3, I_4$ cover $I$. Moreover, for every $(I_i,
  T_i) \in \Sigma$, we see that $O$ is \iclosed{I_i}, so $O$ is
  \sticonfluent{s}{T_i}{I_I} for every $s \in I_i$. Thus, $O$ is
  \sTISconfluent{}.
\end{example}

\subsection{System Model}
Now, we describe the system model used to replicate a segmented
\invariantconfluent{} object without any coordination within a segment and with
only a small amount of coordination when transitioning between segments. As
before, we replicate an object $O$ across a set $p_1, \ldots, p_n$ of $n$
servers each of which manages a replica $s_i \in S$ of the object. Every server
begins with $s_0$, $T$, $I$, and $\Sigma$. Moreover, at any given point in
time, a server designates one of the segments in $\Sigma$ as its
\defword{active segment}.  Initially, every server chooses the first segment
$(I_i, T_i) \in \Sigma$ such that $I_i(s_0)$ to be its active segment. We'll
see momentarily the significance of the active segment.

As before, servers repeatedly perform one of two actions: execute a transaction
or merge states with another server. Merging states in the segmented
\invariantconfluence{} system model is identical to merging states in the
\invariantconfluence{} system model. A server $p_i$ sends its state $s_i$ to
another server $p_j$ which \emph{must} merge $s_i$ into its state $s_j$.
%
Transaction execution in the new system model, on the other hand, is a bit more
involved. Consider a server $s_i$ with active segment $(I_i, T_i)$. A client
can request that $p_i$ execute a transaction $t \in T$. We consider what
happens when $t \in T_i$ and $t \notin T_i$ separately.

If $t \notin T_i$, then $p_i$ initiates a round of global coordination to
execute $t$. During global coordination, every server temporarily stops
processing transactions and transitions to state $s = s_1 \join \ldots \join
s_n$, the join of every server's state. Then, every server speculatively
executes $t$ transitioning to state $t(s)$. If $t(s)$ violates the invariant
(i.e.  $\lnot I(t(s))$), then every server aborts $t$ and reverts to state $s$.
Then, $p_i$ replies to the client. If $t(s)$ satisfies the invariant (i.e.
$I(t(s))$), then every server commits $t$ and remains in state $t(s)$. Every
server then chooses the first segment $(I_i, T_i) \in \Sigma$ such that
$I_i(t(s))$ to be the new active segment.  Note that such a segment is
guaranteed to exist because the segment invariants cover $I$. Moreover,
$\Sigma$ is ordered, so every server will deterministically pick the same
active segment. In fact, an invariant of the system model is that at any given
point of normal processing, every server has the same active segment.

Otherwise, if $t \in T_i$, then $p_i$ executes $t$ immediately and without
coordination.  If $t(s_i)$ satisfies the \emph{active} invariant (i.e.
$I_i(t(s_i))$), then $p_i$ commits $t$, stays in state $t(s_i)$, and replies to
the client. If $t(s_i)$ violates the \emph{global} invariant (i.e. $\lnot
I(t(s_i))$), then $p_i$ aborts $t$, reverts to state $s_i$, and replies to the
client. If $t(s_i)$ satisfies the global invariant but violates the active
invariant (i.e.  $I(t(s_i))$ but $\lnot I_i(t(s_i))$), then $p_i$ reverts to
state $s_i$ and initiates a round of global coordination to execute $t$, as
described in the previous paragraph.  Transaction execution is summarized in
\algoref{TxnExecution}.

\begin{algorithm}[t]
  \caption{%
    Transaction execution in the segmented \invariantconfluence{} system model}%
  \algolabel{TxnExecution}
  \begin{algorithmic}
    \If{$t \notin T_i$}
      \State Execute $t$ with global coordination
    \Else{}
      \If{$I_i(t(s_i))$} Commit $t$
      \ElsIf{$\lnot I(t(s_i))$} Abort $t$
      \Else{} Execute $t$ with global coordination
      \EndIf{}
    \EndIf{}
  \end{algorithmic}
\end{algorithm}

This system model guarantees that all replicas of a segmented
\invariantconfluent{} object always satisfy the invariant. All servers begin in
the same initial state and with the same active segment. Thus, because $O$ is
\invariantconfluent{} with respect to every segment, servers can execute
transactions within the active segment without any coordination and guarantee
that the invariant is never violated. Any operation that would violate the
assumptions of the \invariantconfluence{} system model (e.g.\ executing a
transaction that's not permitted in the active segment or executing a permitted
transaction that leads to a state outside the active segment) triggers a global
coordination. Globally coordinated transactions are only executed if they
maintain the invariant. Moreover, if a globally coordinated transaction leads
to another segment, the coordination ensures that all servers begin in the same
start state and with the same active replica, reestablishing the assumptions of
the \invariantconfluence{} system model.

\subsection{Interactive Decision Procedure}
In order for us to determine whether or not an object $O$ is \sTISconfluent{},
we have to determine whether or not $O$ is \invariantconfluent{} within each
segment $(I_i, T_i) \in \Sigma$. That is, we have determine if $O$ is
\sticonfluent{s}{T_i}{I_i} confluent for every state $s \in I_i$. Ideally, we
could leverage \algoref{InteractiveDecisionProcedure}, invoking it once per
segment. Unfortunately, \algoref{InteractiveDecisionProcedure} can only be used
to determine if $O$ is \sticonfluent{s}{T_i}{I_i} for a \emph{particular} state
$s \in I_i$, not for \emph{every} state $s \in I_i$. Thus, we would have to
invoke \algoref{InteractiveDecisionProcedure} $|I_i|$ times for every segment
$(I_i, T_i)$, which is clearly infeasible given that $I_i$ can be large or even
infinite.

Instead, we develop a new decision procedure that can be used to determine if
an object is \sticonfluent{s}{T}{I} for every state $s \in I$. To do so, we
need to extend the notion of reachability to a notion of coreachability and
then generalize \thmref{IclosureEquivalentIconfluence}. Two states $s_1, s_2
\in I$ are \defword{coreachable} with respect to $T$ and $I$, abbreviated
\defword{\TIcoreachable}, if there exists some state $s_0 \in I$ such that
$s_1$ and $s_2$ are both \sTIreachable{}.

\begin{theorem}\thmlabel{ExtendedIclosureEquivalentIconfluence}
  Consider an object $O = (S, \join)$, a set of transactions $T$, and an
  invariant $I$. If every pair of states in the invariant are \TIcoreachable{},
  then
  \[
    \text{$O$ is \Iclosed{}}
    \iff
    \text{$O$ is \sticonfluent{s}{T}{I} for every $s \in I$}
  \]
\end{theorem}
\begin{proof}
  The proof of the forward direction is exactly the same as the proof of
  \thmref{IclosureImpliesIconfluence}. Transactions always maintain the
  invariant, so if merge does as well, then every reachable state must satisfy
  the invariant.
  %
  For the reverse direction, consider two arbitrary states $s_1, s_2 \in I$.
  The two points are \TIcoreachable{}, so there exists some state $s_0$ from
  which they can be reached. $O$ is \sTIconfluent{} and $s_1 \join s_2$ is
  \sTIreachable{}, so it satisfies the invariant.
\end{proof}

Using \thmref{ExtendedIclosureEquivalentIconfluence}, we develop
\algoref{ArbitraryStartInteractiveDecisionProcedure}: a natural generalization
of \algoref{InteractiveDecisionProcedure}.
\algoref{InteractiveDecisionProcedure} iteratively refines the set of
\emph{reachable} states whereas
\algoref{ArbitraryStartInteractiveDecisionProcedure} iteratively refines the
set of \emph{coreachable} states, but otherwise, the core of the two algorithms
is the same.\footnote{%
  Another small difference is that \IsIclosed{} behaves differently in
  \algoref{InteractiveDecisionProcedure} and
  \algoref{ArbitraryStartInteractiveDecisionProcedure}. In
  \algoref{ArbitraryStartInteractiveDecisionProcedure}, \IsIclosed{} returns a
  triple (closed, $s_1$, $s_2$). If closed is false, then $s_1, s_2 \in I$
  are two states not in $NR$ such that $I(s_1)$ and $I(s_2)$ but $\lnot I(s_1
  \join s_2)$. If no such states exist, then closed is true, and $s_1$ and
  $s_2$ are null.
}
Now, a segmented \invariantconfluence{} decision procedure, can simply invoke
\algoref{ArbitraryStartInteractiveDecisionProcedure} once on each segment.

\newcommand{\algocomment}[1]{\State \textcolor{flatdenim}{\texttt{//} #1}}
\begin{algorithm}[t]
  \caption{%
    Interactive \invariantconfluence{} decision procedure for arbitrary start
    state $s \in I$
  }%
  \algolabel{ArbitraryStartInteractiveDecisionProcedure}
  \begin{algorithmic}
    \algocomment{Return if $O$ is \sticonfluent{s}{T}{I} for every $s \in I$.}
    \Function{IsInvConfluent}{$O$, $T$, $I$}
      \State \Return \Call{Helper}{$O$, $T$, $I$, $\emptyset$, $\emptyset$}
    \EndFunction

    \State

    \algocomment{$R$ is a set of \TIcoreachable{} states.}
    \algocomment{$NR$ is a set of \TIcounreachable{} states.}
    \Function{Helper}{$O$, $T$, $I$, $R$, $NR$}
      \State closed, $s_1$, $s_2$ $\gets$ \Call{IsIclosed}{$O$, $I$, $NR$}
      \If {closed}
        \Return true
      \EndIf
      \State Augment $R, NR$ with random search and user input
      \If{$(s_1, s_2) \in R$}
        \Return false
      \EndIf
      \State \Return \Call{Helper}{$O$, $T$, $I$, $R$, $NR$}
    \EndFunction
  \end{algorithmic}
\end{algorithm}

\begin{example}\examplelabel{CounreachableExample}
  Let $O = (\ints^3 \times \ints^3, \join)$ be an object that separately keeps positive and negative integer counts (dubbed a
  PN-Counter~\cite{shapiro2011comprehensive}), replicated on three machines.
  Every state $s = $ $(p_1, p_2, p_3),$ $(n_1, n_2, n_3)$ represents the
  integer $(p_1 + p_2 + p_3) - (n_1 + n_2 + n_3)$. To increment or decrement
  the counter, the $i$th server increments $p_i$ or $n_i$ respectively, and
  $\join$ computes an element-wise maximum. Our start state $s_0 = (0, 0, 0),
  (0, 0, 0)$; our set $T$ of transactions consists of increment and decrement;
  and our invariant $I$ is that the value of $s$ is non-negative.

  Applying \algoref{InteractiveDecisionProcedure}, \IsIclosed{} returns false
  with the states $s_1 = (1, 0, 0), (0, 1, 0)$ and $s_2 = (1, 0, 0), (0, 0,
  1)$. Both are reachable, so $O$ is not \sTIconfluent{} and
  \algoref{InteractiveDecisionProcedure} returns false.  The culprit is
  concurrent decrements, which we can forbid in a simple one-segment
  segmentation $\Sigma = (I, T^+)$ where $T^+$ consists only of increment
  transactions. Applying, \algoref{ArbitraryStartInteractiveDecisionProcedure},
  \IsIclosed{} again returns false with same states $s_1$ and $s_2$. This time,
  however, the user recognizes that the two states are not
  \ticoreachable{T^+}{I} (all modifications of $(n_1, n_2, n_3)$ require global
  coordination, so it is impossible for $s_1$ and $s_2$ to differ on these
  values).
  The user refines $NR$ with the observation that two
  states are coreachable if and only if they have the same values of $n_1, n_2,
  n_3$. After this, \IsIclosed{} and
  \algoref{ArbitraryStartInteractiveDecisionProcedure} return true.
\end{example}

\subsection{Discussion}
There are a few things worth noting about segmented
\invariantconfluence{}, its system model, and its decision procedure.
%
First, \invariantconfluence{} is a very coarse-grained property. If an object
is \invariantconfluent{}, then we can replicate it with no coordination. If it
is not \invariantconfluent{}, then we have no guarantees. There's no
in-between.  Segmented \invariantconfluence{}, on the other hand, is a much
more fine-grained property that can be applied to applications with varying
degrees of complexity. Segmented \invariantconfluence{} provides guarantees to
complex applications that require a large number of segments and to simple
applications with a smaller number of segments, whereas \invariantconfluence{}
only provides guarantees to applications that can be segmented into a single
segment.

% Second, segmented \invariantconfluence{} naturally subsumes a distributed
% locking approach to replicating non-\invariantconfluent{} objects.  This
% approach first recognizes which transactions cannot be safely executed
% concurrently and then requires them to acquire a distributed lock before
% executing~\cite{balegas2015putting, gotsman2016cause}. For example, in a
% banking application with the invariant that all balances remain non-negative,
% concurrent deposits are permitted, but concurrent withdrawals can lead to
% invariant violations. Thus, we require that withdrawals acquire a distributed
% lock before executing. This example is exactly the same as
% \exampleref{CounreachableExample} which we handled by simply removing
% withdrawal transactions from our segmentation's set of transactions.

% Second, we can integrate a couple of optimizations into our system model to
% further reduce the amount of coordination it requires. To begin, if a server with
% state $s_i$ and active segment $(I_i, T_i)$ receives a transaction $t \in I_i$
% to execute, and $t(s_i)$ violates the active invariant but not the global
% invariant, instead of initiating a round of global coordination, $p_i$ can
% simply buffer $t$ for re-execution at a later time. While this increases the
% latency required to execute $t$, it's possible that after other transactions
% are executed, re-executing $t$ may lead to a state that either satisfies the
% active invariant or violates the global invariant. In either case, a round of
% global coordination is avoided. Similarly, servers can buffer transactions that
% require global coordination, executing an entire batch of these transactions
% during a single round of global coordination.

Second, we can integrate a couple of optimizations into our system model to
further reduce the amount of coordination it requires. For example, servers can
buffer transactions that require global coordination, executing an entire batch
of these transactions during a single round of global coordination.

Third, a segmented \invariantconfluence{} decision procedure can also leverage
\thmref{LatticeProperty} in addition to
\algoref{ArbitraryStartInteractiveDecisionProcedure}. If an object $O$ meets
criteria (1) - (3), then it is \sticonfluent{s}{T}{I} for every state $s \in
I$.

Fourth, while a segmented \invariantconfluence{} decision procedure can help
decide whether or not an object is segmented \invariantconfluent{}, it cannot
currently help construct a segmentation. In the future, we plan to explore
ways by which we can automatically suggest segmentations to the application
designer.
