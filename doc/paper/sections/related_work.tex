\section{Related Work}
% Other possibilities:
%   - Towards Fast Invariant Preservation in Geo-replicated Systems
%   - Feral Concurrency Control
%   - The CISE Tool: Proving Weakly-Consistent Applications Correct
%   - Declarative Programming over Eventually Consistent Data Stores
%   - Extending Eventually Consistent Cloud Databases for Enforcing Numeric Invariants

\newcommand{\sieve}{\textsc{SIEVE}}
\textbf{RedBlue Consistency and \sieve.}
RedBlue consistency is a consistency model that sits between causal consistency
and linearizability~\cite{li2012making}.
%
With RedBlue consistency, every operation is manually labelled as either red or
blue. All operations are executed with causal consistency, but with the added
restrictions that red operations are executed in a single total order embedded
within the causal ordering.
%
Invariant safety is a sufficient (but not necessary) condition for RedBlue
consistent objects to be \invariantconfluent{} and is an analog of
\invariantclosure{} for an operation-based system model (as opposed to the
state based system model that we use).  In~\cite{li2014automating}, Li et al.\
develop sophisticated techniques for deciding invariant safety that involve
calculating weakest preconditions. These techniques are complementary to our
work and can be used to improve the \invariantclosure{} subroutine used by our
decision procedures.

\textbf{The Homeostasis Protocol.}
The homeostasis protocol~\cite{roy2015homeostasis} uses program analysis to
avoid unnecessary coordination between servers in a \emph{sharded} database
(whereas \invariantconfluence{} targets \emph{replicated} databases). The
protocol generates global invariants, called global treaties, and servers
operate without coordination so long as they preserve the treaty. When the
treaty is violated, the nodes perform a round of coordination and establish a
new treaty. The protocol guarantees that transactions are executed with
observational equivalence with respect to some serial execution of the
transactions. This means that intermediate states may be inconsistent, but
externally observable side effects and the final database state are consistent.
The observational equivalence guaranteed by the homeostasis is stronger than
the guarantees of \invariantconfluence{}. As a result, there are invariants and
workloads that the homeostasis protocol would execute with more coordination
than a segmented \invariantconfluent{} execution. Still, the homeostasis
protocol's mechanism of establishing invariants and operating without
coordination so long as the invariants are maintained is very similar to our
design of segmented \invariantconfluence{}.

\textbf{Explicit Consistency.}
Explicit consistency~\cite{balegas2015towards} is a consistency model that
combines \invariantconfluence{} and causal consistency, similar to RedBlue
consistency with invariant safety. To determine if a workload is amenable to
explicitly consistent replication, Balegas et al.\ determine if all pairs of
transactions can be concurrently executed on the same start state without
violating the invariant~\cite{balegas2015towards}.  This is a sufficient
condition for explicit consistency similar to criterion (3) in
\thmref{LatticeProperty}. Balegas et al.\ also describe a variety of
techniques---like conflict resolution, locking, and escrow
transactions~\cite{o1986escrow}---that can be used to replicate workloads that
do not meet their sufficient conditions. As discussed in \secref{Evaluation},
simple forms of these techniques can be modelled with segmented
\invariantconfluence{}.

\textbf{Token Based Invariant Confluence.}
In~\cite{gotsman2016cause}, Gotsman et al.\ discuss a hybrid token based
consistency model that generalizes a family of consistency models including
causal consistency, sequential consistency, and RedBlue consistency. An
application designer defines a set of tokens and specifies which pairs of
tokens conflict, and transactions acquire some subset of the tokens when they
execute. This allows the application designer to specify which transactions
conflict with one another.  Gotsman et al.\ develop sufficient conditions to
determine whether a given token scheme is sufficient to guarantee that a global
invariant is never broken. This model of consistency is very similar to
segmented \invariantconfluence{}. The token based approach allows users to
specify certain conflicts that are not possible with segmented
\invariantconfluence{}. A segmentation only allows transactions within a
segment to acquire a single self-conflicting lock. However, segmented
\invariantconfluence{} also introduces the notion of invariant segmentation,
which cannot be emulated with the token based approach. For example, it is
difficult to emulate escrow transactions with the token based approach.

\textbf{CRDTs.}
CRDTs~\cite{shapiro2011conflict, shapiro2011comprehensive} are distributed
semilattices with inflationary update methods. Due to their algebraic
properties, CRDTs can be replicated with strong eventual consistency without
the need for any coordination. Our definition of distributed objects and our
\invariantconfluence{} system model are inspired directly by the corresponding
definitions and system models in~\cite{shapiro2011conflict}. Also, an
\invariantconfluent{} object does not necessarily have to be eventually
consistent, so it is natural (though not necessary) to use CRDTs as
\invariantconfluent{} distributed objects, achieving a combination of eventual
consistency and invariant preservation. Our criteria in
\thmref{LatticeProperty} also borrow ideas from CRDTs, exploiting the nice
algebraic properties of semilattices.

\textbf{CALM Theorem.}
Bloom~\cite{alvaro2010boom, alvaro2011consistency, conway2012logic} and its
formalism, Dedalus~\cite{alvaro2011dedalus, alvaro2013declarative}, are
declarative Datalog-based programming languages that are designed to program
distributed systems. The accompanying CALM
theorem~\cite{hellerstein2010declarative, ameloot2013relational} states that if
a program can be written in the monotone fragment of these languages, then
there exists a coordination-free implementation of the program.  The CALM
theorem provides guarantees about the outputs of a program, but it does not
guarantee that invariants are maintained throughout the duration of an
execution. Moreover, Bloom and Dedalus are general purpose programming
languages that can be used to implement a variety of distributed systems (e.g.
a sharded database) which are outside of the scope of \invariantconfluence{}.

\textbf{Program Analysis in Database Systems.}
Using program analysis to improve the performance of database systems is not
new. For example, it has been used to improve the performance of
database-backed web applications~\cite{cheung2014using, wu2016transaction,
ramachandra2012program} and used to improve the performance of optimistic
concurrency control on multi-core machines~\cite{wu2016transaction}. Our work
on \invariantconfluence{} continues the theme of using program analysis to reap
the performance benefits gained from understanding program semantics.
