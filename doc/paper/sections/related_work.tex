\section{Related Work}
We now discuss RedBlue consistency~\cite{li2012making, li2014automating}, the
homeostasis protocol~\cite{roy2015homeostasis}, explicit
consistency~\cite{balegas2015putting}, and token-based
invariant-confluence~\cite{gotsman2016cause}: four very relevant pieces of
work. We also briefly survey other related research in the area of
invariant-confluence including program analysis in database systems, CRDTs, the
CALM theorem, and interactive proof assistants.

\newcommand{\sieve}{\textsc{sieve}}
\paragraph{RedBlue Consistency and \sieve}
RedBlue consistency is a form of consistency that sits between causal
consistency and linearizability~\cite{li2012making}. With RedBlue consistency,
every operation is labelled as red or blue. All operations are executed with
causal consistency, but with the added restrictions that red operations are
executed in a single total order embedded within the causal ordering.
RedBlue consistency is similar to segmented invariant-confluence with a single
segment in which all red operations are removed from the segment's set of
allowable transactions. In~\cite{li2012making}, Li et al.\ define
invariant safety as a sufficient condition for RedBlue consistent objects to be
invariant-confluent. Invariant safety is an analog of invariant-closure for an
operation-based (as opposed-to state based~\cite{shapiro2011conflict}) system
model. In~\cite{li2014automating}, Li et al.\ develop techniques to identify
which operations are invariant safe.  Like invariant-closure, invariant safety
is a sufficient but not necessary condition for invariant-confluence. Li et
al.\ develop sophisticated techniques for deciding invariant safety that
involve calculating weakest preconditions.  These techniques are complementary
to our work and can be used to improve the invariant-closure subroutine used by
our decision procedures.

\paragraph{The Homeostasis Protocol}
The homeostasis protocol~\cite{roy2015homeostasis} uses program analysis to
avoid unnecessary coordination between servers in a sharded database. The
protocol generates global invariants, called global treaties, and servers can
operate without coordination so long as they preserve the treaty. When the
treaty is violated, the nodes perform a round of coordination. The protocol
guarantees that transactions are executed with observational equivalence with
respect to some serial execution of the transactions. This means that
intermediate states may be inconsistent, but externally observable side effects
and the final database state are consistent. The observational equivalence
guaranteed by the homeostasis is stronger than the guarantees of
invariant-confluence. As a result, there are invariants and transaction
workloads which the homeostasis protocol would execute with more coordination
than a segmented invariant-confluent execution. Still, the homeostasis
protocol's mechanism of establishing invariants and operating without
coordination so long as the invariants are maintained is very similar to our
design of segmented invariant-confluence.

\paragraph{Explicit Consistency}
A system provides explicit consistency with respect to a partial ordering of
transactions if every server executes the transactions in a total order that
respects the partial order and in a way such that the invariant is always
preserved~\cite{balegas2015towards}. Thus, explicit consistency is like a
combination of invariant-confluence and causal consistency, similar to RedBlue
consistency. To determine if a workload is amenable to explicitly consistent
replication, Balegas et al. determine if all pairs of transactions can be
concurrently executed on the same start state without violating the invariant~\cite{balegas2015towards}.
This is a sufficient condition for explicit consistency similar to criterion
(3) in \thmref{LatticeProperty}. Balegas et al. also describe a variety of
techniques---like conflict resolution, locking, and escrow
transactions~\cite{o1986escrow}---that can be used to replicate workloads that
do meet their sufficient conditions. Conflict resolution involves choosing an
appropriate merge function; locking can be implemented as a single segment
segmentation with certain transactions disallowed; and escrow transactions can
be emulated with a segmentation that allows for concurrent decrements only when
value being decremented is sufficiently large.

\paragraph{Token-Based Invariant-Confluence}
In~\cite{gotsman2016cause}, Gotsman et al.\ discuss a hybrid token-based
consistency model that generalizes RedBlue consistency. An application designer
defines a set of tokens and specifies which pairs of tokens conflict. Then,
transactions acquire some subset of the tokens. This allows the application
designer to specify which transactions conflict at a fine granularity. With no
tokens, the model is equivalent to causal consistency. With all transactions
acquiring a single self-conflicting token, the model is equivalent to
sequential consistency~\cite{lamport1979make}. With some transactions acquiring
a single self-conflicting token and some transactions acquiring no tokens, the
model is equivalent to RedBlue consistency. Gotsman et al.\ develop sufficient
conditions to determine whether a given token scheme is sufficient to guarantee
that a global invariant is never broken. This model of consistency is very
similar to segmented invariant-confluence. The token-based approach allows
users to specify certain conflicts that are not possible with segmented
invariant-confluence. A segmentation only allows transactions within a segment
to acquire a single self-conflicting lock. However, segmented
invariant-confluence also introduces the notion of segmenting the invariant,
which cannot be emulated with the token based approach. For example,
implementing \exampleref{SegmentedZ2} with the token-based approach is
challenging without being able to segment the top-left quadrant from the
bottom-right quadrant. An interesting direction for future work is to try and
integrate the token-based approach into segmented invariant-confluence.

% Towards Fast Invariant Preservation in Geo-replicated Systems
%
% Feral
% The CISE Tool: Proving Weakly-Consistent Applications Correct
% Declarative Programming over Eventually Consistent Data Stores
% Extending Eventually Consistent Cloud Databases for Enforcing Numeric Invariants

\paragraph{CRDTs}
CRDTs~\cite{shapiro2011conflict, shapiro2011comprehensive} are distributed
semilattices with inflationary update methods. Due to their algebraic
properties, CRDTs can be replicated with strong eventual consistency without
the need for any coordination. Our definition of distributed objects and our
invariant-confluence system model are inspired directly by the corresponding
definitions and system models in~\cite{shapiro2011conflict}. Note that an
invariant-confluent object does not necessarily have to be eventually
consistent. In fact, equipping a distributed object with the trivial merge
function $x \join y = x$ makes the object invariant-confluent with respect
any set of transactions and any invariant, but it makes the object very
inconsistent. Thus, it's natural (though not necessary) to use CRDTs as
invariant-confluent distributed objects, achieving a combination of eventual
consistency and invariant-preservation. Our criteria in
\thmref{LatticeProperty} also borrow ideas from CRDTs, exploiting the nice
algebraic properties of semilattices.

\paragraph{CALM Theorem}
Bloom~\cite{alvaro2010boom, alvaro2011consistency, conway2012logic} and its
formalism, Dedalus~\cite{alvaro2011dedalus, alvaro2013declarative}, are
declarative Datalog-based programming languages that are designed to program
distributed systems. The accompanying CALM
theorem~\cite{hellerstein2010declarative, ameloot2013relational} states that if
a program can be written in the monotone fragment of these languages, then
there exists a coordination-free implementation of the program. Moreover,
monotone programs are invariant to message reordering, whereas non-monotone
programs may experience anomalies if messages are reordered in the network. The
CALM theorem provides guarantees about the outputs of a program, but it does
not guarantee that invariants are maintained throughout the duration of an
execution. Moreover, Bloom and Dedalus are general purpose programming
languages that can be used to implement a variety of distributed systems (e.g.
a sharded database), while invariant-confluence is defined strictly in the
context of replicated data.

\paragraph{Program Analysis in Database Systems}
Using program analysis to improve the performance of database systems is not
new. For example, it has been used to improve the performance of
database-backed web applications~\cite{cheung2014using, wu2016transaction,
ramachandra2012program} and used to improve the performance of optimistic
concurrency control on multi-core machines~\cite{wu2016transaction}. Our work
on invariant-confluence continues the theme of using program analysis to reap
the performance benefits of understanding program semantics that would
otherwise be left on the table.

\paragraph{Proof Assistants}
Our interactive decision procedure bears resemblance to interactive proof
assistants like Coq~\cite{coq2017}, Isabelle~\cite{nipkow2002isabelle},
Agda~\cite{norell2008dependently}, and Idris~\cite{brady2013idris}. Proof
assistants guide users through the process of interactively provided a proof
that witnesses the validity of a formula. Our interactive decision procedure
guides users through the process of interactively generating a set of reachable
and unreachable points that witness the invariant-confluence of an object. An
interesting avenue for future work is to integrate an proof assistant into our
interactive decision procedure. For example, a proof assistant could verify
that a state labelled unreachable by a user is actually unreachable.
