\section{Conclusion}
This paper revolved around two major contributions.
%
First, we developed a deeper understanding of \invariantclosure{} and
\invariantconfluence{} by looking at the two criteria with reachability in
mind. We found that \invariantclosure{} fails to incorporate a notion of
reachability, and using this intuition, we developed conditions under which
\invariantclosure{} and \invariantconfluence{} are equivalent. We implemented
this insight in an interactive \invariantconfluence{} decision procedure
that automatically checks whether an object is \invariantconfluent{}, with the
assistance of a programmer.

Second, we proposed a new consistency model and generalization of
\invariantconfluence{}, segmented \invariantconfluence{}, that can be used to
replicate non-\invariantconfluent{} objects with a small amount of coordination
while still preserving their invariants. We found that segmented
\invariantconfluence{} naturally subsumes existing techniques for maintaining
invariants of replicated objects (e.g.\ locking and escrow transactions), and
we developed an interactive decision procedure for segmented
\invariantconfluence{}.

Through our evaluation, we found that our decision procedures could analyze a
number of realistic workloads, each in less than a second. We also showed that
segmented \invariantconfluence{} can significantly outperform linearizable
replication for low-coordination workloads.

\textbf{Acknowledgments.}
The authors would like to thank Alan Fekete, Peter Alvaro, Alvin Cheung,
Alexandra Meliou, Anthony Tan, and Cristina Teodoropol for fruitful discussion
and feedback.
%
This research is supported in part by DHS Award HSHQDC-16-3-00083, NSF CISE
Expeditions Award CCF-1139158, and gifts from Alibaba, Amazon Web Services, Ant
Financial, CapitalOne, Ericsson, GE, Google, Huawei, Intel, IBM, Microsoft,
Scotiabank, Splunk and VMware.
