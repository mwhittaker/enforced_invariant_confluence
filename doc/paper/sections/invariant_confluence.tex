\section{Invariant-Confluence}
Informally, a replicated object is \defword{invariant-confluent} with respect
to an invariant if every replica of the object is guaranteed to satisfy the
invariant despite the possibility of different replicas being concurrently
modified or merged together~\cite{bailis2014coordination}. In this section, we
make this informal notion of invariant-confluence precise. We begin by
introducing our system model of replicated objects and then present a formal
definition of invariant-confluence.

A \defword{distributed object} $O = (S, \join)$ consists of a set $S$ of
states and a binary merge operator $\join: S \times S \to S$ which merges two
states into one. A \defword{transaction} $t: S \to S$ is a function which maps
one state to another. An \defword{invariant} $I$ is a subset of $S$.
Notationally, we write $I(s)$ to denote that $s$ satisfies the invariant (i.e.
$s \in I$) and $\lnot I(s)$ to denote that $s$ does not satisfy the invariant
(i.e. $s \notin I$).
%
For example, $O = (\ints, +)$ is a distributed object consisting of integers
merged by addition; $t(x) = x + 1$ is a transaction which adds one to a state;
and $\setst{x \in \ints}{x \geq 0}$ is the invariant that states $x$ are
non-negative.

In our system model, a distributed object $O$ is replicated across a set $p_1,
\ldots, p_n$ of $n$ servers. Each server $p_i$ manages a replica $s_i \in S$ of
the replicated object. Every server begins with a start state $s_0 \in S$, a
fixed set $T$ of transactions, and an invariant $I$. Servers can perform one of
two actions.

First, a client can contact a server $p_i$ and request that it execute a
transaction $t \in T$. $p_i$ speculatively executes $t$ transitioning from
state $s_i$ to state $t(s_i)$. If $t(s_i)$ satisfies the invariant---i.e.
$I(t(s_i))$---then $p_i$ commits the transaction and remains in state $t(s_i)$.
Otherwise, $p_i$ aborts the transaction and reverts to state $s_i$.

Second, a server $p_i$ can sent its state $s_i$ to another server $p_j$ with
state $s_j$ causing $p_j$ to transition from state $s_j$  to state $s_i \join
s_j$. Servers periodically merge states with one another in order to keep their
states loosely synchronized\footnote{%
  Notably, if $O$ is a CRDT---i.e. $O$ is a semilattice and every transaction
  $t \in T$ is inflationary---then this periodic merging ensures that $O$ is
  strongly eventually consistent~\cite{shapiro2011conflict}.
}.
Note that unlike with transactions, servers \emph{cannot} abort a merge; server
$p_j$ must transition from $s_j$ to $s_i \join s_j$ whether or not $s_i \join
s_j$ satisfies the invariant.

Informally, $O$ is \defword{invariant-confluent with respect to $s_0$, $T$, and
$I$}, abbreviated \defword{\sTIconfluent{}} if every replica $s_1, \ldots, s_n$
is guaranteed to always satisfy the invariant $I$ in every possible execution
of the system.

To define invariant-confluence formally, we represent a state produced by a
system execution as a simple expression generated by the grammar
\[
  e ::= s \mid t(e) \mid e_1 \join e_2
\]
where $s$ represents a state in $S$ and $t$ represents a transaction in $T$. As
an example, consider the system execution in \figref{SystemExecution} in which
a distributed object is replicated across servers $p_1$, $p_2$, and $p_3$.
Server $p_3$ begins with state $s_0$, transitions to state $s_2$ by executing
transaction $u$, transitions to state $s_5$ by executing transaction $w$, and
then transitions to state $s_7$ by merging with server $p_1$. Similarly, server
$p_1$ ends up with state $s_6$ after executing transactions $t$ and $v$ and
merging with server $p_2$. In \figref{Expression}, we see the abstract syntax
tree of the corresponding expression.

{\begin{figure}[ht]
  \centering

  \tikzstyle{s0color}=[fill=flatred]
  \tikzstyle{s1color}=[fill=flatgreen]
  \tikzstyle{s2color}=[fill=flatdenim]
  \tikzstyle{s3color}=[fill=flatorange]
  \tikzstyle{s4color}=[fill=flatyellow]
  \tikzstyle{s5color}=[fill=flatcyan]
  \tikzstyle{s6color}=[fill=flatpurple]
  \tikzstyle{s7color}=[fill=flatblue]
  \tikzstyle{txnline}=[black, thick, -latex]

  \newcommand{\internaltext}[1]{$\boldsymbol #1$}

  \tikzstyle{phantomstate}=[%
    shape=circle, inner sep=1pt, draw=white, line width=1pt, fill=white]
  \tikzstyle{state}=[%
    shape=circle, inner sep=1pt, draw=black, line width=1pt, text opacity=1,
    fill opacity=0.6]

  \begin{subfigure}[c]{0.5\columnwidth}
    \centering

    \begin{tikzpicture}[yscale=0.7, xscale=1]
      \node (p3) at (0.5, 3) {$p_1$};
      \node (p2) at (0.5, 2) {$p_2$};
      \node (p1) at (0.5, 1) {$p_3$};

      \tikzstyle{pline}=[gray, opacity=0.75, thick, -latex]
      \draw[pline] (p3) to (4.75, 3);
      \draw[pline] (p2) to (4.75, 2);
      \draw[pline] (p1) to (4.75, 1);

      % Top line.
      \node[phantomstate] (s03) at (1, 3) {$\phantom{s_0}$};
      \node[phantomstate] (s13) at (2, 3) {$\phantom{s_1}$};
      \node[phantomstate] (s23) at (3, 3) {$\phantom{s_3}$};
      \node[phantomstate] (s33) at (3.75, 3) {$\phantom{s_6}$};
      \node[state, s0color] (s03) at (1, 3) {\internaltext{s_0}};
      \node[state, s1color] (s13) at (2, 3) {\internaltext{s_1}};
      \node[state, s3color] (s23) at (3, 3) {\internaltext{s_3}};
      \node[state, s6color] (s33) at (3.75, 3) {\internaltext{s_6}};

      % Middle line.
      \node[phantomstate] (s02) at (1, 2) {$\phantom{s_0}$};
      \node[phantomstate] (s12) at (2, 2) {$\phantom{s_2}$};
      \node[phantomstate] (s22) at (3, 2) {$\phantom{s_4}$};
      \node[state, s0color] (s02) at (1, 2) {\internaltext{s_0}};
      \node[state, s2color] (s12) at (2, 2) {\internaltext{s_2}};
      \node[state, s4color] (s22) at (3, 2) {\internaltext{s_4}};

      % Bottom line.
      \node[phantomstate] (s01) at (1, 1) {$\phantom{s_0}$};
      \node[phantomstate] (s11) at (2, 1) {$\phantom{s_2}$};
      \node[phantomstate] (s21) at (3, 1) {$\phantom{s_5}$};
      \node[phantomstate] (s31) at (4.25, 1) {$\phantom{s_7}$};
      \node[state, s0color] (s01) at (1, 1) {\internaltext{s_0}};
      \node[state, s2color] (s11) at (2, 1) {\internaltext{s_2}};
      \node[state, s5color] (s21) at (3, 1) {\internaltext{s_5}};
      \node[state, s7color] (s31) at (4.25, 1) {\internaltext{s_7}};

      \tikzstyle{txntext}=[sloped, above]
      \draw[txnline] (s03) to node[txntext]{$t$} (s13);
      \draw[txnline] (s13) to node[txntext]{$v$} (s23);
      \draw[txnline] (s02) to node[txntext]{$u$} (s12);
      \draw[txnline] (s01) to node[txntext]{$u$} (s11);
      \draw[txnline] (s11) to node[txntext]{$w$} (s21);

      \draw[txnline] (s13) to node[txntext]{} (s22);
      \draw[txnline] (s12) to node[txntext]{} (s22);
      \draw[txnline] (s22) to node[txntext]{} (s33);
      \draw[txnline] (s23) to node[txntext]{} (s33);
      \draw[txnline] (s21) to node[txntext]{} (s31);
      \draw[txnline] (s33) to node[txntext]{} (s31);
    \end{tikzpicture}

    \caption{System Execution}\figlabel{SystemExecution}
  \end{subfigure}%
  \begin{subfigure}[c]{0.5\columnwidth}
    \centering

    \begin{tikzpicture}[scale=0.7]
                         \node[state, s7color, label={[label distance=-0.1cm] 90:$s_7$}] (j1) at (0, 0) {\internaltext{\join}};
      \draw (j1)++(-30:1) node[state, s5color, label={[label distance=-0.1cm] 90:$s_5$}] (w)            {\internaltext{w}};
      \draw (j1)++(210:1) node[state, s6color, label={[label distance=-0.1cm] 90:$s_6$}] (j2)           {\internaltext{\join}};
      \draw (w)++(-90:1)  node[state, s2color, label={[label distance=-0.2cm] 60:$s_2$}] (u1)           {\internaltext{u}};
      \draw (u1)++(-90:1) node[state, s0color]                                           (s4)           {\internaltext{s_0}};
      \draw (j2)++(225:1) node[state, s3color, label={[label distance=-0.1cm] 90:$s_3$}] (v)            {\internaltext{v}};
      \draw (v)++(-90:1)  node[state, s1color, label={[label distance=-0.2cm]120:$s_1$}] (t2)           {\internaltext{t}};
      \draw (t2)++(-90:1) node[state, s0color]                                           (s1)           {\internaltext{s_0}};
      \draw (j2)++(-45:1) node[state, s4color, label={[label distance=-0.1cm] 90:$s_4$}] (j3)           {\internaltext{\join}};
      \draw (j3)++(240:1) node[state, s1color, label={[label distance=-0.2cm]120:$s_1$}] (t3)           {\internaltext{t}};
      \draw (t3)++(-90:1) node[state, s0color]                                           (s2)           {\internaltext{s_0}};
      \draw (j3)++(-60:1) node[state, s2color, label={[label distance=-0.1cm] 90:$s_2$}] (u3)           {\internaltext{u}};
      \draw (u3)++(-90:1) node[state, s0color]                                           (s3)           {\internaltext{s_0}};

      \tikzstyle{astedge}=[thick]
      \draw[astedge] (j1) to (w) to (u1) to (s4);
      \draw[astedge] (j1) to (j2) to (v) to (t2) to (s1);
      \draw[astedge] (j1) to (j2) to (j3) to (t3) to (s2);
      \draw[astedge] (j1) to (j2) to (j3) to (u3) to (s3);
    \end{tikzpicture}

    \caption{Expression}\figlabel{Expression}
  \end{subfigure}

  \caption{A system execution and corresponding expression}
  \figlabel{ExampleExpression}
\end{figure}
}

% - define invariant confluence
%   - define system model
%   - define expression based system model
%   - define invariant confluence
%   - Give example with two integers

% TODO: Give a footnote about the difference between this definition and the original definition.
