\section{Invariant Closure}\seclabel{InvariantClosure}
Our ultimate goal is to write a program that can automatically decide whether a
given distributed object $O$ is \sTIconfluent{}. Such a program has to
automatically prove or disprove that every reachable state satisfies the
invariant. However, automatically reasoning about the possibly infinite set of
reachable states is challenging, especially because transactions and merge
functions can be complex and can be interleaved arbitrarily in an execution.
Due to this complexity, existing systems that aim to automatically decide
\invariantconfluence{} instead focus on deciding a sufficient condition for
\invariantconfluence{}---dubbed \defword{\invariantclosure{}}---that is simpler
to reason about~\cite{li2012making, li2014automating}. In this section, we
define \invariantclosure{} and study why the condition is sufficient but not
necessary.  Armed with this understanding, we present conditions under which it
is both sufficient and necessary.

We say an object $O = (S, \join)$ is \defword{\invariantclosed{}} with respect
to an invariant $I$, abbreviated \defword{\Iclosed{}}, if invariant satisfying
states are closed under merge. That is, for every state $s_1, s_2 \in S$, if
$I(s_1)$ and $I(s_2)$, then $I(s_1 \join s_2)$.

\begin{theorem}\thmlabel{IclosureImpliesIconfluence}
  Given an object $O = (S, \join)$, a start state $s_0 \in S$, a set of
  transactions $T$, and an invariant $I$, if $I(s_0)$ and if $O$ is \Iclosed{},
  then $O$ is \sTIconfluent{}.
\end{theorem}

\thmref{IclosureImpliesIconfluence} states that \invariantclosure{} is
sufficient for \invariantconfluence{}. Intuitively, recall that our system model ensures
that transaction execution preserves the invariant, so if merging states also
preserves the invariant and if our start state satisfies the invariant, then
inductively it is impossible for us to reach a state that doesn't satisfy the
invariant.

This is good news because checking if an object is \invariantclosed{} is more
straightforward than checking if it is \invariantconfluent{}. Existing systems
typically use an SMT solver like Z3 to check if an object is
\invariantclosed{}~\cite{de2008z3, balegas2015putting, gotsman2016cause}. If it
is, then by \thmref{IclosureImpliesIconfluence}, it is \invariantconfluent{}.
Unfortunately, \invariantclosure{} is \emph{not} necessary for
\invariantconfluence{}, so if an object is \emph{not} \invariantclosed{}, these
systems cannot conclude that the object is \emph{not} \invariantconfluent{}.
The reason why \invariantclosure{} is not necessary for \invariantconfluence{}
is best explained through an example.

{\newcommand{\xmin}{-2}
\newcommand{\xmax}{2}
\newcommand{\ymin}{-2}
\newcommand{\ymax}{2}

\tikzstyle{point}=[shape=circle, fill=flatgray, inner sep=2pt]
\tikzstyle{inv}=[line width=0.75pt, draw=black]
\tikzstyle{pointinv}=[point, inv]
\tikzstyle{invregion}=[rounded corners, fill=flatgreen!50, draw=none]
\tikzstyle{reachableregion}=[rounded corners, fill=flatblue!50, draw=none]
\tikzstyle{statelabel}=[anchor=south west, inner sep=1pt]

% Axes.
\newcommand{\xyaxes}{
  \draw[] (\xmin.5, 0) to (\xmax.5, 0);
  \draw[] (0, \ymin.5) to (0, \ymax.5);
  \node at (\xmax + 1, 0) {$x$};
  \node at (0, \ymax + 1) {$y$};
}

% Quadrant 1.
\newcommand{\quadi}[5]{{
  \newcommand{\argstyle}{#1}
  \newcommand{\argxmin}{#2}
  \newcommand{\argxmax}{#3}
  \newcommand{\argymin}{#4}
  \newcommand{\argymax}{#5}
  \foreach \x in {0, ..., \argxmax} {
    \foreach \y in {0, ..., \argymax} {
      \node[\argstyle] (\x-\y) at (\x, \y) {};
    }
  }
}}

% Quadrant 2.
\newcommand{\quadii}[5]{{
  \newcommand{\argstyle}{#1}
  \newcommand{\argxmin}{#2}
  \newcommand{\argxmax}{#3}
  \newcommand{\argymin}{#4}
  \newcommand{\argymax}{#5}
  \foreach \x in {\argxmin, ..., 0} {
    \foreach \y in {0, ..., \argymax} {
      \node[\argstyle] (\x-\y) at (\x, \y) {};
    }
  }
}}

% Quadrant 3.
\newcommand{\quadiii}[5]{{
  \newcommand{\argstyle}{#1}
  \newcommand{\argxmin}{#2}
  \newcommand{\argxmax}{#3}
  \newcommand{\argymin}{#4}
  \newcommand{\argymax}{#5}
  \foreach \x in {\argxmin, ..., 0} {
    \foreach \y in {\argymin, ..., 0} {
      \node[\argstyle] (\x-\y) at (\x, \y) {};
    }
  }
}}

% Quadrant 4.
\newcommand{\quadiv}[5]{{
  \newcommand{\argstyle}{#1}
  \newcommand{\argxmin}{#2}
  \newcommand{\argxmax}{#3}
  \newcommand{\argymin}{#4}
  \newcommand{\argymax}{#5}
  \foreach \x in {0, ..., \argxmax} {
    \foreach \y in {\argymin, ..., 0} {
      \node[\argstyle] (\x-\y) at (\x, \y) {};
    }
  }
}}

% State labels.
\newcommand{\statelabels}{
  \node[statelabel] at (0, 0) {$s_0$};
  \node[statelabel] at (-1, 1) {$s_1$};
  \node[statelabel] at (1, -1) {$s_2$};
  \node[statelabel] at (1, 1) {$s_3$};
}

\begin{figure}[t]
  \centering

  \begin{subfigure}[b]{0.5\columnwidth}
    \centering
    \begin{tikzpicture}[scale=0.4]
      \begin{scope}
        \clip (\xmin.5, \ymax.5) rectangle (\xmax.5, \ymin.5);
        \draw[invregion] (\xmin.9, \ymax.9) rectangle (0.5, -0.5);
        \draw[invregion] (-0.5, 0.5) rectangle (\xmax.9, \ymin.9);
        \draw (0.5, 0.5) to (0.5, \ymax.5);
        \draw (0.5, 0.5) to (\xmax.5, 0.5);
        \draw (-0.5, -0.5) to (-0.5, \ymin.5);
        \draw (-0.5, -0.5) to (\xmin.5, -0.5);
      \end{scope}

      \xyaxes{}
      \quadi{point}{\xmin}{\xmax}{\ymin}{\ymax}
      \quadiii{point}{\xmin}{\xmax}{\ymin}{\ymax}
      \quadii{pointinv}{\xmin}{\xmax}{\ymin}{\ymax}
      \quadiv{pointinv}{\xmin}{\xmax}{\ymin}{\ymax}
      \statelabels{}
    \end{tikzpicture}
    \caption{Invariant}\figlabel{Z2Invariant}
  \end{subfigure}%
  \begin{subfigure}[b]{0.5\columnwidth}
    \centering
    \begin{tikzpicture}[scale=0.4]
      \begin{scope}
        \clip (-1, 1) rectangle (\xmax.5, \ymin.5);
        \draw[reachableregion, draw=black] (-0.5, 0.5) rectangle (\xmax.9, \ymin.9);
      \end{scope}

      \xyaxes{}
      \quadi{point}{\xmin}{\xmax}{\ymin}{\ymax}
      \quadiii{point}{\xmin}{\xmax}{\ymin}{\ymax}
      \quadii{point}{\xmin}{\xmax}{\ymin}{\ymax}
      \quadiv{pointinv}{\xmin}{\xmax}{\ymin}{\ymax}
      \statelabels{}
    \end{tikzpicture}
    \caption{Reachable points}\figlabel{Z2Reachable}
  \end{subfigure}

  \caption{An illustration of \exampleref{Z2}}\figlabel{Z2}
\end{figure}
}

\begin{example}\examplelabel{Z2}
  Let $O = (\ints \times \ints, \join)$ consist of pairs $(x, y)$ of integers
  where $(x_1, y_1) \join (x_2, y_2) = (\max(x_1, x_2), \max(y_1, y_2))$. Our
  start state $s_0 \in \ints \times \ints$ is the point $(0, 0)$. Our set $T$
  of transactions consists of two transactions: $t_{x+1}((x, y)) = (x + 1, y)$
  which increments $x$ and $t_{y-1}((x, y)) = (x, y - 1)$ which decrements $y$.
  Our invariant $I = \setst{(x, y) \in \ints \times \ints}{xy \leq 0}$ consists
  of all points $(x, y)$ where the product of $x$ and $y$ is non-positive.

  The invariant and the set of reachable states are illustrated in \figref{Z2}
  in which we draw each state $(x, y)$ as a point in space. The invariant
  consists of the second and fourth quadrant, while the reachable states consist only
  of the fourth quadrant. From this, it is immediate that the reachable states
  are a subset of the invariant, so $O$ is \invariantconfluent{}. However,
  letting $s_1 = (-1, 1)$ and $s_2 = (1, -1)$, we see that $O$ is not
  \invariantclosed{}. $I(s_1)$ and $I(s_2)$, but letting $s_3 = s_1 \join s_2 =
  (1, 1)$, we see $\lnot I(s_3)$.
\end{example}

In \exampleref{Z2}, $s_1$ and $s_2$ witness the fact that $O$ is not
\invariantclosed{}, but $s_1$ is not reachable. This is not particular to
\exampleref{Z2}. In fact, it is fundamentally the reason why
\invariantclosure{} is not equivalent to \invariantconfluence{}.
\Invariantconfluence{} is, at its core, a property of reachable states, but
\invariantclosure{} is completely ignorant of reachability. As a result,
invariant-satisfying yet unreachable states like $s_1$ are the key hurdle
preventing \invariantclosure{} from being equivalent to \invariantconfluence{}.
This is formalized by \thmref{IclosureEquivalentIconfluence}.

\begin{theorem}\thmlabel{IclosureEquivalentIconfluence}
  Consider an object $O = (S, \join)$, a start state $s_0 \in S$, a set of
  transactions $T$, and an invariant $I$. If the invariant is a subset of the
  reachable states (i.e. $I \subseteq \setst{s \in
  S}{\sTIreachablepredicate{s}}$), then
  \[
    (\text{$I(s_0)$ and $O$ is \Iclosed{}}) \iff \text{$O$ is \sTIconfluent{}}
  \]
\end{theorem}
\begin{proof}
  The forward direction of \thmref{IclosureEquivalentIconfluence} follows
  immediately from \thmref{IclosureImpliesIconfluence}. The backward direction
  holds because any two invariant satisfying states $s_1$ and $s_2$ must be
  reachable (by assumption), so their join $s_1 \join s_2$ is also reachable.
  And because $O$ is \sTIconfluent{}, all reachable points, including $s_1
  \join s_2$, satisfy the invariant.
\end{proof}
