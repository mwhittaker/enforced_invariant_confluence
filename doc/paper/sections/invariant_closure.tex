\section{Invariant-Closure}\seclabel{InvariantClosure}
Our ultimate goal is to write a program that can automatically decide whether a
given distributed object $O$ is \sTIconfluent{}. Such a program has to
automatically prove or disprove that every reachable state satisfies the
invariant. However, automatically reasoning about the possibly infinite set of
reachable states is challenging, especially because transactions and merge
functions can be complex and can be interleaved arbitrarily in an execution.
Due to this complexity, existing systems that aim to automatically decide
invariant-confluence instead focus on deciding a sufficient condition for
invariant-confluence---dubbed \defword{invariant-closure}---that is simpler to
reason about~\cite{li2012making, li2014automating}. In this section, we define
invariant-closure and study why the sufficient condition is not necessary.
Armed with this understanding, we present conditions under which it is both
sufficient and necessary.

% - define invariant-closure and introduce it as a sufficient condition
% - cite existing work to show that people have used invariant closure to
%   reason about invariant-confluence
% - explain why invariant closure can fail and why its not sufficient using the
%   two-integers example
% - state main theorem

