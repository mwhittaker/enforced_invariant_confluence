\section{Interactive \Iconfluence{} Decision Procedure}
Imagine we have an \Iclosed{} decision procedure, and let's revisit
\exampleref{StateBasedNotNecessary}. We have a state-based object $O$, a set of
transactions $T$, and an invariant $I$. We want to know if $T$ is
\Iconfluent{}. Assume we have already determined that $I(s_0)$. We can hand $O$
and $I$ to the \Iclosed{} decision procedure to see if $O$ is \Iclosed{}. If
the decision procedure says yes, then by \clmref{StateBasedSufficient}, we know
that $(O, T)$ is \Iconfluent{}.

However, in our example, $O$ is not \Iclosed{}, so the decision procedure will
say no. \clmref{SufficientAndNecessary} tells us that this means one of two
things. Either (1) $T$ is not \Iconfluent{} or (2) $(O, T)$ is \Iconfluent{},
but there are some invariant satisfying states that are not reachable. If we
think harder about the set of reachable states, we may realize that all
reachable states have a non-negative value of $x$. Thus, we may modify our
invariant from $\setst{(x, y)}{xy \leq 0}$ to $\setst{(x, y)}{xy \leq 0 \land x
\geq 0}$. Now, $O$ is \Iclosed{}! If we hand $O$ and $I$ to the decision
procedure, it will say yes, and we can conclude that $(O, T)$ is \Iconfluent{}.

\newcommand{\comment}[1]{\State \textcolor{flatdenim}{\texttt{//} #1}}
\begin{algorithm}
  \caption{Interactive \Iconfluence{} Decision Procedure}%
  \algolabel{InteractiveDecisionProcedure}
  \begin{algorithmic}
    \Function{IsIconfluent}{$O$, $T$, $I$}
      \If {$\lnot I(s_0)$}
        \State \Return no
      \Else
        \State \Return \Call{Helper}{$O$, $T$, $I$, $\set{s_0}$, $\emptyset$}
      \EndIf
    \EndFunction

    \State

    \comment{$R$ is a set of reachable states.}
    \comment{$NR$ is a set of non-reachable states.}
    \Function{Helper}{$O$, $T$, $I$, $R$, $NR$}
      \State closed, $s_1$, $s_2$ $\gets$ \Call{IsIclosed}{$O$, $I - NR$}
      \If {closed $=$ unknown}
        \State \Return unknown
      \ElsIf {closed $=$ yes}
        \State \Return{yes}
      \Else
        \State Augment $R$ and $NR$ with random search and user input
        \State Make sure $s_1$ and $s_2$ are included in $R$ or $NR$
        \If{$s_1, s_2 \in R$}
          \State \Return no
        \Else
          \State \Return \Call{Helper}{$O$, $T$, $I$, $R$, $NR$}
        \EndIf
      \EndIf
    \EndFunction
  \end{algorithmic}
\end{algorithm}

\newcommand{\IsIconfluent}{\textsc{IsIconfluent}}
\newcommand{\IsIclosed}{\textsc{IsIclosed}}
\newcommand{\Helper}{\textsc{Helper}}

This example motivates \algoref{InteractiveDecisionProcedure}, a general
purpose interactive \Iconfluence{} decision procedure in which a user
iteratively refines the invariant until the decision procedure terminates.
Given an object $O$, a set of transactions $T$, and an invariant $I$, we call
\IsIconfluent$(O, T, I)$ to determine if $(O, T)$ is \Iconfluent{}.

\IsIconfluent{} assumes access to an \Iclosed{} decision procedure
\IsIclosed$(O, I)$ which returns whether $O$ is \Iclosed{} (yes, no, or
unknown). If it returns no, it also returns a counterexample $s_1$ and $s_2$
such that $I(s_1)$ and $I(s_2)$, but $\lnot I(s_1 \join s_2)$. Constructing an
\Iclosed{} decision procedure is hard (undecidable, I think), but we can do
pretty well using an SMT solver like Z3. There is also some existing research
on how to built smart \Iclosed{} decision procedures that we can
leverage~\cite{li2014automating}.

\IsIconfluent{} takes in $O$, $T$, and $I$. Formally, $I$ is set of states
(e.g. $\setst{(x, y)}{x \in \ints, y \in \ints, xy \leq 0}$), but an actual
implementation of \IsIconfluent{} would likely take in $I$ as a logical formula
(e.g. $xy \leq 0$). From now on, whenever we mention a (possibly infinite) set
of states, remember that we can always represent the set as a finite formula.

\IsIconfluent{} calls a helper function \Helper{} which---in addition to $O$,
$T$, and $I$---takes as arguments a set $R$ of reachable states and a set $NR$
of unreachable states. Like \IsIconfluent, \Helper$(O, T, I, R, NR)$ returns
whether $T$ is \Iconfluent. Intuitively, $NR$ helps remove unreachable states
from $I$ moving us closer to the case where $I \subseteq
\setst{e}{\reachable{e}}$.

\Helper{} first checks to see if $O$ is \iclosed{$(I - NR)$}. There are three
cases to consider.
\begin{itemize}
  \item
    If \IsIclosed{} gets stuck and doesn't know whether or not $O$ is
    \iclosed{$(I - NR)$}, then \Helper{} is stuck too and is forced to return
    unknown.
  \item
    If \IsIclosed{} determines that $O$ is \iclosed{$(I - NR)$}, then by
    \clmref{StateBasedSufficient}, $T$ is \iconfluent{$(I - NR)$}, so
    $\setst{e}{\reachable{e}} \subseteq I - NR$. $I - NR \subseteq I$, so $T$
    is therefore also \Iconfluent{}. Thus, \Helper{} return yes.
  \item
    If \IsIclosed{} determines that $O$ is not \iclosed{$(I - NR)$}, then we
    have a counterexample $s_1$ and $s_2$. We want to determine whether $s_1$
    and $s_2$ are reachable or unreachable. We can do so in two ways.

    First, we can randomly generate a set of reachable states and add them to
    $R$.

    Second, we can prompt the user to tell us whether or not the states are
    reachable or unreachable. In addition to labelling $s_1$ and $s_2$ as
    reachable or not, the user can also refine $I$ by augmenting $NR$. Taking
    \exampleref{StateBasedNotNecessary} as an example, the user might look
    through the existing reachable and unreachable states and have an aha
    moment realizing that every point with a negative $x$-coordinate is
    unreachable. They can use this knowledge to add all points with negative
    $x$-coordinates to $NR$.

    After $s_1$ and $s_2$ have been labelled reachable or not, we continue. If
    both $s_1$ and $s_2$ are reachable, then so is $s_1 \join s_2$, but $\lnot
    I(s_1 \join s_2)$. Thus, $(O, T)$ is not \Iconfluent{}, so \Helper{}
    returns no.  Otherwise, one of $s_1$ and $s_2$ is unreachable. Now, we
    recurse.
\end{itemize}

Let's walk through how we might use \IsIconfluent{} on
\exampleref{StateBasedNotNecessary}.
\begin{itemize}
  \item
    \IsIconfluent{} calls \Helper{} with $R = \set{(0, 0)}, NR = \emptyset{}$.

  \item
    \Helper{} determines that $O$ is not \iclosed{$(I - R)$} with
    counterexample $s_1 = (-1, 1)$ and $s_2 = (1, -1)$. We take the third
    branch.

  \item
    We randomly generate some reachable points and update $R$ to $\set{(0, 0),
    (1, 0), (0, -1), (1, -1)}$. Lucky for us, $s_2 \in R$, so we know it is
    reachable. We're not sure about $s_1$, so we ask the user. After some
    thought, the user tells us that $s_1$ is unreachable. We recurse with $R =
    \set{(0, 0), (1, 0), (0, -1), (1, -1)}$ and $NR = \set{(-1, 1)}$.

  \item
    \Helper{} determines that $O$ is not \iclosed{$(I - R)$} with
    counterexample $s_1 = (-1, 2)$ and $s_2 = (10, -10)$. We take the third
    branch.

  \item
    We randomly generate some more reachable points and update $R$ to $\set{(0,
    0), (1, 0), (0, -1), (1, -1), (2, 0), (0, -2)}$. $s_1, s_2 \notin R, NR$,
    so we ask the user to label them. The user puts $s_1$ in $NR$ and $s_2$ in
    $R$. We recurse with $R = \set{(0, 0), (1, 0), (0, -1), (1, -1), (2, 0),
    (0, -2), (10, -10)}$ and $NR = \set{(-1, 1), (-1, 2)}$.

  \item
    \Helper{} determines that $O$ is not \iclosed{$(I - R)$} with
    counterexample $s_1 = (-2, 1)$ and $s_2 = (4, -3)$. We take the third
    branch.

  \item
    We randomly generate some more reachable points and update $R$. $s_1, s_2
    \notin R, NR$, so we ask the user to label them. The user notices a pattern
    in $R$ and $NR$ and after some thought, concludes that every point with
    negative $x$-coordinate is unreachable. They update $NR$ to $-\ints \times
    \ints$. We recurse.

  \item
    \Helper{} determines that $O$ is \iclosed{$(I - R)$} and returns yes!
\end{itemize}

\todo{What if the user labels things incorrectly?}
