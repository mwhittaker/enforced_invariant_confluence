\section{Interactive Invariant-Confluence Decision Procedure}
\newcommand{\IsInvConfluent}{\textsc{IsInvConfluent}}
\newcommand{\IsIclosed}{\textsc{IsIclosed}}
\newcommand{\Helper}{\textsc{Helper}}

Imagine we have an \Iclosed{} decision procedure, and let's revisit
\exampleref{StateBasedNotNecessary}. We have a state-based object $O$, a start
state $s_0$, set of transactions $T$, and an invariant $I$. We want to know if
$O$ is \sTIconfluent{}. We can hand $O$ and $I$ to the \Iclosed{} decision
procedure to see if $O$ is \Iclosed{}. If the decision procedure says yes and
we can conclude that $I(s_0)$, then by \clmref{StateBasedSufficient}, we know
that $O$ is \sTIconfluent{}.

However, in our example, $O$ is not \Iclosed{}, so the decision procedure will
say no. \clmref{SufficientAndNecessary} tells us that this means one of two
things. Either (1) $O$ is \emph{not} \sTIconfluent{} or (2) $O$ is
\sTIconfluent{}, but there are some invariant satisfying states that are not
reachable. If we think harder about the set of reachable states, we may realize
that all reachable states have a non-negative value of $x$. Thus, we may modify
our invariant from $\setst{(x, y)}{xy \leq 0}$ to $\setst{(x, y)}{xy \leq 0
\land x \geq 0}$. Now, $O$ is \Iclosed{}! If we hand $O$ and $I$ to the
decision procedure, it will say yes. Moreover, $I(s_0)$, so we can conclude
that $O$ is \sTIconfluent{}.

\newcommand{\comment}[1]{\State \textcolor{flatdenim}{\texttt{//} #1}}
\begin{algorithm}
  \caption{Interactive Invariant-Confluence Decision Procedure}%
  \algolabel{InteractiveDecisionProcedure}
  \begin{algorithmic}
    \Function{IsInvConfluent}{$O$, $s_0$, $T$, $I$}
      \State
        \Return $I(s_0)$ and
                \Call{Helper}{$O$, $s_0$, $T$, $I$, $\set{s_0}$, $\emptyset$}
    \EndFunction

    \State

    \comment{$R$ is a set of $\sTIreachable{}$ states.}
    \comment{$NR$ is a set of $\sTIunreachable{}$ states.}
    \comment{$I(s_0)$ is a precondition.}
    \Function{Helper}{$O$, $s_0$, $T$, $I$, $R$, $NR$}
      \State closed, $s_1$, $s_2$ $\gets$ \Call{IsIclosed}{$O$, $I - NR$}
      \If {closed}
        \State \Return true
      \Else
        \State Augment $R$ and $NR$ with random search and user input
        \If{$s_1, s_2 \in R$}
          \State \Return false
        \Else
          \State \Return \Call{Helper}{$O$, $s_0$, $T$, $I$, $R$, $NR$}
        \EndIf
      \EndIf
    \EndFunction
  \end{algorithmic}
\end{algorithm}

\tikzstyle{point}=[shape=circle, fill=flatgray, inner sep=1.2pt, draw=black]
\tikzstyle{reachable}=[fill=flatgreen!50, draw=none]
\tikzstyle{unreachable}=[fill=flatred!50, draw=none]
\tikzstyle{invariant}=[fill=flatblue!50, draw=none]
\tikzstyle{nothing}=[fill=white, draw=none]

\newcommand{\pointgrid}[4]{{
  \newcommand{\argxmin}{#1}
  \newcommand{\argxmax}{#2}
  \newcommand{\argymin}{#3}
  \newcommand{\argymax}{#4}

  \draw[] (\argxmin, 0) to (\argxmax, 0);
  \draw[] (0, \argymin) to (0, \argymax);
  \foreach \x in {\argxmin, ..., \argxmax} {
    \foreach \y in {\argymin, ..., \argymax} {
      \node[point] (\x-\y) at (\x, \y) {};
    }
  }
}}

\newcommand{\pointrect}[2]{
  \draw[#1] ($#2 + (-0.51, 0.51)$) rectangle ($#2 + (0.51, -0.51)$);
}

\newcommand{\subfigwidth}{0.48\textwidth}
\newcommand{\subfighspace}{0.3cm}
\newcommand{\tikzhspace}{0.4cm}
\newcommand{\tikzscale}{0.25}
\newcommand{\xmin}{-3}
\newcommand{\xmax}{3}
\newcommand{\ymin}{-3}
\newcommand{\ymax}{3}

\begin{figure*}[ht]
  \centering

  \begin{subfigure}[t]{0.38\textwidth}
    \centering
    \begin{tikzpicture}[scale=\tikzscale]
      \draw[reachable, draw=black] (-0.5, 0.5) rectangle (0.5, -0.5);
      \pointgrid{\xmin}{\xmax}{\ymin}{\ymax}
      \node at (0, \ymax + 1) {$R$};
    \end{tikzpicture}\hspace{\tikzhspace}
    \begin{tikzpicture}[scale=\tikzscale]
      \pointgrid{\xmin}{\xmax}{\ymin}{\ymax}
      \node at (0, \ymax + 1) {$NR$};
    \end{tikzpicture}\hspace{\tikzhspace}
    \begin{tikzpicture}[scale=\tikzscale]
      \draw[invariant] (\xmin.5, \ymax.5) rectangle (0.5, -0.5);
      \draw[invariant] (-0.5, 0.5) rectangle (\xmax.5, \ymin.5);
      \draw (0.5, 0.5) to (0.5, \ymax.5);
      \draw (0.5, 0.5) to (\xmax.5, 0.5);
      \draw (-0.5, -0.5) to (-0.5, \ymin.5);
      \draw (-0.5, -0.5) to (\xmin.5, -0.5);
      \pointgrid{\xmin}{\xmax}{\ymin}{\ymax}
      \node at (0, \ymax + 1) {$I - NR$};
    \end{tikzpicture}
    \caption{%
      \IsInvConfluent{} determines $I(s_0)$ and then calls \Helper{} with $R =
      \set{s_0}$, $NR = \emptyset{}$, and $I = \setst{(x, y)}{xy \leq 0}$.
    }
    \figlabel{DecisionProcedureExampleA}
  \end{subfigure}
  \hspace{\subfighspace}
  \begin{subfigure}[t]{0.58\textwidth}
    \centering
    \begin{tikzpicture}[scale=\tikzscale]
      \draw[reachable, draw=black] (-0.5, 0.5) rectangle (1.5, -1.5);
      \pointgrid{\xmin}{\xmax}{\ymin}{\ymax}
      \node at (0, \ymax + 1) {$R$};
    \end{tikzpicture}\hspace{\tikzhspace}
    \begin{tikzpicture}[scale=\tikzscale]
      \pointrect{unreachable, draw=black}{(-1, 1)}
      \pointgrid{\xmin}{\xmax}{\ymin}{\ymax}
      \node at (0, \ymax + 1) {$NR$};
    \end{tikzpicture}\hspace{\tikzhspace}
    \begin{tikzpicture}[scale=\tikzscale]
      \draw[invariant] (\xmin.5, \ymax.5) rectangle (0.5, -0.5);
      \draw[invariant] (-0.5, 0.5) rectangle (\xmax.5, \ymin.5);
      \draw (0.5, 0.5) to (0.5, \ymax.5);
      \draw (0.5, 0.5) to (\xmax.5, 0.5);
      \draw (-0.5, -0.5) to (-0.5, \ymin.5);
      \draw (-0.5, -0.5) to (\xmin.5, -0.5);
      \pointrect{nothing, draw=black}{(-1, 1)}
      \pointgrid{\xmin}{\xmax}{\ymin}{\ymax}
      \node at (0, \ymax + 1) {$I - NR$};
    \end{tikzpicture}
    \caption{%
      \Helper{} determines that $O$ is not \iclosed{(I - NR)} with
      counterexample $s_1 = (-1, 1)$ and $s_2 = (1, -1)$. \Helper{} randomly
      generates some \sTIreachable{} points and adds them to $R$. Luckily for
      us, $s_2 \in R$, so \Helper{} knows that it is \sTIreachable{}.
      \Helper{} is not sure about $s_1$, so it asks the user. After some
      thought, the user tells \Helper{} that $s_1$ is \sTIunreachable{}, so
      \Helper{} adds $s_1$ to $NR$ and then recurses.
    }
    \figlabel{DecisionProcedureExampleB}
  \end{subfigure}

  \begin{subfigure}[t]{0.38\textwidth}
    \centering
    \begin{tikzpicture}[scale=\tikzscale]
      \draw[reachable, draw=black] (-0.5, 0.5) rectangle (2.5, -2.5);
      \pointrect{reachable, draw=black}{(3, -3)}
      \pointgrid{\xmin}{\xmax}{\ymin}{\ymax}
      \node at (0, \ymax + 1) {$R$};
    \end{tikzpicture}\hspace{\tikzhspace}
    \begin{tikzpicture}[scale=\tikzscale]
      \pointrect{unreachable}{(-1, 1)}
      \pointrect{unreachable}{(-1, 2)}
      \draw (-1.5, 0.5) rectangle (-0.5, 2.5);
      \pointgrid{\xmin}{\xmax}{\ymin}{\ymax}
      \node at (0, \ymax + 1) {$NR$};
    \end{tikzpicture}\hspace{\tikzhspace}
    \begin{tikzpicture}[scale=\tikzscale]
      \draw[invariant] (\xmin.5, \ymax.5) rectangle (0.5, -0.5);
      \draw[invariant] (-0.5, 0.5) rectangle (\xmax.5, \ymin.5);
      \draw (0.5, 0.5) to (0.5, \ymax.5);
      \draw (0.5, 0.5) to (\xmax.5, 0.5);
      \draw (-0.5, -0.5) to (-0.5, \ymin.5);
      \draw (-0.5, -0.5) to (\xmin.5, -0.5);
      \pointrect{nothing}{(-1, 1)}
      \pointrect{nothing}{(-1, 2)}
      \draw (-1.5, 0.5) rectangle (-0.5, 2.5);
      \pointgrid{\xmin}{\xmax}{\ymin}{\ymax}
      \node at (0, \ymax + 1) {$I - NR$};
    \end{tikzpicture}
    \caption{%
      \Helper{} determines that $O$ is not \iclosed{(I - NR)} with
      counterexample $s_1 = (-1, 2)$ and $s_2 = (3, -3)$. \Helper{} randomly
      generates some \sTIreachable{} points and adds them to $R$. $s_1, s_2
      \notin R, NR$, so \Helper{} ask the user to label them. The user puts
      $s_1$ in $NR$ and $s_2$ in $R$. Then, \Helper{} recurses.
    }
    \figlabel{DecisionProcedureExampleC}
  \end{subfigure}
  \hspace{\subfighspace}
  \begin{subfigure}[t]{0.58\textwidth}
    \centering
    \begin{tikzpicture}[scale=\tikzscale]
      \draw[reachable] (-0.5, 0.5) rectangle (2.5, -2.5);
      \pointrect{reachable}{(3, 0)}
      \pointrect{reachable}{(3, -1)}
      \pointrect{reachable}{(0, -3)}
      \pointrect{reachable}{(2, -3)}
      \pointrect{reachable}{(3, -3)}
      \draw (-0.5, 0.5) to (\xmax.5, 0.5);
      \draw (-0.5, 0.5) to (-0.5, \ymin.5);
      \draw (-0.5, -3.5) to ++(1, 0)
                         to ++(0, 1)
                         to ++(1, 0)
                         to ++(0, -1)
                         to ++(2, 0)
                         to ++(0, 1)
                         to ++(-1, 0)
                         to ++(0, 1)
                         to ++(1, 0)
                         to ++(0, 2);
      \pointgrid{\xmin}{\xmax}{\ymin}{\ymax}
      \node at (0, \ymax + 1) {$R$};
    \end{tikzpicture}\hspace{\tikzhspace}
    \begin{tikzpicture}[scale=\tikzscale]
      \draw[unreachable] (\xmin.5, \ymax.5) rectangle (-0.5, \ymin.5);
      \draw (\xmin.5, \ymax.5) to (-0.5, \ymax.5) to
            (-0.5, \ymin.5) to (\xmin.5, \ymin.5);
      \pointgrid{\xmin}{\xmax}{\ymin}{\ymax}
      \node at (0, \ymax + 1) {$NR$};
    \end{tikzpicture}\hspace{\tikzhspace}
    \begin{tikzpicture}[scale=\tikzscale]
      \draw[invariant] (-0.5, 0.5) rectangle (\xmax.5, \ymin.5);
      \draw (-0.5, 0.5) to (\xmax.5, 0.5);
      \draw (-0.5, 0.5) to (-0.5, \ymin.5);
      \pointgrid{\xmin}{\xmax}{\ymin}{\ymax}
      \node at (0, \ymax + 1) {$I - NR$};
    \end{tikzpicture}
    \caption{%
      \Helper{} determines that $O$ is not \iclosed{(I - NR)} with
      counterexample $s_1 = (-2, 1)$ and $s_2 = (1, -1)$. \Helper{} randomly
      generates some \sTIreachable{} points and adds them to $R$. $s_2 \in R$
      but $s_1 \notin R, NR$, so \Helper{} asks the user to label $s_1$. The
      user notices a pattern in $R$ and $NR$ and after some thought, concludes
      that every point with negative $x$-coordinate is \sTIunreachable{}.
      They update $NR$ to $-\ints \times \ints$. Then, \Helper{} recurses.
      \Helper{} determines that $O$ is \iclosed{(I - NR)} and returns true!
    }
    \figlabel{DecisionProcedureExampleD}
  \end{subfigure}

  \vspace{1em}
  \hrule
  \vspace{1em}

  \caption{%\vspace{1em}
    An example of a user interacting with
    \algoref{InteractiveDecisionProcedure} on \exampleref{Z2}.  Each step of
    the visualization shows reachable states $R$ (left), non-reachable states
    $NR$ (middle), and the refined invariant $I - NR$ (right) as the algorithm
    executes
  }
  \figlabel{DecisionProcedureExample}
\end{figure*}


This example motivates \algoref{InteractiveDecisionProcedure}, a general
purpose interactive invariant-confluence decision procedure in which a user
iteratively refines the invariant until the decision procedure terminates.
Given an object $O$, a start state $s_0$, a set of transactions $T$, and an
invariant $I$\footnote{%
  Formally, $I$ is set of states (e.g. $\setst{(x, y)}{x \in \ints, y \in
  \ints, xy \leq 0}$), but an actual implementation of \IsInvConfluent{} would
  likely take in $I$ as a logical formula (e.g. $xy \leq 0$). From now on,
  whenever we mention a (possibly infinite) set of states, remember that we can
  always represent the set as a finite formula.
}, we call \IsInvConfluent$(O, s_0, T, I)$ to determine if $O$ is
\sTIconfluent{}. Now, we describe \algoref{InteractiveDecisionProcedure}. An
example of how to use \algoref{InteractiveDecisionProcedure} on
\exampleref{StateBasedNotNecessary} is given in
\figref{DecisionProcedureExample}.

\IsInvConfluent{} assumes access to two helper decision procedures:
\begin{enumerate}
  \item
    A decision procedure to determine $I(s_0)$.

  \item
    An \Iclosed{} decision procedure \IsIclosed$(O, I)$ which returns a triple
    (closed, $s_1$, $s_2$). closed is a boolean indicating whether $O$ is
    \Iclosed{}. If closed is false, then $s_1$ and $s_2$ are a counterexample
    such that $I(s_1)$ and $I(s_2)$, but $\lnot I(s_1 \join s_2)$.
\end{enumerate}

Constructing these two helper decision procedures is hard (undecidable, I
think), but we can do pretty well using an SMT solver like Z3. There is also
some existing research on how to built smart \Iclosed{} decision procedures
that we can leverage~\cite{li2014automating}. In practice either decision an
return unknown instead of true or false. In this case, \IsInvConfluent{} returns
unknown.

\IsInvConfluent{} first checks that $s_0$ satisfies $I$. If it doesn't, then
\IsInvConfluent{} returns false. If it does then, \IsInvConfluent{} calls a
helper function \Helper{} which---in addition to $O$, $s_0$, $T$, and
$I$---takes as arguments a set $R$ of $\sTIreachable{}$ states and a set $NR$
of $\sTIunreachable{}$ states. Like \IsInvConfluent, \Helper$(O, s_0, T, I, R,
NR)$ returns whether $O$ is \sTIconfluent{} (assuming $R$ and $NR$ are
correct). Intuitively, $NR$ helps remove unreachable states from $I$ moving us
closer to the case where $I \subseteq \setst{s}{\sTIreachable{s}}$.

First, \Helper{} checks to see if $O$ is \iclosed{(I - NR)}. There are two
cases to consider.
\begin{itemize}
  \item
    If \IsIclosed{} determines that $O$ is \iclosed{(I - NR)}, then by
    \clmref{StateBasedSufficient}, $O$ is \sticonfluent{s_0}{T}{(I -
    NR)}, so
    $
      \setst{s}{\stireachable{s_0}{T}{(I-NR)}{s}}
        \subseteq I - NR
        \subseteq I
    $.
    If $NR$ only contains unreachable states, then
    $
      \setst{s}{\sTIreachable{s}} = \setst{s}{\stireachable{s_0}{T}{(I-NR)}{s}}
    $.
    Thus,
    $
      \setst{s}{\sTIreachable{s}} \subseteq I
    $.
    Thus, $O$ \sTIconfluent{}, so \Helper{} return true.

  \item
    If \IsIclosed{} determines that $O$ is not \iclosed{(I - NR)}, then we
    have a counterexample $s_1$ and $s_2$. We want to determine whether $s_1$
    and $s_2$ are $\sTIreachable{}$ or $\sTIunreachable{}$. We can do so in two
    ways.

    First, we can randomly generate a set of reachable states and add them to
    $R$.
    %
    Second, we can prompt the user to tell us whether or not the states are
    $\sTIreachable{}$ or $\sTIunreachable{}$. In addition to labelling $s_1$
    and $s_2$ as $\sTIreachable{}$ or not, the user can also refine $I$ by
    augmenting $R$ and $NR$ arbitrarily (see \figref{DecisionProcedureExample}
    for an example). In this step, we also make sure that $s_0 \notin NR$ since
    we know that $s_0$ is $\sTIreachable{}$.

    After $s_1$ and $s_2$ have been labelled $\sTIreachable{}$ or not, we
    continue. If both $s_1$ and $s_2$ are $\sTIreachable{}$, then so is $s_1
    \join s_2$, but $\lnot I(s_1 \join s_2)$. Thus, $O$ is not \sTIconfluent{},
    so \Helper{} returns false. Otherwise, one of $s_1$ and $s_2$ is
    $\sTIunreachable{}$, so we recurse.
\end{itemize}

\algoref{InteractiveDecisionProcedure} is an \emph{interactive} decision
procedure. It requires that a user classify states as reachable or unreachable.
What happens if a user incorrectly classifies a state? There's two possible
errors that can be made:
\begin{itemize}
  \item \textbf{A user can label an unreachable state as reachable.}
    In this case, the $s_1, s_2 \in R$ check in \Helper{} might be erroneously
    met. In turn, \IsInvConfluent{} might erroneously return no. This is
    inconvenient, but not the end of the world. Moreover, we can modify our
    decision procedure so that \Helper{} requires that whenever a user labels a
    state $s$ as $\sTIreachable{}$, they must also provide an $\sTIreachable{}$
    expression $e$ where $eval(e) = s$.

  \item \textbf{A user can label a reachable state as unreachable.}
    In this case, \IsIclosed$(O, I - NR)$ might return true, even though $O$ is
    not \sTIconfluent. Thus, a user might falsely believe their system to be
    \sTIconfluent{} even though it isn't, and eventually their system might
    enter a state which violates the invariant.

    We can mitigate this in two ways. First, we can aggresively expand $R$. If
    a user ever labels a state $s$ as unreachable, but $s \in R$, we can notify
    the user of their mistake. Second, \Helper{} returns true when $O$ is
    \iclosed{(I - NR)} for some $NR$, so $O$ is \sticonfluent{s_0}{T}{(I-R)}
    (even though it might not be \sTIconfluent{}). Thus, when a user deploys
    their system, they can deploy with the invariant $I - NR$ instead of $NR$.
    If $NR$ is correctly populated, then deploying with $I - NR$ has no effect
    on the system. If $NR$ is incorrectly populated, then some
    $\sTIreachable{}$ states are artificially made unreachable, but the system
    is still guaranteed to never enter a state which violates $I$.
\end{itemize}
