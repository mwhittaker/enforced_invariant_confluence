\section{Interactive \Iconfluence{} Decision Procedure}
Imagine we have an \Iclosed{} decision procedure, and let's revisit
\exampleref{StateBasedNotNecessary}. We have a state-based object $O$, a set of
transactions $T$, and an invariant $I$. We want to know if $T$ is
\Iconfluent{}. Assume we have already determined that $I(s_0)$. We can hand $O$
and $I$ to the \Iclosed{} decision procedure to see if $O$ is \Iclosed{}. If
the decision procedure says yes, then by \clmref{StateBasedSufficient}, we know
that $(O, T)$ is \Iconfluent{}.

However, in our example, $O$ is not \Iclosed{}, so the decision procedure will
say no. \clmref{SufficientAndNecessary} tells us that this means one of two
things. Either (1) $T$ is not \Iconfluent{} or (2) $(O, T)$ is \Iconfluent{},
but there are some invariant satisfying states that are not reachable. If we
think harder about the set of reachable states, we may realize that all
reachable states have a non-negative value of $x$. Thus, we may modify our
invariant from $\setst{(x, y)}{xy \leq 0}$ to $\setst{(x, y)}{xy \leq 0 \land x
\geq 0}$. Now, $O$ is \Iclosed{}! If we hand $O$ and $I$ to the decision
procedure, it will say yes, and we can conclude that $(O, T)$ is \Iconfluent{}.

\newcommand{\comment}[1]{\State \textcolor{flatdenim}{\texttt{//} #1}}
\begin{algorithm}
  \caption{Interactive \Iconfluence{} Decision Procedure}%
  \algolabel{InteractiveDecisionProcedure}
  \begin{algorithmic}
    \Function{IsIconfluent}{$O$, $T$, $I$}
      \If {$\lnot I(s_0)$}
        \State \Return no
      \Else
        \State \Return \Call{Helper}{$O$, $T$, $I$, $\set{s_0}$, $\emptyset$}
      \EndIf
    \EndFunction

    \State

    \comment{$R$ is a set of reachable states.}
    \comment{$NR$ is a set of non-reachable states.}
    \Function{Helper}{$O$, $T$, $I$, $R$, $NR$}
      \State closed, $s_1$, $s_2$ $\gets$ \Call{IsIclosed}{$O$, $I - NR$}
      \If {closed $=$ unknown}
        \State \Return unknown
      \ElsIf {closed $=$ yes}
        \State \Return{yes}
      \Else
        \State Augment $R$ and $NR$ with random search and user input
        \State Make sure $s_1$ and $s_2$ are included in $R$ or $NR$
        \If{$s_1, s_2 \in R$}
          \State \Return no
        \Else
          \State \Return \Call{Helper}{$O$, $T$, $I$, $R$, $NR$}
        \EndIf
      \EndIf
    \EndFunction
  \end{algorithmic}
\end{algorithm}

\newcommand{\IsIconfluent}{\textsc{IsIconfluent}}
\newcommand{\IsIclosed}{\textsc{IsIclosed}}
\newcommand{\Helper}{\textsc{Helper}}

This example motivates \algoref{InteractiveDecisionProcedure}, a general
purpose interactive \Iconfluence{} decision procedure in which a user
iteratively refines the invariant until the decision procedure terminates.
Given an object $O$, a set of transactions $T$, and an invariant $I$, we call
\IsIconfluent$(O, T, I)$ to determine if $(O, T)$ is \Iconfluent{}. Now, we
describe \algoref{InteractiveDecisionProcedure}.  An example of how to use
\algoref{InteractiveDecisionProcedure} on \exampleref{StateBasedNotNecessary}
is given in \figref{DecisionProcedureExample}.

\IsIconfluent{} assumes access to an \Iclosed{} decision procedure
\IsIclosed$(O, I)$ which returns whether $O$ is \Iclosed{} (yes, no, or
unknown). If it returns no, it also returns a counterexample $s_1$ and $s_2$
such that $I(s_1)$ and $I(s_2)$, but $\lnot I(s_1 \join s_2)$. Constructing an
\Iclosed{} decision procedure is hard (undecidable, I think), but we can do
pretty well using an SMT solver like Z3. There is also some existing research
on how to built smart \Iclosed{} decision procedures that we can
leverage~\cite{li2014automating}.

\IsIconfluent{} takes in $O$, $T$, and $I$. Formally, $I$ is set of states
(e.g. $\setst{(x, y)}{x \in \ints, y \in \ints, xy \leq 0}$), but an actual
implementation of \IsIconfluent{} would likely take in $I$ as a logical formula
(e.g. $xy \leq 0$). From now on, whenever we mention a (possibly infinite) set
of states, remember that we can always represent the set as a finite formula.

\IsIconfluent{} calls a helper function \Helper{} which---in addition to $O$,
$T$, and $I$---takes as arguments a set $R$ of reachable states and a set $NR$
of unreachable states. Like \IsIconfluent, \Helper$(O, T, I, R, NR)$ returns
whether $T$ is \Iconfluent. Intuitively, $NR$ helps remove unreachable states
from $I$ moving us closer to the case where $I \subseteq
\setst{e}{\reachable{e}}$.

\Helper{} first checks to see if $O$ is \iclosed{$(I - NR)$}. There are three
cases to consider.
\begin{itemize}
  \item
    If \IsIclosed{} gets stuck and doesn't know whether or not $O$ is
    \iclosed{$(I - NR)$}, then \Helper{} is stuck too and is forced to return
    unknown.
  \item
    If \IsIclosed{} determines that $O$ is \iclosed{$(I - NR)$}, then by
    \clmref{StateBasedSufficient}, $T$ is \iconfluent{$(I - NR)$}, so
    $\setst{e}{\reachable{e}} \subseteq I - NR$. $I - NR \subseteq I$, so $T$
    is therefore also \Iconfluent{}. Thus, \Helper{} return yes.
  \item
    If \IsIclosed{} determines that $O$ is not \iclosed{$(I - NR)$}, then we
    have a counterexample $s_1$ and $s_2$. We want to determine whether $s_1$
    and $s_2$ are reachable or unreachable. We can do so in two ways.

    First, we can randomly generate a set of reachable states and add them to
    $R$.

    Second, we can prompt the user to tell us whether or not the states are
    reachable or unreachable. In addition to labelling $s_1$ and $s_2$ as
    reachable or not, the user can also refine $I$ by augmenting $NR$. Taking
    \exampleref{StateBasedNotNecessary} as an example, the user might look
    through the existing reachable and unreachable states and have an aha
    moment realizing that every point with a negative $x$-coordinate is
    unreachable. They can use this knowledge to add all points with negative
    $x$-coordinates to $NR$.

    After $s_1$ and $s_2$ have been labelled reachable or not, we continue. If
    both $s_1$ and $s_2$ are reachable, then so is $s_1 \join s_2$, but $\lnot
    I(s_1 \join s_2)$. Thus, $(O, T)$ is not \Iconfluent{}, so \Helper{}
    returns no.  Otherwise, one of $s_1$ and $s_2$ is unreachable. Now, we
    recurse.
\end{itemize}

\tikzstyle{point}=[shape=circle, fill=flatgray, inner sep=1.2pt, draw=black]
\tikzstyle{reachable}=[fill=flatgreen!50, draw=none]
\tikzstyle{unreachable}=[fill=flatred!50, draw=none]
\tikzstyle{invariant}=[fill=flatblue!50, draw=none]
\tikzstyle{nothing}=[fill=white, draw=none]

\newcommand{\pointgrid}[4]{{
  \newcommand{\argxmin}{#1}
  \newcommand{\argxmax}{#2}
  \newcommand{\argymin}{#3}
  \newcommand{\argymax}{#4}

  \draw[] (\argxmin, 0) to (\argxmax, 0);
  \draw[] (0, \argymin) to (0, \argymax);
  \foreach \x in {\argxmin, ..., \argxmax} {
    \foreach \y in {\argymin, ..., \argymax} {
      \node[point] (\x-\y) at (\x, \y) {};
    }
  }
}}

\newcommand{\pointrect}[2]{
  \draw[#1] ($#2 + (-0.51, 0.51)$) rectangle ($#2 + (0.51, -0.51)$);
}

\newcommand{\subfigwidth}{0.48\textwidth}
\newcommand{\subfighspace}{0.3cm}
\newcommand{\tikzhspace}{0.4cm}
\newcommand{\tikzscale}{0.25}
\newcommand{\xmin}{-3}
\newcommand{\xmax}{3}
\newcommand{\ymin}{-3}
\newcommand{\ymax}{3}

\begin{figure*}[ht]
  \centering

  \begin{subfigure}[t]{0.38\textwidth}
    \centering
    \begin{tikzpicture}[scale=\tikzscale]
      \draw[reachable, draw=black] (-0.5, 0.5) rectangle (0.5, -0.5);
      \pointgrid{\xmin}{\xmax}{\ymin}{\ymax}
      \node at (0, \ymax + 1) {$R$};
    \end{tikzpicture}\hspace{\tikzhspace}
    \begin{tikzpicture}[scale=\tikzscale]
      \pointgrid{\xmin}{\xmax}{\ymin}{\ymax}
      \node at (0, \ymax + 1) {$NR$};
    \end{tikzpicture}\hspace{\tikzhspace}
    \begin{tikzpicture}[scale=\tikzscale]
      \draw[invariant] (\xmin.5, \ymax.5) rectangle (0.5, -0.5);
      \draw[invariant] (-0.5, 0.5) rectangle (\xmax.5, \ymin.5);
      \draw (0.5, 0.5) to (0.5, \ymax.5);
      \draw (0.5, 0.5) to (\xmax.5, 0.5);
      \draw (-0.5, -0.5) to (-0.5, \ymin.5);
      \draw (-0.5, -0.5) to (\xmin.5, -0.5);
      \pointgrid{\xmin}{\xmax}{\ymin}{\ymax}
      \node at (0, \ymax + 1) {$I - NR$};
    \end{tikzpicture}
    \caption{%
      \IsInvConfluent{} determines $I(s_0)$ and then calls \Helper{} with $R =
      \set{s_0}$, $NR = \emptyset{}$, and $I = \setst{(x, y)}{xy \leq 0}$.
    }
    \figlabel{DecisionProcedureExampleA}
  \end{subfigure}
  \hspace{\subfighspace}
  \begin{subfigure}[t]{0.58\textwidth}
    \centering
    \begin{tikzpicture}[scale=\tikzscale]
      \draw[reachable, draw=black] (-0.5, 0.5) rectangle (1.5, -1.5);
      \pointgrid{\xmin}{\xmax}{\ymin}{\ymax}
      \node at (0, \ymax + 1) {$R$};
    \end{tikzpicture}\hspace{\tikzhspace}
    \begin{tikzpicture}[scale=\tikzscale]
      \pointrect{unreachable, draw=black}{(-1, 1)}
      \pointgrid{\xmin}{\xmax}{\ymin}{\ymax}
      \node at (0, \ymax + 1) {$NR$};
    \end{tikzpicture}\hspace{\tikzhspace}
    \begin{tikzpicture}[scale=\tikzscale]
      \draw[invariant] (\xmin.5, \ymax.5) rectangle (0.5, -0.5);
      \draw[invariant] (-0.5, 0.5) rectangle (\xmax.5, \ymin.5);
      \draw (0.5, 0.5) to (0.5, \ymax.5);
      \draw (0.5, 0.5) to (\xmax.5, 0.5);
      \draw (-0.5, -0.5) to (-0.5, \ymin.5);
      \draw (-0.5, -0.5) to (\xmin.5, -0.5);
      \pointrect{nothing, draw=black}{(-1, 1)}
      \pointgrid{\xmin}{\xmax}{\ymin}{\ymax}
      \node at (0, \ymax + 1) {$I - NR$};
    \end{tikzpicture}
    \caption{%
      \Helper{} determines that $O$ is not \iclosed{(I - NR)} with
      counterexample $s_1 = (-1, 1)$ and $s_2 = (1, -1)$. \Helper{} randomly
      generates some \sTIreachable{} points and adds them to $R$. Luckily for
      us, $s_2 \in R$, so \Helper{} knows that it is \sTIreachable{}.
      \Helper{} is not sure about $s_1$, so it asks the user. After some
      thought, the user tells \Helper{} that $s_1$ is \sTIunreachable{}, so
      \Helper{} adds $s_1$ to $NR$ and then recurses.
    }
    \figlabel{DecisionProcedureExampleB}
  \end{subfigure}

  \begin{subfigure}[t]{0.38\textwidth}
    \centering
    \begin{tikzpicture}[scale=\tikzscale]
      \draw[reachable, draw=black] (-0.5, 0.5) rectangle (2.5, -2.5);
      \pointrect{reachable, draw=black}{(3, -3)}
      \pointgrid{\xmin}{\xmax}{\ymin}{\ymax}
      \node at (0, \ymax + 1) {$R$};
    \end{tikzpicture}\hspace{\tikzhspace}
    \begin{tikzpicture}[scale=\tikzscale]
      \pointrect{unreachable}{(-1, 1)}
      \pointrect{unreachable}{(-1, 2)}
      \draw (-1.5, 0.5) rectangle (-0.5, 2.5);
      \pointgrid{\xmin}{\xmax}{\ymin}{\ymax}
      \node at (0, \ymax + 1) {$NR$};
    \end{tikzpicture}\hspace{\tikzhspace}
    \begin{tikzpicture}[scale=\tikzscale]
      \draw[invariant] (\xmin.5, \ymax.5) rectangle (0.5, -0.5);
      \draw[invariant] (-0.5, 0.5) rectangle (\xmax.5, \ymin.5);
      \draw (0.5, 0.5) to (0.5, \ymax.5);
      \draw (0.5, 0.5) to (\xmax.5, 0.5);
      \draw (-0.5, -0.5) to (-0.5, \ymin.5);
      \draw (-0.5, -0.5) to (\xmin.5, -0.5);
      \pointrect{nothing}{(-1, 1)}
      \pointrect{nothing}{(-1, 2)}
      \draw (-1.5, 0.5) rectangle (-0.5, 2.5);
      \pointgrid{\xmin}{\xmax}{\ymin}{\ymax}
      \node at (0, \ymax + 1) {$I - NR$};
    \end{tikzpicture}
    \caption{%
      \Helper{} determines that $O$ is not \iclosed{(I - NR)} with
      counterexample $s_1 = (-1, 2)$ and $s_2 = (3, -3)$. \Helper{} randomly
      generates some \sTIreachable{} points and adds them to $R$. $s_1, s_2
      \notin R, NR$, so \Helper{} ask the user to label them. The user puts
      $s_1$ in $NR$ and $s_2$ in $R$. Then, \Helper{} recurses.
    }
    \figlabel{DecisionProcedureExampleC}
  \end{subfigure}
  \hspace{\subfighspace}
  \begin{subfigure}[t]{0.58\textwidth}
    \centering
    \begin{tikzpicture}[scale=\tikzscale]
      \draw[reachable] (-0.5, 0.5) rectangle (2.5, -2.5);
      \pointrect{reachable}{(3, 0)}
      \pointrect{reachable}{(3, -1)}
      \pointrect{reachable}{(0, -3)}
      \pointrect{reachable}{(2, -3)}
      \pointrect{reachable}{(3, -3)}
      \draw (-0.5, 0.5) to (\xmax.5, 0.5);
      \draw (-0.5, 0.5) to (-0.5, \ymin.5);
      \draw (-0.5, -3.5) to ++(1, 0)
                         to ++(0, 1)
                         to ++(1, 0)
                         to ++(0, -1)
                         to ++(2, 0)
                         to ++(0, 1)
                         to ++(-1, 0)
                         to ++(0, 1)
                         to ++(1, 0)
                         to ++(0, 2);
      \pointgrid{\xmin}{\xmax}{\ymin}{\ymax}
      \node at (0, \ymax + 1) {$R$};
    \end{tikzpicture}\hspace{\tikzhspace}
    \begin{tikzpicture}[scale=\tikzscale]
      \draw[unreachable] (\xmin.5, \ymax.5) rectangle (-0.5, \ymin.5);
      \draw (\xmin.5, \ymax.5) to (-0.5, \ymax.5) to
            (-0.5, \ymin.5) to (\xmin.5, \ymin.5);
      \pointgrid{\xmin}{\xmax}{\ymin}{\ymax}
      \node at (0, \ymax + 1) {$NR$};
    \end{tikzpicture}\hspace{\tikzhspace}
    \begin{tikzpicture}[scale=\tikzscale]
      \draw[invariant] (-0.5, 0.5) rectangle (\xmax.5, \ymin.5);
      \draw (-0.5, 0.5) to (\xmax.5, 0.5);
      \draw (-0.5, 0.5) to (-0.5, \ymin.5);
      \pointgrid{\xmin}{\xmax}{\ymin}{\ymax}
      \node at (0, \ymax + 1) {$I - NR$};
    \end{tikzpicture}
    \caption{%
      \Helper{} determines that $O$ is not \iclosed{(I - NR)} with
      counterexample $s_1 = (-2, 1)$ and $s_2 = (1, -1)$. \Helper{} randomly
      generates some \sTIreachable{} points and adds them to $R$. $s_2 \in R$
      but $s_1 \notin R, NR$, so \Helper{} asks the user to label $s_1$. The
      user notices a pattern in $R$ and $NR$ and after some thought, concludes
      that every point with negative $x$-coordinate is \sTIunreachable{}.
      They update $NR$ to $-\ints \times \ints$. Then, \Helper{} recurses.
      \Helper{} determines that $O$ is \iclosed{(I - NR)} and returns true!
    }
    \figlabel{DecisionProcedureExampleD}
  \end{subfigure}

  \vspace{1em}
  \hrule
  \vspace{1em}

  \caption{%\vspace{1em}
    An example of a user interacting with
    \algoref{InteractiveDecisionProcedure} on \exampleref{Z2}.  Each step of
    the visualization shows reachable states $R$ (left), non-reachable states
    $NR$ (middle), and the refined invariant $I - NR$ (right) as the algorithm
    executes
  }
  \figlabel{DecisionProcedureExample}
\end{figure*}


% \algoref{InteractiveDecisionProcedure} is an \emph{interactive} decision
% procedure. It requires that a user classify states as reachable or unreachable.
% What happens if a user incorrectly classifies a state? Given a counterexample
% $s_1, s_2$, if a user incorrectly says that one of the unreachable points is
% reachable, then the $s_1, s_2 \in R$ check in \Helper{} might be
% erroneously met. In turn, \IsIconfluent{} might erroneously return no. This is
% inconvenient, but not the end of the world.
% Moreover, we can modify our decision procedure so that \Helper{}
% user provide a \emph{proof} of reachability whenever they label as: a series of transactions and joins that result in the
