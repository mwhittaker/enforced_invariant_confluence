\section{Stratified Invariant-Confluence}
The interactive decision procedure in \algoref{InteractiveDecisionProcedure}
and the sufficient conditions in \clmref{NewStateBasedSufficient} give us two
ways to detect when an object $O$ is invariant-confluent. But what if it's not?
%
Let's revisit \exampleref{StateBasedNotNecessary} but with start state $s_0 =
(-2, 2)$. Now, $O$ is not \sTIconfluent{}. The set of $\sTIreachable{}$ points
(illustrated in \figref{FixingIconfluenceReachable}) is not a subset of $I$
(illustrated in \figref{FixingIconfluenceInvariant}).

This is unfortunate, but what if we're determined to make $O$ \sTIconfluent{}?
Well, then we have to adjust the set of $\sTIreachable{}$ points so that they
are a subset of $I$. We can go about this in three ways:
\begin{itemize}
  \item \textbf{Adjust $s_0$.}
    $O$ is not \sticonfluent{(-2, 2)}{T}{I}, but it is \sticonfluent{(0,
    0)}{T}{I}. Thus, we can change our start state to become
    invariant-confluent. This is illustrated in \figref{FixingIconfluenceS}.

  \item \textbf{Adjust $T$.}
    $O$ is not \sticonfluent{s_0}{\set{t_{x+1}, t_{y-1}}}{I}, but it is
    \sticonfluent{s_0}{\set{t_{x+1}}}{I}. Thus, we can remove some transactions
    to become invariant-confluent. This is illustrated in
    \figref{FixingIconfluenceT}. Instead of outright forbidding some
    transactions, we can instead use locking to limit the concurrent execution
    of certain sets of transactions~\cite{balegas2015putting,
    gotsman2016cause}.

  \item \textbf{Adjust $I$.}
    $O$ is not \sticonfluent{s_0}{T}{\setst{(x, y)}{xy \leq 0}} but it is
    \sticonfluent{s_0}{T}{\setst{(x, y)}{x \leq 0, y \geq 0}}. Thus, we can
    remove some states from $I$ to become invariant-confluent. This is
    illustrated in \figref{FixingIconfluenceI}. We could also add some states
    to $I$ and become invariant-confluent, but it's likely that states are
    excluded from $I$ for good reason.
\end{itemize}

{\input{figs/fixing_iconfluence.fig}}

Unfortunately, none of these three adjustments are completely satisfactory.
There may only be one natural start state, and artificially making some
invariant-satisfying states unreachable---either by eliminating transactions or
restricting the invariant---is most likely a bit of a dirty hack. Moreover,
locking can sometimes be overly restrictive. For example, we can make the
example above invariant-confluent by requiring that both transactions $t_{x+1}$
and $t_{y-1}$ acquire a lock. However, once all processors have a state $(x,
y)$ where $x \geq 0$ and $y \leq 0$, the locks are no longer necessary.

Ideally, we would maintain the same start state, use locking to restrict
concurrently executing transactions as little as possible, and leave all
invariant-satisfying reachable states reachable. Can we achieve this?  Yes, at
least kind of. We'll do so using something we call \defword{stratified
invariant-confluence}. The main idea is that we'll partition the set of
invariant-satisfying states. Depending on the state we're in, we'll either
limit which transactions can execute concurrently, restrict the invariant, or
both.

Consider a state-based object $O = (S, \join)$, a start state $s_0$, a set of
transactions $T$, and an invariant $I$ deployed on $n$ processors $p_1, \ldots,
p_n$. Further assume that $O$ is a semilattice.
%
With stratified invariant-confluence, we construct an \ITpartition{}: an
ordered sequence $P = (I_1, T_1), \ldots, (I_n, T_n)$ of $n$ pairs $(I_i, T_i)$
with the following properties:
\begin{itemize}
  \item
    $I_i \subseteq I$ for $1 \leq i \leq n$

  \item
    $T_i \subseteq T$ for $1 \leq i \leq n$

  \item
    $I = \cup_{i=1}^n I_i$ (note that the $I_i$'s don't have to be disjoint)

  \item
    For all $(I_i, T_i) \in P$ and all states $s \in I_i$, $O$ is
    \sticonfluent{s}{I_i}{T_i}.
\end{itemize}
All processors store a copy of the \ITpartition{} $P$. Moreover, at any given
point in time, a processor $p_i$ will execute with one pair $(I_j, T_j) \in P$
\defword{active}. We'll describe what this means in just a moment. Initially,
every processor $p_i$ is state $s_0$. $p_i$ finds the first pair $(I_j, T_j)
\in P$ such that $I_j(s_0)$ and makes this $(I_j, T_j)$ active. As usual, a
processor can perform one of two actions: execute a transaction or perform a
merge.
\begin{itemize}
  \item
    Merging proceeds as usual. When a process $p_i$ with state $s_i$ receives a
    state $s_j$ from $p_j$, it transitions from $s_i$ to $s_i \join p_j$.

  \item
    Transaction processing is a bit different. Imagine a processor $p_i$ in
    state $s_i$ with pair $(I_j, T_j)$ active wants to execute transaction $t$.

    If $t \in T_j$, then $p_i$ executes $t$ to transition from state $s_i$ to
    state $t(s_i)$. If $I_j(t(s_i))$, then $p_i$ returns to the client
    immediately. If $\lnot I(t(s_i))$, then $p_i$ aborts the transaction and
    returns to the client immediately. If $\lnot I_j(t(s_i))$ and $I(t(s_i))$,
    then $p_i$ initiates a round of global coordination among the processors,
    as described below.  If $t \notin T_j$, then $p_i$ also initiates a round
    of coordination.

    During a global coordination, all processors adopt state state $s = s_1
    \join \cdots \join s_n$. Then, all processors execute $t$. If $t(s)$ does
    not satisfy $I$, then the processors remain in state $s$ and continue
    normal processing.  If $t(s)$ does satisfy $I$, then the processors enter
    state $t(s)$ and choose pair $(I_k, T_k)$ to be active where $(I_k, T_k)$
    is the first pair in $P$ where $I_k(t(s))$ (which is guaranteed to exist
    since $\cup_i I_i = I$).
\end{itemize}

An example of stratified invariant-confluence for the example discussed above
is given in \figref{StratifiedIconfluenceExample}.

{\input{figs/stratified_iconfluence_example.fig}}

Stratified invariant-confluence, like invariant-confluence, guarantees that no
replica ever enters a state that violates the invariant $I$. How does it
guarantee this?
%
Initially, all processors begin in the same state $s$ and with the same active
pair $(I_j, T_j)$, and this invariant is maintained whenever a global
coordination is run. During normal non-coordinated operation, because $O$ is
\sticonfluent{s}{T_j}{I_j}, $\setst{s'}{\stireachable{s}{T_j}{I_j}{s'}}
\subseteq I_j \subseteq I$. Thus, processors stay within $I_j$ which itself
lies within $I$. During a coordinated action, processors transition to state $s
= s_1 \join \cdots \join s_n \in I_j$ and then possibly transition into a new
state $t(s)$ with new active pair $(I_k, T_k)$.

A couple things to note:
\begin{itemize}
  \item
    If $O$ is \sTIconfluent{}, then we can let $P = (I, T)$.

  \item
    Let $P = (I, T')$ where $T' \subseteq T$. This is a special case of the
    token-based approach in~\cite{gotsman2016cause} where every transaction in
    $T - T'$ acquires a single token $\tau$ and every transaction in $T$
    doesn't acquire any token. The token-based approach
    in~\cite{gotsman2016cause} is not completely subsumed by our stratified
    invariant-confluence. For example, stratified invariant-confluence cannot
    emulate multi-token systems. Conversely, stratified invariant-confluence is
    not subsumed by the token-based approach which cannot stratify $I$. Notably
    though, the token-based approach can adjust what tokens a transaction
    acquires based on its state, but this is not used in the paper.
\end{itemize}

\todo{%
  Give more examples where stratified invariant-confluence is helpful.
}
\todo{%
  Give name to things like $P$. Call it something like an $(I, T)$-partition.
}
\todo{Can we modify the decision procedure to help build a stratification?}

\begin{example}
  \todo{%
    Bank account example from~\cite{gotsman2016cause}. This might be hard to
    actually implement since the bank account merge is hard to do with a
    state-based object. This example only acquires a single lock.
  }
\end{example}

\begin{example}
  \todo{%
    Auction example from~\cite{gotsman2016cause}. I think we could automate
    this one. This one acquires a single lock. We don't need the lock once
    we're finalized.
  }
\end{example}

\begin{example}
  \todo{%
    Courseware application from~\cite{gotsman2016cause}. When course is
    non-empty, we can allow any operations at all. When a course is empty, we
    cannot allow concurrent operations on it. This one should hopefully be easy
    to automate.
  }
\end{example}

\begin{example}
  \todo{%
    Ad counter from~\cite{balegas2015putting}. This might be hard like the bank
    account balance. Adding numbers might be tricky in a state-based approach.
  }
\end{example}

\begin{example}
  \todo{%
    Tournament from~\cite{balegas2015putting}. This might be hard. Pretty
    complicated example with set cardinality constraints.
  }
\end{example}
