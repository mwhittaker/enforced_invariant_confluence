\section{Sufficient Conditions}
In this section, we present sufficient (but not always necessary) conditions
for state-based and operation-based invariant confluence. We also explain
circumstances under which the sufficient conditions are also necessary.

\begin{definition}
  We say a state-based object $O = (S, s_0, \join)$ is \defword{\Iclosed{}} if
  for all states $s_1$ and $s_2$, if $I(s_1)$ and $I(s_2)$, then $I(s_1 \join
  s_2)$.
  \[
    I(s_1) \land I(s_2) \implies I(s_1 \join s_2)
  \]
  Similarly, we say on operation-based object is \defword{\Iclosed{}} if for
  all states $s_1$ and $s_2$ and all transactions $t$, if $I(s_1)$, $I(s_2)$,
  and $I(t(s_1)(s_1))$, then $I(t(s_1)(s_2))$.
  \[
    I(s_1) \land I(s_2) \land I(t(s_1)(s_1)) \implies I(t(s_1)(s_2))
  \]
\end{definition}

\begin{claim}\clmlabel{StateBasedSufficient}
  Given a state-based object $O = (S, s_0, \join)$, a set of transactions $T$,
  and an invariant $I$, if $O$ is \Iclosed{} and $I(s_0)$, then $(O, T)$ is
  \Iconfluent.
\end{claim}
\begin{elidableproof}
  Let $e_1, e_2$ be expressions such that $\Irec{e_1}$ and $\Irec{e_2}$. Let
  $s_1 = eval(e_1)$ and $s_2 = eval(e_2)$. Then, $I(s_1)$ and $I(s_2)$, so
  $I(s_1 \join s_2)$ which implies $I(e_1 \join e_2)$. Thus, by
  \clmref{StateBasedTwoIconfluenceDefs}, $(O, T)$ is \Iconfluent{}.
\end{elidableproof}

\begin{claim}\clmlabel{OpBasedSufficient}
  Given a operation-based object $O = (S, s_0)$, a set of transactions $T$, and
  an invariant $I$, if $O$ is \Iclosed{} and $I(s_0)$, then $(O, T)$ is
  \Iconfluent.
\end{claim}
\begin{elidableproof}
  Let $t \in T$ and $e_1, e_2$ be expressions such that $\Irec{e_1}$,
  $\Irec{e_2}$, and $I(t(e_1)(e_1))$. Let $s_1 = eval(e_1)$ and $s_2 =
  eval(s_2)$. Then, $I(s_1)$, $I(s_2)$, and $I(t(s_1)(s_1))$, so
  $I(t(s_1)(s_2))$. This implies $I(t(e_1)(e_2))$, so by
  \clmref{OpBasedTwoIconfluenceDefs}, $(O, T)$ is \Iconfluent.
\end{elidableproof}

As a special case, \clmref{StateBasedSufficient} tells us that a set $T$ of
transactions is \Iconfluent{} if $I$ is monotonic and $O$ is a semilattice.

\begin{claim}\clmlabel{Monotonicity}
  Let $O = (S, s_0, \join)$ be a state-based object where $(S, \join)$ is a
  semilattice. Let $s \leq s' \defeq s \join s' = s'$. Let $T$ be an arbitrary
  set of transactions. Let invariant $I$ be \defword{monotonic}: for all $s,
  s'\in S$, if $s \leq s'$, then $I(s) \implies I(s')$. Also assume $I(s_0)$.
  Then, $T$ is \Iconfluent{}.
\end{claim}
\begin{elidableproof}
  Let $s_1$ and $s_2$ be arbitrary states such that $I(s_1)$ and $I(s_2)$. $s_1
  \leq s_1 \join s_2$, so by monotonicity, $I(s_1 \join s_2)$. Thus, by
  \clmref{StateBasedSufficient}, $T$ is \Iconfluent{}.
\end{elidableproof}

Unfortunately, the converses of \clmref{StateBasedSufficient} and
\clmref{OpBasedSufficient} are not true. Being \Iconfluent{} does not
necessarily imply being \Iclosed{}, as the next two examples show.

\begin{example}\examplelabel{StateBasedNotNecessary}
  Consider the state-based object $O = (S, s_0, \join)$ where $(S, \join)$ is
  $(\ints, \max) \times (\ints, \max)$ and $s_0 = (0, 0)$. Let $T$ consist of
  two transactions $t_{x+1} \defeq \lambda (x, y).\ (x + 1, y)$ and $t_{y-1} =
  \lambda (x, y).\ (x, y - 1)$. Let $I = \setst{(x, y)}{xy \leq 0}$.
  %
  $\setst{(x, y)}{\reachable{(x, y)}} = \nats \times -\nats \subseteq I$, so
  $(O, T)$ is \Iconfluent. However, let $s_1 = (-1, 1)$ and $s_2 = (1, -1)$.
  $I(s_1)$ and $I(s_2)$. However, $s_1 \join s_2 = (1, 1)$, and $\lnot I((1,
  1))$. Thus, $O$ is not \Iclosed{}.
\end{example}

\begin{example}
  Consider the operation-based object $O = (S, s_0)$ where $S$ is $\ints \times
  \ints$ and $s_0 = (0, 0)$. Let $T$ consist of two transactions $t_{x+1}
  \defeq \lambda s.\ \lambda (x, y).\ (x + 1, y)$ and $t_{y-1} = \lambda s.\
  \lambda (x, y).\ (x, y - 1)$. Let $I = \setst{(x, y)}{xy \leq 0}$.
  %
  $\setst{(x, y)}{\reachable{(x, y)}} = \nats \times -\nats \subseteq I$, so
  $(O, T)$ is \Iconfluent. However, let $s_1 = (1, -1)$ and $s_2 = (0, 1)$.
  $I(s_1)$, $I(s_2)$, and $I(t_{x+1}(s_1)(s_1))$. However, $t_{x+1}(s_1)(s_2) =
  (1, 1)$, and $\lnot I((1, 1))$. Thus, $O$ is not \Iclosed{}.
\end{example}

\input{figs/not_necessary.fig}

Let's take a closer look at \exampleref{StateBasedNotNecessary} (illustrated in
\figref{StateBasedNotNecessary}) to understand why being \Iconfluent{} doesn't
imply being \Iclosed{}. Our counterexample considers points $s_1 = (-1, 1)$ and
$s_2 = (1, -1)$. $s_2$ is reachable, but $s_1$ is not. This is not a
coincidence! If we consider an \Iconfluent{} object and set of transactions,
then any two reachable points satisfy the invariant, and the join of the two
points is again reachable, so it must satisfy the invariant as well. The only
way to join two invariant satisfying points and end up \emph{not} satisfying
the invariant is if one or both points are not reachable. Thus, if we know that
$I$ is a subset of reachable points, then being \Iconfluent{} and being
\Iclosed{} are equivalent.

\begin{claim}\clmlabel{SufficientAndNecessary}
  If every invariant satisfying state is reachable, then being \Iconfluent{}
  and being \Iclosed{} are equivalent.
  \[
    I \subseteq \setst{e}{\reachable{e}} \implies
    (\text{$O$ \Iclosed{}} \iff \text{$(O, T)$ \Iconfluent})
  \]
\end{claim}
\begin{elidableproof}
  Consider a state-based object $O = (S, s_0, \join)$, a set of transactions
  $T$, and an invariant $I$ where $I \subseteq \setst{e}{\reachable{e}}$.
  %
  Being \Iclosed{} always implies being \Iconfluent{} whether or not $I
  \subseteq \setst{e}{\reachable{e}}$. Thus, we only have to prove that if $(O,
  T)$ is \Iconfluent{}, then $O$ is \Iclosed. If $(O, T)$ is \Iconfluent{},
  then $I(s_0)$. Next, let $s_1$ and $s_2$ be arbitrary states such that
  $I(s_1)$ and $I(s_2)$. Both states satisfy the invariant, so both are
  reachable. That is, there exist expressions $e_1$ and $e_2$ where $eval(e_1)
  = s_1$  and $eval(e_2) = s_2$. $e_1 \join e_2$ is again reachable, so since
  $T$ is \Iconfluent{}, $I(e_1 \join e_2)$ which implies $I(s_1 \join s_2)$.
  Thus, $O$ is \Iclosed.
\end{elidableproof}

Here are some more examples of transactions $T$ and invariants $I$ that are
\Iconfluent{} but not \Iclosed{}.

\todo{Think of more compelling examples.}

\begin{example}
  \todo{Write this up in full.}
  Imagine we are building a system to group students together. We have a relation
  of (student id, grade) and a map (group id -> set of student ids). We want to
  enforce the invariant that every student in every group is in the same grade
  and that all groups are non-empty. If we start off with an initially empty
  mapping, then there are conflicts. If we start off with at least one student
  mapped into each group, then there are not any conflicts.
\end{example}

\begin{example}
  Consider the state-based object $(S, s_0, \join)$ where $(S, \join) = (\ints,
  \max) \times (\ints, \max)$ and $s_0 = (0, 0)$. Let $T = \set{(x, y) \mapsto
  (x + 1, y), (x, y) \mapsto (x, y - 1)}$ and let $I((x, y)) = xy \leq 0$. The
  reachable states are $\nats \times -\nats$, all of which satisfy $I$, so $T$
  is \Iconfluent{}. However, let $a = (-1, 1)$ and $b = (1, -1)$. $I(a)$ and
  $I(b)$, but $a \join b = (1, 1)$ does not satisfy $I$.
\end{example}

\begin{example}
  Consider the state-based object $(\powerset{\nats}, \set{0}, \cup)$. Let $T =
  \setst{X \mapsto X \cup Y}{Y \subseteq \nats}$ let $I(X) = \exists k \in
  \set{0, 1}.\ \forall x \in X.\ x \equiv k \mod 2$. That is $I(X)$ if $X$
  contains only odd or only even elements. The reachable states are
  $\powerset{2\nats}$, all of which satisfy $I$, so $T$ is \Iconfluent{}.
  However, let $A = \set{0}$ and $Y = \set{1}$. $I(A)$ and $I(B)$, but $A \join
  B = \set{0, 1}$ does not satisfy $I$.
\end{example}

\begin{example}
  Consider a state-based object consisting of two integer 2P-sets $(A_1, R_1)$ and
  $(A_2, R_2)$ with initial state $(\set{1, 3}, \emptyset{})$ and $(\set{1, 2,
  3, 4}, \emptyset{})$. Let $T$ consist of all transactions that either remove
  an element from the first 2P-set (i.e.\ add something to $R_1$) or add
  something to the second 2P-set (i.e.\ add something to $A_2$). Let $I$ be
  that the first 2P-set is a subset of the second (i.e. $(A_1 - R_1) \subseteq
  (A_2 - R_2)$).
  %
  $T$ is \Iconfluent{}. However, let $s_1 = (\set{1}, \set{2}, \set{1},
  \set{2})$ and $s_2 = (\set{2}, \set{1}, \set{2}, \set{1})$. $I(s_1)$ and
  $I(s_2)$, but $s_1 \cup s_2 = (\set{1, 2}, \set{1, 2}, \set{1, 2}, \set{1,
  2})$ which does not satisfy $I$.
\end{example}
