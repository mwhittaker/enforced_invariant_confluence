\section{\dIconfluence{} and \dIreduction{}}
A \defword{join-free} expression $e = t_1(t_2(\ldots t_n(s_0) \ldots))$ is an
expression that does not contain a join.

We say $T$ is \defword{\dIconfluent{}} if for every pair $e_1$ and $e_2$ of
join-free expressions, $\Irec{e_1}$ and $\Irec{e_2}$ implies that $I(e_1 \join
e_2)$. Intuitively, \dIconfluence{} is the same thing as \Iconfluence{} with
the added restriction that subexpressions are join-free.

We say $T$ is \defword{\dIreducible{}} if for every pair $e_1$ and $e_2$ of
join-free expressions, $\Irec{e_1 \join e_2}$ implies that there exists some
join-free expression $e_3$ such that $\Irec{e_3}$ and $eval(e3) = eval(e1 \join
e_2)$.  Intuitively, \dIreduction{} allows us to eliminate join operators by
substituting $e_1 \join e_2$ (which contains 1 join) with $e_3$ (which does not
contain any joins).

\section{\dIconfluence{} and \dIreduction{}}
A \defword{join-free} expression $e = t_1(t_2(\ldots t_n(s_0) \ldots))$ is an
expression that does not contain a join.

We say $T$ is \defword{\dIconfluent{}} if for every pair $e_1$ and $e_2$ of
join-free expressions, $\Irec{e_1}$ and $\Irec{e_2}$ implies that $I(e_1 \join
e_2)$. Intuitively, \dIconfluence{} is the same thing as \Iconfluence{} with
the added restriction that subexpressions are join-free.

We say $T$ is \defword{\dIreducible{}} if for every pair $e_1$ and $e_2$ of
join-free expressions, $\Irec{e_1 \join e_2}$ implies that there exists some join-free expression $e_3$ such that $\Irec{e_3}$ and $eval(e3) = eval(e1 \join
e_2)$.  Intuitively, \dIreduction{} allows us to eliminate join operators by
substituting $e_1 \join e_2$ (which contains 1 join) with $e_3$ (which does not
contain any joins).

% \section{\dIconfluence{} and \dIreduction{}}
A \defword{join-free} expression $e = t_1(t_2(\ldots t_n(s_0) \ldots))$ is an
expression that does not contain a join.

We say $T$ is \defword{\dIconfluent{}} if for every pair $e_1$ and $e_2$ of
join-free expressions, $\Irec{e_1}$ and $\Irec{e_2}$ implies that $I(e_1 \join
e_2)$. Intuitively, \dIconfluence{} is the same thing as \Iconfluence{} with
the added restriction that subexpressions are join-free.

We say $T$ is \defword{\dIreducible{}} if for every pair $e_1$ and $e_2$ of
join-free expressions, $\Irec{e_1 \join e_2}$ implies that there exists some join-free expression $e_3$ such that $\Irec{e_3}$ and $eval(e3) = eval(e1 \join
e_2)$.  Intuitively, \dIreduction{} allows us to eliminate join operators by
substituting $e_1 \join e_2$ (which contains 1 join) with $e_3$ (which does not
contain any joins).

% \section{\dIconfluence{} and \dIreduction{}}
A \defword{join-free} expression $e = t_1(t_2(\ldots t_n(s_0) \ldots))$ is an
expression that does not contain a join.

We say $T$ is \defword{\dIconfluent{}} if for every pair $e_1$ and $e_2$ of
join-free expressions, $\Irec{e_1}$ and $\Irec{e_2}$ implies that $I(e_1 \join
e_2)$. Intuitively, \dIconfluence{} is the same thing as \Iconfluence{} with
the added restriction that subexpressions are join-free.

We say $T$ is \defword{\dIreducible{}} if for every pair $e_1$ and $e_2$ of
join-free expressions, $\Irec{e_1 \join e_2}$ implies that there exists some join-free expression $e_3$ such that $\Irec{e_3}$ and $eval(e3) = eval(e1 \join
e_2)$.  Intuitively, \dIreduction{} allows us to eliminate join operators by
substituting $e_1 \join e_2$ (which contains 1 join) with $e_3$ (which does not
contain any joins).

% \input{diconfluence_direduction.fig}
%
% We can also interpret \dIconfluence{} and \dIreduction{} using the graph-based
% model of \Iconfluence{}, as illustrated in
% \figref{diconfluence_and_direduction}. Intuitively, \dIconfluence{} is the
% property that any two divergent join-free paths that satisfy $I$ can be joined
% together to form a new state that satisfies $I$. \dIreduction{} is the property
% that the join of any two divergent join-free paths that satisfy $I$ can be
% replaced with a single join-free path that satisfies $I$. Both conditions
% involve diamond diagrams, hence the names \dIconfluence{} and \dIreduction{}.

\begin{claim}\clmlabel{diconfluence_direducible_implies_iconfluence}
  If $T$ is \dIconfluent{} and \dIreducible, then it is \Iconfluent.
\end{claim}
\begin{proof}
  First, we prove that for all join-free $e$, if $\Irec{e}$ then there exists a
  join-free $e'$ where $\Irec{e'}$ and $eval(e) = eval(e')$. We proceed by a
  pretty straightforward structural induction on $e$.
  \begin{itemize}
    \item \textbf{Case 1: $e = s_0$.}
      Trivially, $e' = s_0.$

    \item \textbf{Case 2: $e = t(e_1)$.}
      $\Irec{t(e_1)}$, so $\Irec{e_1}$. Thus, by the inductive hypothesis,
      there exists a join-free $e_1'$ such that $\Irec{e_1'}$ and $eval(e_1) =
      eval(e_1')$. Now, let $e' = t(e_1')$ which is clearly join-free.
      \[
        eval(e')
          = eval(t(e_1'))
          = t(eval(e_1'))
          = t(eval(e_1))
          = eval(t(e_1))
          = eval(e)
      \]
      This also implies that $\Irec{e'}$.

    \item \textbf{Case 3: $e = e_1 \join e_2$.}
      $\Irec{e_1 \join e_2}$, so $\Irec{e_1}$ and $\Irec{e_2}$. Thus, by the
      inductive hypothesis, there exist join-free expressions $e_1'$ and $e_2'$
      where $eval(e_1') = eval(e_1)$, $\Irec{e_1'}$, $eval(e_2') = eval(e_2)$,
      and $\Irec{e_2'}$. Let $e'' = e_1' \join e_2'$
      \begin{align*}
        eval(e'')
        &= eval(e_1' \join e_2') \\
        &= eval(e_1') \join eval(e_2') \\
        &= eval(e_1) \join eval(e_2) \\
        &= eval(e_1 \join e_2) \\
        &= eval(e)
      \end{align*}
      This implies that $\Irec{e''}$. Because $T$ is \dIreducible, there exists
      a join-free $e'$ such that $\Irec{e'}$ and $eval(e') = eval(e'') =
      eval(e)$.
  \end{itemize}

  Next, consider arbitrary expressions $e_1$ and $e_2$ where $\Irec{e_1}$ and
  $\Irec{e_2}$. To show that $T$ is \Iconfluent, we must show that $I(e_1 \join
  e_2)$. Using our little lemma above, we know there exists join-free $e_1'$
  and $e_2'$ where $eval(e_1') = eval(e_1)$, $\Irec{e_1'}$, $eval(e_2') =
  eval(e_2)$, and $\Irec{e_2'}$. Because $T$ is \dIconfluent, $I(e_1' \join e_2')$, and
  \[
    eval(e_1' \join e_2')
      = eval(e_1') \join eval(e_2')
      = eval(e_1) \join eval(e_2)
      = eval(e_1 \join e_2)
  \]
  Thus, $I(e_1 \join e_2)$.
\end{proof}

%
% We can also interpret \dIconfluence{} and \dIreduction{} using the graph-based
% model of \Iconfluence{}, as illustrated in
% \figref{diconfluence_and_direduction}. Intuitively, \dIconfluence{} is the
% property that any two divergent join-free paths that satisfy $I$ can be joined
% together to form a new state that satisfies $I$. \dIreduction{} is the property
% that the join of any two divergent join-free paths that satisfy $I$ can be
% replaced with a single join-free path that satisfies $I$. Both conditions
% involve diamond diagrams, hence the names \dIconfluence{} and \dIreduction{}.

\begin{claim}\clmlabel{diconfluence_direducible_implies_iconfluence}
  If $T$ is \dIconfluent{} and \dIreducible, then it is \Iconfluent.
\end{claim}
\begin{proof}
  First, we prove that for all join-free $e$, if $\Irec{e}$ then there exists a
  join-free $e'$ where $\Irec{e'}$ and $eval(e) = eval(e')$. We proceed by a
  pretty straightforward structural induction on $e$.
  \begin{itemize}
    \item \textbf{Case 1: $e = s_0$.}
      Trivially, $e' = s_0.$

    \item \textbf{Case 2: $e = t(e_1)$.}
      $\Irec{t(e_1)}$, so $\Irec{e_1}$. Thus, by the inductive hypothesis,
      there exists a join-free $e_1'$ such that $\Irec{e_1'}$ and $eval(e_1) =
      eval(e_1')$. Now, let $e' = t(e_1')$ which is clearly join-free.
      \[
        eval(e')
          = eval(t(e_1'))
          = t(eval(e_1'))
          = t(eval(e_1))
          = eval(t(e_1))
          = eval(e)
      \]
      This also implies that $\Irec{e'}$.

    \item \textbf{Case 3: $e = e_1 \join e_2$.}
      $\Irec{e_1 \join e_2}$, so $\Irec{e_1}$ and $\Irec{e_2}$. Thus, by the
      inductive hypothesis, there exist join-free expressions $e_1'$ and $e_2'$
      where $eval(e_1') = eval(e_1)$, $\Irec{e_1'}$, $eval(e_2') = eval(e_2)$,
      and $\Irec{e_2'}$. Let $e'' = e_1' \join e_2'$
      \begin{align*}
        eval(e'')
        &= eval(e_1' \join e_2') \\
        &= eval(e_1') \join eval(e_2') \\
        &= eval(e_1) \join eval(e_2) \\
        &= eval(e_1 \join e_2) \\
        &= eval(e)
      \end{align*}
      This implies that $\Irec{e''}$. Because $T$ is \dIreducible, there exists
      a join-free $e'$ such that $\Irec{e'}$ and $eval(e') = eval(e'') =
      eval(e)$.
  \end{itemize}

  Next, consider arbitrary expressions $e_1$ and $e_2$ where $\Irec{e_1}$ and
  $\Irec{e_2}$. To show that $T$ is \Iconfluent, we must show that $I(e_1 \join
  e_2)$. Using our little lemma above, we know there exists join-free $e_1'$
  and $e_2'$ where $eval(e_1') = eval(e_1)$, $\Irec{e_1'}$, $eval(e_2') =
  eval(e_2)$, and $\Irec{e_2'}$. Because $T$ is \dIconfluent, $I(e_1' \join e_2')$, and
  \[
    eval(e_1' \join e_2')
      = eval(e_1') \join eval(e_2')
      = eval(e_1) \join eval(e_2)
      = eval(e_1 \join e_2)
  \]
  Thus, $I(e_1 \join e_2)$.
\end{proof}

%
% We can also interpret \dIconfluence{} and \dIreduction{} using the graph-based
% model of \Iconfluence{}, as illustrated in
% \figref{diconfluence_and_direduction}. Intuitively, \dIconfluence{} is the
% property that any two divergent join-free paths that satisfy $I$ can be joined
% together to form a new state that satisfies $I$. \dIreduction{} is the property
% that the join of any two divergent join-free paths that satisfy $I$ can be
% replaced with a single join-free path that satisfies $I$. Both conditions
% involve diamond diagrams, hence the names \dIconfluence{} and \dIreduction{}.

\begin{claim}\clmlabel{diconfluence_direducible_implies_iconfluence}
  If $T$ is \dIconfluent{} and \dIreducible, then it is \Iconfluent.
\end{claim}
\begin{proof}
  First, we prove that for all join-free $e$, if $\Irec{e}$ then there exists a
  join-free $e'$ where $\Irec{e'}$ and $eval(e) = eval(e')$. We proceed by a
  pretty straightforward structural induction on $e$.
  \begin{itemize}
    \item \textbf{Case 1: $e = s_0$.}
      Trivially, $e' = s_0.$

    \item \textbf{Case 2: $e = t(e_1)$.}
      $\Irec{t(e_1)}$, so $\Irec{e_1}$. Thus, by the inductive hypothesis,
      there exists a join-free $e_1'$ such that $\Irec{e_1'}$ and $eval(e_1) =
      eval(e_1')$. Now, let $e' = t(e_1')$ which is clearly join-free.
      \[
        eval(e')
          = eval(t(e_1'))
          = t(eval(e_1'))
          = t(eval(e_1))
          = eval(t(e_1))
          = eval(e)
      \]
      This also implies that $\Irec{e'}$.

    \item \textbf{Case 3: $e = e_1 \join e_2$.}
      $\Irec{e_1 \join e_2}$, so $\Irec{e_1}$ and $\Irec{e_2}$. Thus, by the
      inductive hypothesis, there exist join-free expressions $e_1'$ and $e_2'$
      where $eval(e_1') = eval(e_1)$, $\Irec{e_1'}$, $eval(e_2') = eval(e_2)$,
      and $\Irec{e_2'}$. Let $e'' = e_1' \join e_2'$
      \begin{align*}
        eval(e'')
        &= eval(e_1' \join e_2') \\
        &= eval(e_1') \join eval(e_2') \\
        &= eval(e_1) \join eval(e_2) \\
        &= eval(e_1 \join e_2) \\
        &= eval(e)
      \end{align*}
      This implies that $\Irec{e''}$. Because $T$ is \dIreducible, there exists
      a join-free $e'$ such that $\Irec{e'}$ and $eval(e') = eval(e'') =
      eval(e)$.
  \end{itemize}

  Next, consider arbitrary expressions $e_1$ and $e_2$ where $\Irec{e_1}$ and
  $\Irec{e_2}$. To show that $T$ is \Iconfluent, we must show that $I(e_1 \join
  e_2)$. Using our little lemma above, we know there exists join-free $e_1'$
  and $e_2'$ where $eval(e_1') = eval(e_1)$, $\Irec{e_1'}$, $eval(e_2') =
  eval(e_2)$, and $\Irec{e_2'}$. Because $T$ is \dIconfluent, $I(e_1' \join e_2')$, and
  \[
    eval(e_1' \join e_2')
      = eval(e_1') \join eval(e_2')
      = eval(e_1) \join eval(e_2)
      = eval(e_1 \join e_2)
  \]
  Thus, $I(e_1 \join e_2)$.
\end{proof}


We can also interpret \dIconfluence{} and \dIreduction{} using the graph-based
model of \Iconfluence{}, as illustrated in
\figref{diconfluence_and_direduction}. Intuitively, \dIconfluence{} is the
property that any two divergent join-free paths that satisfy $I$ can be joined
together to form a new state that satisfies $I$. \dIreduction{} is the property
that the join of any two divergent join-free paths that satisfy $I$ can be
replaced with a single join-free path that satisfies $I$. Both conditions
involve diamond diagrams, hence the names \dIconfluence{} and \dIreduction{}.

\begin{claim}\clmlabel{diconfluence_direducible_implies_iconfluence}
  If $T$ is \dIconfluent{} and \dIreducible, then it is \Iconfluent.
\end{claim}
\begin{proof}
  First, we prove that for all join-free $e$, if $\Irec{e}$ then there exists a
  join free $e'$ where $\Irec{e'}$ and $eval(e) = eval(e')$. We proceed by a
  pretty straightforward structural induction on $e$.
  \begin{itemize}
    \item \textbf{Case 1: $e = s_0$.}
      Trivially, $e' = s_0.$

    \item \textbf{Case 2: $e = t(e_1)$.}
      $\Irec{t(e_1)}$, so $\Irec{e_1}$. Thus, by the inductive hypothesis,
      there exists a join-free $e_1'$ such that $\Irec{e_1'}$ and $eval(e_1) =
      eval(e_1')$. Now, let $e' = t(e_1')$ which is clearly join-free.
      \[
        eval(e')
          = eval(t(e_1'))
          = t(eval(e_1'))
          = t(eval(e_1))
          = eval(t(e_1))
          = eval(e)
      \]
      This also implies that $\Irec{e'}$.

    \item \textbf{Case 3: $e = e_1 \join e_2$.}
      $\Irec{e_1 \join e_2}$, so $\Irec{e_1}$ and $\Irec{e_2}$. Thus, by the
      inductive hypothesis, there exist join-free expressions $e_1'$ and $e_2'$
      where $eval(e_1') = eval(e_1)$, $\Irec{e_1'}$, $eval(e_2') = eval(e_2)$,
      and $\Irec{e_2'}$. Let $e'' = e_1' \join e_2'$
      \[
        eval(e'')
          = eval(e_1' \join e_2')
          = eval(e_1') \join eval(e_2')
          = eval(e_1) \join eval(e_2)
          = eval(e_1 \join e_2)
          = eval(e)
      \]
      This implies that $\Irec{e''}$. Because $T$ is \dIreducible, there exists
      a join-free $e'$ such that $\Irec{e'}$ and $eval(e') = eval(e'') =
      eval(e)$.
  \end{itemize}

  Next, consider arbitrary expressions $e_1$ and $e_2$ where $\Irec{e_1}$ and
  $\Irec{e_2}$. To show that $T$ is \Iconfluent, we must show that $I(e_1 \join
  e_2)$. Using our little lemma above, we know there exists join-free $e_1'$
  and $e_2'$ where $eval(e_1') = eval(e_1)$, $\Irec{e_1'}$, $eval(e_2') =
  eval(e_2)$, and $\Irec{e_2'}$. Because $T$ is \dIconfluent, $I(e_1' \join e_2')$, and
  \[
    eval(e_1' \join e_2')
      = eval(e_1') \join eval(e_2')
      = eval(e_1) \join eval(e_2)
      = eval(e_1 \join e_2)
  \]
  Thus, $I(e_1 \join e_2)$.
\end{proof}
