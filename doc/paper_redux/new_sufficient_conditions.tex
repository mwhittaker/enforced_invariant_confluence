\section{Sufficient Conditions for \Iconfluence{}}
\newcommand{\Icommutativity}{$I$-commutativity}
\newcommand{\Ijoinisapply}{$I$-join-is-apply}
\newcommand{\nestedapply}[3]{{#1}_1({#1}_2(\ldots ({#1}_#2(#3)) \ldots))}
\newcommand{\bart}[1]{\overline{t_{#1}}}
\newcommand{\baru}[1]{\overline{u_{#1}}}

In this section, we prove that the following three conditions imply
\Iconfluence.
\begin{enumerate}
  \item
    $(S, \join)$ is a semilattice.

  \item
    Every transaction $t \in T$ is of the form $t(s) = s \join s_t$ for some
    $s_t \in S$.

  \item
    For all $u, t \in T$, and for all $s \in S$, if $I(s)$, $I(t(s))$, and
    $I(u(s))$, then $I(t(s) \join u(s))$.
\end{enumerate}

Using \clmref{DiconfluenceDireducibleImpliesIconfluence}, it suffices to
show that the three properties imply \dIconfluence{} and \dIreduction{}. We do
so in \clmref{ThreePropsDiconfluence} and \clmref{ThreePropsDireducible}.

\begin{claim}\clmlabel{ThreePropsDiconfluence}
  \Icommutativity, \Ijoinisapply, and \onedIconfluence{} imply \dIconfluence.
\end{claim}
\begin{proof}
  Consider arbitrary join-free expressions $e_1$ and $e_2$ where $\Irec{e_1}$
  and $\Irec{e_2}$. We want to show that $I(e_1 \join e_2)$. Because $e_1$ and
  $e_2$ are join-free, they are of the following form:
  \begin{mathpar}
    e_1 = \nestedapply{t}{n}{s_0}

    e_2 = \nestedapply{u}{m}{s_0}
  \end{mathpar}
  where $n, m \in \nats$ and $t_i, u_j \in T$. As a syntactic convenience, let
  $\bart{i}(e) = \nestedapply{t}{i}{e}$. Similarly define $\baru{j}(e)$.
  Then,
  \begin{mathpar}
    e_1 = \bart{n}(s_0)

    e_2 = \baru{m}(s_0)
  \end{mathpar}
  We prove by strong mathematical induction on m that for all i + j = m, I(barti join baruj).kjkj

\end{proof}

\begin{claim}\clmlabel{ThreePropsDireducible}
  \Icommutativity, \Ijoinisapply, and \onedIconfluence{} imply \dIreduction.
\end{claim}
\begin{proof}

\end{proof}

TODO(mwhittaker): Give examples of when these conditions are met.
TODO(mwhittaker): Explain that 1-I-confluence is necessary.
