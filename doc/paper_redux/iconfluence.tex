\section{\Iconfluence{}}
A \emph{distributed object} is a triple $(S, s_0, \join)$ where $S$ is a set of
states, $s_0 \in S$ is a designated start state, and $\join: S \times S \to S$
is a commutative binary merge operator. A \emph{transaction} $t: S \to S$ is a
function that maps one state to another. An \emph{invariant} $I: S \to
\set{\text{true}, \text{false}}$ is a predicate on states. For example,
$(\nats, +)$ is a distributed object, $t(x) = 2x$ is a transaction, and $I(x) =
x \geq 0$ is an invariant.

There are three ways to think about \Iconfluence{}: a process-based approach, a
graph-based approach, and an expression-based approach. All three approaches
are illustrated in \figref{threeconfluencemodels} and are described in more
detail below.

\documentclass[xcolor={dvipsnames,svgnames,table}]{beamer}

\usepackage[noend]{algpseudocode}
\usepackage{algorithm}
\usepackage{amsmath}
\usepackage{centernot}
\usepackage{etoolbox}
\usepackage{invariantconfluence}
\usepackage{mathpartir}
\usepackage{mathtools}
\usepackage{pervasives}
\usepackage{pgfpages}
\usepackage{python}
\usepackage{slides}
\usepackage{stmaryrd}
\usepackage{tikz}
\usetikzlibrary{arrows}
\usetikzlibrary{backgrounds}
\usetikzlibrary{calc}
\usetikzlibrary{decorations.pathreplacing}
\usetikzlibrary{positioning}

\title{Interactive Checks for Coordination Avoidance}
\author{Michael Whittaker, Joseph M. Hellerstein}
\date{}

% https://gist.github.com/andrejbauer/ac361549ac2186be0cdb
% \setbeameroption{hide notes} % Only slides
% \setbeameroption{show only notes} % Only notes
\setbeameroption{show notes on second screen=right} % Both
\setbeamertemplate{note page}{\pagecolor{yellow!10}\insertnote}

\begin{document}
\begin{frame}
  \maketitle
\end{frame}

{\begin{frame}
  \tikzstyle{pic}=[inner sep=0pt]
  \tikzstyle{thought}=[cloud, draw, aspect=4]

  \begin{center}
    \begin{tikzpicture}[xscale=3, yscale=3]
      \node[pic] at (0, 0) {%
        \includegraphics[width=0.2\textwidth]{assets/person.jpg}
      };
      \pause

      \node[thought, aspect=1] at (-0.5, 0.5) {};
      \node[thought] at (-1, 1) {Weak Consistency?};
      \pause

      \node[thought, aspect=1] at (0.5, 0.5) {};
      \node[thought] at (1, 1) {Strong Consistency?};
      \pause

      \node[anchor=west] at (-1.75, 0.25) {%
        \textcolor{flatgreen}{\texttt{+}} Super fast
      };
      \node[anchor=west] at (-1.75, 0) {%
        \textcolor{flatred}{\texttt{-}} Hard to reason about
      };
      \pause

      \node[anchor=west] at (0.3, 0.25) {%
        \textcolor{flatgreen}{\texttt{+}} Easy to reason about
      };
      \node[anchor=west] at (0.3, 0) {%
        \textcolor{flatred}{\texttt{-}} Sacrifices availability
      };
    \end{tikzpicture}
  \end{center}

  \note{%
    Imagine you're an application developer and you have some data that you
    want to replicate. But, you're not sure whether to replicate the data with
    weak consistency or with strong consistency. \\[12pt]

    You can implement weak consistency super fast, but weakly consistent
    systems are hard to reason about. Strongly consistent systems are much
    easier to reason about, but they come at the cost of availability and
    performance. \\[12pt]
  }
\end{frame}

\begin{frame}
  \tikzstyle{pic}=[inner sep=0pt]
  \tikzstyle{quote}=[fill=gray!30, inner sep=12pt]

  \begin{center}
    \begin{tikzpicture}[xscale=1.3]
      \node[pic] (bailis) at (0, 0) {%
        \includegraphics[width=0.25\textwidth]{assets/bailis.jpg}
      };
      \node[quote, text width=0.8\textwidth, align=left] at (0, 3) {
        \Large\textit{Why not have both?} Invariant confluence!
      };
      \path[fill=gray!30] (bailis.north) -- ++(0, 1.5) -- ++(1, 0) -- (bailis.north);
    \end{tikzpicture}
  \end{center}

  \note{%
    Fortunately, Peter Bailis saves the day and reminds us of when in 2014, he
    and some other smart folks including Alan and my advisor Joe defined
    invariant confluence as a way to get the benefits of both weak and strong
    consistency.

    Invariant confluence will be the topic of this talk, so let's build some
    intuition on what invariant confluence is.
  }
\end{frame}
}
{\newcommand{\bankdistance}{4}
\tikzstyle{bank}=[
  draw,
  minimum width=1.5cm,
  minimum height=1cm,
  font=\huge
]
\tikzstyle{op}=[
  -latex,
  ultra thick
]

\newcommand{\drawbanks}[6]{{
  \newcommand{\banka}{#1}
  \newcommand{\bankb}{#2}
  \newcommand{\bankaa}{#3}
  \newcommand{\bankbb}{#4}
  \newcommand{\bankaaa}{#5}
  \newcommand{\bankbbb}{#6}

  \node[bank, label=above:{Replica $A$}] (a1) at (0,              0) {\banka};
  \node[bank                           ] (a2) at (0,             -3) {\bankaa};
  \node[bank                           ] (a3) at (0,             -6) {\bankaaa};
  \node[bank, label=above:{Replica $B$}] (b1) at (\bankdistance,  0) {\bankb};
  \node[bank                           ] (b2) at (\bankdistance, -3) {\bankbb};
  \node[bank                           ] (b3) at (\bankdistance, -6) {\bankbbb};

  \draw[op] (a1) -- (a2);
  \draw[op] (a2) -- (a3);
  \draw[op] (b1) -- (b2);
  \draw[op] (b2) -- (b3);
}}

\newcommand{\zigzag}[2]{{
  \newcommand{\banka}{#1}
  \newcommand{\bankb}{#2}

  \draw[red, op] ($(\banka) + (0, -0.7)$) -- ($(\bankb) + (0, -1.0)$);
  \draw[red, op] ($(\bankb) + (0, -1.0)$) -- ($(\banka) + (0, -1.3)$);
  \draw[red, op] ($(\banka) + (0, -1.3)$) -- ($(\bankb) + (0, -1.6)$);
  \draw[red, op] ($(\bankb) + (0, -1.6)$) -- ($(\banka) + (0, -1.9)$);
}}

\begin{frame}{Replicated Bank Account}
  \begin{center}
    \begin{tikzpicture}
      \node[bank, label=above:{Replica $A$}] (a1) at (0, 0) {\$50};
      \node[bank, label=above:{Replica $B$}] (b1) at (\bankdistance, 0) {\$50};
    \end{tikzpicture}
  \end{center}
\end{frame}

\begin{frame}{Coordinating}
  \begin{center}
    \begin{tikzpicture}
      \drawbanks{\$50}{\$50}
                {\$60}{\$60}
                {\$80}{\$80}

      \draw[op] ($(a1) + (-2, 0)$) -- (a1) node[midway, above] {$+\$10$};
      \draw[op] ($(b1) + (+2, 0)$) -- (b1) node[midway, above] {$+\$20$};
      \draw[op] (a2) -- ($(a2) + (-2, 0)$) node[midway, above] {OK};
      \draw[op] (b3) -- ($(b3) + (+2, 0)$) node[midway, above] {OK};

      \zigzag{a1}{b1}
      \zigzag{a2}{b2}
    \end{tikzpicture}
  \end{center}
\end{frame}

\begin{frame}{Coordinating}
  \begin{center}
    \begin{tikzpicture}
      \drawbanks{\$50}{\$50}
                {\$20}{\$20}
                {\$20}{\$20}

      \draw[op] ($(a1) + (-2, 0)$) -- (a1) node[midway, above] {$-\$30$};
      \draw[op] ($(b1) + (+2, 0)$) -- (b1) node[midway, above] {$-\$40$};
      \draw[op] (a2) -- ($(a2) + (-2, 0)$) node[midway, above] {OK};
      \draw[op] (b3) -- ($(b3) + (+2, 0)$) node[midway, above] {NO};

      \zigzag{a1}{b1}
      \zigzag{a2}{b2}
    \end{tikzpicture}
  \end{center}
\end{frame}

\begin{frame}{Avoiding Coordination}
  \begin{center}
    \begin{tikzpicture}
      \drawbanks{\$50}{\$50}
                {\$60}{\$70}
                {\$80}{\$80}

      \draw[op] ($(a1) + (-2, 0)$) -- (a1) node[midway, above] {$+\$10$};
      \draw[op] ($(b1) + (+2, 0)$) -- (b1) node[midway, above] {$+\$20$};
      \draw[op] (a2) -- ($(a2) + (-2, 0)$) node[midway, above] {OK};
      \draw[op] (b2) -- ($(b2) + (+2, 0)$) node[midway, above] {OK};

      \draw[dashed, op] (a2) -- (b3) node[near start, sloped, above] {$+\$10$};
      \draw[dashed, op] (b2) -- (a3) node[near start, sloped, above] {$+\$20$};
    \end{tikzpicture}
  \end{center}
\end{frame}

\begin{frame}{Avoiding Coordination}
  \begin{center}
    \begin{tikzpicture}
      \drawbanks{\$50}{\$50}
                {\$20}{\$10}
                {-\$20}{-\$20}

      \draw[op] ($(a1) + (-2, 0)$) -- (a1) node[midway, above] {$-\$30$};
      \draw[op] ($(b1) + (+2, 0)$) -- (b1) node[midway, above] {$-\$40$};
      \draw[op] (a2) -- ($(a2) + (-2, 0)$) node[midway, above] {OK};
      \draw[op] (b2) -- ($(b2) + (+2, 0)$) node[midway, above] {OK};

      \draw[dashed, op] (a2) -- (b3) node[near start, sloped, above] {$-\$30$};
      \draw[dashed, op] (b2) -- (a3) node[near start, sloped, above] {$-\$40$};
    \end{tikzpicture}
  \end{center}
\end{frame}

\begin{frame}
  \begin{center}
    \Huge
    Deposits don't require coordination to maintain invariants, but withdrawals
    do.
  \end{center}
\end{frame}
}
{\documentclass[xcolor={dvipsnames,svgnames,table}]{beamer}

\usepackage[noend]{algpseudocode}
\usepackage{algorithm}
\usepackage{amsmath}
\usepackage{centernot}
\usepackage{etoolbox}
\usepackage{invariantconfluence}
\usepackage{mathpartir}
\usepackage{mathtools}
\usepackage{pervasives}
\usepackage{pgfpages}
\usepackage{python}
\usepackage{slides}
\usepackage{stmaryrd}
\usepackage{tikz}
\usetikzlibrary{arrows}
\usetikzlibrary{backgrounds}
\usetikzlibrary{calc}
\usetikzlibrary{decorations.pathreplacing}
\usetikzlibrary{positioning}

\title{Interactive Checks for Coordination Avoidance}
\author{Michael Whittaker, Joseph M. Hellerstein}
\date{}

% https://gist.github.com/andrejbauer/ac361549ac2186be0cdb
% \setbeameroption{hide notes} % Only slides
% \setbeameroption{show only notes} % Only notes
\setbeameroption{show notes on second screen=right} % Both
\setbeamertemplate{note page}{\pagecolor{yellow!10}\insertnote}

\begin{document}
\begin{frame}
  \maketitle
\end{frame}

{\begin{frame}
  \tikzstyle{pic}=[inner sep=0pt]
  \tikzstyle{thought}=[cloud, draw, aspect=4]

  \begin{center}
    \begin{tikzpicture}[xscale=3, yscale=3]
      \node[pic] at (0, 0) {%
        \includegraphics[width=0.2\textwidth]{assets/person.jpg}
      };
      \pause

      \node[thought, aspect=1] at (-0.5, 0.5) {};
      \node[thought] at (-1, 1) {Weak Consistency?};
      \pause

      \node[thought, aspect=1] at (0.5, 0.5) {};
      \node[thought] at (1, 1) {Strong Consistency?};
      \pause

      \node[anchor=west] at (-1.75, 0.25) {%
        \textcolor{flatgreen}{\texttt{+}} Super fast
      };
      \node[anchor=west] at (-1.75, 0) {%
        \textcolor{flatred}{\texttt{-}} Hard to reason about
      };
      \pause

      \node[anchor=west] at (0.3, 0.25) {%
        \textcolor{flatgreen}{\texttt{+}} Easy to reason about
      };
      \node[anchor=west] at (0.3, 0) {%
        \textcolor{flatred}{\texttt{-}} Sacrifices availability
      };
    \end{tikzpicture}
  \end{center}

  \note{%
    Imagine you're an application developer and you have some data that you
    want to replicate. But, you're not sure whether to replicate the data with
    weak consistency or with strong consistency. \\[12pt]

    You can implement weak consistency super fast, but weakly consistent
    systems are hard to reason about. Strongly consistent systems are much
    easier to reason about, but they come at the cost of availability and
    performance. \\[12pt]
  }
\end{frame}

\begin{frame}
  \tikzstyle{pic}=[inner sep=0pt]
  \tikzstyle{quote}=[fill=gray!30, inner sep=12pt]

  \begin{center}
    \begin{tikzpicture}[xscale=1.3]
      \node[pic] (bailis) at (0, 0) {%
        \includegraphics[width=0.25\textwidth]{assets/bailis.jpg}
      };
      \node[quote, text width=0.8\textwidth, align=left] at (0, 3) {
        \Large\textit{Why not have both?} Invariant confluence!
      };
      \path[fill=gray!30] (bailis.north) -- ++(0, 1.5) -- ++(1, 0) -- (bailis.north);
    \end{tikzpicture}
  \end{center}

  \note{%
    Fortunately, Peter Bailis saves the day and reminds us of when in 2014, he
    and some other smart folks including Alan and my advisor Joe defined
    invariant confluence as a way to get the benefits of both weak and strong
    consistency.

    Invariant confluence will be the topic of this talk, so let's build some
    intuition on what invariant confluence is.
  }
\end{frame}
}
{\newcommand{\bankdistance}{4}
\tikzstyle{bank}=[
  draw,
  minimum width=1.5cm,
  minimum height=1cm,
  font=\huge
]
\tikzstyle{op}=[
  -latex,
  ultra thick
]

\newcommand{\drawbanks}[6]{{
  \newcommand{\banka}{#1}
  \newcommand{\bankb}{#2}
  \newcommand{\bankaa}{#3}
  \newcommand{\bankbb}{#4}
  \newcommand{\bankaaa}{#5}
  \newcommand{\bankbbb}{#6}

  \node[bank, label=above:{Replica $A$}] (a1) at (0,              0) {\banka};
  \node[bank                           ] (a2) at (0,             -3) {\bankaa};
  \node[bank                           ] (a3) at (0,             -6) {\bankaaa};
  \node[bank, label=above:{Replica $B$}] (b1) at (\bankdistance,  0) {\bankb};
  \node[bank                           ] (b2) at (\bankdistance, -3) {\bankbb};
  \node[bank                           ] (b3) at (\bankdistance, -6) {\bankbbb};

  \draw[op] (a1) -- (a2);
  \draw[op] (a2) -- (a3);
  \draw[op] (b1) -- (b2);
  \draw[op] (b2) -- (b3);
}}

\newcommand{\zigzag}[2]{{
  \newcommand{\banka}{#1}
  \newcommand{\bankb}{#2}

  \draw[red, op] ($(\banka) + (0, -0.7)$) -- ($(\bankb) + (0, -1.0)$);
  \draw[red, op] ($(\bankb) + (0, -1.0)$) -- ($(\banka) + (0, -1.3)$);
  \draw[red, op] ($(\banka) + (0, -1.3)$) -- ($(\bankb) + (0, -1.6)$);
  \draw[red, op] ($(\bankb) + (0, -1.6)$) -- ($(\banka) + (0, -1.9)$);
}}

\begin{frame}{Replicated Bank Account}
  \begin{center}
    \begin{tikzpicture}
      \node[bank, label=above:{Replica $A$}] (a1) at (0, 0) {\$50};
      \node[bank, label=above:{Replica $B$}] (b1) at (\bankdistance, 0) {\$50};
    \end{tikzpicture}
  \end{center}
\end{frame}

\begin{frame}{Coordinating}
  \begin{center}
    \begin{tikzpicture}
      \drawbanks{\$50}{\$50}
                {\$60}{\$60}
                {\$80}{\$80}

      \draw[op] ($(a1) + (-2, 0)$) -- (a1) node[midway, above] {$+\$10$};
      \draw[op] ($(b1) + (+2, 0)$) -- (b1) node[midway, above] {$+\$20$};
      \draw[op] (a2) -- ($(a2) + (-2, 0)$) node[midway, above] {OK};
      \draw[op] (b3) -- ($(b3) + (+2, 0)$) node[midway, above] {OK};

      \zigzag{a1}{b1}
      \zigzag{a2}{b2}
    \end{tikzpicture}
  \end{center}
\end{frame}

\begin{frame}{Coordinating}
  \begin{center}
    \begin{tikzpicture}
      \drawbanks{\$50}{\$50}
                {\$20}{\$20}
                {\$20}{\$20}

      \draw[op] ($(a1) + (-2, 0)$) -- (a1) node[midway, above] {$-\$30$};
      \draw[op] ($(b1) + (+2, 0)$) -- (b1) node[midway, above] {$-\$40$};
      \draw[op] (a2) -- ($(a2) + (-2, 0)$) node[midway, above] {OK};
      \draw[op] (b3) -- ($(b3) + (+2, 0)$) node[midway, above] {NO};

      \zigzag{a1}{b1}
      \zigzag{a2}{b2}
    \end{tikzpicture}
  \end{center}
\end{frame}

\begin{frame}{Avoiding Coordination}
  \begin{center}
    \begin{tikzpicture}
      \drawbanks{\$50}{\$50}
                {\$60}{\$70}
                {\$80}{\$80}

      \draw[op] ($(a1) + (-2, 0)$) -- (a1) node[midway, above] {$+\$10$};
      \draw[op] ($(b1) + (+2, 0)$) -- (b1) node[midway, above] {$+\$20$};
      \draw[op] (a2) -- ($(a2) + (-2, 0)$) node[midway, above] {OK};
      \draw[op] (b2) -- ($(b2) + (+2, 0)$) node[midway, above] {OK};

      \draw[dashed, op] (a2) -- (b3) node[near start, sloped, above] {$+\$10$};
      \draw[dashed, op] (b2) -- (a3) node[near start, sloped, above] {$+\$20$};
    \end{tikzpicture}
  \end{center}
\end{frame}

\begin{frame}{Avoiding Coordination}
  \begin{center}
    \begin{tikzpicture}
      \drawbanks{\$50}{\$50}
                {\$20}{\$10}
                {-\$20}{-\$20}

      \draw[op] ($(a1) + (-2, 0)$) -- (a1) node[midway, above] {$-\$30$};
      \draw[op] ($(b1) + (+2, 0)$) -- (b1) node[midway, above] {$-\$40$};
      \draw[op] (a2) -- ($(a2) + (-2, 0)$) node[midway, above] {OK};
      \draw[op] (b2) -- ($(b2) + (+2, 0)$) node[midway, above] {OK};

      \draw[dashed, op] (a2) -- (b3) node[near start, sloped, above] {$-\$30$};
      \draw[dashed, op] (b2) -- (a3) node[near start, sloped, above] {$-\$40$};
    \end{tikzpicture}
  \end{center}
\end{frame}

\begin{frame}
  \begin{center}
    \Huge
    Deposits don't require coordination to maintain invariants, but withdrawals
    do.
  \end{center}
\end{frame}
}
{\documentclass[xcolor={dvipsnames,svgnames,table}]{beamer}

\usepackage[noend]{algpseudocode}
\usepackage{algorithm}
\usepackage{amsmath}
\usepackage{centernot}
\usepackage{etoolbox}
\usepackage{invariantconfluence}
\usepackage{mathpartir}
\usepackage{mathtools}
\usepackage{pervasives}
\usepackage{pgfpages}
\usepackage{python}
\usepackage{slides}
\usepackage{stmaryrd}
\usepackage{tikz}
\usetikzlibrary{arrows}
\usetikzlibrary{backgrounds}
\usetikzlibrary{calc}
\usetikzlibrary{decorations.pathreplacing}
\usetikzlibrary{positioning}

\title{Interactive Checks for Coordination Avoidance}
\author{Michael Whittaker, Joseph M. Hellerstein}
\date{}

% https://gist.github.com/andrejbauer/ac361549ac2186be0cdb
% \setbeameroption{hide notes} % Only slides
% \setbeameroption{show only notes} % Only notes
\setbeameroption{show notes on second screen=right} % Both
\setbeamertemplate{note page}{\pagecolor{yellow!10}\insertnote}

\begin{document}
\begin{frame}
  \maketitle
\end{frame}

{\input{sections/four_wise_men.tex}}
{\input{sections/bank_example.tex}}
{\input{sections/iconfluence.tex}}
{\input{sections/iclosure.tex}}
{\input{sections/decision_procedure.tex}}
{\input{sections/segmented_iconfluence.tex}}
{\input{sections/evaluation.tex}}
\end{document}
}
{\begin{frame}
  \Huge
  \begin{center}
    Goal: develop an invariant-confluence decision procedure.
  \end{center}

  \note{%
    Now that we've defined invariant confluence, we can turn our attention to
    the main goal of our paper. And that is to develop an invariant-confluence
    decision procedure. As you saw in the previos example, determining whether
    or not a distributed object is invariant confluent is not easy to do by
    hand, so we'd like to develop a decision procedure that can automatically
    check whether something is invariant confluent for us.
  }
\end{frame}

\begin{frame}
  \begin{center}
    \begin{tikzpicture}[xscale=3, yscale=2]
      \node (confluence) at (2, 0) {\Large invariant confluence};
      \pause

      \node (hard) at (2, 1) {hard to check :(};
      \draw[-latex, thick] (hard) to (confluence);
      \pause

      \node (closure) at (0, 0) {\Large invariant closure};
      \node at ($(confluence)!0.5!(closure)$) {\Large $\implies$};
      \pause

      \node (easy) at (0, 1) {easy to check :)};
      \draw[-latex, thick] (easy) to (closure);
    \end{tikzpicture}
  \end{center}

  \note{%
    Unfortunately, developing an invariant confluence decision procedure
    straight up is not easy. Invariant confluence is fundamentally a property
    about reachable states but reasoning automatically about reachable states
    is hard. \\[12pt]

    Instead of reasoning about invariant confluence directly then, we'll look
    at a sufficient condition for invariant confluence called invariant
    closure. \\[12pt]

    Unlike invariant confluence, we'll see that invariant closure is easy to
    check. \\[12pt]
  }
\end{frame}

\begin{frame}
  \Large
  We say a set $S$ is \defword{closed under $f$} if for every $x, y \in S, f(x,
  y) \in S$. \pause For example,
  \begin{itemize}
    \item Even numbers are closed under addition.
    \item Odd numbers are \emph{not} closed under addition.
  \end{itemize}

  \note{%
    First, a quick refresher on what it meas for a set to be closed. We say a
    set $S$ is closed under a binary operator $f$ if for every $x$ and $y$ in
    $S$, $f(x, y)$ is also in $S$. For example, even numbers are closed under
    addition because the sum of any two even numbers is even. But odd numbers
    are not closed under addition because the sum of two odd numbers may not
    be odd.
  }
\end{frame}

\begin{frame}
  \Large
  \begin{itemize}
    \item
      $O = (S, \join)$ is \defword{\invariantclosed{}} with respect to an
      invariant $I$ if invariant satisfying states are closed under merge.
    \pause\item
      For every state $s_1, s_2 \in S$, if $I(s_1)$ and $I(s_2)$, then $I(s_1
      \join s_2)$.
  \end{itemize}

  \note{%
    We say that an object $O$ is invariant closed if invariant satisfying
    states are closed under merge. That is, if an object is invariant closed,
    then for every pair of states $s_1$ and $s_2$, if $s_1$ and $s_2$ satisfy
    the invariant then so does $s_1 \join s_2$.
  }
\end{frame}

\begin{frame}
  \Large
  \[
    \text{invariant closure} \implies \text{invariant confluence}
  \]

  \pause

  Why?

  \pause
  Transactions maintain the invariant. If merging does as well, then the
  invariant is always maintained.

  \note{%
    As I mentioned earlier, invariant closure is a sufficient condition for
    invariant confluence. I'll refer you to the paper for the proof, but it's a
    very simple proof.
  }
\end{frame}

\begin{frame}
  \Large
  Checking invariant closure is more straightforward.

  \vspace{0.5in}
  \pause

  \begin{tikzpicture}[xscale=6]
    \node[draw] (formula) at (0, 0) {
      $
      \begin{aligned}
        \forall x_1, & y_1, x_2, y_2.\, \\
        \quad & x_1y_1 \leq 0 \land x_2y_2 \leq 0 \implies \\
        \quad & \max(x_1, x_2)\max(y_1, y_2) \leq 0
      \end{aligned}
      $
    };
    \pause
    \node (smt) at (1, 0) {SMT Solver};
    \draw[-latex, ultra thick] (formula) to (smt);
  \end{tikzpicture}

  \note{%
    The good thing about invariant closure is that it's much easier to check
    automatically. Remember that example we had with pairs of
    integers? We can pose whether or not that object is invariant closed as
    this formula, and we can pass this formula directly to an SMT solver to
    figure out whether it's invariant closed.
  }
\end{frame}

\begin{frame}
  \begin{center}
    \begin{tikzpicture}[xscale=3, yscale=2]
      \node (confluence) at (2, 0) {\Large invariant confluence};
      \node (closure) at (0, 0) {\Large invariant closure};
      \node at ($(confluence)!0.5!(closure)$) {\Large $\implies$};
      \node (hard) at (2, 1) {hard to check :(};
      \draw[-latex, thick] (hard) to (confluence);
      \node (easy) at (0, 1) {easy to check :)};
      \draw[-latex, thick] (easy) to (closure);
    \end{tikzpicture}
  \end{center}

  \note{%
    To decide whether an object is invariant confluent, then, we can take the
    object and ask if it's invariant closed. If it is, then it's also invariant
    confluent and we're done. If it's not invariant closed, then what do we
    know?
  }
\end{frame}

\begin{frame}
  \Large
  \[
    \text{invariant closure}
    \xLeftarrow{\phantom{aa}?\phantom{aa}}
    \text{invariant confluence}
  \]

  \note{%
    Well, that depends on whether invariant confluence implies invariant
    closure.
  }
\end{frame}

\input{runningexample.tex}

\begin{frame}
  \begin{columns}
    \begin{column}{0.5\textwidth}
      \centering
      \begin{tikzpicture}[scale=1]
        \begin{scope}
          \clip (\xmin.5, \ymax.5) rectangle (\xmax.5, \ymin.5);
          \draw[invregion] (\xmin.9, \ymax.9) rectangle (0.5, -0.5);
          \draw[invregion] (-0.5, 0.5) rectangle (\xmax.9, \ymin.9);
          \draw (0.5, 0.5) to (0.5, \ymax.5);
          \draw (0.5, 0.5) to (\xmax.5, 0.5);
          \draw (-0.5, -0.5) to (-0.5, \ymin.5);
          \draw (-0.5, -0.5) to (\xmin.5, -0.5);
        \end{scope}

        \xyaxes{}
        \quadi{point}{\xmin}{\xmax}{\ymin}{\ymax}
        \quadiii{point}{\xmin}{\xmax}{\ymin}{\ymax}
        \quadii{pointinv}{\xmin}{\xmax}{\ymin}{\ymax}
        \quadiv{pointinv}{\xmin}{\xmax}{\ymin}{\ymax}
      \end{tikzpicture}

      {\Huge Invariant}
    \end{column}
    \begin{column}{0.5\textwidth}
      \centering
      \begin{tikzpicture}[scale=1]
        \begin{scope}
          \clip (-1, 1) rectangle (\xmax.5, \ymin.5);
          \draw[reachableregion, draw=black] (-0.5, 0.5) rectangle (\xmax.9, \ymin.9);
        \end{scope}

        \xyaxes{}
        \quadi{point}{\xmin}{\xmax}{\ymin}{\ymax}
        \quadiii{point}{\xmin}{\xmax}{\ymin}{\ymax}
        \quadii{point}{\xmin}{\xmax}{\ymin}{\ymax}
        \quadiv{pointinv}{\xmin}{\xmax}{\ymin}{\ymax}
      \end{tikzpicture}

      {\Huge Reachable}
    \end{column}
  \end{columns}

  \note{%
    To see if it does, let's revisit the example from before. Recall that in
    this example, our object is invariant confluent. The set of reachable
    states is a subset of the invariant.
  }
\end{frame}

\begin{frame}
  \begin{columns}
    \begin{column}{0.5\textwidth}
      \centering
      \begin{tikzpicture}[scale=1]
        \begin{scope}
          \clip (\xmin.5, \ymax.5) rectangle (\xmax.5, \ymin.5);
          \draw[invregion] (\xmin.9, \ymax.9) rectangle (0.5, -0.5);
          \draw[invregion] (-0.5, 0.5) rectangle (\xmax.9, \ymin.9);
          \draw (0.5, 0.5) to (0.5, \ymax.5);
          \draw (0.5, 0.5) to (\xmax.5, 0.5);
          \draw (-0.5, -0.5) to (-0.5, \ymin.5);
          \draw (-0.5, -0.5) to (\xmin.5, -0.5);
        \end{scope}

        \xyaxes{}
        \quadi{point}{\xmin}{\xmax}{\ymin}{\ymax}
        \quadiii{point}{\xmin}{\xmax}{\ymin}{\ymax}
        \quadii{pointinv}{\xmin}{\xmax}{\ymin}{\ymax}
        \quadiv{pointinv}{\xmin}{\xmax}{\ymin}{\ymax}

        \draw[-latex, ultra thick] (-1, 2) to (2, 2);
        \draw[-latex, ultra thick] (2, -1) to (2, 2);
      \end{tikzpicture}

      {\Huge Invariant}
    \end{column}
    \begin{column}{0.5\textwidth}
      \centering
      \begin{tikzpicture}[scale=1]
        \begin{scope}
          \clip (-1, 1) rectangle (\xmax.5, \ymin.5);
          \draw[reachableregion, draw=black] (-0.5, 0.5) rectangle (\xmax.9, \ymin.9);
        \end{scope}

        \xyaxes{}
        \quadi{point}{\xmin}{\xmax}{\ymin}{\ymax}
        \quadiii{point}{\xmin}{\xmax}{\ymin}{\ymax}
        \quadii{point}{\xmin}{\xmax}{\ymin}{\ymax}
        \quadiv{pointinv}{\xmin}{\xmax}{\ymin}{\ymax}
      \end{tikzpicture}

      {\Huge Reachable}
    \end{column}
  \end{columns}

  \note{%
    But, note that our object is not invariant closed. The points $(-1, 2)$ and
    $(2, -1)$ both satisfy the invariant, but if we merge them we get the point
    $(2, 2)$, and the point $(2, 2)$ does not satisfy the invariant.
  }
\end{frame}

\begin{frame}
  \begin{center}
    \begin{tikzpicture}[xscale=3, yscale=2]
      \node (confluence) at (2, 0) {\Large invariant confluence};
      \node (closure) at (0, 0) {\Large invariant closure};
      \node at ($(confluence)!0.5!(closure)$) {\Large $\centernot\impliedby$};
      \node (hard) at (2, 1) {hard to check :(};
      \draw[-latex, thick] (hard) to (confluence);
      \node (easy) at (0, 1) {easy to check :)};
      \draw[-latex, thick] (easy) to (closure);
    \end{tikzpicture}
  \end{center}

  \note{%
    Thus, our object is invariant confluent but not invariant closed, so
    invariant confluence does not imply invariant closure. This is unfortunate.
    We can check for invariant closure but not invariant confluence, so we'd
    like the two to be equivalent. \\[12pt]

    Are there any situations in which the two do happen to be equivalent?
    \\[12pt]
  }
\end{frame}

\begin{frame}
  \begin{columns}
    \begin{column}{0.5\textwidth}
      \centering
      \begin{tikzpicture}[scale=1]
        \begin{scope}
          \clip (\xmin.5, \ymax.5) rectangle (\xmax.5, \ymin.5);
          \draw[invregion] (\xmin.9, \ymax.9) rectangle (0.5, -0.5);
          \draw[invregion] (-0.5, 0.5) rectangle (\xmax.9, \ymin.9);
          \draw (0.5, 0.5) to (0.5, \ymax.5);
          \draw (0.5, 0.5) to (\xmax.5, 0.5);
          \draw (-0.5, -0.5) to (-0.5, \ymin.5);
          \draw (-0.5, -0.5) to (\xmin.5, -0.5);
        \end{scope}

        \xyaxes{}
        \quadi{point}{\xmin}{\xmax}{\ymin}{\ymax}
        \quadiii{point}{\xmin}{\xmax}{\ymin}{\ymax}
        \quadii{pointinv}{\xmin}{\xmax}{\ymin}{\ymax}
        \quadiv{pointinv}{\xmin}{\xmax}{\ymin}{\ymax}

        \draw[-latex, ultra thick] (-1, 2) to (2, 2);
        \draw[-latex, ultra thick] (2, -1) to (2, 2);
      \end{tikzpicture}

      {\Huge Invariant}
    \end{column}
    \begin{column}{0.5\textwidth}
      \centering
      \begin{tikzpicture}[scale=1]
        \begin{scope}
          \clip (-1, 1) rectangle (\xmax.5, \ymin.5);
          \draw[reachableregion, draw=black] (-0.5, 0.5) rectangle (\xmax.9, \ymin.9);
        \end{scope}

        \xyaxes{}
        \quadi{point}{\xmin}{\xmax}{\ymin}{\ymax}
        \quadiii{point}{\xmin}{\xmax}{\ymin}{\ymax}
        \quadii{point}{\xmin}{\xmax}{\ymin}{\ymax}
        \quadiv{pointinv}{\xmin}{\xmax}{\ymin}{\ymax}
      \end{tikzpicture}

      {\Huge Reachable}
    \end{column}
  \end{columns}

  \note{%
    Well, let's take a look at our example again. We noticed that our object is
    not invariant closed because we can merge $(-1, 2)$ and $(2, -1)$ to
    violate the invariant. But, notice that the point $(-1, 2)$ is not
    reachable.
  }
\end{frame}

\begin{frame}
  \Large
  If $I$ is a subset of reachable states, then
  \[
    \text{invariant closure} \iff \text{invariant confluence}
  \]

  \pause

  Why?

  \pause

  Forward direction is the same as before. For backwards direction, $I$ and
  reachable states are equal. Reachable states are closed under merge, so
  therefore so is $I$.

  \note{%
    Turns out, this is not a coincidence. An invariant confluent object may not
    be invariant closed but only because we're merging points that are
    unreachable. If the invariant is a subset of the reachable states---that
    is, if every invariant satisfying point is reachable---then invariant
    closure and invariant confluence are equivalent. \\[12pt]

    Again, I'll refer you to the paper for a proof. \\[12pt]
  }
\end{frame}

\begin{frame}
  \Large
  If $I$ is a subset of reachable states, then
  \begin{center}
    \begin{tikzpicture}[xscale=3, yscale=2]
      \node (confluence) at (2, 0) {\Large invariant confluence};
      \node (closure) at (0, 0) {\Large invariant closure};
      \node at ($(confluence)!0.5!(closure)$) {\Large $\iff$};
      \node (hard) at (2, 1) {hard to check :(};
      \draw[-latex, thick] (hard) to (confluence);
      \node (easy) at (0, 1) {easy to check :)};
      \draw[-latex, thick] (easy) to (closure);
    \end{tikzpicture}
  \end{center}

  \note{%
    This is good news. Invariant confluence is hard to check, but invariant
    closure is easy to check. And when the invariant is a subset of the
    reachable states, the two are equivalent. In this case, invariant
    confluence also becomes easy to check.
  }
\end{frame}
}
{\begin{frame}
  \begin{center}
    \Huge
    Main idea: prune the invariant until it's a subset of reachable states.
    Then check for invariant closure.
  \end{center}
\end{frame}

\newcommand{\algocomment}[1]{\State \textcolor{flatdenim}{\texttt{//} #1}}

\begin{frame}
  \begin{algorithmic}
    \algocomment{Return if $O$ is \sTIconfluent{}.}
    \Function{IsInvConfluent}{$O$, $s_0$, $T$, $I$}
      \State
      \Return $I(s_0)$ and
      \Call{Helper}{$O$, $s_0$, $T$, $I$, $\set{s_0}$, $\emptyset$}
    \EndFunction

    \State

    \algocomment{$R$ is a set of \sTIreachable{} states.}
    \algocomment{$NR$ is a set of \sTIunreachable{} states.}
    \algocomment{$I(s_0)$ is a precondition.}
    \Function{Helper}{$O$, $s_0$, $T$, $I$, $R$, $NR$}
      \State closed, $s_1$, $s_2$ $\gets$ \Call{IsIclosed}{$O$, $I - NR$}
      \If {closed}
        \Return true
      \EndIf
      \State Augment $R, NR$ with random search and user input
      \If{$s_1, s_2 \in R$}
        \Return false
      \EndIf
      \State \Return \Call{Helper}{$O$, $s_0$, $T$, $I$, $R$, $NR$}
    \EndFunction
  \end{algorithmic}
\end{frame}

\tikzstyle{point}=[shape=circle, fill=flatgray, inner sep=2pt, draw=black]
\tikzstyle{reachable}=[fill=flatgreen!50, draw=none]
\tikzstyle{unreachable}=[fill=flatred!50, draw=none]
\tikzstyle{invariant}=[fill=flatblue!50, draw=none]
\tikzstyle{nothing}=[fill=white, draw=none]

\newcommand{\pointgrid}[4]{{
  \newcommand{\argxmin}{#1}
  \newcommand{\argxmax}{#2}
  \newcommand{\argymin}{#3}
  \newcommand{\argymax}{#4}

  \draw[] (\argxmin, 0) to (\argxmax, 0);
  \draw[] (0, \argymin) to (0, \argymax);
  \foreach \x in {\argxmin, ..., \argxmax} {
    \foreach \y in {\argymin, ..., \argymax} {
      \node[point] (\x-\y) at (\x, \y) {};
    }
  }
}}

\newcommand{\pointrect}[2]{
  \draw[#1] ($#2 + (-0.51, 0.51)$) rectangle ($#2 + (0.51, -0.51)$);
}

\newcommand{\subfigwidth}{0.48\textwidth}
\newcommand{\subfighspace}{0.3cm}
\newcommand{\tikzhspace}{0.4cm}
\newcommand{\tikzscale}{0.5}
\newcommand{\xmin}{-3}
\newcommand{\xmax}{3}
\newcommand{\ymin}{-3}
\newcommand{\ymax}{3}

\begin{frame}
  \begin{columns}
    \begin{column}{0.3\textwidth}
      \centering
      \begin{tikzpicture}[scale=\tikzscale]
        \draw[reachable, draw=black] (-0.5, 0.5) rectangle (0.5, -0.5);
        \pointgrid{\xmin}{\xmax}{\ymin}{\ymax}
        \node at (0, \ymax + 1) {$R$};
      \end{tikzpicture}
    \end{column}
    \begin{column}{0.3\textwidth}
      \centering
      \begin{tikzpicture}[scale=\tikzscale]
        \pointgrid{\xmin}{\xmax}{\ymin}{\ymax}
        \node at (0, \ymax + 1) {$NR$};
      \end{tikzpicture}
    \end{column}
    \begin{column}{0.3\textwidth}
      \centering
      \begin{tikzpicture}[scale=\tikzscale]
        \draw[invariant] (\xmin.5, \ymax.5) rectangle (0.5, -0.5);
        \draw[invariant] (-0.5, 0.5) rectangle (\xmax.5, \ymin.5);
        \draw (0.5, 0.5) to (0.5, \ymax.5);
        \draw (0.5, 0.5) to (\xmax.5, 0.5);
        \draw (-0.5, -0.5) to (-0.5, \ymin.5);
        \draw (-0.5, -0.5) to (\xmin.5, -0.5);
        \pointgrid{\xmin}{\xmax}{\ymin}{\ymax}
        \node at (0, \ymax + 1) {$I - NR$};

        \pause
        \node[fill opacity=0.75, text opacity=1, fill=white] at (-1, 1) {$s_1$};
        \node[fill opacity=0.75, text opacity=1, fill=white] at (1, -1) {$s_2$};
      \end{tikzpicture}
    \end{column}
  \end{columns}
\end{frame}

\begin{frame}
  \begin{columns}
    \begin{column}{0.3\textwidth}
      \begin{tikzpicture}[scale=\tikzscale]
        \draw[reachable, draw=black] (-0.5, 0.5) rectangle (1.5, -1.5);
        \pointgrid{\xmin}{\xmax}{\ymin}{\ymax}
        \node at (0, \ymax + 1) {$R$};
      \end{tikzpicture}
    \end{column}
    \begin{column}{0.3\textwidth}
      \begin{tikzpicture}[scale=\tikzscale]
        \pointrect{unreachable, draw=black}{(-1, 1)}
        \pointgrid{\xmin}{\xmax}{\ymin}{\ymax}
        \node at (0, \ymax + 1) {$NR$};
      \end{tikzpicture}
    \end{column}
    \begin{column}{0.3\textwidth}
      \begin{tikzpicture}[scale=\tikzscale]
        \draw[invariant] (\xmin.5, \ymax.5) rectangle (0.5, -0.5);
        \draw[invariant] (-0.5, 0.5) rectangle (\xmax.5, \ymin.5);
        \draw (0.5, 0.5) to (0.5, \ymax.5);
        \draw (0.5, 0.5) to (\xmax.5, 0.5);
        \draw (-0.5, -0.5) to (-0.5, \ymin.5);
        \draw (-0.5, -0.5) to (\xmin.5, -0.5);
        \pointrect{nothing, draw=black}{(-1, 1)}
        \pointgrid{\xmin}{\xmax}{\ymin}{\ymax}
        \node at (0, \ymax + 1) {$I - NR$};

        \pause
        \node[fill opacity=0.75, text opacity=1, fill=white] at (-1, 2) {$s_1$};
        \node[fill opacity=0.75, text opacity=1, fill=white] at (3, -3) {$s_2$};
      \end{tikzpicture}
    \end{column}
  \end{columns}
\end{frame}

\begin{frame}
  \begin{columns}
    \begin{column}{0.3\textwidth}
      \begin{tikzpicture}[scale=\tikzscale]
        \draw[reachable, draw=black] (-0.5, 0.5) rectangle (2.5, -2.5);
        \pointrect{reachable, draw=black}{(3, -3)}
        \pointgrid{\xmin}{\xmax}{\ymin}{\ymax}
        \node at (0, \ymax + 1) {$R$};
      \end{tikzpicture}
    \end{column}
    \begin{column}{0.3\textwidth}
      \begin{tikzpicture}[scale=\tikzscale]
        \pointrect{unreachable}{(-1, 1)}
        \pointrect{unreachable}{(-1, 2)}
        \draw (-1.5, 0.5) rectangle (-0.5, 2.5);
        \pointgrid{\xmin}{\xmax}{\ymin}{\ymax}
        \node at (0, \ymax + 1) {$NR$};
      \end{tikzpicture}
    \end{column}
    \begin{column}{0.3\textwidth}
      \begin{tikzpicture}[scale=\tikzscale]
        \draw[invariant] (\xmin.5, \ymax.5) rectangle (0.5, -0.5);
        \draw[invariant] (-0.5, 0.5) rectangle (\xmax.5, \ymin.5);
        \draw (0.5, 0.5) to (0.5, \ymax.5);
        \draw (0.5, 0.5) to (\xmax.5, 0.5);
        \draw (-0.5, -0.5) to (-0.5, \ymin.5);
        \draw (-0.5, -0.5) to (\xmin.5, -0.5);
        \pointrect{nothing}{(-1, 1)}
        \pointrect{nothing}{(-1, 2)}
        \draw (-1.5, 0.5) rectangle (-0.5, 2.5);
        \pointgrid{\xmin}{\xmax}{\ymin}{\ymax}
        \node at (0, \ymax + 1) {$I - NR$};

        \pause
        \node[fill opacity=0.75, text opacity=1, fill=white] at (-2, 1) {$s_1$};
        \node[fill opacity=0.75, text opacity=1, fill=white] at (1, -1) {$s_2$};
      \end{tikzpicture}
    \end{column}
  \end{columns}
\end{frame}

\begin{frame}
  \begin{columns}
    \begin{column}{0.3\textwidth}
      \begin{tikzpicture}[scale=\tikzscale]
        \draw[reachable] (-0.5, 0.5) rectangle (2.5, -2.5);
        \pointrect{reachable}{(3, 0)}
        \pointrect{reachable}{(3, -1)}
        \pointrect{reachable}{(0, -3)}
        \pointrect{reachable}{(2, -3)}
        \pointrect{reachable}{(3, -3)}
        \draw (-0.5, 0.5) to (\xmax.5, 0.5);
        \draw (-0.5, 0.5) to (-0.5, \ymin.5);
        \draw (-0.5, -3.5) to ++(1, 0)
        to ++(0, 1)
        to ++(1, 0)
        to ++(0, -1)
        to ++(2, 0)
        to ++(0, 1)
        to ++(-1, 0)
        to ++(0, 1)
        to ++(1, 0)
        to ++(0, 2);
        \pointgrid{\xmin}{\xmax}{\ymin}{\ymax}
        \node at (0, \ymax + 1) {$R$};
      \end{tikzpicture}
    \end{column}
    \begin{column}{0.3\textwidth}
      \begin{tikzpicture}[scale=\tikzscale]
        \draw[unreachable] (\xmin.5, \ymax.5) rectangle (-0.5, \ymin.5);
        \draw (\xmin.5, \ymax.5) to (-0.5, \ymax.5) to
        (-0.5, \ymin.5) to (\xmin.5, \ymin.5);
        \pointgrid{\xmin}{\xmax}{\ymin}{\ymax}
        \node at (0, \ymax + 1) {$NR$};
      \end{tikzpicture}
    \end{column}
    \begin{column}{0.3\textwidth}
      \begin{tikzpicture}[scale=\tikzscale]
        \draw[invariant] (-0.5, 0.5) rectangle (\xmax.5, \ymin.5);
        \draw (-0.5, 0.5) to (\xmax.5, 0.5);
        \draw (-0.5, 0.5) to (-0.5, \ymin.5);
        \pointgrid{\xmin}{\xmax}{\ymin}{\ymax}
        \node at (0, \ymax + 1) {$I - NR$};
      \end{tikzpicture}
    \end{column}
  \end{columns}
\end{frame}

\begin{frame}
  \begin{algorithmic}
    \algocomment{Return if $O$ is \sTIconfluent{}.}
    \Function{IsInvConfluent}{$O$, $s_0$, $T$, $I$}
      \State
      \Return $I(s_0)$ and
      \Call{Helper}{$O$, $s_0$, $T$, $I$, $\set{s_0}$, $\emptyset$}
    \EndFunction

    \State

    \algocomment{$R$ is a set of \sTIreachable{} states.}
    \algocomment{$NR$ is a set of \sTIunreachable{} states.}
    \algocomment{$I(s_0)$ is a precondition.}
    \Function{Helper}{$O$, $s_0$, $T$, $I$, $R$, $NR$}
      \State closed, $s_1$, $s_2$ $\gets$ \Call{IsIclosed}{$O$, $I - NR$}
      \If {closed}
        \Return true
      \EndIf
      \State Augment $R, NR$ with random search and user input
      \If{$s_1, s_2 \in R$}
        \Return false
      \EndIf
      \State \Return \Call{Helper}{$O$, $s_0$, $T$, $I$, $R$, $NR$}
    \EndFunction
  \end{algorithmic}
\end{frame}
}
{\begin{frame}
  \begin{enumerate}
    \item
      Use the interactive decision procedure to check if object is invariant
      confluent.
    \pause\item
      If it is, deploy it with weak consistency.
    \pause\item
      If it's not, then...? \pause deploy with strong consistency?
  \end{enumerate}
\end{frame}

\newcommand{\dy}{0.2}
\begin{frame}
  \begin{center}
    \begin{tikzpicture}
      \draw[ultra thick, latex-latex] (0, 0) to (8, 0);
      \draw[thick] (1, -\dy) to (1, \dy);
      \draw[thick] (4, -\dy) to (4, \dy);
      \draw[thick] (7, -\dy) to (7, \dy);
      \node[align=center] at (9, 0) {Invariant\\confluence-ness};
      \pause
      \node[rotate=-45, anchor=north west] at (7, -\dy) {Invariant Confluent};
      \pause
      \node[rotate=-45, anchor=north west] at (4, -\dy) {Kinda Invariant Confluent};
      \pause
      \node[rotate=-45, anchor=north west] at (1, -\dy) {Not at all Invariant Confluent};

      \pause
      \draw[ultra thick, -latex, red] (7, 1) to (7, \dy);
      \node[align=center, fill=white] at (7, 1.5) {Weak\\Consistency};

      \pause
      \draw[decorate, decoration={brace, amplitude=12pt, raise=4pt},
            ultra thick, blue] (1, \dy) to (6.9, \dy);
      \node[align=center] at (4, 1.5) {Strong\\Consistency};
    \end{tikzpicture}
  \end{center}
\end{frame}

\begin{frame}
  \begin{center}
    \begin{tikzpicture}
      \draw[ultra thick, latex-latex] (0, 0) to (8, 0);
      \draw[thick] (1, -\dy) to (1, \dy);
      \draw[thick] (4, -\dy) to (4, \dy);
      \draw[thick] (7, -\dy) to (7, \dy);
      \node[align=center] at (9, 0) {Invariant\\confluence-ness};
      \node[rotate=-45, anchor=north west] at (7, -\dy) {Invariant Confluent};
      \node[rotate=-45, anchor=north west] at (4, -\dy) {Kinda Invariant Confluent};
      \node[rotate=-45, anchor=north west] at (1, -\dy) {Not at all Invariant Confluent};

      \draw[ultra thick, -latex, red] (7, 1) to (7, \dy);
      \node[align=center, fill=white] at (7, 1.5) {Weak\\Consistency};
      \draw[ultra thick, -latex, red] (4, 1) to (4, \dy);
      \node[align=center, fill=white] at (4, 1.5) {Weakish\\Consistency};
      \draw[ultra thick, -latex, red] (1, 1) to (1, \dy);
      \node[align=center, fill=white] at (1, 1.5) {Strong\\Consistency};
    \end{tikzpicture}
  \end{center}
\end{frame}

\begin{frame}
  \begin{center}
    \Large
    \defword{Segmented invariant confluence}: divide state space into segments;
    operate with weak consistency within segments and strong consistency across
    segments.
  \end{center}
\end{frame}

\begin{frame}
  A \defword{segmentation} $\Sigma = (I_1, T_1), \ldots, (I_n, T_n)$ is a
  sequence (not a set) of $n$ segments $(I_i, T_i)$.
  \pause
  $O$ is \defword{segmented \invariantconfluent{}} with respect to $s_0$, $T$,
  $I$, and $\Sigma$, abbreviated \defword{\sTISconfluent}, if the following
  conditions hold:
  \begin{itemize}
    \pause\item
      The start state satisfies the invariant (i.e. $I(s_0)$).

    \pause\item
      $I$ is covered by the invariants in $\Sigma$.

    \pause\item
      $O$ is \invariantconfluent{} within each segment. That is, for every $(I_i,
      T_i) \in \Sigma$ and for every state $s \in I_i$, $O$ is
      \sticonfluent{s}{T_i}{I_i}.
  \end{itemize}
\end{frame}

{
\newcommand{\xmin}{-2}
\newcommand{\xmax}{2}
\newcommand{\ymin}{-2}
\newcommand{\ymax}{2}

% Axes.
\newcommand{\xyaxes}{
  \draw[] (\xmin.5, 0) to (\xmax.5, 0);
  \draw[] (0, \ymin.5) to (0, \ymax.5);
  \node at (\xmax + 1, 0) {$x$};
  \node at (0, \ymax + 1) {$y$};
}

% Quadrant 1.
\newcommand{\quadi}[5]{{
  \newcommand{\argstyle}{#1}
  \newcommand{\argxmin}{#2}
  \newcommand{\argxmax}{#3}
  \newcommand{\argymin}{#4}
  \newcommand{\argymax}{#5}
  \foreach \x in {0, ..., \argxmax} {
    \foreach \y in {0, ..., \argymax} {
      \node[\argstyle] (\x-\y) at (\x, \y) {};
    }
  }
}}

% Quadrant 2.
\newcommand{\quadii}[5]{{
  \newcommand{\argstyle}{#1}
  \newcommand{\argxmin}{#2}
  \newcommand{\argxmax}{#3}
  \newcommand{\argymin}{#4}
  \newcommand{\argymax}{#5}
  \foreach \x in {\argxmin, ..., 0} {
    \foreach \y in {0, ..., \argymax} {
      \node[\argstyle] (\x-\y) at (\x, \y) {};
    }
  }
}}

% Quadrant 3.
\newcommand{\quadiii}[5]{{
  \newcommand{\argstyle}{#1}
  \newcommand{\argxmin}{#2}
  \newcommand{\argxmax}{#3}
  \newcommand{\argymin}{#4}
  \newcommand{\argymax}{#5}
  \foreach \x in {\argxmin, ..., 0} {
    \foreach \y in {\argymin, ..., 0} {
      \node[\argstyle] (\x-\y) at (\x, \y) {};
    }
  }
}}

% Quadrant 4.
\newcommand{\quadiv}[5]{{
  \newcommand{\argstyle}{#1}
  \newcommand{\argxmin}{#2}
  \newcommand{\argxmax}{#3}
  \newcommand{\argymin}{#4}
  \newcommand{\argymax}{#5}
  \foreach \x in {0, ..., \argxmax} {
    \foreach \y in {\argymin, ..., 0} {
      \node[\argstyle] (\x-\y) at (\x, \y) {};
    }
  }
}}

% State labels.
\newcommand{\statelabels}{
  \node[statelabel] at (0, 0) {$s_0$};
  \node[statelabel] at (-1, 1) {$s_1$};
  \node[statelabel] at (1, -1) {$s_2$};
  \node[statelabel] at (1, 1) {$s_3$};
}

\tikzstyle{point}=[shape=circle, fill=flatgray, inner sep=3pt]
\tikzstyle{inv}=[line width=0.75pt, draw=black]
\tikzstyle{pointinv}=[point, inv]
\tikzstyle{invregion}=[rounded corners, fill=flatgreen!50, draw=none]
\tikzstyle{reachableregion}=[rounded corners, fill=flatblue!50, draw=none]
\tikzstyle{statelabel}=[anchor=south west, inner sep=1pt]

\begin{frame}
  \begin{columns}
    \begin{column}{0.5\textwidth}
      \centering
      \begin{tikzpicture}[scale=1]
        \begin{scope}
          \clip (\xmin.5, \ymax.5) rectangle (\xmax.5, \ymin.5);
          \draw[invregion] (\xmin.9, \ymax.9) rectangle (0.5, -0.5);
          \draw[invregion] (-0.5, 0.5) rectangle (\xmax.9, \ymin.9);
          \draw (0.5, 0.5) to (0.5, \ymax.5);
          \draw (0.5, 0.5) to (\xmax.5, 0.5);
          \draw (-0.5, -0.5) to (-0.5, \ymin.5);
          \draw (-0.5, -0.5) to (\xmin.5, -0.5);
        \end{scope}

        \xyaxes{}
        \quadi{point}{\xmin}{\xmax}{\ymin}{\ymax}
        \quadiii{point}{\xmin}{\xmax}{\ymin}{\ymax}
        \quadii{pointinv}{\xmin}{\xmax}{\ymin}{\ymax}
        \quadiv{pointinv}{\xmin}{\xmax}{\ymin}{\ymax}
      \end{tikzpicture}

      {\Huge Invariant}
    \end{column}
    \begin{column}{0.5\textwidth}
      \pause
      \centering
      \begin{tikzpicture}[scale=1]
        \begin{scope}
          \clip (-3, 3) rectangle (\xmax.5, \ymin.5);
          \draw[reachableregion, draw=black] (-2.5, 2.5) rectangle (\xmax.9, \ymin.9);
        \end{scope}

        \xyaxes{}
        \quadi{pointinv}{\xmin}{\xmax}{\ymin}{\ymax}
        \quadiii{pointinv}{\xmin}{\xmax}{\ymin}{\ymax}
        \quadii{pointinv}{\xmin}{\xmax}{\ymin}{\ymax}
        \quadiv{pointinv}{\xmin}{\xmax}{\ymin}{\ymax}
      \end{tikzpicture}

      {\Huge Reachable}
    \end{column}
  \end{columns}
\end{frame}
}

{
\tikzstyle{point}=[shape=circle, fill=flatgray, inner sep=2pt, draw=black]
\tikzstyle{region}=[draw=none]
\tikzstyle{region1}=[region, fill=flatred!50]
\tikzstyle{region2}=[region, fill=flatgreen!50]
\tikzstyle{region3}=[region, fill=flatblue!50]
\tikzstyle{region4}=[region, fill=flatpurple!50]

\newcommand{\pointgrid}[4]{{
  \newcommand{\argxmin}{#1}
  \newcommand{\argxmax}{#2}
  \newcommand{\argymin}{#3}
  \newcommand{\argymax}{#4}

  \draw[] (\argxmin, 0) to (\argxmax, 0);
  \draw[] (0, \argymin) to (0, \argymax);
  \foreach \x in {\argxmin, ..., \argxmax} {
    \foreach \y in {\argymin, ..., \argymax} {
      \node[point] (\x-\y) at (\x, \y) {};
    }
  }
}}

\newcommand{\subfigwidth}{0.24\columnwidth}
\newcommand{\subfighspace}{0.3cm}
\newcommand{\tikzhspace}{0.4cm}
\newcommand{\tikzscale}{0.75}
\newcommand{\xmin}{-2}
\newcommand{\xmax}{2}
\newcommand{\ymin}{-2}
\newcommand{\ymax}{2}

\begin{frame}
  \begin{center}
    \begin{tikzpicture}[scale=\tikzscale]
      \draw[white] (-3, -3) to (3, 3);
      \draw[region1] (\xmin.5, \ymax.5) rectangle (-0.5, 0.5);
      \draw (-0.5, 0.5) to (\xmin.5, 0.5);
      \draw (-0.5, 0.5) to (-0.5, \ymax.5);
      \pointgrid{\xmin}{\xmax}{\ymin}{\ymax}
    \end{tikzpicture}%
    \hspace{0.1in}%
    \begin{tikzpicture}[scale=\tikzscale]
      \draw[white] (-3, -3) to (3, 3);
      \draw[region2] (-0.5, 0.5) rectangle (\xmax.5, \ymin.5);
      \draw (-0.5, 0.5) to (\xmax.5, 0.5);
      \draw (-0.5, 0.5) to (-0.5, \ymin.5);
      \pointgrid{\xmin}{\xmax}{\ymin}{\ymax}
    \end{tikzpicture}

    \begin{tikzpicture}[scale=\tikzscale]
      \draw[white] (-3, -3) to (3, 3);
      \draw[region3] (-0.5, \ymax.5) rectangle (0.5, \ymin.5);
      \draw (-0.5, \ymax.5) to (-0.5, \ymin.5);
      \draw (0.5, \ymax.5) to (0.5, \ymin.5);
      \pointgrid{\xmin}{\xmax}{\ymin}{\ymax}
    \end{tikzpicture}%
    \hspace{0.1in}%
    \begin{tikzpicture}[scale=\tikzscale]
      \draw[white] (-3, -3) to (3, 3);
      \draw[region4] (\xmin.5, 0.5) rectangle (\xmax.5, -0.5);
      \draw (\xmax.5, -0.5) to (\xmin.5, -0.5);
      \draw (\xmax.5, 0.5) to (\xmin.5, 0.5);
      \pointgrid{\xmin}{\xmax}{\ymin}{\ymax}
    \end{tikzpicture}
  \end{center}
\end{frame}
}
}
{\section{Evaluation}\seclabel{Evaluation}
In this section, we describe and evaluate Lucy: a prototype implementation of
our decision procedures and system models.
% We evaluate Lucy by answering the following questions:
% \begin{itemize}
%   \item
%     How practical is the interactive invariant-confluence decision procedure?
%     Can we use it to classify real-world transactions and invariants?
%   \item
%     How practical is segmented invariant-confluence? Are real-world workloads
%     amenable to segmentation?
%   \item
%     How efficient is the interactive invariant-confluence decision procedure?
%   \item
%     How efficiently can we replicate a segmented invariant-confluent object as
%     compared to alternative approaches like replicating with weak or strong
%     consistency?
%   \item
%     How does the performance of replicating a segmented invariant-confluence
%     object vary as we vary the workload, segmentation, and replication factor?
% \end{itemize}

\subsection{Implementation}
Lucy includes an implementation of the interactive decision procedure described
in \algoref{InteractiveDecisionProcedure}, an implementation of a decision
procedure which checks criteria (1) - (4) from \thmref{LatticeProperty}, and an
implementation of the decision procedure described in
\algoref{ArbitraryStartInteractiveDecisionProcedure}. The decision procedures
are implemented in roughly 2,500 lines of Python. Users specify objects,
transactions, and invariants in a small Python DSL and interact with the
interactive decision procedures using an interactive Python console. We use
Z3~\cite{de2008z3} to implement our invariant-closure decision procedure,
compiling an object and invariant into a formula that is satisfiable if and
only if the object is \emph{not} invariant-closed. If the object is
invariant-closed, then Z3 concludes that the formula is unsatisfiable.
Otherwise, if the object is not invariant-closed, then Z3 produces a
counterexample witnessing the satisfiability of the formula.

Lucy also includes an implementation of the invariant-confluence and
segmented-invariant confluence system models in roughly 3,500 lines of C++.
Users specify objects, transactions, invariants, and segmentations in C++. Lucy
then replicates the objects using segmented invariant-confluence (or
invariant-confluence if the segmentation contains a single segment without any
disallowed transactions). Clients send every transaction request to a randomly
selected server. When a server receives a transaction request, it executes
\algoref{TxnExecution} to attempt to execute the transaction locally. If the
transaction requires global coordination, then the server forwards the
transaction request to a predetermined leader. When the leader receives a
transaction request, it sends a coordination request to all other servers. When
a server receives a coordination request from the leader, it stops processing
transactions and sends the leader its state in a coordination reply. When the
leader receives coordination replies from all other servers, it executes the
transaction, and then sends its state to the other servers. When a server
receives a new state, it adopts the state, computes its new active segment, and
resumes normal processing. After every 100 transactions processed, a server
sends a merge request to a randomly selected server.
% Merge requests are tagged with a monotonically increasing epoch number that is
% incremented by the master after every round of global coordination. This allows
% servers to discard merge requests from previous epochs.

Lucy can also replicate an object with eventual consistency and with
linearizability. With eventual consistency, clients send every transaction
request to a randomly selected server. The server executes the transaction
locally and returns immediately to the client, sending merge requests after
every 100 transactions. With linearizability, clients send every transaction
request to a predetermined leader. The leader relays the transaction request to
all other servers, and when the leader receives replies from them, it executes
the transaction and replies to the client. This communication pattern mimics
the ``normal operation'' of state machine replication protocols
\cite{lamport1998part, liskov2012viewstamped}.
% In \secref{SegmentedInvariantConfluenceEval}, we compare the performance of
% replicating with segmented invariant-confluence against the performance of
% replicating with eventual consistency and linearizability.

Because fault-tolerance is largely an orthogonal concern to
invariant-confluence, Lucy is implemented without fault-tolerance. It would be
straightforward to add fault-tolerance to Lucy, but it would not affect our
discussions or evaluation, so we leave it for future work.
% Moreover, users currently have to specify their workloads in Python (for the
% decision procedures) and C++ (for the runtime). In the future, we plan on
% removing this redundancy.

\subsection{Decision Procedures}
We now evaluate the practicality and efficiency of our decision procedure
prototypes. We begin by demonstrating the decision procedure on a handful of
simple, yet practical examples. We then discuss how our tool can be used to
analyze the TPC-C benchmark.

\example[$\ints^2$]\examplelabel{TwoIntsEval}
We begin with a minimal working example. Consider again our recurring example
of $\ints^2$ from \exampleref{Z2}. The Python code used to describe the object,
transactions, and invariant is given in \figref{Z2Code}. When we call
\python{checker.check()}, the interactive decision procedure produces a
counterexample in less than a tenth of a second.  After we label the
counterexample and refine the invariant with $y \leq 0$, the interactive
decision procedure determines that the object is invariant-confluent, again, in
less than a tenth of a second. Note that the object is invariant-confluent but
\emph{not} invariant-closed, so prior work that relies on invariant-closure
alone to determine invariant-confluence would not be able to identify this
example as invariant-confluent.

\begin{figure}[ht]
  \begin{Python}[gobble=4]
    checker = InteractiveInvariantConfluenceChecker()
    x = checker.int_max('x', 0) # An int, x, merged by max.
    y = checker.int_max('y', 0) # An int, y, merged by max.
    checker.add_transaction('increment_x', [x.assign(x + 1)])
    checker.add_transaction('decrement_y', [y.assign(y - 1)])
    checker.add_invariant(x * y <= 0)
    checker.check()
  \end{Python}
  \caption{\exampleref{TwoIntsEval} specification}\figlabel{Z2Code}
\end{figure}

\example[Foreign Keys]\examplelabel{ForeignKeysEval}
A 2P-Set $X = (A_X, R_X)$ is a set CRDT composed of a set of additions $A_X$
and a set of removals $R_X$~\cite{shapiro2011comprehensive}. We view the state
of the set $X$ as the difference $A_X - R_X$ of the addition and removal sets.
To add an element $x$ to the set, we add $x$ to $A_X$. Similarly, to remove $x$
from the set, we add it to $R_X$. The merge of two 2P-sets is a pairwise union
(i.e. $(A_X, R_X) \join (A_Y, R_Y) = (A_X \cup A_Y, R_X \cup R_Y)$).

We can use 2P-sets to model a simple relational database with foreign key
constraints. Let object $O = (X, Y) = ((A_X, R_X), (A_Y, R_Y))$ consist of a
pair of two 2P-Sets $X$ and $Y$, which we view as relations. Our invariant $X
\subseteq Y$ (i.e. $(A_X - R_X) \subseteq (A_Y - R_Y)$) models a foreign key
constraint from $X$ to $Y$. We ran our decision procedure on the object with
initial state $((\emptyset, \emptyset), (\emptyset, \emptyset))$ and
with transactions that allow arbitrary insertions and deletions into $X$ and
$Y$. After less than a tenth of a second, the decision procedure produced a
reachable counterexample witnessing the fact that the object is not
invariant-confluent. A concurrent insertion into $X$ and deletion from $Y$ can
lead to a state which violates the invariant. This object is not
invariant-confluent and therefore not invariant-closed. Thus, previous tools
depending on invariant-closure as a sufficient condition would be unable to
conclude definitively that the object is \emph{not} invariant-confluent.

We also reran the decision procedure, but this time with insertions into $X$
and deletions from $Y$ disallowed. In less than a tenth of a second, the
decision procedure correctly deduced that the object is now
invariant-confluent. These results were manually proven
in~\cite{bailis2014coordination}, but our tool was able to confirm them
automatically in a negligible amount of time.

\example[Auction]\examplelabel{AuctionEval}
We now consider a simple auction system introduced in~\cite{gotsman2016cause}.
Our object consists of a set $B$ of integer-valued bids and a optional winning
bid $w$. Initially, $B = \emptyset$ and $w = \bot$ (indicating that there is no
winning bid yet) and we merge states by taking the union of $B$ and the maximum
of $w$ (where $\bot < n$ for all integers $n$). One transaction $t_b$ places a bid
$b$ by inserting it into $B$. Another transaction $t_\text{close}$ closes the
auction and sets $w$ equal to the largest bid in $B$. Our invariant is that if
the auction is closed (i.e.\ $w \neq \bot$), then $w = \max(B)$. We ran our
decision procedure on this example and in a third of a second, it produced a
reachable counterexample witnessing the fact that the object is not
invariant-confluent.  If we concurrently close the auction and place a large
bid, then we can end up in a state in which the auction is closed, but there is
a bid in $B$ larger than $w$.

We then segmented our object as follows. The first segment $(\setst{(B, w)}{w =
\bot}, \setst{t_b}{b \in \ints})$ allows bidding so long as the bid is open.
The second segment $(\setst{B, w}{w \neq \bot} \cap I, \emptyset)$ includes all
auctions which have already been closed and forbids all transactions. This
segmentation captures the intuition that bids should be permitted only when the
auction is open. We ran our segmented invariant-confluence decision procedure
on this example, and it was able to deduce without any human interaction that
the example was segmented invariant-confluent in less than a tenth of a second.

\example[Escrow Transactions]\examplelabel{EscrowTransactionsEval}
Escrow transactions are a concurrency control technique that allows a database
to execute transactions that increment and decrement numeric values with more
concurrency than is otherwise possible with general-purpose techniques like
two-phase locking~\cite{o1986escrow}. The main idea is that a portion of the
numeric value is put in escrow, after which a transaction can freely decrement
the value so long as it is not decremented by more than the amount that has
been escrowed. We show how segmented invariant-confluence can be used to
implement escrow transactions.

Consider again the PN-Counter $s = (p_1, p_2, p_3), (n_1, n_2, n_3)$ from
\exampleref{CounreachableExample} replicated on three servers with transactions
to increment and decrement the PN-Counter. In
\exampleref{CounreachableExample}, we found that concurrent decrements violate
invariant-confluence which led us to a segmentation which prohibited concurrent
decrements. We now propose a new segmentation with escrow amount $k$ that will
allow us to perform concurrent decrements that respect the escrowed value. The
first segment $(\setst{(p_1, p_2, p_3), (n_1, n_2, n_3)}{p_1, p_2, p_3 \geq k
\land n_1, n_2, n_3 \leq k}, T)$ allows for concurrent increments and
decrements so long as every $p_i \geq k$ and every $n_i \leq k$. Intuitively,
this segment represents the situation in which every server has escrowed a
value of $k$. They can decrement freely, so long as they don't exceed their
escrow budget of $k$. The second segment is the one presented in
\exampleref{CounreachableExample} which prohibits concurrent decrements. We ran
our decision procedure on this example and it concluded that it was segmented
invariant-confluent in less than a tenth of a second and without any human
interaction.

\example[TPC-C]\examplelabel{TpccEval}
TPC-C is a ubiquitous OLTP database benchmark with a workload that models a
simple warehousing application~\cite{difallah2013oltp}. The TPC-C specification
outlines twelve ``consistency requirements'' (read invariants) that govern the
warehousing application. In~\cite{bailis2014coordination}, Bailis et al.\
categorize the invariants into one of three types:
\begin{enumerate}
  \item
    Three of the twelve invariants are \textbf{foreign key constraints}.  As
    discussed in \exampleref{ForeignKeysEval}, our decision procedures can
    automatically verify conditions under which foreign key constraints are
    invariant-confluent.

  \item
    \newcommand{\ttt}[1]{{\smaller \texttt{#1}}}
    Seven of the twelve invariants involve \textbf{maintaining arithmetic
    relationships between relations}. Our decision procedures can correctly
    identify these as invariant-confluent. Consider, for example, invariant 1
    which dictates that a warehouse's year to date balance \ttt{W\_YTD} is
    equal to the sum of the district year to date balances \ttt{D\_YTD} of the
    twenty districts that are associated with the warehouse. The Payment
    transaction randomly selects a district and increments \ttt{W\_YTD} and
    \ttt{D\_YTD} by a randomly generated amount. We model this workload with a
    PN-Counter for \ttt{W\_YTD} and twenty PN-Counters for the twenty instances
    of \ttt{D\_YTD}. We applied Lucy to this workload, and it determined that
    the workload was invariant-confluent in less than a second.

  \item
    Two of the twelve invariants involve generating \textbf{sequential and
    unique identifiers}. This workload is \emph{not} invariant-confluent, but
    when the sequentiality requirement is dropped, it is invariant-confluent.
    Unfortunately, modelling unique key generation in our theoretical framework
    (and in our prototype implementation) is not possible because we have thus
    far tacitly assumed that transaction execution is deterministic. We leave a
    generalization of our theory to non-deterministic transactions for future
    work.
\end{enumerate}

\subsection{Segmented Invariant Confluence}%
\seclabel{SegmentedInvariantConfluenceEval}
Now, we evaluate the performance of replicating an object with segmented
invariant-confluence as compared to replicating it with eventual consistency or
linearizability.
% We begin with two benchmarks that demonstrate the same concept: the performance
% of segmented invariant-confluent replication varies with the amount of global
% coordination induced by either (a) performing a transaction that is disallowed
% within a segment or (b) transitioning between segments.

\begin{figure}[ht]
  \centering

  \begin{subfigure}[c]{\columnwidth}
    \centering
    \includegraphics[width=\columnwidth]{figures/vary_withdraws.pdf}
  \end{subfigure}
  \begin{subfigure}[c]{\columnwidth}
    \includegraphics[width=\columnwidth]{figures/vary_segments.pdf}
  \end{subfigure}

  \caption{%
    Segmented invariant-confluent replication throughput versus coordination,
    induced by executing disallowed transactions (top) and by transitioning
    across segments (bottom).
  }\figlabel{ThroughputVsGlobalSyncs}
\end{figure}

\begin{benchmark}\benchlabel{VaryWithdraws}
Consider again the PN-Counter from \exampleref{CounreachableExample} and the
corresponding transactions, invariants, and single-segment segmentation that
forbids concurrent decrements. We replicate this object on 32 servers
deployed on 32 m5.xlarge EC2 instances within the same availability zone.  Each
server has three colocated clients that issue deposit and withdrawal
transactions. We replicate the object with eventual consistency, segmented
invariant-confluence, and linearizability and measure the system's total
throughput as we vary the fraction of client requests that are withdrawals. The
results are shown in the top of \figref{ThroughputVsGlobalSyncs}.

Both eventually consistent replication and linearizable replication are
unaffected by the workload, achieving roughly 700,000 and 7,000 transactions
per second respectively.
%
% Expectedly, eventually consistent replication significantly outperforms
% linearizable replication because (a) transactions can be sent to any server
% (not just the leader) and (b) servers do not coordinate with each other at all.
%
Segmented invariant-confluent replication performs well for low-withdrawal
workloads and performs increasingly poorly as we increase the fraction of
withdrawal transactions, eventually performing worse than linearizable
replication. For example, with 5\% withdrawal transactions, segmented
invariant-confluent replication performs an order of magnitude better than
linearizable replication; with 50\% withdrawals, it performs as well; and with
100\% withdrawals, it performs two times worse.

These results are expected. Deposit transactions can execute without any
coordination while withdrawal transactions require global coordination. As we
increase the fraction of withdrawals, we increase the amount of coordination
that the system has to perform which in turn drastically decreases the
throughput. These results also offer two insights:
%
First, for low-withdrawal workloads, segmented invariant-confluent replication
achieves a compromise between strong and weak consistency. It guarantees that
invariants are maintained (which is impossible with eventual consistency if the
object is not invariant-confluent) with performance many times better than
strongly consistent replication.
%
Second, segmented invariant-confluent replication is poorly suited to workloads
that require a large amount of coordination. For workloads without much inherit
concurrency (e.g.\ withdraw-mostly workloads), maintaining invariants is best
done with strong consistency. It provides stronger guarantees with better
performance.
\end{benchmark}

\begin{benchmark}\benchlabel{VarySegmentLength}
  Consider again the object, transactions, and invariants from \exampleref{Z2}
  and \exampleref{SegmentedZ2}. As with \benchref{VaryWithdraws}, we replicate
  the object across 32 servers. Clients issue 50\% increment $x$ transactions,
  and 50\% decrement $y$ transactions. We consider a ``checkerboard''
  segmentation $\Sigma_n = \setst{(I_{i, j}, T)}{i, j \in \ints}$ where segment
  invariant $I_{i, j}$ consists of the square of points $\setst{(x, y)}{ni \leq
  x < n(i + 1), nj \leq y < n(j + 1)}$ with side length $n$. For example,
  $\Sigma_1$ places each point in its own segment, $\Sigma_2$ tessellates
  $\ints^2$ with 2x2 squares, $\Sigma_3$ tessellates $\ints^2$ with 3x3
  squares, and so on. We measure the throughput of the object replicated with
  eventually consistent, segmented invariant-confluent, and linearizable
  replication as we vary the segment side length $n$. The results are shown in
  the bottom \figref{ThroughputVsGlobalSyncs}.

  This benchmark tells the same tale as \benchref{VaryWithdraws}. Eventual
  consistency and linearizability are unaffected by workload, and eventual
  consistency outperforms linearizability by roughly two orders of magnitude.
  In this example, the segmented invariant-confluent replication only requires
  coordination when transitioning between segment boundaries, so as we increase
  the segment side length, the throughput of the system increases
  significantly.
\end{benchmark}

\begin{figure}[ht]
  \centering
  \includegraphics[width=\columnwidth]{figures/vary_nodes.pdf}
  \caption{%
    Throughput of eventually consistent, segmented invariant-confluent, and
    linearizable replication measured against the number of
    nodes.
  }\figlabel{VaryNodes}
\end{figure}

\begin{benchmark}
  In this benchmark, we measure the scale-out of segmented invariant-confluent
  replication. We repeat \benchref{VaryWithdraws} with a 10\% withdrawal rate,
  but this time we vary the number of servers we use to replicate our object.
  When we replicate with $n$ servers, we use $3n$ clients (the $3$ colocated
  clients on each server) as part of the workload. The results are shown in
  \figref{VaryNodes}.

  Eventually consistent replication scales perfectly with the number of nodes,
  confirming the results in~\cite{bailis2014coordination}.
  %
  % With eventually consistent replication, servers do not coordinate at all, so
  % they are completely unaffected by the number of servers.
  %
  Linearizable replication, on the other hand, scales up to about 3-5 servers
  before performance begins to decrease. These numbers are consistent with
  typical deployments of state-machine replication protocols like
  Paxos~\cite{chandra2007paxos}.
  %
  % Because all messages are sent to the leader, the leader becomes the
  % bottleneck as the number of servers and clients increases. Moreover, the
  % leader must wait for responses from more servers, increasing the latency of
  % the slowest response which in turn decreases throughput.
  %
  Segmented invariant-confluent replication scales up to about 6-8 servers
  before succumbing to the same scalability bottlenecks as linearizable
  replication.

  These results highlight the importance of coordination avoidance in
  distributed databases. While segmented invariant-confluent replication scales
  out slightly better than linearizable replication, both scale significantly
  worse than eventually consistent replication even for a very low (i.e.\ 10\%)
  withdrawal workload. This demonstrates that even a small amount of
  coordination can significantly reduce the scalability of a system.
\end{benchmark}
}
\end{document}
}
{\begin{frame}
  \Huge
  \begin{center}
    Goal: develop an invariant-confluence decision procedure.
  \end{center}

  \note{%
    Now that we've defined invariant confluence, we can turn our attention to
    the main goal of our paper. And that is to develop an invariant-confluence
    decision procedure. As you saw in the previos example, determining whether
    or not a distributed object is invariant confluent is not easy to do by
    hand, so we'd like to develop a decision procedure that can automatically
    check whether something is invariant confluent for us.
  }
\end{frame}

\begin{frame}
  \begin{center}
    \begin{tikzpicture}[xscale=3, yscale=2]
      \node (confluence) at (2, 0) {\Large invariant confluence};
      \pause

      \node (hard) at (2, 1) {hard to check :(};
      \draw[-latex, thick] (hard) to (confluence);
      \pause

      \node (closure) at (0, 0) {\Large invariant closure};
      \node at ($(confluence)!0.5!(closure)$) {\Large $\implies$};
      \pause

      \node (easy) at (0, 1) {easy to check :)};
      \draw[-latex, thick] (easy) to (closure);
    \end{tikzpicture}
  \end{center}

  \note{%
    Unfortunately, developing an invariant confluence decision procedure
    straight up is not easy. Invariant confluence is fundamentally a property
    about reachable states but reasoning automatically about reachable states
    is hard. \\[12pt]

    Instead of reasoning about invariant confluence directly then, we'll look
    at a sufficient condition for invariant confluence called invariant
    closure. \\[12pt]

    Unlike invariant confluence, we'll see that invariant closure is easy to
    check. \\[12pt]
  }
\end{frame}

\begin{frame}
  \Large
  We say a set $S$ is \defword{closed under $f$} if for every $x, y \in S, f(x,
  y) \in S$. \pause For example,
  \begin{itemize}
    \item Even numbers are closed under addition.
    \item Odd numbers are \emph{not} closed under addition.
  \end{itemize}

  \note{%
    First, a quick refresher on what it meas for a set to be closed. We say a
    set $S$ is closed under a binary operator $f$ if for every $x$ and $y$ in
    $S$, $f(x, y)$ is also in $S$. For example, even numbers are closed under
    addition because the sum of any two even numbers is even. But odd numbers
    are not closed under addition because the sum of two odd numbers may not
    be odd.
  }
\end{frame}

\begin{frame}
  \Large
  \begin{itemize}
    \item
      $O = (S, \join)$ is \defword{\invariantclosed{}} with respect to an
      invariant $I$ if invariant satisfying states are closed under merge.
    \pause\item
      For every state $s_1, s_2 \in S$, if $I(s_1)$ and $I(s_2)$, then $I(s_1
      \join s_2)$.
  \end{itemize}

  \note{%
    We say that an object $O$ is invariant closed if invariant satisfying
    states are closed under merge. That is, if an object is invariant closed,
    then for every pair of states $s_1$ and $s_2$, if $s_1$ and $s_2$ satisfy
    the invariant then so does $s_1 \join s_2$.
  }
\end{frame}

\begin{frame}
  \Large
  \[
    \text{invariant closure} \implies \text{invariant confluence}
  \]

  \pause

  Why?

  \pause
  Transactions maintain the invariant. If merging does as well, then the
  invariant is always maintained.

  \note{%
    As I mentioned earlier, invariant closure is a sufficient condition for
    invariant confluence. I'll refer you to the paper for the proof, but it's a
    very simple proof.
  }
\end{frame}

\begin{frame}
  \Large
  Checking invariant closure is more straightforward.

  \vspace{0.5in}
  \pause

  \begin{tikzpicture}[xscale=6]
    \node[draw] (formula) at (0, 0) {
      $
      \begin{aligned}
        \forall x_1, & y_1, x_2, y_2.\, \\
        \quad & x_1y_1 \leq 0 \land x_2y_2 \leq 0 \implies \\
        \quad & \max(x_1, x_2)\max(y_1, y_2) \leq 0
      \end{aligned}
      $
    };
    \pause
    \node (smt) at (1, 0) {SMT Solver};
    \draw[-latex, ultra thick] (formula) to (smt);
  \end{tikzpicture}

  \note{%
    The good thing about invariant closure is that it's much easier to check
    automatically. Remember that example we had with pairs of
    integers? We can pose whether or not that object is invariant closed as
    this formula, and we can pass this formula directly to an SMT solver to
    figure out whether it's invariant closed.
  }
\end{frame}

\begin{frame}
  \begin{center}
    \begin{tikzpicture}[xscale=3, yscale=2]
      \node (confluence) at (2, 0) {\Large invariant confluence};
      \node (closure) at (0, 0) {\Large invariant closure};
      \node at ($(confluence)!0.5!(closure)$) {\Large $\implies$};
      \node (hard) at (2, 1) {hard to check :(};
      \draw[-latex, thick] (hard) to (confluence);
      \node (easy) at (0, 1) {easy to check :)};
      \draw[-latex, thick] (easy) to (closure);
    \end{tikzpicture}
  \end{center}

  \note{%
    To decide whether an object is invariant confluent, then, we can take the
    object and ask if it's invariant closed. If it is, then it's also invariant
    confluent and we're done. If it's not invariant closed, then what do we
    know?
  }
\end{frame}

\begin{frame}
  \Large
  \[
    \text{invariant closure}
    \xLeftarrow{\phantom{aa}?\phantom{aa}}
    \text{invariant confluence}
  \]

  \note{%
    Well, that depends on whether invariant confluence implies invariant
    closure.
  }
\end{frame}

\newcommand{\xmin}{-2}
\newcommand{\xmax}{2}
\newcommand{\ymin}{-2}
\newcommand{\ymax}{2}

% Axes.
\newcommand{\xyaxes}{
  \draw[] (\xmin.5, 0) to (\xmax.5, 0);
  \draw[] (0, \ymin.5) to (0, \ymax.5);
  \node at (\xmax + 1, 0) {$x$};
  \node at (0, \ymax + 1) {$y$};
}

% Quadrant 1.
\newcommand{\quadi}[5]{{
  \newcommand{\argstyle}{#1}
  \newcommand{\argxmin}{#2}
  \newcommand{\argxmax}{#3}
  \newcommand{\argymin}{#4}
  \newcommand{\argymax}{#5}
  \foreach \x in {0, ..., \argxmax} {
    \foreach \y in {0, ..., \argymax} {
      \node[\argstyle] (\x-\y) at (\x, \y) {};
    }
  }
}}

% Quadrant 2.
\newcommand{\quadii}[5]{{
  \newcommand{\argstyle}{#1}
  \newcommand{\argxmin}{#2}
  \newcommand{\argxmax}{#3}
  \newcommand{\argymin}{#4}
  \newcommand{\argymax}{#5}
  \foreach \x in {\argxmin, ..., 0} {
    \foreach \y in {0, ..., \argymax} {
      \node[\argstyle] (\x-\y) at (\x, \y) {};
    }
  }
}}

% Quadrant 3.
\newcommand{\quadiii}[5]{{
  \newcommand{\argstyle}{#1}
  \newcommand{\argxmin}{#2}
  \newcommand{\argxmax}{#3}
  \newcommand{\argymin}{#4}
  \newcommand{\argymax}{#5}
  \foreach \x in {\argxmin, ..., 0} {
    \foreach \y in {\argymin, ..., 0} {
      \node[\argstyle] (\x-\y) at (\x, \y) {};
    }
  }
}}

% Quadrant 4.
\newcommand{\quadiv}[5]{{
  \newcommand{\argstyle}{#1}
  \newcommand{\argxmin}{#2}
  \newcommand{\argxmax}{#3}
  \newcommand{\argymin}{#4}
  \newcommand{\argymax}{#5}
  \foreach \x in {0, ..., \argxmax} {
    \foreach \y in {\argymin, ..., 0} {
      \node[\argstyle] (\x-\y) at (\x, \y) {};
    }
  }
}}

% State labels.
\newcommand{\statelabels}{
  \node[statelabel] at (0, 0) {$s_0$};
  \node[statelabel] at (-1, 1) {$s_1$};
  \node[statelabel] at (1, -1) {$s_2$};
  \node[statelabel] at (1, 1) {$s_3$};
}

\tikzstyle{point}=[shape=circle, fill=flatgray, inner sep=3pt]
\tikzstyle{inv}=[line width=0.75pt, draw=black]
\tikzstyle{pointinv}=[point, inv]
\tikzstyle{invregion}=[rounded corners, fill=flatgreen!50, draw=none]
\tikzstyle{reachableregion}=[rounded corners, fill=flatblue!50, draw=none]
\tikzstyle{statelabel}=[anchor=south west, inner sep=1pt]



\begin{frame}
  \begin{columns}
    \begin{column}{0.5\textwidth}
      \centering
      \begin{tikzpicture}[scale=1]
        \begin{scope}
          \clip (\xmin.5, \ymax.5) rectangle (\xmax.5, \ymin.5);
          \draw[invregion] (\xmin.9, \ymax.9) rectangle (0.5, -0.5);
          \draw[invregion] (-0.5, 0.5) rectangle (\xmax.9, \ymin.9);
          \draw (0.5, 0.5) to (0.5, \ymax.5);
          \draw (0.5, 0.5) to (\xmax.5, 0.5);
          \draw (-0.5, -0.5) to (-0.5, \ymin.5);
          \draw (-0.5, -0.5) to (\xmin.5, -0.5);
        \end{scope}

        \xyaxes{}
        \quadi{point}{\xmin}{\xmax}{\ymin}{\ymax}
        \quadiii{point}{\xmin}{\xmax}{\ymin}{\ymax}
        \quadii{pointinv}{\xmin}{\xmax}{\ymin}{\ymax}
        \quadiv{pointinv}{\xmin}{\xmax}{\ymin}{\ymax}
      \end{tikzpicture}

      {\Huge Invariant}
    \end{column}
    \begin{column}{0.5\textwidth}
      \centering
      \begin{tikzpicture}[scale=1]
        \begin{scope}
          \clip (-1, 1) rectangle (\xmax.5, \ymin.5);
          \draw[reachableregion, draw=black] (-0.5, 0.5) rectangle (\xmax.9, \ymin.9);
        \end{scope}

        \xyaxes{}
        \quadi{point}{\xmin}{\xmax}{\ymin}{\ymax}
        \quadiii{point}{\xmin}{\xmax}{\ymin}{\ymax}
        \quadii{point}{\xmin}{\xmax}{\ymin}{\ymax}
        \quadiv{pointinv}{\xmin}{\xmax}{\ymin}{\ymax}
      \end{tikzpicture}

      {\Huge Reachable}
    \end{column}
  \end{columns}

  \note{%
    To see if it does, let's revisit the example from before. Recall that in
    this example, our object is invariant confluent. The set of reachable
    states is a subset of the invariant.
  }
\end{frame}

\begin{frame}
  \begin{columns}
    \begin{column}{0.5\textwidth}
      \centering
      \begin{tikzpicture}[scale=1]
        \begin{scope}
          \clip (\xmin.5, \ymax.5) rectangle (\xmax.5, \ymin.5);
          \draw[invregion] (\xmin.9, \ymax.9) rectangle (0.5, -0.5);
          \draw[invregion] (-0.5, 0.5) rectangle (\xmax.9, \ymin.9);
          \draw (0.5, 0.5) to (0.5, \ymax.5);
          \draw (0.5, 0.5) to (\xmax.5, 0.5);
          \draw (-0.5, -0.5) to (-0.5, \ymin.5);
          \draw (-0.5, -0.5) to (\xmin.5, -0.5);
        \end{scope}

        \xyaxes{}
        \quadi{point}{\xmin}{\xmax}{\ymin}{\ymax}
        \quadiii{point}{\xmin}{\xmax}{\ymin}{\ymax}
        \quadii{pointinv}{\xmin}{\xmax}{\ymin}{\ymax}
        \quadiv{pointinv}{\xmin}{\xmax}{\ymin}{\ymax}

        \draw[-latex, ultra thick] (-1, 2) to (2, 2);
        \draw[-latex, ultra thick] (2, -1) to (2, 2);
      \end{tikzpicture}

      {\Huge Invariant}
    \end{column}
    \begin{column}{0.5\textwidth}
      \centering
      \begin{tikzpicture}[scale=1]
        \begin{scope}
          \clip (-1, 1) rectangle (\xmax.5, \ymin.5);
          \draw[reachableregion, draw=black] (-0.5, 0.5) rectangle (\xmax.9, \ymin.9);
        \end{scope}

        \xyaxes{}
        \quadi{point}{\xmin}{\xmax}{\ymin}{\ymax}
        \quadiii{point}{\xmin}{\xmax}{\ymin}{\ymax}
        \quadii{point}{\xmin}{\xmax}{\ymin}{\ymax}
        \quadiv{pointinv}{\xmin}{\xmax}{\ymin}{\ymax}
      \end{tikzpicture}

      {\Huge Reachable}
    \end{column}
  \end{columns}

  \note{%
    But, note that our object is not invariant closed. The points $(-1, 2)$ and
    $(2, -1)$ both satisfy the invariant, but if we merge them we get the point
    $(2, 2)$, and the point $(2, 2)$ does not satisfy the invariant.
  }
\end{frame}

\begin{frame}
  \begin{center}
    \begin{tikzpicture}[xscale=3, yscale=2]
      \node (confluence) at (2, 0) {\Large invariant confluence};
      \node (closure) at (0, 0) {\Large invariant closure};
      \node at ($(confluence)!0.5!(closure)$) {\Large $\centernot\impliedby$};
      \node (hard) at (2, 1) {hard to check :(};
      \draw[-latex, thick] (hard) to (confluence);
      \node (easy) at (0, 1) {easy to check :)};
      \draw[-latex, thick] (easy) to (closure);
    \end{tikzpicture}
  \end{center}

  \note{%
    Thus, our object is invariant confluent but not invariant closed, so
    invariant confluence does not imply invariant closure. This is unfortunate.
    We can check for invariant closure but not invariant confluence, so we'd
    like the two to be equivalent. \\[12pt]

    Are there any situations in which the two do happen to be equivalent?
    \\[12pt]
  }
\end{frame}

\begin{frame}
  \begin{columns}
    \begin{column}{0.5\textwidth}
      \centering
      \begin{tikzpicture}[scale=1]
        \begin{scope}
          \clip (\xmin.5, \ymax.5) rectangle (\xmax.5, \ymin.5);
          \draw[invregion] (\xmin.9, \ymax.9) rectangle (0.5, -0.5);
          \draw[invregion] (-0.5, 0.5) rectangle (\xmax.9, \ymin.9);
          \draw (0.5, 0.5) to (0.5, \ymax.5);
          \draw (0.5, 0.5) to (\xmax.5, 0.5);
          \draw (-0.5, -0.5) to (-0.5, \ymin.5);
          \draw (-0.5, -0.5) to (\xmin.5, -0.5);
        \end{scope}

        \xyaxes{}
        \quadi{point}{\xmin}{\xmax}{\ymin}{\ymax}
        \quadiii{point}{\xmin}{\xmax}{\ymin}{\ymax}
        \quadii{pointinv}{\xmin}{\xmax}{\ymin}{\ymax}
        \quadiv{pointinv}{\xmin}{\xmax}{\ymin}{\ymax}

        \draw[-latex, ultra thick] (-1, 2) to (2, 2);
        \draw[-latex, ultra thick] (2, -1) to (2, 2);
      \end{tikzpicture}

      {\Huge Invariant}
    \end{column}
    \begin{column}{0.5\textwidth}
      \centering
      \begin{tikzpicture}[scale=1]
        \begin{scope}
          \clip (-1, 1) rectangle (\xmax.5, \ymin.5);
          \draw[reachableregion, draw=black] (-0.5, 0.5) rectangle (\xmax.9, \ymin.9);
        \end{scope}

        \xyaxes{}
        \quadi{point}{\xmin}{\xmax}{\ymin}{\ymax}
        \quadiii{point}{\xmin}{\xmax}{\ymin}{\ymax}
        \quadii{point}{\xmin}{\xmax}{\ymin}{\ymax}
        \quadiv{pointinv}{\xmin}{\xmax}{\ymin}{\ymax}
      \end{tikzpicture}

      {\Huge Reachable}
    \end{column}
  \end{columns}

  \note{%
    Well, let's take a look at our example again. We noticed that our object is
    not invariant closed because we can merge $(-1, 2)$ and $(2, -1)$ to
    violate the invariant. But, notice that the point $(-1, 2)$ is not
    reachable.
  }
\end{frame}

\begin{frame}
  \Large
  If $I$ is a subset of reachable states, then
  \[
    \text{invariant closure} \iff \text{invariant confluence}
  \]

  \pause

  Why?

  \pause

  Forward direction is the same as before. For backwards direction, $I$ and
  reachable states are equal. Reachable states are closed under merge, so
  therefore so is $I$.

  \note{%
    Turns out, this is not a coincidence. An invariant confluent object may not
    be invariant closed but only because we're merging points that are
    unreachable. If the invariant is a subset of the reachable states---that
    is, if every invariant satisfying point is reachable---then invariant
    closure and invariant confluence are equivalent. \\[12pt]

    Again, I'll refer you to the paper for a proof. \\[12pt]
  }
\end{frame}

\begin{frame}
  \Large
  If $I$ is a subset of reachable states, then
  \begin{center}
    \begin{tikzpicture}[xscale=3, yscale=2]
      \node (confluence) at (2, 0) {\Large invariant confluence};
      \node (closure) at (0, 0) {\Large invariant closure};
      \node at ($(confluence)!0.5!(closure)$) {\Large $\iff$};
      \node (hard) at (2, 1) {hard to check :(};
      \draw[-latex, thick] (hard) to (confluence);
      \node (easy) at (0, 1) {easy to check :)};
      \draw[-latex, thick] (easy) to (closure);
    \end{tikzpicture}
  \end{center}

  \note{%
    This is good news. Invariant confluence is hard to check, but invariant
    closure is easy to check. And when the invariant is a subset of the
    reachable states, the two are equivalent. In this case, invariant
    confluence also becomes easy to check.
  }
\end{frame}
}
{\begin{frame}
  \begin{center}
    \Huge
    Main idea: prune the invariant until it's a subset of reachable states.
    Then check for invariant closure.
  \end{center}
\end{frame}

\newcommand{\algocomment}[1]{\State \textcolor{flatdenim}{\texttt{//} #1}}

\begin{frame}
  \begin{algorithmic}
    \algocomment{Return if $O$ is \sTIconfluent{}.}
    \Function{IsInvConfluent}{$O$, $s_0$, $T$, $I$}
      \State
      \Return $I(s_0)$ and
      \Call{Helper}{$O$, $s_0$, $T$, $I$, $\set{s_0}$, $\emptyset$}
    \EndFunction

    \State

    \algocomment{$R$ is a set of \sTIreachable{} states.}
    \algocomment{$NR$ is a set of \sTIunreachable{} states.}
    \algocomment{$I(s_0)$ is a precondition.}
    \Function{Helper}{$O$, $s_0$, $T$, $I$, $R$, $NR$}
      \State closed, $s_1$, $s_2$ $\gets$ \Call{IsIclosed}{$O$, $I - NR$}
      \If {closed}
        \Return true
      \EndIf
      \State Augment $R, NR$ with random search and user input
      \If{$s_1, s_2 \in R$}
        \Return false
      \EndIf
      \State \Return \Call{Helper}{$O$, $s_0$, $T$, $I$, $R$, $NR$}
    \EndFunction
  \end{algorithmic}
\end{frame}

\tikzstyle{point}=[shape=circle, fill=flatgray, inner sep=2pt, draw=black]
\tikzstyle{reachable}=[fill=flatgreen!50, draw=none]
\tikzstyle{unreachable}=[fill=flatred!50, draw=none]
\tikzstyle{invariant}=[fill=flatblue!50, draw=none]
\tikzstyle{nothing}=[fill=white, draw=none]

\newcommand{\pointgrid}[4]{{
  \newcommand{\argxmin}{#1}
  \newcommand{\argxmax}{#2}
  \newcommand{\argymin}{#3}
  \newcommand{\argymax}{#4}

  \draw[] (\argxmin, 0) to (\argxmax, 0);
  \draw[] (0, \argymin) to (0, \argymax);
  \foreach \x in {\argxmin, ..., \argxmax} {
    \foreach \y in {\argymin, ..., \argymax} {
      \node[point] (\x-\y) at (\x, \y) {};
    }
  }
}}

\newcommand{\pointrect}[2]{
  \draw[#1] ($#2 + (-0.51, 0.51)$) rectangle ($#2 + (0.51, -0.51)$);
}

\newcommand{\subfigwidth}{0.48\textwidth}
\newcommand{\subfighspace}{0.3cm}
\newcommand{\tikzhspace}{0.4cm}
\newcommand{\tikzscale}{0.5}
\newcommand{\xmin}{-3}
\newcommand{\xmax}{3}
\newcommand{\ymin}{-3}
\newcommand{\ymax}{3}

\begin{frame}
  \begin{columns}
    \begin{column}{0.3\textwidth}
      \centering
      \begin{tikzpicture}[scale=\tikzscale]
        \draw[reachable, draw=black] (-0.5, 0.5) rectangle (0.5, -0.5);
        \pointgrid{\xmin}{\xmax}{\ymin}{\ymax}
        \node at (0, \ymax + 1) {$R$};
      \end{tikzpicture}
    \end{column}
    \begin{column}{0.3\textwidth}
      \centering
      \begin{tikzpicture}[scale=\tikzscale]
        \pointgrid{\xmin}{\xmax}{\ymin}{\ymax}
        \node at (0, \ymax + 1) {$NR$};
      \end{tikzpicture}
    \end{column}
    \begin{column}{0.3\textwidth}
      \centering
      \begin{tikzpicture}[scale=\tikzscale]
        \draw[invariant] (\xmin.5, \ymax.5) rectangle (0.5, -0.5);
        \draw[invariant] (-0.5, 0.5) rectangle (\xmax.5, \ymin.5);
        \draw (0.5, 0.5) to (0.5, \ymax.5);
        \draw (0.5, 0.5) to (\xmax.5, 0.5);
        \draw (-0.5, -0.5) to (-0.5, \ymin.5);
        \draw (-0.5, -0.5) to (\xmin.5, -0.5);
        \pointgrid{\xmin}{\xmax}{\ymin}{\ymax}
        \node at (0, \ymax + 1) {$I - NR$};

        \pause
        \node[fill opacity=0.75, text opacity=1, fill=white] at (-1, 1) {$s_1$};
        \node[fill opacity=0.75, text opacity=1, fill=white] at (1, -1) {$s_2$};
      \end{tikzpicture}
    \end{column}
  \end{columns}
\end{frame}

\begin{frame}
  \begin{columns}
    \begin{column}{0.3\textwidth}
      \begin{tikzpicture}[scale=\tikzscale]
        \draw[reachable, draw=black] (-0.5, 0.5) rectangle (1.5, -1.5);
        \pointgrid{\xmin}{\xmax}{\ymin}{\ymax}
        \node at (0, \ymax + 1) {$R$};
      \end{tikzpicture}
    \end{column}
    \begin{column}{0.3\textwidth}
      \begin{tikzpicture}[scale=\tikzscale]
        \pointrect{unreachable, draw=black}{(-1, 1)}
        \pointgrid{\xmin}{\xmax}{\ymin}{\ymax}
        \node at (0, \ymax + 1) {$NR$};
      \end{tikzpicture}
    \end{column}
    \begin{column}{0.3\textwidth}
      \begin{tikzpicture}[scale=\tikzscale]
        \draw[invariant] (\xmin.5, \ymax.5) rectangle (0.5, -0.5);
        \draw[invariant] (-0.5, 0.5) rectangle (\xmax.5, \ymin.5);
        \draw (0.5, 0.5) to (0.5, \ymax.5);
        \draw (0.5, 0.5) to (\xmax.5, 0.5);
        \draw (-0.5, -0.5) to (-0.5, \ymin.5);
        \draw (-0.5, -0.5) to (\xmin.5, -0.5);
        \pointrect{nothing, draw=black}{(-1, 1)}
        \pointgrid{\xmin}{\xmax}{\ymin}{\ymax}
        \node at (0, \ymax + 1) {$I - NR$};

        \pause
        \node[fill opacity=0.75, text opacity=1, fill=white] at (-1, 2) {$s_1$};
        \node[fill opacity=0.75, text opacity=1, fill=white] at (3, -3) {$s_2$};
      \end{tikzpicture}
    \end{column}
  \end{columns}
\end{frame}

\begin{frame}
  \begin{columns}
    \begin{column}{0.3\textwidth}
      \begin{tikzpicture}[scale=\tikzscale]
        \draw[reachable, draw=black] (-0.5, 0.5) rectangle (2.5, -2.5);
        \pointrect{reachable, draw=black}{(3, -3)}
        \pointgrid{\xmin}{\xmax}{\ymin}{\ymax}
        \node at (0, \ymax + 1) {$R$};
      \end{tikzpicture}
    \end{column}
    \begin{column}{0.3\textwidth}
      \begin{tikzpicture}[scale=\tikzscale]
        \pointrect{unreachable}{(-1, 1)}
        \pointrect{unreachable}{(-1, 2)}
        \draw (-1.5, 0.5) rectangle (-0.5, 2.5);
        \pointgrid{\xmin}{\xmax}{\ymin}{\ymax}
        \node at (0, \ymax + 1) {$NR$};
      \end{tikzpicture}
    \end{column}
    \begin{column}{0.3\textwidth}
      \begin{tikzpicture}[scale=\tikzscale]
        \draw[invariant] (\xmin.5, \ymax.5) rectangle (0.5, -0.5);
        \draw[invariant] (-0.5, 0.5) rectangle (\xmax.5, \ymin.5);
        \draw (0.5, 0.5) to (0.5, \ymax.5);
        \draw (0.5, 0.5) to (\xmax.5, 0.5);
        \draw (-0.5, -0.5) to (-0.5, \ymin.5);
        \draw (-0.5, -0.5) to (\xmin.5, -0.5);
        \pointrect{nothing}{(-1, 1)}
        \pointrect{nothing}{(-1, 2)}
        \draw (-1.5, 0.5) rectangle (-0.5, 2.5);
        \pointgrid{\xmin}{\xmax}{\ymin}{\ymax}
        \node at (0, \ymax + 1) {$I - NR$};

        \pause
        \node[fill opacity=0.75, text opacity=1, fill=white] at (-2, 1) {$s_1$};
        \node[fill opacity=0.75, text opacity=1, fill=white] at (1, -1) {$s_2$};
      \end{tikzpicture}
    \end{column}
  \end{columns}
\end{frame}

\begin{frame}
  \begin{columns}
    \begin{column}{0.3\textwidth}
      \begin{tikzpicture}[scale=\tikzscale]
        \draw[reachable] (-0.5, 0.5) rectangle (2.5, -2.5);
        \pointrect{reachable}{(3, 0)}
        \pointrect{reachable}{(3, -1)}
        \pointrect{reachable}{(0, -3)}
        \pointrect{reachable}{(2, -3)}
        \pointrect{reachable}{(3, -3)}
        \draw (-0.5, 0.5) to (\xmax.5, 0.5);
        \draw (-0.5, 0.5) to (-0.5, \ymin.5);
        \draw (-0.5, -3.5) to ++(1, 0)
        to ++(0, 1)
        to ++(1, 0)
        to ++(0, -1)
        to ++(2, 0)
        to ++(0, 1)
        to ++(-1, 0)
        to ++(0, 1)
        to ++(1, 0)
        to ++(0, 2);
        \pointgrid{\xmin}{\xmax}{\ymin}{\ymax}
        \node at (0, \ymax + 1) {$R$};
      \end{tikzpicture}
    \end{column}
    \begin{column}{0.3\textwidth}
      \begin{tikzpicture}[scale=\tikzscale]
        \draw[unreachable] (\xmin.5, \ymax.5) rectangle (-0.5, \ymin.5);
        \draw (\xmin.5, \ymax.5) to (-0.5, \ymax.5) to
        (-0.5, \ymin.5) to (\xmin.5, \ymin.5);
        \pointgrid{\xmin}{\xmax}{\ymin}{\ymax}
        \node at (0, \ymax + 1) {$NR$};
      \end{tikzpicture}
    \end{column}
    \begin{column}{0.3\textwidth}
      \begin{tikzpicture}[scale=\tikzscale]
        \draw[invariant] (-0.5, 0.5) rectangle (\xmax.5, \ymin.5);
        \draw (-0.5, 0.5) to (\xmax.5, 0.5);
        \draw (-0.5, 0.5) to (-0.5, \ymin.5);
        \pointgrid{\xmin}{\xmax}{\ymin}{\ymax}
        \node at (0, \ymax + 1) {$I - NR$};
      \end{tikzpicture}
    \end{column}
  \end{columns}
\end{frame}

\begin{frame}
  \begin{algorithmic}
    \algocomment{Return if $O$ is \sTIconfluent{}.}
    \Function{IsInvConfluent}{$O$, $s_0$, $T$, $I$}
      \State
      \Return $I(s_0)$ and
      \Call{Helper}{$O$, $s_0$, $T$, $I$, $\set{s_0}$, $\emptyset$}
    \EndFunction

    \State

    \algocomment{$R$ is a set of \sTIreachable{} states.}
    \algocomment{$NR$ is a set of \sTIunreachable{} states.}
    \algocomment{$I(s_0)$ is a precondition.}
    \Function{Helper}{$O$, $s_0$, $T$, $I$, $R$, $NR$}
      \State closed, $s_1$, $s_2$ $\gets$ \Call{IsIclosed}{$O$, $I - NR$}
      \If {closed}
        \Return true
      \EndIf
      \State Augment $R, NR$ with random search and user input
      \If{$s_1, s_2 \in R$}
        \Return false
      \EndIf
      \State \Return \Call{Helper}{$O$, $s_0$, $T$, $I$, $R$, $NR$}
    \EndFunction
  \end{algorithmic}
\end{frame}
}
{\begin{frame}
  \begin{enumerate}
    \item
      Use the interactive decision procedure to check if object is invariant
      confluent.
    \pause\item
      If it is, deploy it with weak consistency.
    \pause\item
      If it's not, then...? \pause deploy with strong consistency?
  \end{enumerate}
\end{frame}

\newcommand{\dy}{0.2}
\begin{frame}
  \begin{center}
    \begin{tikzpicture}
      \draw[ultra thick, latex-latex] (0, 0) to (8, 0);
      \draw[thick] (1, -\dy) to (1, \dy);
      \draw[thick] (4, -\dy) to (4, \dy);
      \draw[thick] (7, -\dy) to (7, \dy);
      \node[align=center] at (9, 0) {Invariant\\confluence-ness};
      \pause
      \node[rotate=-45, anchor=north west] at (7, -\dy) {Invariant Confluent};
      \pause
      \node[rotate=-45, anchor=north west] at (4, -\dy) {Kinda Invariant Confluent};
      \pause
      \node[rotate=-45, anchor=north west] at (1, -\dy) {Not at all Invariant Confluent};

      \pause
      \draw[ultra thick, -latex, red] (7, 1) to (7, \dy);
      \node[align=center, fill=white] at (7, 1.5) {Weak\\Consistency};

      \pause
      \draw[decorate, decoration={brace, amplitude=12pt, raise=4pt},
            ultra thick, blue] (1, \dy) to (6.9, \dy);
      \node[align=center] at (4, 1.5) {Strong\\Consistency};
    \end{tikzpicture}
  \end{center}
\end{frame}

\begin{frame}
  \begin{center}
    \begin{tikzpicture}
      \draw[ultra thick, latex-latex] (0, 0) to (8, 0);
      \draw[thick] (1, -\dy) to (1, \dy);
      \draw[thick] (4, -\dy) to (4, \dy);
      \draw[thick] (7, -\dy) to (7, \dy);
      \node[align=center] at (9, 0) {Invariant\\confluence-ness};
      \node[rotate=-45, anchor=north west] at (7, -\dy) {Invariant Confluent};
      \node[rotate=-45, anchor=north west] at (4, -\dy) {Kinda Invariant Confluent};
      \node[rotate=-45, anchor=north west] at (1, -\dy) {Not at all Invariant Confluent};

      \draw[ultra thick, -latex, red] (7, 1) to (7, \dy);
      \node[align=center, fill=white] at (7, 1.5) {Weak\\Consistency};
      \draw[ultra thick, -latex, red] (4, 1) to (4, \dy);
      \node[align=center, fill=white] at (4, 1.5) {Weakish\\Consistency};
      \draw[ultra thick, -latex, red] (1, 1) to (1, \dy);
      \node[align=center, fill=white] at (1, 1.5) {Strong\\Consistency};
    \end{tikzpicture}
  \end{center}
\end{frame}

\begin{frame}
  \begin{center}
    \Large
    \defword{Segmented invariant confluence}: divide state space into segments;
    operate with weak consistency within segments and strong consistency across
    segments.
  \end{center}
\end{frame}

\begin{frame}
  A \defword{segmentation} $\Sigma = (I_1, T_1), \ldots, (I_n, T_n)$ is a
  sequence (not a set) of $n$ segments $(I_i, T_i)$.
  \pause
  $O$ is \defword{segmented \invariantconfluent{}} with respect to $s_0$, $T$,
  $I$, and $\Sigma$, abbreviated \defword{\sTISconfluent}, if the following
  conditions hold:
  \begin{itemize}
    \pause\item
      The start state satisfies the invariant (i.e. $I(s_0)$).

    \pause\item
      $I$ is covered by the invariants in $\Sigma$.

    \pause\item
      $O$ is \invariantconfluent{} within each segment. That is, for every $(I_i,
      T_i) \in \Sigma$ and for every state $s \in I_i$, $O$ is
      \sticonfluent{s}{T_i}{I_i}.
  \end{itemize}
\end{frame}

{
\newcommand{\xmin}{-2}
\newcommand{\xmax}{2}
\newcommand{\ymin}{-2}
\newcommand{\ymax}{2}

% Axes.
\newcommand{\xyaxes}{
  \draw[] (\xmin.5, 0) to (\xmax.5, 0);
  \draw[] (0, \ymin.5) to (0, \ymax.5);
  \node at (\xmax + 1, 0) {$x$};
  \node at (0, \ymax + 1) {$y$};
}

% Quadrant 1.
\newcommand{\quadi}[5]{{
  \newcommand{\argstyle}{#1}
  \newcommand{\argxmin}{#2}
  \newcommand{\argxmax}{#3}
  \newcommand{\argymin}{#4}
  \newcommand{\argymax}{#5}
  \foreach \x in {0, ..., \argxmax} {
    \foreach \y in {0, ..., \argymax} {
      \node[\argstyle] (\x-\y) at (\x, \y) {};
    }
  }
}}

% Quadrant 2.
\newcommand{\quadii}[5]{{
  \newcommand{\argstyle}{#1}
  \newcommand{\argxmin}{#2}
  \newcommand{\argxmax}{#3}
  \newcommand{\argymin}{#4}
  \newcommand{\argymax}{#5}
  \foreach \x in {\argxmin, ..., 0} {
    \foreach \y in {0, ..., \argymax} {
      \node[\argstyle] (\x-\y) at (\x, \y) {};
    }
  }
}}

% Quadrant 3.
\newcommand{\quadiii}[5]{{
  \newcommand{\argstyle}{#1}
  \newcommand{\argxmin}{#2}
  \newcommand{\argxmax}{#3}
  \newcommand{\argymin}{#4}
  \newcommand{\argymax}{#5}
  \foreach \x in {\argxmin, ..., 0} {
    \foreach \y in {\argymin, ..., 0} {
      \node[\argstyle] (\x-\y) at (\x, \y) {};
    }
  }
}}

% Quadrant 4.
\newcommand{\quadiv}[5]{{
  \newcommand{\argstyle}{#1}
  \newcommand{\argxmin}{#2}
  \newcommand{\argxmax}{#3}
  \newcommand{\argymin}{#4}
  \newcommand{\argymax}{#5}
  \foreach \x in {0, ..., \argxmax} {
    \foreach \y in {\argymin, ..., 0} {
      \node[\argstyle] (\x-\y) at (\x, \y) {};
    }
  }
}}

% State labels.
\newcommand{\statelabels}{
  \node[statelabel] at (0, 0) {$s_0$};
  \node[statelabel] at (-1, 1) {$s_1$};
  \node[statelabel] at (1, -1) {$s_2$};
  \node[statelabel] at (1, 1) {$s_3$};
}

\tikzstyle{point}=[shape=circle, fill=flatgray, inner sep=3pt]
\tikzstyle{inv}=[line width=0.75pt, draw=black]
\tikzstyle{pointinv}=[point, inv]
\tikzstyle{invregion}=[rounded corners, fill=flatgreen!50, draw=none]
\tikzstyle{reachableregion}=[rounded corners, fill=flatblue!50, draw=none]
\tikzstyle{statelabel}=[anchor=south west, inner sep=1pt]

\begin{frame}
  \begin{columns}
    \begin{column}{0.5\textwidth}
      \centering
      \begin{tikzpicture}[scale=1]
        \begin{scope}
          \clip (\xmin.5, \ymax.5) rectangle (\xmax.5, \ymin.5);
          \draw[invregion] (\xmin.9, \ymax.9) rectangle (0.5, -0.5);
          \draw[invregion] (-0.5, 0.5) rectangle (\xmax.9, \ymin.9);
          \draw (0.5, 0.5) to (0.5, \ymax.5);
          \draw (0.5, 0.5) to (\xmax.5, 0.5);
          \draw (-0.5, -0.5) to (-0.5, \ymin.5);
          \draw (-0.5, -0.5) to (\xmin.5, -0.5);
        \end{scope}

        \xyaxes{}
        \quadi{point}{\xmin}{\xmax}{\ymin}{\ymax}
        \quadiii{point}{\xmin}{\xmax}{\ymin}{\ymax}
        \quadii{pointinv}{\xmin}{\xmax}{\ymin}{\ymax}
        \quadiv{pointinv}{\xmin}{\xmax}{\ymin}{\ymax}
      \end{tikzpicture}

      {\Huge Invariant}
    \end{column}
    \begin{column}{0.5\textwidth}
      \pause
      \centering
      \begin{tikzpicture}[scale=1]
        \begin{scope}
          \clip (-3, 3) rectangle (\xmax.5, \ymin.5);
          \draw[reachableregion, draw=black] (-2.5, 2.5) rectangle (\xmax.9, \ymin.9);
        \end{scope}

        \xyaxes{}
        \quadi{pointinv}{\xmin}{\xmax}{\ymin}{\ymax}
        \quadiii{pointinv}{\xmin}{\xmax}{\ymin}{\ymax}
        \quadii{pointinv}{\xmin}{\xmax}{\ymin}{\ymax}
        \quadiv{pointinv}{\xmin}{\xmax}{\ymin}{\ymax}
      \end{tikzpicture}

      {\Huge Reachable}
    \end{column}
  \end{columns}
\end{frame}
}

{
\tikzstyle{point}=[shape=circle, fill=flatgray, inner sep=2pt, draw=black]
\tikzstyle{region}=[draw=none]
\tikzstyle{region1}=[region, fill=flatred!50]
\tikzstyle{region2}=[region, fill=flatgreen!50]
\tikzstyle{region3}=[region, fill=flatblue!50]
\tikzstyle{region4}=[region, fill=flatpurple!50]

\newcommand{\pointgrid}[4]{{
  \newcommand{\argxmin}{#1}
  \newcommand{\argxmax}{#2}
  \newcommand{\argymin}{#3}
  \newcommand{\argymax}{#4}

  \draw[] (\argxmin, 0) to (\argxmax, 0);
  \draw[] (0, \argymin) to (0, \argymax);
  \foreach \x in {\argxmin, ..., \argxmax} {
    \foreach \y in {\argymin, ..., \argymax} {
      \node[point] (\x-\y) at (\x, \y) {};
    }
  }
}}

\newcommand{\subfigwidth}{0.24\columnwidth}
\newcommand{\subfighspace}{0.3cm}
\newcommand{\tikzhspace}{0.4cm}
\newcommand{\tikzscale}{0.75}
\newcommand{\xmin}{-2}
\newcommand{\xmax}{2}
\newcommand{\ymin}{-2}
\newcommand{\ymax}{2}

\begin{frame}
  \begin{center}
    \begin{tikzpicture}[scale=\tikzscale]
      \draw[white] (-3, -3) to (3, 3);
      \draw[region1] (\xmin.5, \ymax.5) rectangle (-0.5, 0.5);
      \draw (-0.5, 0.5) to (\xmin.5, 0.5);
      \draw (-0.5, 0.5) to (-0.5, \ymax.5);
      \pointgrid{\xmin}{\xmax}{\ymin}{\ymax}
    \end{tikzpicture}%
    \hspace{0.1in}%
    \begin{tikzpicture}[scale=\tikzscale]
      \draw[white] (-3, -3) to (3, 3);
      \draw[region2] (-0.5, 0.5) rectangle (\xmax.5, \ymin.5);
      \draw (-0.5, 0.5) to (\xmax.5, 0.5);
      \draw (-0.5, 0.5) to (-0.5, \ymin.5);
      \pointgrid{\xmin}{\xmax}{\ymin}{\ymax}
    \end{tikzpicture}

    \begin{tikzpicture}[scale=\tikzscale]
      \draw[white] (-3, -3) to (3, 3);
      \draw[region3] (-0.5, \ymax.5) rectangle (0.5, \ymin.5);
      \draw (-0.5, \ymax.5) to (-0.5, \ymin.5);
      \draw (0.5, \ymax.5) to (0.5, \ymin.5);
      \pointgrid{\xmin}{\xmax}{\ymin}{\ymax}
    \end{tikzpicture}%
    \hspace{0.1in}%
    \begin{tikzpicture}[scale=\tikzscale]
      \draw[white] (-3, -3) to (3, 3);
      \draw[region4] (\xmin.5, 0.5) rectangle (\xmax.5, -0.5);
      \draw (\xmax.5, -0.5) to (\xmin.5, -0.5);
      \draw (\xmax.5, 0.5) to (\xmin.5, 0.5);
      \pointgrid{\xmin}{\xmax}{\ymin}{\ymax}
    \end{tikzpicture}
  \end{center}
\end{frame}
}
}
{\section{Evaluation}\seclabel{Evaluation}
In this section, we describe and evaluate Lucy: a prototype implementation of
our decision procedures and system models.
% We evaluate Lucy by answering the following questions:
% \begin{itemize}
%   \item
%     How practical is the interactive invariant-confluence decision procedure?
%     Can we use it to classify real-world transactions and invariants?
%   \item
%     How practical is segmented invariant-confluence? Are real-world workloads
%     amenable to segmentation?
%   \item
%     How efficient is the interactive invariant-confluence decision procedure?
%   \item
%     How efficiently can we replicate a segmented invariant-confluent object as
%     compared to alternative approaches like replicating with weak or strong
%     consistency?
%   \item
%     How does the performance of replicating a segmented invariant-confluence
%     object vary as we vary the workload, segmentation, and replication factor?
% \end{itemize}

\subsection{Implementation}
Lucy includes an implementation of the interactive decision procedure described
in \algoref{InteractiveDecisionProcedure}, an implementation of a decision
procedure which checks criteria (1) - (4) from \thmref{LatticeProperty}, and an
implementation of the decision procedure described in
\algoref{ArbitraryStartInteractiveDecisionProcedure}. The decision procedures
are implemented in roughly 2,500 lines of Python. Users specify objects,
transactions, and invariants in a small Python DSL and interact with the
interactive decision procedures using an interactive Python console. We use
Z3~\cite{de2008z3} to implement our invariant-closure decision procedure,
compiling an object and invariant into a formula that is satisfiable if and
only if the object is \emph{not} invariant-closed. If the object is
invariant-closed, then Z3 concludes that the formula is unsatisfiable.
Otherwise, if the object is not invariant-closed, then Z3 produces a
counterexample witnessing the satisfiability of the formula.

Lucy also includes an implementation of the invariant-confluence and
segmented-invariant confluence system models in roughly 3,500 lines of C++.
Users specify objects, transactions, invariants, and segmentations in C++. Lucy
then replicates the objects using segmented invariant-confluence (or
invariant-confluence if the segmentation contains a single segment without any
disallowed transactions). Clients send every transaction request to a randomly
selected server. When a server receives a transaction request, it executes
\algoref{TxnExecution} to attempt to execute the transaction locally. If the
transaction requires global coordination, then the server forwards the
transaction request to a predetermined leader. When the leader receives a
transaction request, it sends a coordination request to all other servers. When
a server receives a coordination request from the leader, it stops processing
transactions and sends the leader its state in a coordination reply. When the
leader receives coordination replies from all other servers, it executes the
transaction, and then sends its state to the other servers. When a server
receives a new state, it adopts the state, computes its new active segment, and
resumes normal processing. After every 100 transactions processed, a server
sends a merge request to a randomly selected server.
% Merge requests are tagged with a monotonically increasing epoch number that is
% incremented by the master after every round of global coordination. This allows
% servers to discard merge requests from previous epochs.

Lucy can also replicate an object with eventual consistency and with
linearizability. With eventual consistency, clients send every transaction
request to a randomly selected server. The server executes the transaction
locally and returns immediately to the client, sending merge requests after
every 100 transactions. With linearizability, clients send every transaction
request to a predetermined leader. The leader relays the transaction request to
all other servers, and when the leader receives replies from them, it executes
the transaction and replies to the client. This communication pattern mimics
the ``normal operation'' of state machine replication protocols
\cite{lamport1998part, liskov2012viewstamped}.
% In \secref{SegmentedInvariantConfluenceEval}, we compare the performance of
% replicating with segmented invariant-confluence against the performance of
% replicating with eventual consistency and linearizability.

Because fault-tolerance is largely an orthogonal concern to
invariant-confluence, Lucy is implemented without fault-tolerance. It would be
straightforward to add fault-tolerance to Lucy, but it would not affect our
discussions or evaluation, so we leave it for future work.
% Moreover, users currently have to specify their workloads in Python (for the
% decision procedures) and C++ (for the runtime). In the future, we plan on
% removing this redundancy.

\subsection{Decision Procedures}
We now evaluate the practicality and efficiency of our decision procedure
prototypes. We begin by demonstrating the decision procedure on a handful of
simple, yet practical examples. We then discuss how our tool can be used to
analyze the TPC-C benchmark.

\example[$\ints^2$]\examplelabel{TwoIntsEval}
We begin with a minimal working example. Consider again our recurring example
of $\ints^2$ from \exampleref{Z2}. The Python code used to describe the object,
transactions, and invariant is given in \figref{Z2Code}. When we call
\python{checker.check()}, the interactive decision procedure produces a
counterexample in less than a tenth of a second.  After we label the
counterexample and refine the invariant with $y \leq 0$, the interactive
decision procedure determines that the object is invariant-confluent, again, in
less than a tenth of a second. Note that the object is invariant-confluent but
\emph{not} invariant-closed, so prior work that relies on invariant-closure
alone to determine invariant-confluence would not be able to identify this
example as invariant-confluent.

\begin{figure}[ht]
  \begin{Python}[gobble=4]
    checker = InteractiveInvariantConfluenceChecker()
    x = checker.int_max('x', 0) # An int, x, merged by max.
    y = checker.int_max('y', 0) # An int, y, merged by max.
    checker.add_transaction('increment_x', [x.assign(x + 1)])
    checker.add_transaction('decrement_y', [y.assign(y - 1)])
    checker.add_invariant(x * y <= 0)
    checker.check()
  \end{Python}
  \caption{\exampleref{TwoIntsEval} specification}\figlabel{Z2Code}
\end{figure}

\example[Foreign Keys]\examplelabel{ForeignKeysEval}
A 2P-Set $X = (A_X, R_X)$ is a set CRDT composed of a set of additions $A_X$
and a set of removals $R_X$~\cite{shapiro2011comprehensive}. We view the state
of the set $X$ as the difference $A_X - R_X$ of the addition and removal sets.
To add an element $x$ to the set, we add $x$ to $A_X$. Similarly, to remove $x$
from the set, we add it to $R_X$. The merge of two 2P-sets is a pairwise union
(i.e. $(A_X, R_X) \join (A_Y, R_Y) = (A_X \cup A_Y, R_X \cup R_Y)$).

We can use 2P-sets to model a simple relational database with foreign key
constraints. Let object $O = (X, Y) = ((A_X, R_X), (A_Y, R_Y))$ consist of a
pair of two 2P-Sets $X$ and $Y$, which we view as relations. Our invariant $X
\subseteq Y$ (i.e. $(A_X - R_X) \subseteq (A_Y - R_Y)$) models a foreign key
constraint from $X$ to $Y$. We ran our decision procedure on the object with
initial state $((\emptyset, \emptyset), (\emptyset, \emptyset))$ and
with transactions that allow arbitrary insertions and deletions into $X$ and
$Y$. After less than a tenth of a second, the decision procedure produced a
reachable counterexample witnessing the fact that the object is not
invariant-confluent. A concurrent insertion into $X$ and deletion from $Y$ can
lead to a state which violates the invariant. This object is not
invariant-confluent and therefore not invariant-closed. Thus, previous tools
depending on invariant-closure as a sufficient condition would be unable to
conclude definitively that the object is \emph{not} invariant-confluent.

We also reran the decision procedure, but this time with insertions into $X$
and deletions from $Y$ disallowed. In less than a tenth of a second, the
decision procedure correctly deduced that the object is now
invariant-confluent. These results were manually proven
in~\cite{bailis2014coordination}, but our tool was able to confirm them
automatically in a negligible amount of time.

\example[Auction]\examplelabel{AuctionEval}
We now consider a simple auction system introduced in~\cite{gotsman2016cause}.
Our object consists of a set $B$ of integer-valued bids and a optional winning
bid $w$. Initially, $B = \emptyset$ and $w = \bot$ (indicating that there is no
winning bid yet) and we merge states by taking the union of $B$ and the maximum
of $w$ (where $\bot < n$ for all integers $n$). One transaction $t_b$ places a bid
$b$ by inserting it into $B$. Another transaction $t_\text{close}$ closes the
auction and sets $w$ equal to the largest bid in $B$. Our invariant is that if
the auction is closed (i.e.\ $w \neq \bot$), then $w = \max(B)$. We ran our
decision procedure on this example and in a third of a second, it produced a
reachable counterexample witnessing the fact that the object is not
invariant-confluent.  If we concurrently close the auction and place a large
bid, then we can end up in a state in which the auction is closed, but there is
a bid in $B$ larger than $w$.

We then segmented our object as follows. The first segment $(\setst{(B, w)}{w =
\bot}, \setst{t_b}{b \in \ints})$ allows bidding so long as the bid is open.
The second segment $(\setst{B, w}{w \neq \bot} \cap I, \emptyset)$ includes all
auctions which have already been closed and forbids all transactions. This
segmentation captures the intuition that bids should be permitted only when the
auction is open. We ran our segmented invariant-confluence decision procedure
on this example, and it was able to deduce without any human interaction that
the example was segmented invariant-confluent in less than a tenth of a second.

\example[Escrow Transactions]\examplelabel{EscrowTransactionsEval}
Escrow transactions are a concurrency control technique that allows a database
to execute transactions that increment and decrement numeric values with more
concurrency than is otherwise possible with general-purpose techniques like
two-phase locking~\cite{o1986escrow}. The main idea is that a portion of the
numeric value is put in escrow, after which a transaction can freely decrement
the value so long as it is not decremented by more than the amount that has
been escrowed. We show how segmented invariant-confluence can be used to
implement escrow transactions.

Consider again the PN-Counter $s = (p_1, p_2, p_3), (n_1, n_2, n_3)$ from
\exampleref{CounreachableExample} replicated on three servers with transactions
to increment and decrement the PN-Counter. In
\exampleref{CounreachableExample}, we found that concurrent decrements violate
invariant-confluence which led us to a segmentation which prohibited concurrent
decrements. We now propose a new segmentation with escrow amount $k$ that will
allow us to perform concurrent decrements that respect the escrowed value. The
first segment $(\setst{(p_1, p_2, p_3), (n_1, n_2, n_3)}{p_1, p_2, p_3 \geq k
\land n_1, n_2, n_3 \leq k}, T)$ allows for concurrent increments and
decrements so long as every $p_i \geq k$ and every $n_i \leq k$. Intuitively,
this segment represents the situation in which every server has escrowed a
value of $k$. They can decrement freely, so long as they don't exceed their
escrow budget of $k$. The second segment is the one presented in
\exampleref{CounreachableExample} which prohibits concurrent decrements. We ran
our decision procedure on this example and it concluded that it was segmented
invariant-confluent in less than a tenth of a second and without any human
interaction.

\example[TPC-C]\examplelabel{TpccEval}
TPC-C is a ubiquitous OLTP database benchmark with a workload that models a
simple warehousing application~\cite{difallah2013oltp}. The TPC-C specification
outlines twelve ``consistency requirements'' (read invariants) that govern the
warehousing application. In~\cite{bailis2014coordination}, Bailis et al.\
categorize the invariants into one of three types:
\begin{enumerate}
  \item
    Three of the twelve invariants are \textbf{foreign key constraints}.  As
    discussed in \exampleref{ForeignKeysEval}, our decision procedures can
    automatically verify conditions under which foreign key constraints are
    invariant-confluent.

  \item
    \newcommand{\ttt}[1]{{\smaller \texttt{#1}}}
    Seven of the twelve invariants involve \textbf{maintaining arithmetic
    relationships between relations}. Our decision procedures can correctly
    identify these as invariant-confluent. Consider, for example, invariant 1
    which dictates that a warehouse's year to date balance \ttt{W\_YTD} is
    equal to the sum of the district year to date balances \ttt{D\_YTD} of the
    twenty districts that are associated with the warehouse. The Payment
    transaction randomly selects a district and increments \ttt{W\_YTD} and
    \ttt{D\_YTD} by a randomly generated amount. We model this workload with a
    PN-Counter for \ttt{W\_YTD} and twenty PN-Counters for the twenty instances
    of \ttt{D\_YTD}. We applied Lucy to this workload, and it determined that
    the workload was invariant-confluent in less than a second.

  \item
    Two of the twelve invariants involve generating \textbf{sequential and
    unique identifiers}. This workload is \emph{not} invariant-confluent, but
    when the sequentiality requirement is dropped, it is invariant-confluent.
    Unfortunately, modelling unique key generation in our theoretical framework
    (and in our prototype implementation) is not possible because we have thus
    far tacitly assumed that transaction execution is deterministic. We leave a
    generalization of our theory to non-deterministic transactions for future
    work.
\end{enumerate}

\subsection{Segmented Invariant Confluence}%
\seclabel{SegmentedInvariantConfluenceEval}
Now, we evaluate the performance of replicating an object with segmented
invariant-confluence as compared to replicating it with eventual consistency or
linearizability.
% We begin with two benchmarks that demonstrate the same concept: the performance
% of segmented invariant-confluent replication varies with the amount of global
% coordination induced by either (a) performing a transaction that is disallowed
% within a segment or (b) transitioning between segments.

\begin{figure}[ht]
  \centering

  \begin{subfigure}[c]{\columnwidth}
    \centering
    \includegraphics[width=\columnwidth]{figures/vary_withdraws.pdf}
  \end{subfigure}
  \begin{subfigure}[c]{\columnwidth}
    \includegraphics[width=\columnwidth]{figures/vary_segments.pdf}
  \end{subfigure}

  \caption{%
    Segmented invariant-confluent replication throughput versus coordination,
    induced by executing disallowed transactions (top) and by transitioning
    across segments (bottom).
  }\figlabel{ThroughputVsGlobalSyncs}
\end{figure}

\begin{benchmark}\benchlabel{VaryWithdraws}
Consider again the PN-Counter from \exampleref{CounreachableExample} and the
corresponding transactions, invariants, and single-segment segmentation that
forbids concurrent decrements. We replicate this object on 32 servers
deployed on 32 m5.xlarge EC2 instances within the same availability zone.  Each
server has three colocated clients that issue deposit and withdrawal
transactions. We replicate the object with eventual consistency, segmented
invariant-confluence, and linearizability and measure the system's total
throughput as we vary the fraction of client requests that are withdrawals. The
results are shown in the top of \figref{ThroughputVsGlobalSyncs}.

Both eventually consistent replication and linearizable replication are
unaffected by the workload, achieving roughly 700,000 and 7,000 transactions
per second respectively.
%
% Expectedly, eventually consistent replication significantly outperforms
% linearizable replication because (a) transactions can be sent to any server
% (not just the leader) and (b) servers do not coordinate with each other at all.
%
Segmented invariant-confluent replication performs well for low-withdrawal
workloads and performs increasingly poorly as we increase the fraction of
withdrawal transactions, eventually performing worse than linearizable
replication. For example, with 5\% withdrawal transactions, segmented
invariant-confluent replication performs an order of magnitude better than
linearizable replication; with 50\% withdrawals, it performs as well; and with
100\% withdrawals, it performs two times worse.

These results are expected. Deposit transactions can execute without any
coordination while withdrawal transactions require global coordination. As we
increase the fraction of withdrawals, we increase the amount of coordination
that the system has to perform which in turn drastically decreases the
throughput. These results also offer two insights:
%
First, for low-withdrawal workloads, segmented invariant-confluent replication
achieves a compromise between strong and weak consistency. It guarantees that
invariants are maintained (which is impossible with eventual consistency if the
object is not invariant-confluent) with performance many times better than
strongly consistent replication.
%
Second, segmented invariant-confluent replication is poorly suited to workloads
that require a large amount of coordination. For workloads without much inherit
concurrency (e.g.\ withdraw-mostly workloads), maintaining invariants is best
done with strong consistency. It provides stronger guarantees with better
performance.
\end{benchmark}

\begin{benchmark}\benchlabel{VarySegmentLength}
  Consider again the object, transactions, and invariants from \exampleref{Z2}
  and \exampleref{SegmentedZ2}. As with \benchref{VaryWithdraws}, we replicate
  the object across 32 servers. Clients issue 50\% increment $x$ transactions,
  and 50\% decrement $y$ transactions. We consider a ``checkerboard''
  segmentation $\Sigma_n = \setst{(I_{i, j}, T)}{i, j \in \ints}$ where segment
  invariant $I_{i, j}$ consists of the square of points $\setst{(x, y)}{ni \leq
  x < n(i + 1), nj \leq y < n(j + 1)}$ with side length $n$. For example,
  $\Sigma_1$ places each point in its own segment, $\Sigma_2$ tessellates
  $\ints^2$ with 2x2 squares, $\Sigma_3$ tessellates $\ints^2$ with 3x3
  squares, and so on. We measure the throughput of the object replicated with
  eventually consistent, segmented invariant-confluent, and linearizable
  replication as we vary the segment side length $n$. The results are shown in
  the bottom \figref{ThroughputVsGlobalSyncs}.

  This benchmark tells the same tale as \benchref{VaryWithdraws}. Eventual
  consistency and linearizability are unaffected by workload, and eventual
  consistency outperforms linearizability by roughly two orders of magnitude.
  In this example, the segmented invariant-confluent replication only requires
  coordination when transitioning between segment boundaries, so as we increase
  the segment side length, the throughput of the system increases
  significantly.
\end{benchmark}

\begin{figure}[ht]
  \centering
  \includegraphics[width=\columnwidth]{figures/vary_nodes.pdf}
  \caption{%
    Throughput of eventually consistent, segmented invariant-confluent, and
    linearizable replication measured against the number of
    nodes.
  }\figlabel{VaryNodes}
\end{figure}

\begin{benchmark}
  In this benchmark, we measure the scale-out of segmented invariant-confluent
  replication. We repeat \benchref{VaryWithdraws} with a 10\% withdrawal rate,
  but this time we vary the number of servers we use to replicate our object.
  When we replicate with $n$ servers, we use $3n$ clients (the $3$ colocated
  clients on each server) as part of the workload. The results are shown in
  \figref{VaryNodes}.

  Eventually consistent replication scales perfectly with the number of nodes,
  confirming the results in~\cite{bailis2014coordination}.
  %
  % With eventually consistent replication, servers do not coordinate at all, so
  % they are completely unaffected by the number of servers.
  %
  Linearizable replication, on the other hand, scales up to about 3-5 servers
  before performance begins to decrease. These numbers are consistent with
  typical deployments of state-machine replication protocols like
  Paxos~\cite{chandra2007paxos}.
  %
  % Because all messages are sent to the leader, the leader becomes the
  % bottleneck as the number of servers and clients increases. Moreover, the
  % leader must wait for responses from more servers, increasing the latency of
  % the slowest response which in turn decreases throughput.
  %
  Segmented invariant-confluent replication scales up to about 6-8 servers
  before succumbing to the same scalability bottlenecks as linearizable
  replication.

  These results highlight the importance of coordination avoidance in
  distributed databases. While segmented invariant-confluent replication scales
  out slightly better than linearizable replication, both scale significantly
  worse than eventually consistent replication even for a very low (i.e.\ 10\%)
  withdrawal workload. This demonstrates that even a small amount of
  coordination can significantly reduce the scalability of a system.
\end{benchmark}
}
\end{document}


\subsection{Process-Based}
The process-based model is most similar to the process model described by
Shapiro et al.\ in~\cite{shapiro2011conflict}. We are given a set
$\seq{p}{1}{n}$ of $n$ processors each of which begins with state $s_0$.
Processors can execute a transaction $t \in T$ to transition from state $s$ to
state $t(s)$, or they can send their state $s$ to another processor with state
$s'$ causing the other processor to transition from state $s'$ to state $s
\join s'$.

Consider any two processors $p_i$ and $p_j$ with states $s_i^0, \ldots, s_i^n$
and $s_j^0, \ldots, s_j^m$ such that every state $s_i^0, \ldots, s_i^n, s_j^0,
\ldots, s_j^m$ satisfies $I$. $T$ is invariant-confluent with respect to $I$,
abbreviated \Iconfluent{}, if $s_i^n \join s_j^m$ is guaranteed to satisfy $I$.

\subsection{Graph-Based}
The graph-based approach is most similar to the model used by Bailis et al.\
in~\cite{bailis2014coordination}. We are given a directed acyclic graph in
which vertices can have an arbitrary number of children but at most two
parents. The vertices are labelled with states, and the edges are either
labelled with transactions or are unlabelled if they correspond to a join. The
graph is really just an alternate way of representing an execution in the
process-based model, but with duplicate states collapsed into a single vertex.
As with the process-based model, we say $T$ is \Iconfluent{} if the join of any
two invariant satisfying states with invariant satisfying ancestors is
guaranteed to satisfy the invariant.

\subsection{Expression-Based}
The expression-based approach deals with expressions built from $s_0$,
transactions in $T$, and $\join$. That is, expressions $e$ are built from the
grammar
\[
  e ::= s_0 \mid t(e) \mid e_1 \join e_2
\]
where $t$ corresponds to a transaction in $T$. We can evaluate an expression,
denoted $eval(e)$, in the obvious way:
\begin{mathpar}
  eval(s_0) \defeq s_0

  eval(t(e)) \defeq t(eval(e))

  eval(e_1 \join e_2) \defeq eval(e_1) \join eval(e_2)
\end{mathpar}

We say an expression $e$ satisfies $I$ if $I(eval(e))$. We say an expression
recursively satisfies $I$, denoted $\Irec{e}$, if $e$ and all of $e$'s children
satisfy $I$. That is,
\begin{mathpar}
  \Irec{s_0} \defeq I(s_0)

  \Irec{t(e)} \defeq I(t(e)) \land I(e)

  \Irec{e_1 \join e_2} \defeq I(e_1 \join e_2) \land I(e_1) \land I(e_2)
\end{mathpar}

We say $T$ is \Iconfluent, if for all expressions $e_1, e_2$, if $\Irec{e_1}$
and $\Irec{e_2}$, then $I(e_1 \join e_2)$.
