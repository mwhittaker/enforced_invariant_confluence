\begin{frame}
  \Huge
  \begin{center}
    Goal: develop an invariant-confluence decision procedure.
  \end{center}
\end{frame}

\begin{frame}
  \Huge
  \begin{center}
    Reasoning about reachable states is hard.
  \end{center}
\end{frame}

\begin{frame}
  \Large
  We say a set $S$ is \defword{closed under $f$} if for every $x, y \in S, f(x,
  y) \in S$. \pause For example,
  \begin{itemize}
    \item Even numbers are closed under addition.
    \item Odd numbers are \emph{not} closed under addition.
  \end{itemize}
\end{frame}

\begin{frame}
  \Large
  $O = (S, \join)$ is \defword{\invariantclosed{}} with respect to an invariant
  $I$, abbreviated \defword{\Iclosed{}}, if invariant satisfying states are
  closed under merge. \pause That is, for every state $s_1, s_2 \in S$, if
  $I(s_1)$ and $I(s_2)$, then $I(s_1 \join s_2)$.
\end{frame}

\begin{frame}
  \Large
  \[
    \text{invariant closure} \implies \text{invariant confluence}
  \]
  \pause
  Transactions perserve invariants. If merging does too, then all reachable
  states satisfy the invariant.
\end{frame}

\begin{frame}
  \Large
  \[
    \text{invariant closure} \implies \text{invariant confluence}
  \]
  \pause
  Transactions perserve invariants. If merging does too, then all reachable
  states satisfy the invariant.
\end{frame}

\begin{frame}
  \Large
  Checking invariant closure is more straightforward.
  \pause
  \begin{align*}
    \forall x_1, & y_1, x_2, y_2.\, \\
    \quad & x_1y_1 \leq 0 \land x_2y_2 \leq 0 \implies \\
    \quad & \max(x_1, x_2)\max(y_1, y_2) \leq 0
  \end{align*}
\end{frame}

\begin{frame}
  \Large
  \[
    \text{invariant closure} \implies \text{invariant confluence}
  \]
\end{frame}

\begin{frame}
  \Large
  \[
    \text{invariant closure}
    \xLeftarrow{\phantom{a}?\phantom{a}}
    \text{invariant confluence}
  \]
\end{frame}

\newcommand{\xmin}{-2}
\newcommand{\xmax}{2}
\newcommand{\ymin}{-2}
\newcommand{\ymax}{2}

% Axes.
\newcommand{\xyaxes}{
  \draw[] (\xmin.5, 0) to (\xmax.5, 0);
  \draw[] (0, \ymin.5) to (0, \ymax.5);
  \node at (\xmax + 1, 0) {$x$};
  \node at (0, \ymax + 1) {$y$};
}

% Quadrant 1.
\newcommand{\quadi}[5]{{
  \newcommand{\argstyle}{#1}
  \newcommand{\argxmin}{#2}
  \newcommand{\argxmax}{#3}
  \newcommand{\argymin}{#4}
  \newcommand{\argymax}{#5}
  \foreach \x in {0, ..., \argxmax} {
    \foreach \y in {0, ..., \argymax} {
      \node[\argstyle] (\x-\y) at (\x, \y) {};
    }
  }
}}

% Quadrant 2.
\newcommand{\quadii}[5]{{
  \newcommand{\argstyle}{#1}
  \newcommand{\argxmin}{#2}
  \newcommand{\argxmax}{#3}
  \newcommand{\argymin}{#4}
  \newcommand{\argymax}{#5}
  \foreach \x in {\argxmin, ..., 0} {
    \foreach \y in {0, ..., \argymax} {
      \node[\argstyle] (\x-\y) at (\x, \y) {};
    }
  }
}}

% Quadrant 3.
\newcommand{\quadiii}[5]{{
  \newcommand{\argstyle}{#1}
  \newcommand{\argxmin}{#2}
  \newcommand{\argxmax}{#3}
  \newcommand{\argymin}{#4}
  \newcommand{\argymax}{#5}
  \foreach \x in {\argxmin, ..., 0} {
    \foreach \y in {\argymin, ..., 0} {
      \node[\argstyle] (\x-\y) at (\x, \y) {};
    }
  }
}}

% Quadrant 4.
\newcommand{\quadiv}[5]{{
  \newcommand{\argstyle}{#1}
  \newcommand{\argxmin}{#2}
  \newcommand{\argxmax}{#3}
  \newcommand{\argymin}{#4}
  \newcommand{\argymax}{#5}
  \foreach \x in {0, ..., \argxmax} {
    \foreach \y in {\argymin, ..., 0} {
      \node[\argstyle] (\x-\y) at (\x, \y) {};
    }
  }
}}

% State labels.
\newcommand{\statelabels}{
  \node[statelabel] at (0, 0) {$s_0$};
  \node[statelabel] at (-1, 1) {$s_1$};
  \node[statelabel] at (1, -1) {$s_2$};
  \node[statelabel] at (1, 1) {$s_3$};
}

\tikzstyle{point}=[shape=circle, fill=flatgray, inner sep=3pt]
\tikzstyle{inv}=[line width=0.75pt, draw=black]
\tikzstyle{pointinv}=[point, inv]
\tikzstyle{invregion}=[rounded corners, fill=flatgreen!50, draw=none]
\tikzstyle{reachableregion}=[rounded corners, fill=flatblue!50, draw=none]
\tikzstyle{statelabel}=[anchor=south west, inner sep=1pt]

\begin{frame}
  \begin{columns}
    \begin{column}{0.5\textwidth}
      \centering
      \begin{tikzpicture}[scale=1]
        \begin{scope}
          \clip (\xmin.5, \ymax.5) rectangle (\xmax.5, \ymin.5);
          \draw[invregion] (\xmin.9, \ymax.9) rectangle (0.5, -0.5);
          \draw[invregion] (-0.5, 0.5) rectangle (\xmax.9, \ymin.9);
          \draw (0.5, 0.5) to (0.5, \ymax.5);
          \draw (0.5, 0.5) to (\xmax.5, 0.5);
          \draw (-0.5, -0.5) to (-0.5, \ymin.5);
          \draw (-0.5, -0.5) to (\xmin.5, -0.5);
        \end{scope}

        \xyaxes{}
        \quadi{point}{\xmin}{\xmax}{\ymin}{\ymax}
        \quadiii{point}{\xmin}{\xmax}{\ymin}{\ymax}
        \quadii{pointinv}{\xmin}{\xmax}{\ymin}{\ymax}
        \quadiv{pointinv}{\xmin}{\xmax}{\ymin}{\ymax}
      \end{tikzpicture}

      {\Huge Invariant}
    \end{column}
    \begin{column}{0.5\textwidth}
      \centering
      \begin{tikzpicture}[scale=1]
        \begin{scope}
          \clip (-1, 1) rectangle (\xmax.5, \ymin.5);
          \draw[reachableregion, draw=black] (-0.5, 0.5) rectangle (\xmax.9, \ymin.9);
        \end{scope}

        \xyaxes{}
        \quadi{point}{\xmin}{\xmax}{\ymin}{\ymax}
        \quadiii{point}{\xmin}{\xmax}{\ymin}{\ymax}
        \quadii{point}{\xmin}{\xmax}{\ymin}{\ymax}
        \quadiv{pointinv}{\xmin}{\xmax}{\ymin}{\ymax}
      \end{tikzpicture}

      {\Huge Reachable}
    \end{column}
  \end{columns}
\end{frame}

\begin{frame}
  \begin{columns}
    \begin{column}{0.5\textwidth}
      \centering
      \begin{tikzpicture}[scale=1]
        \begin{scope}
          \clip (\xmin.5, \ymax.5) rectangle (\xmax.5, \ymin.5);
          \draw[invregion] (\xmin.9, \ymax.9) rectangle (0.5, -0.5);
          \draw[invregion] (-0.5, 0.5) rectangle (\xmax.9, \ymin.9);
          \draw (0.5, 0.5) to (0.5, \ymax.5);
          \draw (0.5, 0.5) to (\xmax.5, 0.5);
          \draw (-0.5, -0.5) to (-0.5, \ymin.5);
          \draw (-0.5, -0.5) to (\xmin.5, -0.5);
        \end{scope}

        \xyaxes{}
        \quadi{point}{\xmin}{\xmax}{\ymin}{\ymax}
        \quadiii{point}{\xmin}{\xmax}{\ymin}{\ymax}
        \quadii{pointinv}{\xmin}{\xmax}{\ymin}{\ymax}
        \quadiv{pointinv}{\xmin}{\xmax}{\ymin}{\ymax}

        \draw[-latex, ultra thick] (-1, 2) to (2, 2);
        \draw[-latex, ultra thick] (2, -1) to (2, 2);
      \end{tikzpicture}

      {\Huge Invariant}
    \end{column}
    \begin{column}{0.5\textwidth}
      \centering
      \begin{tikzpicture}[scale=1]
        \begin{scope}
          \clip (-1, 1) rectangle (\xmax.5, \ymin.5);
          \draw[reachableregion, draw=black] (-0.5, 0.5) rectangle (\xmax.9, \ymin.9);
        \end{scope}

        \xyaxes{}
        \quadi{point}{\xmin}{\xmax}{\ymin}{\ymax}
        \quadiii{point}{\xmin}{\xmax}{\ymin}{\ymax}
        \quadii{point}{\xmin}{\xmax}{\ymin}{\ymax}
        \quadiv{pointinv}{\xmin}{\xmax}{\ymin}{\ymax}
      \end{tikzpicture}

      {\Huge Reachable}
    \end{column}
  \end{columns}
\end{frame}

\begin{frame}
  \Large
  \[
    \text{invariant closure} \centernot\impliedby \text{invariant confluence}
  \]
\end{frame}

\begin{frame}
  \begin{columns}
    \begin{column}{0.5\textwidth}
      \centering
      \begin{tikzpicture}[scale=1]
        \begin{scope}
          \clip (\xmin.5, \ymax.5) rectangle (\xmax.5, \ymin.5);
          \draw[invregion] (\xmin.9, \ymax.9) rectangle (0.5, -0.5);
          \draw[invregion] (-0.5, 0.5) rectangle (\xmax.9, \ymin.9);
          \draw (0.5, 0.5) to (0.5, \ymax.5);
          \draw (0.5, 0.5) to (\xmax.5, 0.5);
          \draw (-0.5, -0.5) to (-0.5, \ymin.5);
          \draw (-0.5, -0.5) to (\xmin.5, -0.5);
        \end{scope}

        \xyaxes{}
        \quadi{point}{\xmin}{\xmax}{\ymin}{\ymax}
        \quadiii{point}{\xmin}{\xmax}{\ymin}{\ymax}
        \quadii{pointinv}{\xmin}{\xmax}{\ymin}{\ymax}
        \quadiv{pointinv}{\xmin}{\xmax}{\ymin}{\ymax}

        \draw[-latex, ultra thick] (-1, 2) to (2, 2);
        \draw[-latex, ultra thick] (2, -1) to (2, 2);
      \end{tikzpicture}

      {\Huge Invariant}
    \end{column}
    \begin{column}{0.5\textwidth}
      \centering
      \begin{tikzpicture}[scale=1]
        \begin{scope}
          \clip (-1, 1) rectangle (\xmax.5, \ymin.5);
          \draw[reachableregion, draw=black] (-0.5, 0.5) rectangle (\xmax.9, \ymin.9);
        \end{scope}

        \xyaxes{}
        \quadi{point}{\xmin}{\xmax}{\ymin}{\ymax}
        \quadiii{point}{\xmin}{\xmax}{\ymin}{\ymax}
        \quadii{point}{\xmin}{\xmax}{\ymin}{\ymax}
        \quadiv{pointinv}{\xmin}{\xmax}{\ymin}{\ymax}
      \end{tikzpicture}

      {\Huge Reachable}
    \end{column}
  \end{columns}
\end{frame}

\begin{frame}
  \Large
  If $I$ is a subset of reachable states, then
  \[
    \text{invariant closure} \iff \text{invariant confluence}
  \]
  \pause
  \begin{itemize}
    \item
      Forward direction: same as before.
    \item
      Backward direction: $I$ is equal to the set of reachable states.
      Reachable states are closed under merge, so so is $I$.
  \end{itemize}
\end{frame}
