\begin{frame}
  \tikzstyle{pic}=[inner sep=0pt]
  \tikzstyle{thought}=[cloud, draw, aspect=4]

  \begin{center}
    \begin{tikzpicture}[xscale=3, yscale=3]
      \node[pic] at (0, 0) {%
        \includegraphics[width=0.2\textwidth]{assets/person.jpg}
      };
      \pause

      \node[thought, aspect=1] at (-0.5, 0.5) {};
      \node[thought] at (-1, 1) {Weak Consistency?};
      \pause

      \node[thought, aspect=1] at (0.5, 0.5) {};
      \node[thought] at (1, 1) {Strong Consistency?};
      \pause

      \node[anchor=west] at (-1.75, 0.25) {%
        \textcolor{flatgreen}{\texttt{+}} Super fast
      };
      \node[anchor=west] at (-1.75, 0) {%
        \textcolor{flatred}{\texttt{-}} Hard to reason about
      };
      \pause

      \node[anchor=west] at (0.3, 0.25) {%
        \textcolor{flatgreen}{\texttt{+}} Easy to reason about
      };
      \node[anchor=west] at (0.3, 0) {%
        \textcolor{flatred}{\texttt{-}} Sacrifices availability
      };
    \end{tikzpicture}
  \end{center}

  \note{%
    Imagine you're an application developer and you have some data that you
    want to replicate. But, you're not sure whether to replicate the data with
    weak consistency or with strong consistency. \\[12pt]

    You can implement weak consistency super fast, but weakly consistent
    systems are hard to reason about. Strongly consistent systems are much
    easier to reason about, but they come at the cost of availability and
    performance. \\[12pt]
  }
\end{frame}

\begin{frame}
  \tikzstyle{pic}=[inner sep=0pt]
  \tikzstyle{quote}=[fill=gray!30, inner sep=12pt]

  \begin{center}
    \begin{tikzpicture}[xscale=1.3]
      \node[pic] (bailis) at (0, 0) {%
        \includegraphics[width=0.25\textwidth]{assets/bailis.jpg}
      };
      \node[quote, text width=0.8\textwidth, align=left] at (0, 3) {
        \Large\textit{Why not have both?} Invariant confluence!
      };
      \path[fill=gray!30] (bailis.north) -- ++(0, 1.5) -- ++(1, 0) -- (bailis.north);
    \end{tikzpicture}
  \end{center}

  \note{%
    Fortunately, Peter Bailis saves the day and reminds us of when in 2014, he
    and some other smart folks including Alan and my advisor Joe defined
    invariant confluence as a way to get the benefits of both weak and strong
    consistency.

    Invariant confluence will be the topic of this talk, so let's build some
    intuition on what invariant confluence is.
  }
\end{frame}
