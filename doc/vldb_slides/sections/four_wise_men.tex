\tikzstyle{pic}=[inner sep=0pt]
\tikzstyle{quote}=[fill=gray!30, inner sep=12pt]

\newcommand{\vogels}{
  \node[pic] (vogels) at (0, 0) {
    \includegraphics[
      width=0.22\textwidth,
      trim=0 75 0 25,
      clip
    ]{assets/vogels.jpg}
  };
}

\newcommand{\hoare}{
  \node[pic, right=0.2cm of vogels] (hoare) {
    \includegraphics[width=0.22\textwidth]{assets/hoare.jpg}
  };
}

\newcommand{\lamport}{
  \node[pic, right=0.2cm of hoare] (lamport) {
    \includegraphics[
      width=0.22\textwidth,
      trim=250 700 300 200,
      clip
    ]{assets/lamport.jpg}
  };
}

\newcommand{\brewer}{
  \node[pic, right=0.2cm of lamport] (brewer) {
    \includegraphics[
      width=0.22\textwidth,
      trim=100 300 100 100,
      clip
    ]{assets/brewer.jpg}
  };
}

\begin{frame}
  \note{%
    Hi everyone! Thanks so much for attending. I'm Michael, and today I'll be
    talking about ``Interactive Checks for Coordination Avoidance''. This is
    joint work with my advisor at UC Berkeley, Joe Hellerstein. \\[12pt]

    % I'd like to begin my talk with a short fairy tale. The fairy tale involves
    % an application developer who has some data they need to replicate.
    % Unfortunately, the application developer cannot decide whether to replicate
    % with weak consistency or strong consistency. The application developer
    % stays up all night trying to decide but eventually falls asleep at their
    % desk. In their sleep, they are visited by four wise men. And so, the fairy
    % tale begins:

    Imagine you're an application developer and you're trying to decide what
    kind of consistency you'd like to use to replicate some of your data.
  }
\end{frame}

\begin{frame}
  \begin{center}
    \begin{tikzpicture}[xscale=1.3]
      \vogels
      \node[quote, text width=0.8\textwidth, align=left] at (0, 3) {
        \Large\textit{
        Go with weak consistency. It's super duper fast and good enough for
        many applications!
        % Weak. Weak. It's weak that you're seeking.
        % \pause
        % Please learn that if you yearn your performance to peak, then turn no
        % further than weak.
      }
      };
      \path[fill=gray!30] (vogels.north) -- ++(0, 1) -- ++(0.5, 0) -- (vogels.north);
    \end{tikzpicture}
  \end{center}

  \note{%
    Someone like Werner Vogels would probably tell you to use weak consistency.
    You can implement weak consistency super duper fast, and for many
    applications, weak consistency is good enough. \\[12pt]

    % ``Weak. Weak. It's weak that you're seeking'', squeaked Werner Vogels.
    % His voice trembled, creaking, as he kept speaking.
    % ``Please learn that if you yearn your performance to peak,
    % then turn no further than weak.'' critiqued Werner.
    %
    % Werner Vogels has a good point. You can implement weak consistency super
    % duper fast, and for many applications, weak consistency is good enough.
    % But, let's continue.
  }
\end{frame}

\begin{frame}
  \begin{center}
    \begin{tikzpicture}[xscale=1.3]
      \vogels
      \hoare
      \node[quote, text width=0.8\textwidth, align=left] at (1, 3) {
        \Large\textit{
          Weakly consistent systems are so hard to reason about!
          % No. No. No more.
          % \pause
          % Weak is not easy. Weak is just treason. Please, these anomalies make
          % it so hard to reason.
        }
      };
      \path[fill=gray!30] (hoare.north) -- ++(0, 1) -- ++(0.5, 0) -- (hoare.north);
    \end{tikzpicture}
  \end{center}

  \note{%
    Tony Hoare would encourage us not to settle for weak consistency because a
    weakly consistent system is hard to reason about.

    % ``No, no, no more'' groaned Tony Hoare.
    % He moaned, he lamented, his soul was tormented
    % ``Weak is not easy. Weak is just treason.
    % Please, these anomalies, make it so hard to reason.''
    %
    % You see, Tony Hoare encourages us not to settle for weak consistency
    % because a weakly consistent system is very hard to reason about.
  }
\end{frame}

\begin{frame}
  \begin{center}
    \begin{tikzpicture}[xscale=1.3]
      \vogels
      \hoare
      \lamport
      \node[quote, text width=0.8\textwidth, align=left] at (2, 3) {
        \Large\textit{
          Go with a strongly consistent system! I think there are some good
          algorithms for that kind of stuff.
          % You see, I agree.
          % \pause
          % Weak is just wrong, not right for long. To defend your end users, you
          % must go with strong.
        }
      };
      \path[fill=gray!30] (lamport.north) -- ++(0, 0.5) -- ++(0.25, 0) -- (lamport.north);
    \end{tikzpicture}
  \end{center}

  \note{%
    Leslie Lamport would agree. He would encourage us to choose strong
    consistency. Strongly consistent systems are much easier to reason about.
    After all a strongly consistent system to end users looks exactly like a
    system that's not even distributed.

    % ``You see, I agree'', says he, the ever mesmerizing Leslie.
    % ``Weak is just wrong, not right for long.
    % To defend your end users, you must go with strong.''
    %
    % Leslie Lamport encourages us to choose strong consistency becausesStrongly
    % consistent systems are much easier to reason about. After all a strongly
    % consistent system to end users looks exactly like a system that's not even
    % distributed.
  }
\end{frame}

\begin{frame}
  \begin{center}
    \begin{tikzpicture}[xscale=1.3]
      \vogels
      \hoare
      \lamport
      \brewer
      \node[quote, text width=0.8\textwidth, align=left] at (3, 3) {
        \Large\textit{
          Not so fast!
          % Alas, not so fast!
          % \pause
          % Consistency weak but not weary, strong but not nearly as fast as you
          % think when you hear my CAP theory.
        }
      };
      \path[fill=gray!30] (brewer.north) -- ++(0, 1) -- ++(-1, 0) -- (brewer.north);
    \end{tikzpicture}
  \end{center}

  \note{%
    Eric Brewer would counter that strong consistency isn't perfect. Strong
    consistency comes at the price of availability and performance. So we're in
    a bit of a pickle. Weak consistency is too confusing and strong consistency
    is too heavy handed.

    %
    % ``Alas, not so fast'', Brewer the dapper chap clapped back.
    % ``Consistency weak but not weary, strong but not nearly
    % as fast as you think when you hear my CAP theory.''
    %
    % Eric Brewer is right. Surely, strong consistency comes at the price of
    % availability and performance. So our application developer is in a bit of a
    % pickle. Weak consistency is too confusing and strong consistency is too
    % heavy handed. Fortunately, a hero emerges to save the day.
  }
\end{frame}

\begin{frame}
  \begin{center}
    \begin{tikzpicture}[xscale=1.3]
      \node[pic] (bailis) at (0, 0) {\includegraphics[width=0.25\textwidth]{assets/bailis.jpg}};
      \node[quote, text width=0.8\textwidth, align=left] at (0, 3) {
        \Large\textit{Why not have both?}
      };
      \path[fill=gray!30] (bailis.north) -- ++(0, 1.5) -- ++(1, 0) -- (bailis.north);
    \end{tikzpicture}
  \end{center}

  \note{%
    % Out teetered Peter, with swagger demeanor.
    % ``Why not have both?'' quoth the Peter.
    % The end.

    Fortunately, Peter Bailis can save the and reminding us of when in 2014, he
    and some other smart folks including Alan defined invariant confluence as a
    way to get the benefits of both weak and strong consistency. An invariant
    confluent object can be replicated without coordination while maintaining
    its application invariants.

    Invariant confluence will be the topic of this talk, so let's build some
    intuition on what invariant confluence is.
  }
\end{frame}
