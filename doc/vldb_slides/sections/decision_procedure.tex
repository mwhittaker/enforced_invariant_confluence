\begin{frame}
  \begin{center}
    \Huge
    Main idea: prune the invariant until it's a subset of reachable states.
    Then check for invariant closure.
  \end{center}

  \note{%
    With this result in hand, we're now ready to develop an invariant
    confluence decision procedure. The main idea is to prune the invariant
    until it's a subset of reachable states. Once it is, invariant closure and
    invariant confluence are identical, so we can check for invariant
    confluence by checking for invariant closure.
  }
\end{frame}

\newcommand{\algocomment}[1]{\State \textcolor{flatdenim}{\texttt{//} #1}}

\begin{frame}
  \begin{algorithmic}
    \algocomment{Return if $O$ is \sTIconfluent{}.}
    \Function{IsInvConfluent}{$O$, $s_0$, $T$, $I$}
      \State
      \Return $I(s_0)$ and
      \Call{Helper}{$O$, $s_0$, $T$, $I$, $\set{s_0}$, $\emptyset$}
    \EndFunction

    \State

    \algocomment{$R$ is a set of \sTIreachable{} states.}
    \algocomment{$NR$ is a set of \sTIunreachable{} states.}
    \algocomment{$I(s_0)$ is a precondition.}
    \Function{Helper}{$O$, $s_0$, $T$, $I$, $R$, $NR$}
      \State closed, $s_1$, $s_2$ $\gets$ \Call{IsIclosed}{$O$, $I - NR$}
      \If {closed}
        \Return true
      \EndIf
      \State Augment $R, NR$ with random search and user input
      \If{$s_1, s_2 \in R$}
        \Return false
      \EndIf
      \State \Return \Call{Helper}{$O$, $s_0$, $T$, $I$, $R$, $NR$}
    \EndFunction
  \end{algorithmic}

  \note{%
    Unfortunately, I don't have time to walk through the decision procedure in
    full, but you can find all the details in our paper.
  }
\end{frame}

\tikzstyle{point}=[shape=circle, fill=flatgray, inner sep=2pt, draw=black]
\tikzstyle{reachable}=[fill=flatgreen!50, draw=none]
\tikzstyle{unreachable}=[fill=flatred!50, draw=none]
\tikzstyle{invariant}=[fill=flatblue!50, draw=none]
\tikzstyle{nothing}=[fill=white, draw=none]

\newcommand{\pointgrid}[4]{{
  \newcommand{\argxmin}{#1}
  \newcommand{\argxmax}{#2}
  \newcommand{\argymin}{#3}
  \newcommand{\argymax}{#4}

  \draw[] (\argxmin, 0) to (\argxmax, 0);
  \draw[] (0, \argymin) to (0, \argymax);
  \foreach \x in {\argxmin, ..., \argxmax} {
    \foreach \y in {\argymin, ..., \argymax} {
      \node[point] (\x-\y) at (\x, \y) {};
    }
  }
}}

\newcommand{\pointrect}[2]{
  \draw[#1] ($#2 + (-0.51, 0.51)$) rectangle ($#2 + (0.51, -0.51)$);
}

\newcommand{\subfigwidth}{0.48\textwidth}
\newcommand{\subfighspace}{0.3cm}
\newcommand{\tikzhspace}{0.4cm}
\newcommand{\tikzscale}{0.5}
\newcommand{\xmin}{-3}
\newcommand{\xmax}{3}
\newcommand{\ymin}{-3}
\newcommand{\ymax}{3}

\begin{frame}
  \begin{columns}
    \begin{column}{0.3\textwidth}
      \centering
      \begin{tikzpicture}[scale=\tikzscale]
        \draw[reachable, draw=black] (-0.5, 0.5) rectangle (0.5, -0.5);
        \pointgrid{\xmin}{\xmax}{\ymin}{\ymax}
        \node at (0, \ymax + 1) {$R$};
      \end{tikzpicture}
    \end{column}
    \begin{column}{0.3\textwidth}
      \centering
      \begin{tikzpicture}[scale=\tikzscale]
        \pointgrid{\xmin}{\xmax}{\ymin}{\ymax}
        \node at (0, \ymax + 1) {$NR$};
      \end{tikzpicture}
    \end{column}
    \begin{column}{0.3\textwidth}
      \centering
      \begin{tikzpicture}[scale=\tikzscale]
        \draw[invariant] (\xmin.5, \ymax.5) rectangle (0.5, -0.5);
        \draw[invariant] (-0.5, 0.5) rectangle (\xmax.5, \ymin.5);
        \draw (0.5, 0.5) to (0.5, \ymax.5);
        \draw (0.5, 0.5) to (\xmax.5, 0.5);
        \draw (-0.5, -0.5) to (-0.5, \ymin.5);
        \draw (-0.5, -0.5) to (\xmin.5, -0.5);
        \pointgrid{\xmin}{\xmax}{\ymin}{\ymax}
        \node at (0, \ymax + 1) {$I - NR$};

        \pause
        \node[fill opacity=0.75, text opacity=1, fill=white] at (-1, 1) {$s_1$};
        \node[fill opacity=0.75, text opacity=1, fill=white] at (1, -1) {$s_2$};
      \end{tikzpicture}
    \end{column}
  \end{columns}

  \note{%
    At a high level, our decision procedure maintains a set of reachable states
    and a set of unreachable states. It grows the two sets using random search
    and user feedback until the pruned invariant is a subset of reachable
    states. At this point, invariant closure and invariant confluence are
    identical, the decision procedure uses an SMT solver to check for invariant
    closure, and terminates.

    % I will give you a very high level overview of the decision procedure
    % though. Our decision procedure maintains two sets: $R$, a set of reachable
    % states, and $NR$, a set of non reachable states. It also implicitly
    % maintains $I - NR$, which is the pruned invariant. It's the invariant with
    % non-reachable states pruned from it.
  }
\end{frame}

\begin{frame}
  \begin{columns}
    \begin{column}{0.3\textwidth}
      \begin{tikzpicture}[scale=\tikzscale]
        \draw[reachable, draw=black] (-0.5, 0.5) rectangle (1.5, -1.5);
        \pointgrid{\xmin}{\xmax}{\ymin}{\ymax}
        \node at (0, \ymax + 1) {$R$};
      \end{tikzpicture}
    \end{column}
    \begin{column}{0.3\textwidth}
      \begin{tikzpicture}[scale=\tikzscale]
        \pointrect{unreachable, draw=black}{(-1, 1)}
        \pointgrid{\xmin}{\xmax}{\ymin}{\ymax}
        \node at (0, \ymax + 1) {$NR$};
      \end{tikzpicture}
    \end{column}
    \begin{column}{0.3\textwidth}
      \begin{tikzpicture}[scale=\tikzscale]
        \draw[invariant] (\xmin.5, \ymax.5) rectangle (0.5, -0.5);
        \draw[invariant] (-0.5, 0.5) rectangle (\xmax.5, \ymin.5);
        \draw (0.5, 0.5) to (0.5, \ymax.5);
        \draw (0.5, 0.5) to (\xmax.5, 0.5);
        \draw (-0.5, -0.5) to (-0.5, \ymin.5);
        \draw (-0.5, -0.5) to (\xmin.5, -0.5);
        \pointrect{nothing, draw=black}{(-1, 1)}
        \pointgrid{\xmin}{\xmax}{\ymin}{\ymax}
        \node at (0, \ymax + 1) {$I - NR$};

        \pause
        \node[fill opacity=0.75, text opacity=1, fill=white] at (-1, 2) {$s_1$};
        \node[fill opacity=0.75, text opacity=1, fill=white] at (3, -3) {$s_2$};
      \end{tikzpicture}
    \end{column}
  \end{columns}

  \note{%
    % The decision procedure then repeatedly finds pairs of points that violate
    % invariant closure and grows the set of reachable and not reachable points
    % by random search and by asking the user. Here, for example, we see the sets
    % $R$ and $NR$ growing each iteration of the algorithm.
  }
\end{frame}

\begin{frame}
  \begin{columns}
    \begin{column}{0.3\textwidth}
      \begin{tikzpicture}[scale=\tikzscale]
        \draw[reachable, draw=black] (-0.5, 0.5) rectangle (2.5, -2.5);
        \pointrect{reachable, draw=black}{(3, -3)}
        \pointgrid{\xmin}{\xmax}{\ymin}{\ymax}
        \node at (0, \ymax + 1) {$R$};
      \end{tikzpicture}
    \end{column}
    \begin{column}{0.3\textwidth}
      \begin{tikzpicture}[scale=\tikzscale]
        \pointrect{unreachable}{(-1, 1)}
        \pointrect{unreachable}{(-1, 2)}
        \draw (-1.5, 0.5) rectangle (-0.5, 2.5);
        \pointgrid{\xmin}{\xmax}{\ymin}{\ymax}
        \node at (0, \ymax + 1) {$NR$};
      \end{tikzpicture}
    \end{column}
    \begin{column}{0.3\textwidth}
      \begin{tikzpicture}[scale=\tikzscale]
        \draw[invariant] (\xmin.5, \ymax.5) rectangle (0.5, -0.5);
        \draw[invariant] (-0.5, 0.5) rectangle (\xmax.5, \ymin.5);
        \draw (0.5, 0.5) to (0.5, \ymax.5);
        \draw (0.5, 0.5) to (\xmax.5, 0.5);
        \draw (-0.5, -0.5) to (-0.5, \ymin.5);
        \draw (-0.5, -0.5) to (\xmin.5, -0.5);
        \pointrect{nothing}{(-1, 1)}
        \pointrect{nothing}{(-1, 2)}
        \draw (-1.5, 0.5) rectangle (-0.5, 2.5);
        \pointgrid{\xmin}{\xmax}{\ymin}{\ymax}
        \node at (0, \ymax + 1) {$I - NR$};

        \pause
        \node[fill opacity=0.75, text opacity=1, fill=white] at (-2, 1) {$s_1$};
        \node[fill opacity=0.75, text opacity=1, fill=white] at (1, -1) {$s_2$};
      \end{tikzpicture}
    \end{column}
  \end{columns}
\end{frame}

\begin{frame}
  \begin{columns}
    \begin{column}{0.3\textwidth}
      \begin{tikzpicture}[scale=\tikzscale]
        \draw[reachable] (-0.5, 0.5) rectangle (2.5, -2.5);
        \pointrect{reachable}{(3, 0)}
        \pointrect{reachable}{(3, -1)}
        \pointrect{reachable}{(0, -3)}
        \pointrect{reachable}{(2, -3)}
        \pointrect{reachable}{(3, -3)}
        \draw (-0.5, 0.5) to (\xmax.5, 0.5);
        \draw (-0.5, 0.5) to (-0.5, \ymin.5);
        \draw (-0.5, -3.5) to ++(1, 0)
        to ++(0, 1)
        to ++(1, 0)
        to ++(0, -1)
        to ++(2, 0)
        to ++(0, 1)
        to ++(-1, 0)
        to ++(0, 1)
        to ++(1, 0)
        to ++(0, 2);
        \pointgrid{\xmin}{\xmax}{\ymin}{\ymax}
        \node at (0, \ymax + 1) {$R$};
      \end{tikzpicture}
    \end{column}
    \begin{column}{0.3\textwidth}
      \begin{tikzpicture}[scale=\tikzscale]
        \draw[unreachable] (\xmin.5, \ymax.5) rectangle (-0.5, \ymin.5);
        \draw (\xmin.5, \ymax.5) to (-0.5, \ymax.5) to
        (-0.5, \ymin.5) to (\xmin.5, \ymin.5);
        \pointgrid{\xmin}{\xmax}{\ymin}{\ymax}
        \node at (0, \ymax + 1) {$NR$};
      \end{tikzpicture}
    \end{column}
    \begin{column}{0.3\textwidth}
      \begin{tikzpicture}[scale=\tikzscale]
        \draw[invariant] (-0.5, 0.5) rectangle (\xmax.5, \ymin.5);
        \draw (-0.5, 0.5) to (\xmax.5, 0.5);
        \draw (-0.5, 0.5) to (-0.5, \ymin.5);
        \pointgrid{\xmin}{\xmax}{\ymin}{\ymax}
        \node at (0, \ymax + 1) {$I - NR$};
      \end{tikzpicture}
    \end{column}
  \end{columns}

  \note{%
    % Eventually, the set of nonreachable points is large enough that the pruned
    % invariant is a subset of the reachable states. At this point, invariant
    % closure and invariant confluence are identical, so the algorithm checks for
    % invariant closure and terminates.
  }
\end{frame}
