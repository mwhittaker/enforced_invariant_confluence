\begin{frame}
  \Large
  \begin{itemize}
    \item
      A \defword{distributed object} $O = (S, \join)$ consists of a set $S$ of
      states and a binary merge operator $\join: S \times S \to S$ that merges
      two states into one.

    \pause\item
      Example: $O = (\nats, \max)$.

    \pause\item
      Example: $O = (\setst{X}{X \subseteq \nats}, \cup)$.
  \end{itemize}
\end{frame}

\begin{frame}
  \Large
  \begin{itemize}
    \item
      A \defword{transaction} $t: S \to S$ is a function that maps one state to
      another.

    \pause\item
      Example: $t(x) = x + 1 : \nats \to \nats$.
  \end{itemize}
\end{frame}

\begin{frame}
  \Large
  \begin{itemize}
    \item
      An \defword{invariant} $I$ is a subset of $S$.

    \pause\item
      Example: $I = \setst{x \in \ints}{x \geq 0}$.

    \pause\item
      Notation: $I(s)$ and $\lnot I(s)$.
  \end{itemize}
\end{frame}

\newcommand{\internaltext}[1]{$\boldsymbol #1$}

\begin{frame}
  \tikzstyle{s0color}=[fill=flatred]
  \tikzstyle{s1color}=[fill=flatgreen]
  \tikzstyle{s2color}=[fill=flatdenim]
  \tikzstyle{s3color}=[fill=flatorange]
  \tikzstyle{s4color}=[fill=flatyellow]
  \tikzstyle{s5color}=[fill=flatcyan]
  \tikzstyle{s6color}=[fill=flatpurple]
  \tikzstyle{s7color}=[fill=flatblue]
  \tikzstyle{txnline}=[black, thick, -latex]
  \tikzstyle{phantomstate}=[%
    shape=circle, inner sep=2pt, draw=white, line width=1pt, fill=white]
  \tikzstyle{state}=[%
    shape=circle, inner sep=2pt, draw=black, line width=1pt, text opacity=1,
    fill opacity=0.6]
  \tikzstyle{txntext}=[sloped, above]

  \begin{center}
    \Large
    \begin{tikzpicture}[yscale=2, xscale=2]
      \node (p3) at (0.5, 3) {$p_1$};
      \node (p2) at (0.5, 2) {$p_2$};
      \node (p1) at (0.5, 1) {$p_3$};

      \tikzstyle{pline}=[gray, opacity=0.75, thick, -latex]
      \draw[pline] (p3) to (4.75, 3);
      \draw[pline] (p2) to (4.75, 2);
      \draw[pline] (p1) to (4.75, 1);

      \pause

      % Initial states.
      \node[phantomstate] (s03) at (1, 3) {$\phantom{s_0}$};
      \node[state, s0color] (s03) at (1, 3) {\internaltext{s_0}};
      \node[phantomstate] (s02) at (1, 2) {$\phantom{s_0}$};
      \node[state, s0color] (s02) at (1, 2) {\internaltext{s_0}};
      \node[phantomstate] (s01) at (1, 1) {$\phantom{s_0}$};
      \node[state, s0color] (s01) at (1, 1) {\internaltext{s_0}};

      \pause

      % Top line t.
      \node[phantomstate] (s13) at (2, 3) {$\phantom{s_1}$};
      \node[state, s1color] (s13) at (2, 3) {\internaltext{s_1}};
      \draw[txnline] (s03) to node[txntext]{$t$} (s13);

      \pause

      % Middle and bottom u.
      \node[phantomstate] (s12) at (2, 2) {$\phantom{s_2}$};
      \node[state, s2color] (s12) at (2, 2) {\internaltext{s_2}};
      \node[phantomstate] (s11) at (2, 1) {$\phantom{s_2}$};
      \node[state, s2color] (s11) at (2, 1) {\internaltext{s_2}};
      \draw[txnline] (s02) to node[txntext]{$u$} (s12);
      \draw[txnline] (s01) to node[txntext]{$u$} (s11);

      \pause

      % Merge.
      \node[phantomstate] (s22) at (3, 2) {$\phantom{s_4}$};
      \node[state, s4color] (s22) at (3, 2) {\internaltext{s_4}};
      \draw[txnline] (s13) to node[txntext]{} (s22);
      \draw[txnline] (s12) to node[txntext]{} (s22);

      \pause

      % The rest.
      \node[phantomstate] (s23) at (3, 3) {$\phantom{s_3}$};
      \node[phantomstate] (s33) at (3.75, 3) {$\phantom{s_6}$};
      \node[state, s3color] (s23) at (3, 3) {\internaltext{s_3}};
      \node[state, s6color] (s33) at (3.75, 3) {\internaltext{s_6}};
      \node[phantomstate] (s21) at (3, 1) {$\phantom{s_5}$};
      \node[phantomstate] (s31) at (4.25, 1) {$\phantom{s_7}$};
      \node[state, s5color] (s21) at (3, 1) {\internaltext{s_5}};
      \node[state, s7color] (s31) at (4.25, 1) {\internaltext{s_7}};
      \draw[txnline] (s13) to node[txntext]{$v$} (s23);
      \draw[txnline] (s11) to node[txntext]{$w$} (s21);
      \draw[txnline] (s22) to node[txntext]{} (s33);
      \draw[txnline] (s23) to node[txntext]{} (s33);
      \draw[txnline] (s21) to node[txntext]{} (s31);
      \draw[txnline] (s33) to node[txntext]{} (s31);
    \end{tikzpicture}
  \end{center}
\end{frame}

\begin{frame}{Expression-Based Formalism}
\Huge
\[
  e ::= s \mid t(e) \mid e_1 \join e_2
\]
\end{frame}

\begin{frame}
  \tikzstyle{s0color}=[fill=flatred]
  \tikzstyle{s1color}=[fill=flatgreen]
  \tikzstyle{s2color}=[fill=flatdenim]
  \tikzstyle{s3color}=[fill=flatorange]
  \tikzstyle{s4color}=[fill=flatyellow]
  \tikzstyle{s5color}=[fill=flatcyan]
  \tikzstyle{s6color}=[fill=flatpurple]
  \tikzstyle{s7color}=[fill=flatblue]
  \tikzstyle{txnline}=[black, thick, -latex]
  \tikzstyle{phantomstate}=[%
    shape=circle, inner sep=1pt, draw=white, line width=1pt, fill=white]
  \tikzstyle{state}=[%
    shape=circle, inner sep=1pt, draw=black, line width=1pt, text opacity=1,
    fill opacity=0.6]

  \begin{columns}
    \begin{column}{0.5\textwidth}
      \centering
      \Large
      \begin{tikzpicture}[xscale=1.25, yscale=1.25]
        \node (p3) at (0.5, 3) {$p_1$};
        \node (p2) at (0.5, 2) {$p_2$};
        \node (p1) at (0.5, 1) {$p_3$};

        \tikzstyle{pline}=[gray, opacity=0.75, thick, -latex]
        \draw[pline] (p3) to (4.75, 3);
        \draw[pline] (p2) to (4.75, 2);
        \draw[pline] (p1) to (4.75, 1);

        % Top line.
        \node[phantomstate] (s03) at (1, 3) {$\phantom{s_0}$};
        \node[phantomstate] (s13) at (2, 3) {$\phantom{s_1}$};
        \node[phantomstate] (s23) at (3, 3) {$\phantom{s_3}$};
        \node[phantomstate] (s33) at (3.75, 3) {$\phantom{s_6}$};
        \node[state, s0color] (s03) at (1, 3) {\internaltext{s_0}};
        \node[state, s1color] (s13) at (2, 3) {\internaltext{s_1}};
        \node[state, s3color] (s23) at (3, 3) {\internaltext{s_3}};
        \node[state, s6color] (s33) at (3.75, 3) {\internaltext{s_6}};

        % Middle line.
        \node[phantomstate] (s02) at (1, 2) {$\phantom{s_0}$};
        \node[phantomstate] (s12) at (2, 2) {$\phantom{s_2}$};
        \node[phantomstate] (s22) at (3, 2) {$\phantom{s_4}$};
        \node[state, s0color] (s02) at (1, 2) {\internaltext{s_0}};
        \node[state, s2color] (s12) at (2, 2) {\internaltext{s_2}};
        \node[state, s4color] (s22) at (3, 2) {\internaltext{s_4}};

        % Bottom line.
        \node[phantomstate] (s01) at (1, 1) {$\phantom{s_0}$};
        \node[phantomstate] (s11) at (2, 1) {$\phantom{s_2}$};
        \node[phantomstate] (s21) at (3, 1) {$\phantom{s_5}$};
        \node[phantomstate] (s31) at (4.25, 1) {$\phantom{s_7}$};
        \node[state, s0color] (s01) at (1, 1) {\internaltext{s_0}};
        \node[state, s2color] (s11) at (2, 1) {\internaltext{s_2}};
        \node[state, s5color] (s21) at (3, 1) {\internaltext{s_5}};
        \node[state, s7color] (s31) at (4.25, 1) {\internaltext{s_7}};

        \tikzstyle{txntext}=[sloped, above]
        \draw[txnline] (s03) to node[txntext]{$t$} (s13);
        \draw[txnline] (s13) to node[txntext]{$v$} (s23);
        \draw[txnline] (s02) to node[txntext]{$u$} (s12);
        \draw[txnline] (s01) to node[txntext]{$u$} (s11);
        \draw[txnline] (s11) to node[txntext]{$w$} (s21);

        \draw[txnline] (s13) to node[txntext]{} (s22);
        \draw[txnline] (s12) to node[txntext]{} (s22);
        \draw[txnline] (s22) to node[txntext]{} (s33);
        \draw[txnline] (s23) to node[txntext]{} (s33);
        \draw[txnline] (s21) to node[txntext]{} (s31);
        \draw[txnline] (s33) to node[txntext]{} (s31);
      \end{tikzpicture}
    \end{column}
    \begin{column}{0.5\textwidth}
      \centering
      \Large
      \pause
      \begin{tikzpicture}[scale=1.25]
                           \node[state, s7color, label={[label distance=-0.1cm] 90:$s_7$}] (j1) at (0, 0) {\internaltext{\join}};
        \draw (j1)++(-30:1) node[state, s5color, label={[label distance=-0.1cm] 90:$s_5$}] (w)            {\internaltext{w}};
        \draw (j1)++(210:1) node[state, s6color, label={[label distance=-0.1cm] 90:$s_6$}] (j2)           {\internaltext{\join}};
        \draw (w)++(-90:1)  node[state, s2color, label={[label distance=-0.2cm] 60:$s_2$}] (u1)           {\internaltext{u}};
        \draw (u1)++(-90:1) node[state, s0color]                                           (s4)           {\internaltext{s_0}};
        \draw (j2)++(225:1) node[state, s3color, label={[label distance=-0.1cm] 90:$s_3$}] (v)            {\internaltext{v}};
        \draw (v)++(-90:1)  node[state, s1color, label={[label distance=-0.2cm]120:$s_1$}] (t2)           {\internaltext{t}};
        \draw (t2)++(-90:1) node[state, s0color]                                           (s1)           {\internaltext{s_0}};
        \draw (j2)++(-45:1) node[state, s4color, label={[label distance=-0.1cm] 90:$s_4$}] (j3)           {\internaltext{\join}};
        \draw (j3)++(240:1) node[state, s1color, label={[label distance=-0.2cm]120:$s_1$}] (t3)           {\internaltext{t}};
        \draw (t3)++(-90:1) node[state, s0color]                                           (s2)           {\internaltext{s_0}};
        \draw (j3)++(-60:1) node[state, s2color, label={[label distance=-0.1cm] 90:$s_2$}] (u3)           {\internaltext{u}};
        \draw (u3)++(-90:1) node[state, s0color]                                           (s3)           {\internaltext{s_0}};

        \tikzstyle{astedge}=[thick]
        \draw[astedge] (j1) to (w) to (u1) to (s4);
        \draw[astedge] (j1) to (j2) to (v) to (t2) to (s1);
        \draw[astedge] (j1) to (j2) to (j3) to (t3) to (s2);
        \draw[astedge] (j1) to (j2) to (j3) to (u3) to (s3);
      \end{tikzpicture}
    \end{column}
  \end{columns}

  {
    \Large
    \[
      s_7 = (v(t(s_0)) \join (t(s_0) \join u(s_0))) \join w(u(s_0))
    \]
  }
\end{frame}

\begin{frame}
  \Large
  \begin{mathpar}
    \inferrule{ }{\sTIreachablepredicate{s_0}}

    \inferrule{\sTIreachablepredicate{e} \\ I(t(e))}
              {\sTIreachablepredicate{t(e)}}

    \inferrule{\sTIreachablepredicate{e_1} \\ \sTIreachablepredicate{e_2}}
              {\sTIreachablepredicate{e_1 \join e_2}}
  \end{mathpar}
\end{frame}

\begin{frame}
  \Large
  An example:
  \begin{itemize}
    \item $O = (\ints, \max)$
    \item $T = \set{t(x) = x + 1}$
    \item $I = \setst{x \in \ints}{x \geq 0}$
    \item $s_0 = 42$
  \end{itemize}

  What states are reachable?
\end{frame}

\begin{frame}
  \Large
  $O$ is \defword{\invariantconfluent} with respect to $s_0$, $T$, and $I$,
  abbreviated \defword{\sTIconfluent{}}, if all reachable states satisfy the
  invariant:
  \[
    \setst{s \in S}{\sTIreachablepredicate{s}} \subseteq I
  \]
\end{frame}

\begin{frame}
  \Large
  An example:
  \begin{itemize}
    \item $O = (\ints \times \ints, \max \times \max)$
    \pause \item $t_x((x, y)) = (x + 1, y)$
    \pause \item $t_y((x, y)) = (x, y + 1)$
    \pause \item $T = \set{t_x, t_y}$
    \pause \item $I = \setst{(x, y)}{xy \leq 0}$
    \pause \item $s_0 = (0, 0)$
  \end{itemize}
  Is $O$ \sTIconfluent{}?
\end{frame}

\newcommand{\xmin}{-2}
\newcommand{\xmax}{2}
\newcommand{\ymin}{-2}
\newcommand{\ymax}{2}

% Axes.
\newcommand{\xyaxes}{
  \draw[] (\xmin.5, 0) to (\xmax.5, 0);
  \draw[] (0, \ymin.5) to (0, \ymax.5);
  \node at (\xmax + 1, 0) {$x$};
  \node at (0, \ymax + 1) {$y$};
}

% Quadrant 1.
\newcommand{\quadi}[5]{{
  \newcommand{\argstyle}{#1}
  \newcommand{\argxmin}{#2}
  \newcommand{\argxmax}{#3}
  \newcommand{\argymin}{#4}
  \newcommand{\argymax}{#5}
  \foreach \x in {0, ..., \argxmax} {
    \foreach \y in {0, ..., \argymax} {
      \node[\argstyle] (\x-\y) at (\x, \y) {};
    }
  }
}}

% Quadrant 2.
\newcommand{\quadii}[5]{{
  \newcommand{\argstyle}{#1}
  \newcommand{\argxmin}{#2}
  \newcommand{\argxmax}{#3}
  \newcommand{\argymin}{#4}
  \newcommand{\argymax}{#5}
  \foreach \x in {\argxmin, ..., 0} {
    \foreach \y in {0, ..., \argymax} {
      \node[\argstyle] (\x-\y) at (\x, \y) {};
    }
  }
}}

% Quadrant 3.
\newcommand{\quadiii}[5]{{
  \newcommand{\argstyle}{#1}
  \newcommand{\argxmin}{#2}
  \newcommand{\argxmax}{#3}
  \newcommand{\argymin}{#4}
  \newcommand{\argymax}{#5}
  \foreach \x in {\argxmin, ..., 0} {
    \foreach \y in {\argymin, ..., 0} {
      \node[\argstyle] (\x-\y) at (\x, \y) {};
    }
  }
}}

% Quadrant 4.
\newcommand{\quadiv}[5]{{
  \newcommand{\argstyle}{#1}
  \newcommand{\argxmin}{#2}
  \newcommand{\argxmax}{#3}
  \newcommand{\argymin}{#4}
  \newcommand{\argymax}{#5}
  \foreach \x in {0, ..., \argxmax} {
    \foreach \y in {\argymin, ..., 0} {
      \node[\argstyle] (\x-\y) at (\x, \y) {};
    }
  }
}}

% State labels.
\newcommand{\statelabels}{
  \node[statelabel] at (0, 0) {$s_0$};
  \node[statelabel] at (-1, 1) {$s_1$};
  \node[statelabel] at (1, -1) {$s_2$};
  \node[statelabel] at (1, 1) {$s_3$};
}

\tikzstyle{point}=[shape=circle, fill=flatgray, inner sep=3pt]
\tikzstyle{inv}=[line width=0.75pt, draw=black]
\tikzstyle{pointinv}=[point, inv]
\tikzstyle{invregion}=[rounded corners, fill=flatgreen!50, draw=none]
\tikzstyle{reachableregion}=[rounded corners, fill=flatblue!50, draw=none]
\tikzstyle{statelabel}=[anchor=south west, inner sep=1pt]

\begin{frame}
  \begin{center}
    \begin{tikzpicture}
      \xyaxes{}
      \quadi{point}{\xmin}{\xmax}{\ymin}{\ymax}
      \quadiii{point}{\xmin}{\xmax}{\ymin}{\ymax}
      \quadii{point}{\xmin}{\xmax}{\ymin}{\ymax}
      \quadiv{point}{\xmin}{\xmax}{\ymin}{\ymax}

      \pause
      \draw[-latex, ultra thick] (0,0) to node[below] {$t_x$} (1, 0);
      \pause
      \draw[-latex, ultra thick] (0,0) to node[left] {$t_y$} (0, 1);
    \end{tikzpicture}
  \end{center}
\end{frame}

\begin{frame}
  \begin{center}
    \begin{tikzpicture}
      \xyaxes{}
      \quadi{point}{\xmin}{\xmax}{\ymin}{\ymax}
      \quadiii{point}{\xmin}{\xmax}{\ymin}{\ymax}
      \quadii{point}{\xmin}{\xmax}{\ymin}{\ymax}
      \quadiv{point}{\xmin}{\xmax}{\ymin}{\ymax}

      \node[pointinv] at (0, -1) {};
      \node[pointinv] at (2, -2) {};
      \draw[-latex, ultra thick] (0,-1) to (2, -1);
      \draw[-latex, ultra thick] (2,-2) to (2, -1);
    \end{tikzpicture}
  \end{center}
\end{frame}

\begin{frame}
  \begin{columns}
    \begin{column}{0.5\textwidth}
      \centering
      \begin{tikzpicture}[scale=1]
        \begin{scope}
          \clip (\xmin.5, \ymax.5) rectangle (\xmax.5, \ymin.5);
          \draw[invregion] (\xmin.9, \ymax.9) rectangle (0.5, -0.5);
          \draw[invregion] (-0.5, 0.5) rectangle (\xmax.9, \ymin.9);
          \draw (0.5, 0.5) to (0.5, \ymax.5);
          \draw (0.5, 0.5) to (\xmax.5, 0.5);
          \draw (-0.5, -0.5) to (-0.5, \ymin.5);
          \draw (-0.5, -0.5) to (\xmin.5, -0.5);
        \end{scope}

        \xyaxes{}
        \quadi{point}{\xmin}{\xmax}{\ymin}{\ymax}
        \quadiii{point}{\xmin}{\xmax}{\ymin}{\ymax}
        \quadii{pointinv}{\xmin}{\xmax}{\ymin}{\ymax}
        \quadiv{pointinv}{\xmin}{\xmax}{\ymin}{\ymax}
      \end{tikzpicture}

      {\Huge Invariant}
    \end{column}
    \begin{column}{0.5\textwidth}
      \pause
      \centering
      \begin{tikzpicture}[scale=1]
        \begin{scope}
          \clip (-1, 1) rectangle (\xmax.5, \ymin.5);
          \draw[reachableregion, draw=black] (-0.5, 0.5) rectangle (\xmax.9, \ymin.9);
        \end{scope}

        \xyaxes{}
        \quadi{point}{\xmin}{\xmax}{\ymin}{\ymax}
        \quadiii{point}{\xmin}{\xmax}{\ymin}{\ymax}
        \quadii{point}{\xmin}{\xmax}{\ymin}{\ymax}
        \quadiv{pointinv}{\xmin}{\xmax}{\ymin}{\ymax}
      \end{tikzpicture}

      {\Huge Reachable}
    \end{column}
  \end{columns}
\end{frame}

\begin{frame}
  \Huge
  \begin{center}
    Goal: develop an invariant-confluence decision procedure.
  \end{center}
\end{frame}
