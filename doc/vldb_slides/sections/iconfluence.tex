\begin{frame}
  \Large
  \begin{itemize}
    \item
      A \defword{distributed object} $O = (S, \join)$.
    \pause\item
      Example: $O = (\ints, \max)$
    \pause\item
      A \defword{transaction} $t: S \to S$.
    \pause\item
      Example: $t(x) = x + 1 : \ints \to \ints$.
    \pause\item
      An \defword{invariant} $I$ is a subset of $S$.
    \pause\item
      Example: $I = \setst{x \in \ints}{x \geq 0}$.
  \end{itemize}

  \note{%
    Now that we have some intuition on invariant confluence, let's define
    invariant confluence carefully. \\[12pt]

    Intuitively, a distributed object is some object that we want to replicate.
    It can be something complex like a database or something simple like a
    list, a set, or an integer. Formally, we say a distributed object $O$ is a
    pair of a set $S$ of states and a binary merge operator that I'll pronounce
    ``join''. \\[12pt]

    For example, the set of integers merged by the join operator max is
    a distributed object. \\[12pt]

    Next, a transaction as a function that maps one state to another. For
    example, if our set of states is the set of integers, then the function $t$
    that maps an integer $x$ to the integer $x + 1$ is a transaction. \\[12pt]

    Finally, an invariant $I$ is a subset of the set of states $S$. For
    example, if our set of states is the set of integers, then the set of
    integers greater than or equal to zero is an invariant. \\[12pt]
  }
\end{frame}

\newcommand{\internaltext}[1]{$\boldsymbol #1$}
\tikzstyle{s0color}=[fill=flatred]
\tikzstyle{s1color}=[fill=flatgreen]
\tikzstyle{s2color}=[fill=flatdenim]
\tikzstyle{s3color}=[fill=flatorange]
\tikzstyle{s4color}=[fill=flatyellow]
\tikzstyle{s5color}=[fill=flatcyan]
\tikzstyle{s6color}=[fill=flatpurple]
\tikzstyle{s7color}=[fill=flatblue]
\tikzstyle{txnline}=[black, thick, -latex]
\tikzstyle{phantomstate}=[%
  shape=circle, inner sep=2pt, draw=white, line width=1pt, fill=white]
\tikzstyle{state}=[%
  shape=circle, inner sep=2pt, draw=black, line width=1pt, text opacity=1,
  fill opacity=0.6]
\tikzstyle{txntext}=[sloped, above]
\tikzstyle{pline}=[gray, opacity=0.75, thick, -latex]

\begin{frame}
  \begin{center}
    \Large
    \begin{tikzpicture}[yscale=2, xscale=2]
      \node (p3) at (0.5, 3) {$p_1$};
      \node (p2) at (0.5, 2) {$p_2$};
      \node (p1) at (0.5, 1) {$p_3$};

      \draw[pline] (p3) to (4.75, 3);
      \draw[pline] (p2) to (4.75, 2);
      \draw[pline] (p1) to (4.75, 1);

      \pause

      % Initial states.
      \node[phantomstate] (s03) at (1, 3) {$\phantom{s_0}$};
      \node[state, s0color] (s03) at (1, 3) {\internaltext{s_0}};
      \node[phantomstate] (s02) at (1, 2) {$\phantom{s_0}$};
      \node[state, s0color] (s02) at (1, 2) {\internaltext{s_0}};
      \node[phantomstate] (s01) at (1, 1) {$\phantom{s_0}$};
      \node[state, s0color] (s01) at (1, 1) {\internaltext{s_0}};
      \pause

      % Top line t.
      \node[phantomstate] (s13) at (2, 3) {$\phantom{s_1}$};
      \node[state, s1color] (s13) at (2, 3) {\internaltext{s_1}};
      \draw[txnline] (s03) to node[txntext]{$t$} (s13);
      \pause

      % Note about aborting.
      \tikzstyle{fade}=[%
        fill=white, opacity=0.5, minimum width=4.5in, minimum height=3in]
      \tikzstyle{callout}=[%
        draw, ultra thick, fill=white, align=center, text width=4in]
      \only<4>{
        \node[fade] at (3, 2) {};
        \node[callout] at (3, 2) {
          \Huge Replicas abort transactions that violate the invariant.
        };
      }
      \pause

      % Middle and bottom u.
      \node[phantomstate] (s12) at (2, 2) {$\phantom{s_2}$};
      \node[state, s2color] (s12) at (2, 2) {\internaltext{s_2}};
      \node[phantomstate] (s11) at (2, 1) {$\phantom{s_2}$};
      \node[state, s2color] (s11) at (2, 1) {\internaltext{s_2}};
      \draw[txnline] (s02) to node[txntext]{$u$} (s12);
      \draw[txnline] (s01) to node[txntext]{$u$} (s11);
      \pause

      % Merge.
      \node[phantomstate] (s22) at (3, 2) {$\phantom{s_4}$};
      \node[state, s4color] (s22) at (3, 2) {\internaltext{s_4}};
      \draw[txnline] (s13) to node[txntext]{} (s22);
      \draw[txnline] (s12) to node[txntext]{} (s22);
      \pause

      % Note about merging.
      \only<7>{
        \node[fade] at (3, 2) {};
        \node[callout] at (3, 2) {
          \Huge Replicas cannot abort merges.
        };
      }
      \pause

      % The rest.
      \node[phantomstate] (s23) at (3, 3) {$\phantom{s_3}$};
      \node[phantomstate] (s33) at (3.75, 3) {$\phantom{s_6}$};
      \node[state, s3color] (s23) at (3, 3) {\internaltext{s_3}};
      \node[state, s6color] (s33) at (3.75, 3) {\internaltext{s_6}};
      \node[phantomstate] (s21) at (3, 1) {$\phantom{s_5}$};
      \node[phantomstate] (s31) at (4.25, 1) {$\phantom{s_7}$};
      \node[state, s5color] (s21) at (3, 1) {\internaltext{s_5}};
      \node[state, s7color] (s31) at (4.25, 1) {\internaltext{s_7}};
      \draw[txnline] (s13) to node[txntext]{$v$} (s23);
      \draw[txnline] (s11) to node[txntext]{$w$} (s21);
      \draw[txnline] (s22) to node[txntext]{} (s33);
      \draw[txnline] (s23) to node[txntext]{} (s33);
      \draw[txnline] (s21) to node[txntext]{} (s31);
      \draw[txnline] (s33) to node[txntext]{} (s31);
    \end{tikzpicture}
  \end{center}

  \note{%
    \scriptsize

    We'll consider the scenario in which we replicate our distributed object
    across some number replicas. Here, we show three replicas $p_1$, $p_2$, and
    $p_3$. \\[12pt]

    Each replica is initialized with an initial state $s_0$ as well as a set of
    transactions $T$ and an invariant $I$. \\[12pt]

    Replicas repeatedly perform one of two actions. First, a replica can
    execute a transaction. For example, here, replica $p_1$ executes
    transaction $t$ taking it from state $s_0$ to state $s_1$. One very
    important thing to note about transactions is that if a replica executes a
    transaction and the new state does not satisfy the invariant, then the
    replica aborts the transaction and rolls back its state. Replicas only
    execute transactions if they maintain the invariant. We saw this in the
    bank account example when a replica aborted a withdrawal if there were
    insufficient funds. \\[12pt]

    Here, we see that replica $p_2$ and $p_3$ both execute transaction $u$
    leading to state $s_2$. \\[12pt]

    In addition to executing transactions, replicas can also periodically send
    their state to another replica to get merged. For example, here we see that
    replica $p_1$ sends its state $s_1$ to replica $p_2$. $p_2$ merges the two
    states using the join operator to get state $s_4$. Here, it's critical to
    note that unlike with transactions, replicas cannot abort a merge. They
    can abort transactions, but they cannot abort a merge. \\[12pt]

    And here, we see that the replicas continue to execute transactions and
    merge their states with one another. You'll note that in our model,
    replicas communicate with each other every so often to merge, but otherwise
    do not coordinate. \\[12pt]
  }
\end{frame}

\begin{frame}
  \Huge
  A state is \defword{reachable} if there exists some execution of our system
  in which some replica enters the state.

  \note{%
    We say that a state is reachable if there exists some execution in which
    some replica enters the state.
  }
\end{frame}

\begin{frame}
  \begin{center}
    \Large
    \begin{tikzpicture}[yscale=2, xscale=2]
      \node (p3) at (0.5, 3) {$p_1$};
      \node (p2) at (0.5, 2) {$p_2$};
      \node (p1) at (0.5, 1) {$p_3$};

      \draw[pline] (p3) to (4.75, 3);
      \draw[pline] (p2) to (4.75, 2);
      \draw[pline] (p1) to (4.75, 1);

      \node[phantomstate] (s01) at (1, 1) {$\phantom{s_0}$};
      \node[phantomstate] (s02) at (1, 2) {$\phantom{s_0}$};
      \node[phantomstate] (s03) at (1, 3) {$\phantom{s_0}$};
      \node[phantomstate] (s11) at (2, 1) {$\phantom{s_2}$};
      \node[phantomstate] (s12) at (2, 2) {$\phantom{s_2}$};
      \node[phantomstate] (s13) at (2, 3) {$\phantom{s_1}$};
      \node[phantomstate] (s21) at (3, 1) {$\phantom{s_5}$};
      \node[phantomstate] (s22) at (3, 2) {$\phantom{s_4}$};
      \node[phantomstate] (s23) at (3, 3) {$\phantom{s_3}$};
      \node[phantomstate] (s31) at (4.25, 1) {$\phantom{s_7}$};
      \node[phantomstate] (s33) at (3.75, 3) {$\phantom{s_6}$};
      \node[state, s0color] (s01) at (1, 1) {\internaltext{s_0}};
      \node[state, s0color] (s02) at (1, 2) {\internaltext{s_0}};
      \node[state, s0color] (s03) at (1, 3) {\internaltext{s_0}};
      \node[state, s1color] (s13) at (2, 3) {\internaltext{s_1}};
      \node[state, s2color] (s11) at (2, 1) {\internaltext{s_2}};
      \node[state, s2color] (s12) at (2, 2) {\internaltext{s_2}};
      \node[state, s3color] (s23) at (3, 3) {\internaltext{s_3}};
      \node[state, s4color] (s22) at (3, 2) {\internaltext{s_4}};
      \node[state, s5color] (s21) at (3, 1) {\internaltext{s_5}};
      \node[state, s6color] (s33) at (3.75, 3) {\internaltext{s_6}};
      \node[state, s7color] (s31) at (4.25, 1) {\internaltext{s_7}};
      \draw[txnline] (s01) to node[txntext]{$u$} (s11);
      \draw[txnline] (s02) to node[txntext]{$u$} (s12);
      \draw[txnline] (s03) to node[txntext]{$t$} (s13);
      \draw[txnline] (s11) to node[txntext]{$w$} (s21);
      \draw[txnline] (s12) to node[txntext]{} (s22);
      \draw[txnline] (s13) to node[txntext]{$v$} (s23);
      \draw[txnline] (s13) to node[txntext]{} (s22);
      \draw[txnline] (s21) to node[txntext]{} (s31);
      \draw[txnline] (s22) to node[txntext]{} (s33);
      \draw[txnline] (s23) to node[txntext]{} (s33);
      \draw[txnline] (s33) to node[txntext]{} (s31);
    \end{tikzpicture}
  \end{center}

  \note{%
    For example, in this diagram, we see states $s_0$, $s_1$, $s_2$, all the
    way to $s_7$ are all reachable.
  }
\end{frame}

\begin{frame}
  \Large
  $O$ is \defword{\invariantconfluent} with respect to $s_0$, $T$, and $I$ if
  all reachable states satisfy the invariant:
  \[
    \setst{s \in S}{\sTIreachablepredicate{s}} \subseteq I
  \]

  \note{%
    Finally, an object is invariant confluent with respect to some start state
    $s_0$, set of transactions $T$, and invariant $I$ which I'll abbreviate to
    \sTIconfluent{} if all reachable states satisfy the invariant. That is, if
    set of reachable states is a subset of the invariant.

    In other words, if an object is invariant confluent, then it's impossible
    for us to reach a state that doesn't satisfy the invariant.

    Any questions?
  }
\end{frame}

\newcommand{\xmin}{-2}
\newcommand{\xmax}{2}
\newcommand{\ymin}{-2}
\newcommand{\ymax}{2}

% Axes.
\newcommand{\xyaxes}{
  \draw[] (\xmin.5, 0) to (\xmax.5, 0);
  \draw[] (0, \ymin.5) to (0, \ymax.5);
  \node at (\xmax + 1, 0) {$x$};
  \node at (0, \ymax + 1) {$y$};
}

% Quadrant 1.
\newcommand{\quadi}[5]{{
  \newcommand{\argstyle}{#1}
  \newcommand{\argxmin}{#2}
  \newcommand{\argxmax}{#3}
  \newcommand{\argymin}{#4}
  \newcommand{\argymax}{#5}
  \foreach \x in {0, ..., \argxmax} {
    \foreach \y in {0, ..., \argymax} {
      \node[\argstyle] (\x-\y) at (\x, \y) {};
    }
  }
}}

% Quadrant 2.
\newcommand{\quadii}[5]{{
  \newcommand{\argstyle}{#1}
  \newcommand{\argxmin}{#2}
  \newcommand{\argxmax}{#3}
  \newcommand{\argymin}{#4}
  \newcommand{\argymax}{#5}
  \foreach \x in {\argxmin, ..., 0} {
    \foreach \y in {0, ..., \argymax} {
      \node[\argstyle] (\x-\y) at (\x, \y) {};
    }
  }
}}

% Quadrant 3.
\newcommand{\quadiii}[5]{{
  \newcommand{\argstyle}{#1}
  \newcommand{\argxmin}{#2}
  \newcommand{\argxmax}{#3}
  \newcommand{\argymin}{#4}
  \newcommand{\argymax}{#5}
  \foreach \x in {\argxmin, ..., 0} {
    \foreach \y in {\argymin, ..., 0} {
      \node[\argstyle] (\x-\y) at (\x, \y) {};
    }
  }
}}

% Quadrant 4.
\newcommand{\quadiv}[5]{{
  \newcommand{\argstyle}{#1}
  \newcommand{\argxmin}{#2}
  \newcommand{\argxmax}{#3}
  \newcommand{\argymin}{#4}
  \newcommand{\argymax}{#5}
  \foreach \x in {0, ..., \argxmax} {
    \foreach \y in {\argymin, ..., 0} {
      \node[\argstyle] (\x-\y) at (\x, \y) {};
    }
  }
}}

% State labels.
\newcommand{\statelabels}{
  \node[statelabel] at (0, 0) {$s_0$};
  \node[statelabel] at (-1, 1) {$s_1$};
  \node[statelabel] at (1, -1) {$s_2$};
  \node[statelabel] at (1, 1) {$s_3$};
}

\tikzstyle{point}=[shape=circle, fill=flatgray, inner sep=3pt]
\tikzstyle{inv}=[line width=0.75pt, draw=black]
\tikzstyle{pointinv}=[point, inv]
\tikzstyle{invregion}=[rounded corners, fill=flatgreen!50, draw=none]
\tikzstyle{reachableregion}=[rounded corners, fill=flatblue!50, draw=none]
\tikzstyle{statelabel}=[anchor=south west, inner sep=1pt]



\begin{frame}
  \Large
  An example:
  \begin{itemize}
    \item $O = (\ints \times \ints, \max \times \max)$
    \pause \item $t_x((x, y)) = (x + 1, y)$
    \item $t_y((x, y)) = (x, y - 1)$
    \item $T = \set{t_x, t_y}$
    \pause \item $I = \setst{(x, y)}{xy \leq 0}$
    \pause \item $s_0 = (0, 0)$
  \end{itemize}
  Is $O$ \sTIconfluent{}?

  \note{%
    Let's take a look at an example. Consider the distributed object conisting
    of pairs of integers $(x, y)$ merged pairwise by max. We consider a
    transaction $t_x$ that increments $x$ part of a pair, and a transaction
    $t_y$ that decrements the $y$ part. Our invariant is that the product of
    $x$ and $y$ is non-positive.  Our start state is the pair $(0, 0)$. Is $O$
    invariant confluent with respect to this start state, set of transactions,
    and invariant?
  }
\end{frame}

\begin{frame}
  \begin{center}
    \begin{tikzpicture}
      \xyaxes{}
      \quadi{point}{\xmin}{\xmax}{\ymin}{\ymax}
      \quadiii{point}{\xmin}{\xmax}{\ymin}{\ymax}
      \quadii{point}{\xmin}{\xmax}{\ymin}{\ymax}
      \quadiv{point}{\xmin}{\xmax}{\ymin}{\ymax}

      \pause
      \draw[-latex, ultra thick] (0,0) to node[above] {$t_x$} (1, 0);
      \pause
      \draw[-latex, ultra thick] (0,0) to node[left] {$t_y$} (0, -1);
    \end{tikzpicture}
  \end{center}

  \note{%
    Well, this one is a bit hard to think about, but we can make the example
    more concrete if we think about it geometrically. Every state is a pair of
    integers, which we can think of as a point on a plane like this.

    The transaction $t_x$ takes in a point and spits out the point one to the
    right. The transaction $t_y$ takes in a point and spits out the point one
    below.
  }
\end{frame}

\begin{frame}
  \begin{center}
    \begin{tikzpicture}
      \xyaxes{}
      \quadi{point}{\xmin}{\xmax}{\ymin}{\ymax}
      \quadiii{point}{\xmin}{\xmax}{\ymin}{\ymax}
      \quadii{point}{\xmin}{\xmax}{\ymin}{\ymax}
      \quadiv{point}{\xmin}{\xmax}{\ymin}{\ymax}

      \node[pointinv] at (0, -1) {};
      \node[pointinv] at (2, -2) {};
      \draw[-latex, ultra thick] (0,-1) to (2, -1);
      \draw[-latex, ultra thick] (2,-2) to (2, -1);
    \end{tikzpicture}
  \end{center}

  \note{%
    The merge function, pairwise max, takes two points and spits out the upper
    right corner of the rectangle formed by the two points. Here, we see the
    point $(0, -1)$ and $(2, -2)$ merge to the point $(2, -1)$.
  }
\end{frame}

\begin{frame}
  \begin{columns}
    \begin{column}{0.5\textwidth}
      \centering
      \begin{tikzpicture}[scale=1]
        \begin{scope}
          \clip (\xmin.5, \ymax.5) rectangle (\xmax.5, \ymin.5);
          \draw[invregion] (\xmin.9, \ymax.9) rectangle (0.5, -0.5);
          \draw[invregion] (-0.5, 0.5) rectangle (\xmax.9, \ymin.9);
          \draw (0.5, 0.5) to (0.5, \ymax.5);
          \draw (0.5, 0.5) to (\xmax.5, 0.5);
          \draw (-0.5, -0.5) to (-0.5, \ymin.5);
          \draw (-0.5, -0.5) to (\xmin.5, -0.5);
        \end{scope}

        \xyaxes{}
        \quadi{point}{\xmin}{\xmax}{\ymin}{\ymax}
        \quadiii{point}{\xmin}{\xmax}{\ymin}{\ymax}
        \quadii{pointinv}{\xmin}{\xmax}{\ymin}{\ymax}
        \quadiv{pointinv}{\xmin}{\xmax}{\ymin}{\ymax}
      \end{tikzpicture}

      {\Huge Invariant} \\
      $\setst{(x, y)}{xy \leq 0}$
    \end{column}
    \begin{column}{0.5\textwidth}
      \pause
      \centering
      \begin{tikzpicture}[scale=1]
        \begin{scope}
          \clip (-1, 1) rectangle (\xmax.5, \ymin.5);
          \draw[reachableregion, draw=black] (-0.5, 0.5) rectangle (\xmax.9, \ymin.9);
        \end{scope}

        \xyaxes{}
        \quadi{point}{\xmin}{\xmax}{\ymin}{\ymax}
        \quadiii{point}{\xmin}{\xmax}{\ymin}{\ymax}
        \quadii{point}{\xmin}{\xmax}{\ymin}{\ymax}
        \quadiv{pointinv}{\xmin}{\xmax}{\ymin}{\ymax}
      \end{tikzpicture}

      {\Huge Reachable}
    \end{column}
  \end{columns}

  \note{%
    Our invariant was the set of points whose product is non-positive. Now, we
    see that this is just the second and fourth quadrant of the plane where one
    of $x$ or $y$ is negative but not both.

    Our start state is the origin. Starting at the origin, walk right using
    $t_x$, walking down using $t_y$, and merging points together, what states
    can we reach?  We can reach every point in the fourth quadrant. We can
    reach every point in the fourth quadrant from the origin by walking right,
    down, and merging points.

    Now, we see the set of points that satisfy the invariant and the set of
    points that are reachable, and the set of reachable points is a subset of
    the invariant satisfying points. Thus, our object is invariant confluent.
    We can only reach points that satisfy the invariant. We can never reach a
    point that does not satisfy the invariant.

    Any questions?
  }
\end{frame}
