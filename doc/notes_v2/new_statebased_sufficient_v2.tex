\section{New Sufficient State-Based Conditions}
\newcommand{\stimergereduction}[3]{$\Diamond(#1, #2, #3)$-merge-reduction}
\newcommand{\stimergereducible}[3]{$\Diamond(#1, #2, #3)$-merge-reducible}
\newcommand{\stimergereducibility}[3]{$\Diamond(#1, #2, #3)$-merge-reducibility}
\newcommand{\sTImergereduction}{\stimergereduction{s_0}{T}{I}}
\newcommand{\sTImergereducible}{\stimergereducible{s_0}{T}{I}}
\newcommand{\sTImergereducibility}{\stimergereducibility{s_0}{T}{I}}

\begin{definition}
  A \defword{join-free} expression $e = t_1(t_2(\ldots t_n(s) \ldots))$ is an
  expression that does not contain a join.
\end{definition}

\begin{definition}
  $O$ is \defword{merge-reducible with respect to $s_0$, $T$, and $I$},
  abbreviated \defword{\sTImergereducible{}}, if for every pair $e_1$ and $e_2$
  of join-free $\sTIreachable{}$ expressions, there exists some join-free
  $\sTIreachable{}$ expression $e_3$ such that $eval(e_3) = eval(e1 \join
  e_2)$.
\end{definition}

\begin{claim}\clmlabel{ReducibleImpliesIconfluent}
  If $I(s_0)$ and $O$ is \sTImergereducible, then $O$ is \sTIconfluent.
\end{claim}
\begin{elidableproof}
  Assume that $I(s_0)$ and that $O$ is \sTImergereducible{}. We prove by structural
  induction on $e$, that for all $\sTIreachable{}$ expressions $e$, there
  exists a join-free $\sTIreachable{}$ expression $e'$ such that $eval(e) =
  eval(e')$.
  \begin{itemize}
    \item \textbf{Case 1: $e = s_0$.}
      Trivially, $e' = s_0$.

    \item \textbf{Case 2: $e = t(e_1)$.}
      $e_1$ is $\sTIreachable{}$, so by the inductive hypothesis, there exists
      a join-free $\sTIreachable{}$ expression $e_1'$ such that $eval(e_1) =
      eval(e_1')$. $t(e_1)$ is $\sTIreachable{}$, so $I(t(e_1))$. $eval(e_1) =
      eval(e_1')$, so $I(t(e_1'))$. Thus, $t(e_1')$ is $\sTIreachable{}$ (and join
      free), so we can let $e' = t(e_1')$.

    \item \textbf{Case 3: $e = e_1 \join e_2$.}
      $e_1$ and $e_2$ are $\sTIreachable{}$, so by the inductive hypothesis,
      there exists equivalent join-free $\sTIreachable{}$ expressions $e_1'$
      and $e_2'$. $O$ is \sTImergereducible{}, so it's immediate that there exists
      an $e'$.
  \end{itemize}

  Consider an arbitrary $\sTIreachable{}$ expression $e$ and it's equivalent
  join-free $\sTIreachable{}$ counterpart $e'$. $e'$ is either $s_0$ or
  $t(e'')$.  In either case, it satisfies the invariant, so $O$ is
  \sTIconfluent{}.
\end{elidableproof}

\begin{claim}\clmlabel{NewStateBasedSufficient}
  Consider a state-based object $O = (S, \join)$, a start state $s_0$, a set of
  transactions $T$, and an invariant $I$. $T$ is \sTIconfluent{} if the
  following criteria are met:
  \begin{enumerate}
    \item
      $O$ is a semilattice.

    \item
      For every $t \in T$, there exists some $s_t \in S$ such that $t(s) = s
      \join s_t$.

    \item
      For every pair of transactions $t_1, t_2 \in T$ and for all states $s \in
      S$, if $I(s)$, $I(t_1(s))$, and $I(t_2(s))$, then $I(t_1(s) \join
      t_2(s))$.

    \item
      $I(s_0)$
  \end{enumerate}
\end{claim}
\begin{elidableproof}
  \newcommand{\bart}[1]{\overline{t_{#1}}}
  \newcommand{\baru}[1]{\overline{u_{#1}}}
  \newcommand{\bartu}[2]{\bart{#1}(\baru{#2}(s_0))}

  To show that $T$ is \sTIconfluent{}, it suffices to show that $T$ is
  \sTImergereducible{} (by \clmref{ReducibleImpliesIconfluent}).  Let
  \[
    e_1 = t_n(t_{n-1}(\ldots (t_1(s_0))\ldots))
    \quad \text{and} \quad
    e_2 = u_n(u_{m-1}(\ldots (u_1(s_0))\ldots))
  \]
  be two arbitrary join-free $\sTIreachable{}$ expressions that recursively.
  For ease of notation, let
  \[
    \bart{i} = t_i(\ldots(t_1(s_0))\ldots)
    \quad\text{and}\quad
    \baru{j} = u_j(\ldots(u_1(s_0))\ldots)
  \]
  We prove by strong induction on $k \in \nats$ that if $k = i + j$ where $0
  \leq i \leq n$ and $0 \leq j \leq m$, $\bartu{i}{j}$ is $\sTIreachable{}$ and
  $eval(\bartu{i}{j}) = eval(\bart{i}(s_0) \join \bart{j}(s_0))$.
  \begin{itemize}
    \item \textbf{Case $k = 0$.}
      $i = j = 0$, so $\bartu{0}{0} = s_0$ which is trivially $\sTIreachable{}$
      and equivalent to $\bart{0}(s_0) \join \baru{0}(s_0) = s_0 \join s_0$
      which evaluates to $s_0$.

    \item \textbf{Case $k = 1$.}
      Without loss of generality, assume $i = 1$ and $j = 0$. Then,
      $
        \bartu{1}{0} = t_1(s_0)
      $ which is $\sTIreachable{}$ because it is a subexpression of $\bart{n}$
      which is $\sTIreachable{}$. Moreover, it is equivalent to $\bart{1}(s_0)
      \join \baru{0}(s_0) = t_1(s_0) \join s_0$ which evaluates to $s_{t_1}
      \join s_0 \join s_0 = s_{t_1} \join s_0 = t_1(s_0)$.

    \item \textbf{Case $k \geq 2$.}
      If $i = 0$, then $j = k$ and $\baru{k}(s_0)$ is $\sTIreachable{}$ because
      it is a subexpression of $\baru{m}(s_0)$. Also, it is equivalent to
      $\bart{0}(s_0) \join \baru{k}(s_0)$ which evaluates to $\baru{k}(s_0)$.
      %
      If $j = 0$, then $i = k$ and $\bart{k}(s_0)$ is $\sTIreachable{}$ because
      it is a subexpression of $\bart{n}(s_0)$. Also, it is equivalent to
      $\bart{k}(s_0) \join \baru{0}(s_0)$ which evaluates to $\bart{k}(s_0)$.

      Otherwise, $i, j > 1$. Let
      \begin{align*}
        e_{i-1,j-1} &= \bartu{i-1}{j-1} \\
        e_{i,j-1} &= \bartu{i}{j-1} \\
        e_{i-1,j} &= \bartu{i-1}{j}
      \end{align*}
      By the inductive hypothesis, $e_{i-1,j-1}$, $e_{i,j-1}$, and $e_{i-1,j}$
      are all $\sTIreachable{}$. By condition 3 (with $s = eval(e_{i-1,j-1})$,
      $t_1$ = $t_i$, and $t_2 = u_j$), $I(e_{i,j-1} \join e_{i-1,j})$. Thus,
      $I(\bartu{i}{j})$, so $\bartu{i}{j}$ is $\sTIreachable{}$.
  \end{itemize}
\end{elidableproof}
