\section{Related Work}
% Other possibilities:
%   - Towards Fast Invariant Preservation in Geo-replicated Systems
%   - Feral Concurrency Control
%   - The CISE Tool: Proving Weakly-Consistent Applications Correct
%   - Declarative Programming over Eventually Consistent Data Stores
%   - Extending Eventually Consistent Cloud Databases for Enforcing Numeric Invariants
% In this section, we detail four alternate approaches to maintaining invariants
% without coordination: RedBlue consistency and Sieve, the homeostasis and
% demarcation protocols, explicit consistency, and token based invariant
% confluence. We then survey other bodies of related work.

\newcommand{\sieve}{\textsc{SIEVE}}
RedBlue consistency~\cite{li2012making}, is a consistency model that sits
between causal consistency and linearizability.
%
% With RedBlue consistency, every operation is manually labelled as either red or
% blue. All operations are executed with causal consistency, but with the added
% restrictions that red operations are executed in a single total order embedded
% within the causal ordering.
%
In~\cite{li2012making}, Li et al.\ introduce invariant safety as a
sufficient (but not necessary) condition for RedBlue consistent objects to be
\invariantconfluent{}. Invariant safety is an analog of \invariantclosure{}.
In~\cite{li2014automating}, Li et al.\ develop sophisticated techniques for
deciding invariant safety that involve calculating weakest preconditions.
These techniques are complementary to our work and can be used to improve the
\invariantclosure{} subroutine used by our decision procedures.
%
% In contrast with these techniques, our \invariantconfluence{} decision
% procedures can determine the \invariantconfluence{} of objects that are
% \emph{not} invariant safe.

% \textbf{The Demarcation and Homeostasis Protocols.}
The homeostasis protocol~\cite{roy2015homeostasis}, a generalization of the
demarcation protocol~\cite{barbara1994demarcation}, uses program analysis to
avoid unnecessary coordination between servers in a \emph{sharded} database
(whereas \invariantconfluence{} targets \emph{replicated} databases).
% The
% protocol guarantees that transactions are executed with observational
% equivalence with respect to some serial execution of the transactions. This
% means that intermediate states may be inconsistent, but externally observable
% side effects and the final database state are consistent.  The observational
% equivalence guaranteed by the homeostasis protocol is stronger than the
% guarantees of \invariantconfluence{}. As a result, there are invariants and
% workloads that the homeostasis protocol would execute with more coordination
% than a segmented \invariantconfluent{} execution. Moreover, the homeostasis and
% demarcation protocols' mechanism of establishing global invariants and
% operating without coordination so long as the invariants are maintained is very
% similar to our design of segmented \invariantconfluence{}.

% \textbf{Explicit Consistency.}
Explicit consistency~\cite{balegas2015towards} is a consistency model that
combines \invariantconfluence{} and causal consistency, similar to RedBlue
consistency with invariant safety.
% To determine if a workload is amenable to
% explicitly consistent replication, Balegas et al.\ determine if all pairs of
% transactions can be concurrently executed on the same start state without
% violating the invariant~\cite{balegas2015towards}. Balegas et al.\ argue that
% this is a sufficient condition for explicit consistency. It is similar to
% criterion (3) in \thmref{LatticeProperty}. In our work, we take a step further
% and explore sufficient \emph{and necessary} conditions for
% \invariantconfluence{}.
%
Balegas et al.\ also describe a variety of techniques---like conflict
resolution, locking, and escrow transactions~\cite{o1986escrow}---that can be
used to replicate workloads that do not meet their sufficient conditions.
Segmented \invariantconfluence{} is a formalism that can be
used to model simple forms of these techniques.

% \textbf{Token Based Invariant Confluence.}
In~\cite{gotsman2016cause}, Gotsman et al.\ discuss a hybrid token based
consistency model that generalizes a family of consistency models including
causal consistency, sequential consistency, and RedBlue consistency.
% An
% application designer defines a set of tokens and specifies which pairs of
% tokens conflict, and transactions acquire some subset of the tokens when they
% execute. This allows the application designer to specify which transactions
% conflict with one another.  Gotsman et al.\ develop sufficient conditions to
% determine whether a given token scheme is sufficient to guarantee that a global
% invariant is never broken.
The token based approach allows users to
specify certain conflicts that are not possible with segmented
\invariantconfluence{}.
% because a segmentation only allows transactions within a
% segment to acquire a single self-conflicting lock.
However, segmented
\invariantconfluence{} also introduces the notion of invariant segmentation,
which cannot be emulated with the token based approach.
For example, it is
difficult to emulate escrow transactions with the token based approach.

% \textbf{Serializable Distributed Databases.}
% In \secref{Evaluation}, we saw that segmented \invariantconfluent{} replication
% vastly outperforms linearizable replication for low coordination workloads, and
% it performs comparably or worse for medium and high coordination workloads.
% Distributed databases like Calvin~\cite{thomson2012calvin},
% Janus~\cite{mu2016consolidating}, and TAPIR~\cite{zhang2015building} employ
% algorithmic optimizations to implement serializable transactions with high
% throughput and low latency.
% %
% While segmented \invariantconfluent{} replication will likely always outperform
% serializable replication for low coordination workloads, these databases make
% serializable replication the most performant option for executing workloads
% that require a modest amount of coordination.

% \textbf{Branch and Merge.}
% Bayou~\cite{terry1995managing}, Dynamo~\cite{decandia2007dynamo}, and
% TARDiS~\cite{crooks2016tardis} all take a branch and merge approach to
% maintaining distributed invariants without coordination.
% With this approach,
% servers execute transactions without any coordination but keep track of the
% causal dependencies between transactions. Periodically, two servers merge
% states and invoke a user defined merge function to reconcile the divergent
% states.
% This approach does \emph{not} provide any formal guarantees that
% invariants are maintained.
% Its correctness depends on the correctness of the
% potentially complex user defined merge functions.

% \textbf{CRDTs.}
% CRDTs~\cite{shapiro2011conflict, shapiro2011comprehensive} are distributed
% semilattices with inflationary update methods. Due to their algebraic
% properties, CRDTs can be replicated with strong eventual consistency without
% the need for any coordination. Our definition of distributed objects and our
% \invariantconfluence{} system model are inspired directly by the corresponding
% definitions and system models in~\cite{shapiro2011conflict}.

%
% CRDTs are eventually consistent but may not preserve invariants. Conversely,
% \invariantconfluent{} objects preserve invariants but may not be eventually
% consistent. Thus, it is natural (though not necessary) to use CRDTs as
% distributed objects. If a CRDT is determined to be \invariantconfluent{} with
% respect to a particular invariant and set of transactions, then it achieves a
% combination of strong eventual consistency and invariant preservation. Any CRDT
% (e.g., counters, sets, graphs, sequences) can be used for this purpose.
% %
% Finally, our criteria in \thmref{LatticeProperty} also borrow ideas from CRDTs,
% exploiting the algebraic properties of semilattices.

% \textbf{CALM Theorem.}
% Bloom~\cite{alvaro2010boom, alvaro2011consistency, conway2012logic} and its
% formalism, Dedalus~\cite{alvaro2011dedalus, alvaro2013declarative}, are
% declarative Datalog-based programming languages that are designed to program
% distributed systems. The accompanying CALM
% theorem~\cite{hellerstein2010declarative, ameloot2013relational} states that if
% and only if a program can be written in the monotone fragment of these
% languages, then there exists a consistent, coordination-free implementation of
% the program.  The CALM theorem provides guarantees about the consistency of
% program outputs. It does not directly capture our notions of transactions or
% invariant maintenance during program execution.  Moreover, Bloom and Dedalus
% are general-purpose programming languages that can be used to implement a
% variety of distributed systems that are outside of the scope of
% \invariantconfluence{}.
