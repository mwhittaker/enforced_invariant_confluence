\section{Interactive Decision Procedure}\seclabel{InteractiveDecisionProcedure}
\newcommand{\Helper}{\textsc{Helper}}
\newcommand{\IsIclosed}{\textsc{IsIclosed}}
\newcommand{\IsInvConfluent}{\textsc{IsInvConfluent}}

\newcommand{\algocomment}[1]{\State \textcolor{flatdenim}{\texttt{//} #1}}
\begin{algorithm}[t]
  \caption{Interactive \invariantconfluence{} decision procedure}%
  \algolabel{InteractiveDecisionProcedure}
  \begin{algorithmic}
    \algocomment{Return if $O$ is \sTIconfluent{}.}
    \Function{IsInvConfluent}{$O$, $s_0$, $T$, $I$}
      \State
        \Return $I(s_0)$ and
        \Call{Helper}{$O$, $s_0$, $T$, $I$, $\set{s_0}$, $\emptyset$}
    \EndFunction

    \State

    \algocomment{$R$ is a set of \sTIreachable{} states.}
    \algocomment{$NR$ is a set of \sTIunreachable{} states.}
    \algocomment{$I(s_0)$ is a precondition.}
    \Function{Helper}{$O$, $s_0$, $T$, $I$, $R$, $NR$}
      \State closed, $s_1$, $s_2$ $\gets$ \Call{IsIclosed}{$O$, $I - NR$}
      \If {closed}
        \Return true
      \EndIf
      \State Augment $R, NR$ with random search and user input
      \If{$s_1, s_2 \in R$}
        \Return false
      \EndIf
      \State \Return \Call{Helper}{$O$, $s_0$, $T$, $I$, $R$, $NR$}
    \EndFunction
  \end{algorithmic}
\end{algorithm}

\thmref{IclosureEquivalentIconfluence} tells us that if all invariant
satisfying points are reachable, then \invariantclosure{} and
\invariantconfluence{} are equivalent. In this section, we present the
interactive \invariantconfluence{} decision procedure shown in
\algoref{InteractiveDecisionProcedure}, that takes advantage of this result.

\subsection{The Decision Procedure}
A user provides \algoref{InteractiveDecisionProcedure} with an object $O = (S,
\join)$, a start state $s_0$, a set of transactions $T$, and an invariant
$I$. The user then interacts with \algoref{InteractiveDecisionProcedure} to
iteratively eliminate unreachable states from the invariant. Meanwhile, the
algorithm leverages an \invariantclosure{} decision procedure to either (a)
conclude that $O$ is or is not \sTIconfluent{} or (b) provide counterexamples
to the user to help them eliminate unreachable states. After all unreachable
states have been eliminated from the invariant,
\thmref{IclosureEquivalentIconfluence} allows us to reduce the problem of
\invariantconfluence{} directly to the problem of \invariantclosure{}, and the
algorithm terminates.
%
We now describe \algoref{InteractiveDecisionProcedure} in detail. An example of
how to use \algoref{InteractiveDecisionProcedure} on \exampleref{Z2} is given
in \figref{DecisionProcedureExample}.

{\tikzstyle{point}=[shape=circle, fill=flatgray, inner sep=1.2pt, draw=black]
\tikzstyle{reachable}=[fill=flatgreen!50, draw=none]
\tikzstyle{unreachable}=[fill=flatred!50, draw=none]
\tikzstyle{invariant}=[fill=flatblue!50, draw=none]
\tikzstyle{nothing}=[fill=white, draw=none]

\newcommand{\pointgrid}[4]{{
  \newcommand{\argxmin}{#1}
  \newcommand{\argxmax}{#2}
  \newcommand{\argymin}{#3}
  \newcommand{\argymax}{#4}

  \draw[] (\argxmin, 0) to (\argxmax, 0);
  \draw[] (0, \argymin) to (0, \argymax);
  \foreach \x in {\argxmin, ..., \argxmax} {
    \foreach \y in {\argymin, ..., \argymax} {
      \node[point] (\x-\y) at (\x, \y) {};
    }
  }
}}

\newcommand{\pointrect}[2]{
  \draw[#1] ($#2 + (-0.51, 0.51)$) rectangle ($#2 + (0.51, -0.51)$);
}

\newcommand{\subfigwidth}{0.48\textwidth}
\newcommand{\subfighspace}{0.3cm}
\newcommand{\tikzhspace}{0.4cm}
\newcommand{\tikzscale}{0.25}
\newcommand{\xmin}{-3}
\newcommand{\xmax}{3}
\newcommand{\ymin}{-3}
\newcommand{\ymax}{3}

\begin{figure*}[ht]
  \centering

  \begin{subfigure}[t]{0.38\textwidth}
    \centering
    \begin{tikzpicture}[scale=\tikzscale]
      \draw[reachable, draw=black] (-0.5, 0.5) rectangle (0.5, -0.5);
      \pointgrid{\xmin}{\xmax}{\ymin}{\ymax}
      \node at (0, \ymax + 1) {$R$};
    \end{tikzpicture}\hspace{\tikzhspace}
    \begin{tikzpicture}[scale=\tikzscale]
      \pointgrid{\xmin}{\xmax}{\ymin}{\ymax}
      \node at (0, \ymax + 1) {$NR$};
    \end{tikzpicture}\hspace{\tikzhspace}
    \begin{tikzpicture}[scale=\tikzscale]
      \draw[invariant] (\xmin.5, \ymax.5) rectangle (0.5, -0.5);
      \draw[invariant] (-0.5, 0.5) rectangle (\xmax.5, \ymin.5);
      \draw (0.5, 0.5) to (0.5, \ymax.5);
      \draw (0.5, 0.5) to (\xmax.5, 0.5);
      \draw (-0.5, -0.5) to (-0.5, \ymin.5);
      \draw (-0.5, -0.5) to (\xmin.5, -0.5);
      \pointgrid{\xmin}{\xmax}{\ymin}{\ymax}
      \node at (0, \ymax + 1) {$I - NR$};
    \end{tikzpicture}
    \caption{%
      \IsInvConfluent{} determines $I(s_0)$ and then calls \Helper{} with $R =
      \set{s_0}$, $NR = \emptyset{}$, and $I = \setst{(x, y)}{xy \leq 0}$.
    }
    \figlabel{DecisionProcedureExampleA}
  \end{subfigure}
  \hspace{\subfighspace}
  \begin{subfigure}[t]{0.58\textwidth}
    \centering
    \begin{tikzpicture}[scale=\tikzscale]
      \draw[reachable, draw=black] (-0.5, 0.5) rectangle (1.5, -1.5);
      \pointgrid{\xmin}{\xmax}{\ymin}{\ymax}
      \node at (0, \ymax + 1) {$R$};
    \end{tikzpicture}\hspace{\tikzhspace}
    \begin{tikzpicture}[scale=\tikzscale]
      \pointrect{unreachable, draw=black}{(-1, 1)}
      \pointgrid{\xmin}{\xmax}{\ymin}{\ymax}
      \node at (0, \ymax + 1) {$NR$};
    \end{tikzpicture}\hspace{\tikzhspace}
    \begin{tikzpicture}[scale=\tikzscale]
      \draw[invariant] (\xmin.5, \ymax.5) rectangle (0.5, -0.5);
      \draw[invariant] (-0.5, 0.5) rectangle (\xmax.5, \ymin.5);
      \draw (0.5, 0.5) to (0.5, \ymax.5);
      \draw (0.5, 0.5) to (\xmax.5, 0.5);
      \draw (-0.5, -0.5) to (-0.5, \ymin.5);
      \draw (-0.5, -0.5) to (\xmin.5, -0.5);
      \pointrect{nothing, draw=black}{(-1, 1)}
      \pointgrid{\xmin}{\xmax}{\ymin}{\ymax}
      \node at (0, \ymax + 1) {$I - NR$};
    \end{tikzpicture}
    \caption{%
      \Helper{} determines that $O$ is not \iclosed{(I - NR)} with
      counterexample $s_1 = (-1, 1)$ and $s_2 = (1, -1)$. \Helper{} randomly
      generates some \sTIreachable{} points and adds them to $R$. Luckily for
      us, $s_2 \in R$, so \Helper{} knows that it is \sTIreachable{}.
      \Helper{} is not sure about $s_1$, so it asks the user. After some
      thought, the user tells \Helper{} that $s_1$ is \sTIunreachable{}, so
      \Helper{} adds $s_1$ to $NR$ and then recurses.
    }
    \figlabel{DecisionProcedureExampleB}
  \end{subfigure}

  \begin{subfigure}[t]{0.38\textwidth}
    \centering
    \begin{tikzpicture}[scale=\tikzscale]
      \draw[reachable, draw=black] (-0.5, 0.5) rectangle (2.5, -2.5);
      \pointrect{reachable, draw=black}{(3, -3)}
      \pointgrid{\xmin}{\xmax}{\ymin}{\ymax}
      \node at (0, \ymax + 1) {$R$};
    \end{tikzpicture}\hspace{\tikzhspace}
    \begin{tikzpicture}[scale=\tikzscale]
      \pointrect{unreachable}{(-1, 1)}
      \pointrect{unreachable}{(-1, 2)}
      \draw (-1.5, 0.5) rectangle (-0.5, 2.5);
      \pointgrid{\xmin}{\xmax}{\ymin}{\ymax}
      \node at (0, \ymax + 1) {$NR$};
    \end{tikzpicture}\hspace{\tikzhspace}
    \begin{tikzpicture}[scale=\tikzscale]
      \draw[invariant] (\xmin.5, \ymax.5) rectangle (0.5, -0.5);
      \draw[invariant] (-0.5, 0.5) rectangle (\xmax.5, \ymin.5);
      \draw (0.5, 0.5) to (0.5, \ymax.5);
      \draw (0.5, 0.5) to (\xmax.5, 0.5);
      \draw (-0.5, -0.5) to (-0.5, \ymin.5);
      \draw (-0.5, -0.5) to (\xmin.5, -0.5);
      \pointrect{nothing}{(-1, 1)}
      \pointrect{nothing}{(-1, 2)}
      \draw (-1.5, 0.5) rectangle (-0.5, 2.5);
      \pointgrid{\xmin}{\xmax}{\ymin}{\ymax}
      \node at (0, \ymax + 1) {$I - NR$};
    \end{tikzpicture}
    \caption{%
      \Helper{} determines that $O$ is not \iclosed{(I - NR)} with
      counterexample $s_1 = (-1, 2)$ and $s_2 = (3, -3)$. \Helper{} randomly
      generates some \sTIreachable{} points and adds them to $R$. $s_1, s_2
      \notin R, NR$, so \Helper{} ask the user to label them. The user puts
      $s_1$ in $NR$ and $s_2$ in $R$. Then, \Helper{} recurses.
    }
    \figlabel{DecisionProcedureExampleC}
  \end{subfigure}
  \hspace{\subfighspace}
  \begin{subfigure}[t]{0.58\textwidth}
    \centering
    \begin{tikzpicture}[scale=\tikzscale]
      \draw[reachable] (-0.5, 0.5) rectangle (2.5, -2.5);
      \pointrect{reachable}{(3, 0)}
      \pointrect{reachable}{(3, -1)}
      \pointrect{reachable}{(0, -3)}
      \pointrect{reachable}{(2, -3)}
      \pointrect{reachable}{(3, -3)}
      \draw (-0.5, 0.5) to (\xmax.5, 0.5);
      \draw (-0.5, 0.5) to (-0.5, \ymin.5);
      \draw (-0.5, -3.5) to ++(1, 0)
                         to ++(0, 1)
                         to ++(1, 0)
                         to ++(0, -1)
                         to ++(2, 0)
                         to ++(0, 1)
                         to ++(-1, 0)
                         to ++(0, 1)
                         to ++(1, 0)
                         to ++(0, 2);
      \pointgrid{\xmin}{\xmax}{\ymin}{\ymax}
      \node at (0, \ymax + 1) {$R$};
    \end{tikzpicture}\hspace{\tikzhspace}
    \begin{tikzpicture}[scale=\tikzscale]
      \draw[unreachable] (\xmin.5, \ymax.5) rectangle (-0.5, \ymin.5);
      \draw (\xmin.5, \ymax.5) to (-0.5, \ymax.5) to
            (-0.5, \ymin.5) to (\xmin.5, \ymin.5);
      \pointgrid{\xmin}{\xmax}{\ymin}{\ymax}
      \node at (0, \ymax + 1) {$NR$};
    \end{tikzpicture}\hspace{\tikzhspace}
    \begin{tikzpicture}[scale=\tikzscale]
      \draw[invariant] (-0.5, 0.5) rectangle (\xmax.5, \ymin.5);
      \draw (-0.5, 0.5) to (\xmax.5, 0.5);
      \draw (-0.5, 0.5) to (-0.5, \ymin.5);
      \pointgrid{\xmin}{\xmax}{\ymin}{\ymax}
      \node at (0, \ymax + 1) {$I - NR$};
    \end{tikzpicture}
    \caption{%
      \Helper{} determines that $O$ is not \iclosed{(I - NR)} with
      counterexample $s_1 = (-2, 1)$ and $s_2 = (1, -1)$. \Helper{} randomly
      generates some \sTIreachable{} points and adds them to $R$. $s_2 \in R$
      but $s_1 \notin R, NR$, so \Helper{} asks the user to label $s_1$. The
      user notices a pattern in $R$ and $NR$ and after some thought, concludes
      that every point with negative $x$-coordinate is \sTIunreachable{}.
      They update $NR$ to $-\ints \times \ints$. Then, \Helper{} recurses.
      \Helper{} determines that $O$ is \iclosed{(I - NR)} and returns true!
    }
    \figlabel{DecisionProcedureExampleD}
  \end{subfigure}

  \vspace{1em}
  \hrule
  \vspace{1em}

  \caption{%\vspace{1em}
    An example of a user interacting with
    \algoref{InteractiveDecisionProcedure} on \exampleref{Z2}.  Each step of
    the visualization shows reachable states $R$ (left), non-reachable states
    $NR$ (middle), and the refined invariant $I - NR$ (right) as the algorithm
    executes
  }
  \figlabel{DecisionProcedureExample}
\end{figure*}
}

\IsInvConfluent{} assumes access to an \invariantclosure{} decision procedure
\IsIclosed$(O, I)$. \IsIclosed$(O, I)$ returns a triple (closed, $s_1$, $s_2$).
closed is a boolean indicating whether $O$ is \Iclosed{}. If closed is true,
then $s_1$ and $s_2$ are null. If closed is false, then $s_1$ and $s_2$ are a
counterexample witnessing the fact that $O$ is not \Iclosed{}. That is,
$I(s_1)$ and $I(s_2)$, but $\lnot I(s_1 \join s_2)$ (e.g., $s_1$ and $s_2$ from
\exampleref{Z2}).  As we mentioned earlier, we can (and do) implement the
\invariantclosure{} decision procedure using an SMT solver like
Z3~\cite{de2008z3}.

\IsInvConfluent{} first checks that $s_0$ satisfies the invariant. $s_0$ is
reachable, so if it does not satisfy the invariant, then $O$ is not
\sTIconfluent{} and \IsInvConfluent{} returns false. Otherwise,
\IsInvConfluent{} calls a helper function \Helper{} that---in addition to $O$,
$s_0$, $T$, and $I$---takes as arguments a set $R$ of \sTIreachable{} states
and a set $NR$ of \sTIunreachable{} states. Like \IsInvConfluent, \Helper$(O,
s_0, T, I, R, NR)$ returns whether $O$ is \sTIconfluent{} (assuming $R$ and
$NR$ are correct).  As \algoref{InteractiveDecisionProcedure} executes, $NR$ is
iteratively increased, which removes unreachable states from $I$ until $I$ is a
subset of $\setst{s \in S}{\sTIreachablepredicate{s}}$.

First, \Helper{} checks to see if $O$ is \iclosed{(I - NR)}.
%
If \IsIclosed{} determines that $O$ is \iclosed{(I - NR)}, then by
\thmref{IclosureImpliesIconfluence}, $O$ is \sticonfluent{s_0}{T}{I-NR}, so
\[
  \setst{s \in S}{\stireachablepredicate{s_0}{T}{I-NR}{s}}
    \subseteq I - NR
    \subseteq I
\]
Because $NR$ only contains \sTIunreachable{} states, then the set of
\sTIreachable{} states is equal to set of \stireachable{s_0}{T}{I-NR}{} states
which, as we just showed, is a subset of $I$.  Thus, $O$ is \sTIconfluent{}, so
\Helper{} returns true.

If \IsIclosed{} determines that $O$ is \emph{not} \iclosed{(I - NR)}, then we
have a counterexample $s_1, s_2$. We want to determine whether $s_1$ and $s_2$
are reachable or unreachable. We can do so in two ways.
%
First, we can randomly generate a set of reachable states and add them to $R$.
If $s_1$ or $s_2$ is in $R$, then we know they are reachable.
%
Second, we can prompt the user to tell us directly whether or not the states
are reachable or unreachable.

In addition to labelling $s_1$ and $s_2$ as reachable or unreachable, the user
can also refine $I$ by augmenting $R$ and $NR$ arbitrarily (see
\figref{DecisionProcedureExample} for an example). In this step, we also make
sure that $s_0 \notin NR$ since we know that $s_0$ is reachable.

After $s_1$ and $s_2$ have been labelled as \sTIreachable{} or not, we
continue. If both $s_1$ and $s_2$ are \sTIreachable{}, then so is $s_1 \join
s_2$, but $\lnot I(s_1 \join s_2)$. Thus, $O$ is not \sTIconfluent{}, so
\Helper{} returns false. Otherwise, one of $s_1$ and $s_2$ is
\sTIunreachable{}, so we recurse.

\Helper{} recurses only when one of $s_1$ or $s_2$ is unreachable, so $NR$
grows after every recursive invocation of \Helper{}. Similarly, $R$ continues
to grow as \Helper{} randomly explores the set of reachable states. As the user
sees more and more examples of unreachable and reachable states, it often becomes
easier and easier for them to recognize patterns that define which states are
reachable and which are not. As a result, it becomes easier for a user to
augment $NR$ and eliminate a large number of unreachable states from the
invariant. Once $NR$ has been sufficiently augmented to the point that $I - NR$
is a subset of the reachable states, \thmref{IclosureEquivalentIconfluence}
guarantees that the algorithm will terminate after one more call to \IsIclosed.

\subsection{Limitations}
Our interactive \invariantconfluence{} decision procedure has two
limitations. First, \algoref{InteractiveDecisionProcedure} requires an
\invariantclosure{} decision procedure, but determining \invariantclosure{} is
undecidable in general.
%
\begin{techreport}

  For example, let $O_p = (S, \join)$ where $S$ is the set of all programs and
  $s_1 \join s_2 = p$ for some fixed program $p$. Letting $I$ be the set of all
  programs that terminate, determining if $O_p$ is \Iclosed{} is tantamount to
  determining if $p$ terminates.
\end{techreport}
%
In practice, we can implement an \invariantclosure{} decision procedure using
an SMT solver like Z3 that works well on simple objects, invariants, and merge
operators (e.g., integers, tuples, infinite sets, bit vectors, linear
constraints, basic arithmetic, tuple projection, basic set operations, bit
arithmetic). But, SMT solvers are mostly unable to analyze more complex
constructs (e.g., finite lists~\cite{kroning2009proposal}, graphs, recursive
algebraic data types, nonlinear constraints, merge operators that contain loops
or recursion).

Second, \algoref{InteractiveDecisionProcedure} relies on a user to identify
unreachable states. As we saw in \figref{DecisionProcedureExample}, the set of
unreachable states can sometimes be clear, especially if there's a noticeable
pattern in the set of reachable states. However, if the set of transactions is
large or complex or if the merge operator is complex, then reasoning about
unreachable states can be difficult.  Unlike with reachable states---where
verifying that a state is reachable only requires thinking of a single way to
reach the state---verifying that a state is unreachable requires a programmer
to reason about a large number of system executions and conclude that
\emph{none} of them can lead to the state. In the future, we plan on exploring
ways to help a user reason about unreachable states and ways to discover sets of
unreachable states automatically.

\begin{techreport}
  \subsection{Tolerating User Error}
  \algoref{InteractiveDecisionProcedure} is an \emph{interactive} decision
  procedure. It requires that a user classify states as reachable or
  unreachable.  To err is human, so what happens when a user incorrectly
  classifies a state?  There are two possible errors that can be made, and
  \algoref{InteractiveDecisionProcedure} can be made robust to both.

  \textbf{A user can label an unreachable state as reachable.}
  In this case, \Helper{} might erroneously find $s_1$ and $s_2 \in R$ and
  return false, concluding that $O$ is not \sTIconfluent{} even when it is.
  This is inconvenient, but not catastrophic. We can modify
  \algoref{InteractiveDecisionProcedure} so that \Helper{} requires that
  whenever a user labels a state $s$ as \sTIreachable{}, they must also provide
  an \sTIreachable{} expression $e$ that evaluates to $s$. Here, $e$ acts a
  proof that certifies that $s$ is indeed reachable. This increases the burden
  on the user but completely eliminates this type of user error.

  \textbf{A user can label a reachable state as unreachable.}
  In this case, \IsIclosed$(O, I - NR)$ might return true, even though $O$ is
  not \sTIconfluent. Thus, a user might falsely believe their distributed
  object to be \sTIconfluent{} even though it isn't, and eventually one replica
  of their distributed object might enter a state that violates the invariant.
  %
  We can mitigate this in two ways. First, we can aggressively expand $R$
  automatically. If a user ever labels a state $s$ as unreachable, but $s \in
  R$, we can notify the user of their mistake. Second, \Helper{} returns true
  when $O$ is \iclosed{(I - NR)} for some $NR$, so $O$ \emph{is}
  \sticonfluent{s_0}{T}{I-NR} (even though it might not be \sTIconfluent{}).
  Thus, when a user replicates their distributed object across a set of
  servers, they can deploy with the invariant $I - NR$ instead of $I$. If $NR$
  is correct and only contains unreachable states, then deploying with $I - NR$
  is equivalent to deploying with $I$. If $NR$ is incorrect and contains some
  \sTIreachable{} states, then some of these states are artificially made
  unreachable, but the system is still guaranteed to never produce a state
  that violates $I$.
\end{techreport}
