\documentclass[9pt]{article}

\usepackage[hidelinks]{hyperref}
\usepackage{enumitem}
\usepackage{float}
\usepackage{invariantconfluence}
\usepackage{multicol}
\usepackage{pervasives}
\usepackage{vldblite}

% Reviewer quotes.
\definecolor{ReviewerDarkGray}{HTML}{37474F}
\newenvironment{reviewerquote}
{\list{}{\leftmargin=\parindent\rightmargin=0in}\item[] \itshape \color{ReviewerDarkGray}}%
{\endlist}

\newtoggle{showfeedback}
\toggletrue{showfeedback}
% \togglefalse{showfeedback}
\NewEnviron{feedback}{
  \iftoggle{showfeedback}{
    \begin{itemize}[leftmargin=*]
      \item[] \itshape \textcolor{ReviewerDarkGray}{\BODY}
    \end{itemize}
  }{}
}

% Turn on the full reviewer comments if the turnonshowfeedback flag was passed
% in from the command line. This was taken from [1].
%
% [1]: https://stackoverflow.com/a/1466610/3187068
% \ifdefined\turnonshowfeedback
%   \toggletrue{showfeedback}
% \else
%   \togglefalse{showfeedback}
% \fi

\begin{document}
\begin{center}
  \ttlfnt Response to Reviewer Comments
\end{center}

\begin{multicols*}{2}
Below, you'll find our response to reviewer comments and our revisions. All
changes made to the paper are shown in \markrevisions{pink}.

%%%%%%%%%%%%%%%%%%%%%%%%%%%%%%%%%%%%%%%%%%%%%%%%%%%%%%%%%%%%%%%%%%%%%%%%%%%%%%%
\section*{Overview}
Our original publication on \invariantconfluence{} is largely theoretical. We
provide new, and arguably improved, definitions of \invariantconfluence{}. We
characterize the relationship between \invariantclosure{} and
\invariantconfluence{}. We exploit the relationship to develop an
\invariantclosure{} decision procedure. We invent the notion of
\mergereducibility{} and characterize its relationship with
\invariantclosure{}. We invent the notion of segmented \invariantconfluence{},
and develop another decision procedure.

When our paper was invited to this journal, we took the opportunity to improve
the rigor and clarity of the core contributions of the paper. We added numerous
proofs to verify the theorems in the paper, and we added many examples and
illustrations providing valuable intuition behind the key results. We also
added a significant amount of new material:
\begin{itemize}
  \item (Section 2.3)
    In our conference paper, we present a definition of invariant confluence
    that is slightly different than the original definition from Bailis et
    al.~\cite{bailis2014coordination}, but we do not comment on the difference.
    We added a subsection (Section 2.3) clarifying that the difference in
    definitions is superficial. We describe Bailis' original definition using
    our formalism and prove that the two definitions are equivalent.

  \item (Section 4.3)
    One of the core contributions of our paper is an interactive invariant
    confluence decision procedure. Here, ``interactive'' means that the
    decision procedure relies on human input to guide the decision procedure.
    It's possible for a person to provide the decision procedure with incorrect
    information. We added a subsection (Section 4.3) describing how to mitigate
    these errors.

  \item (Section 6.4)
    We added a paragraph explaining the relationship between segmented
    invariant confluence and a distributed locking approach to maintaining
    invariants (that can be found in related works).

  \item (Section 6.4)
    We added a paragraph explaining a couple of optimizations that can be made
    to our segmented invariant confluence system model to reduce the frequency
    of global coordination.

  \item (Section 6.4)
    We added a paragraph explaining how a segmented invariant confluence
    decision procedure can leverage our merge reducibility results in addition
    to our results on invariant closure.

  \item (Section 6.4)
    We clarified the fault tolerance limitations of segmented invariant
    confluence. A naive implementation of our segmented invariant confluence
    decision procedure does not guarantee liveness, but there are
    straightforward techniques we can leverage to introduce liveness. We added
    an explanation of this to the paper.

  \item (Section 7)
    Our conference paper presents formal definitions of distributed objects and
    invariant confluence. The definitions revolve around servers gossiping
    state to one another and merging the states together using a binary merge
    operator. This approach is called a ``state-based'' approach. We added a
    section (Section 7) outlining a dual ``operation-based'' approach in which
    transactions, rather than states, are gossiped. This duality between
    state-based and operation-based definitions is well established in the CRDT
    literature.
\end{itemize}

The requested revisions, however, fall mainly into two categories. First, we
have requests to expand our empirical evaluation. Unfortunately, an expanded
evaluation requires a significant amount of work for very little additional
technical contribution. The applications of our theoretical results are
important (which is why we included them in the paper originally), but it is
not the focus of the paper. We believe that future work focused more on the
application of \invariantconfluence{} would leverage the key insights of our
paper and would benefit from the increased clarity and rigor of our journal
submission.

Second, we have requests to complete some of our suggested future work, with
arguments that the two years since our original publication should be enough
time to do so. We disagree that the time since our original publication should
have any bearing on our journal submission. Since our publication, we have been
working in new research areas. We have not spent the last two years working on
\invariantconfluence{}. Moreover, the suggested future work involves a
significant research undertaking that would involve solving many fundamental
research problems with implications well beyond \invariantconfluence{}. We
believe that any such results would more than warrant a publication on their
own.

%%%%%%%%%%%%%%%%%%%%%%%%%%%%%%%%%%%%%%%%%%%%%%%%%%%%%%%%%%%%%%%%%%%%%%%%%%%%%%%
\section*{Reviewer 1}
\paragraph{Reviewer 1: W1}
\begin{feedback}
  More details about the DSL to specify the invariants with more complex
  examples would add a lot of value to the paper
\end{feedback}
We added a link to our DSL implementation on GitHub to the beginning of Section
8. We also added a bit more explanation of the DSL to Section 8.1.  The paper
already has over 10 examples of invariant confluence with varying degrees of
complexity from simple (e.g., Example 10) to complex (e.g., Example 14), and we
feel that additional examples will offer little additional technical insight.

\paragraph{Reviewer 1: W2}
\begin{feedback}
  More experiments with some realistic workloads (like TPCC) to show the
  benefit of the proposed concepts in action are needed
\end{feedback}
We agree that having a number of realistic workloads in the paper would be
beneficial, but specifying, verifying, and explaining every workload requires a
significant amount of work for a marginal increase in the clarity of the paper.
The focus of the paper is the foundations, not the applications of these ideas.

\paragraph{Reviewer 1: W3}
\begin{feedback}
  Even though the automatic segmentation is left for future work, an example
  with a manual mapping of segments in a realistic workload (like TPCC) would
  show-case its potential
\end{feedback}
See above.

\paragraph{Reviewer 1: W4}
\begin{feedback}
  The main weak point of this specific submission is whether it has enough new
  material to to qualify for being an extended version
\end{feedback}
See our Overview section above.

\paragraph{Reviewer 1: D1}
\begin{feedback}
  W4 touches upon the ``new material'' for this extended version of the paper,
  and the above weak points W1, W2 and W3 can all serve as additional material.
\end{feedback}
See W1, W2, W3, and D6.

\paragraph{Reviewer 1: D2}
\begin{feedback}
  W1 will essentially help the reader to quickly see the value of the proposed
  definitions and theorems with some specification examples that will be able
  to capture more interesting use-cases.
\end{feedback}
See W1.

\paragraph{Reviewer 1: D3}
\begin{feedback}
  W2 would help to quantify the benefits that come from being able to have some
  invariant confluent objects in an otherwise strongly consistent setting.
\end{feedback}
See W2.

\paragraph{Reviewer 1: D4}
\begin{feedback}
  W3 would really help to showcase the value of segmentation in practice.
  Ideally, we want to see that in a specific application there is a logical
  mapping between segments and data, and then use this work as an example for
  potential future work that will be fully automated as the authors suggest.
\end{feedback}
See W3.

\paragraph{Reviewer 1: D5}
\begin{feedback}
  In fact, this automation would definitely strength the paper dramatically if
  it were added (but it is not required for this extended version to be enough)
\end{feedback}
We think that automated segmentation would be such a significant contribution
that it would warrant a publication on its own.

\paragraph{Reviewer 1: D6}
\begin{feedback}
  It is clear that the authors added more material to the journal paper,
  however, it does not substantially augment the messages of the paper --
  addressing W1-W3 will help in that direction (for example section 7 ends
  without any enough discussion of the practical connections of the proposed
  idea)
\end{feedback}
See our Overview section above.

\paragraph{Reviewer 1: D7}
\begin{feedback}
  In the intro we read about ``the trade-off between application complexity and
  coordination-freedom'', it would be great to see some practical examples later
  in the paper
\end{feedback}
The trade-off between application complexity and coordination-freedom is shown
in Examples 2, 6, 12, and 13. Very simple examples (Example 2) don't require
any segmentation. More complex initial states (Example 6), more complex
transactions (Example 12), and more complex optimizations (Example 13) require
segmentation. The trade-off we describe in the introduction is written to give
some intuition to the reader, but ultimately the notion of complexity is
subjective, and the number of segments of an application is up to the user.

\paragraph{Reviewer 1: D8}
\begin{feedback}
  In section 4.2, more details about what can be done with respect to help a
  user reason about unreachable states would be great
\end{feedback}
See Reviewer 3: Section 4.2.

\paragraph{Reviewer 1: D9}
\begin{feedback}
  Figure 5 should be moved before figure 6
\end{feedback}
Fixed!

\paragraph{Reviewer 1: D10}
\begin{feedback}
  in page 3 we read ``The culprit is concurrent decrements, which we can forbid
  in a simple one-segment segmentation ...'', this sentence does not parse well
\end{feedback}
We reworded the sentence to be easier to read.

\paragraph{Reviewer 1: D11}
\begin{feedback}
  In 6.4 we read ``This example is exactly the same as Example 8 which we
  handled by simply removing withdrawal transactions from our segmentation's
  set of transactions.'' How can one identify transactions to remove, and is
  the removal practically done?
\end{feedback}
In the current work, users must manually identify transactions to remove. More
generally, identifying a transaction to remove from a segmentation is a special
case of forming a segmentation, and we do not provide any automated assistance
on how to form a segmentation. Is the removal practically done? That depends on
the workload. For some workloads, identifying invariant-violating transactions
can be easy, like the bank withdrawal example (Example 8). Other, more complex
workloads, can be difficult to reason about. Forming segmentations for these
workloads is not easy. We think that automatically forming segmentations is an
interesting avenue for future research.

\paragraph{Reviewer 1: D12}
\begin{feedback}
  In 8.1 we read ``When a server receives a coordination request from the
  leader, it stops processing transactions and sends the leader its state.''
  what happens to the in-flight transactions? Are the cancelled or allowed to
  finish? Is the system simply not admitting new transactions?
\end{feedback}
Thanks for catching this. We added some clarification to Section 8.1.  In
short, in-flight transactions are deferred until after the round of
coordination is done. They aren't cancelled, and they are allowed to finish,
but only after the coordination is done. The entire system stops processing
transactions temporarily. Technically, the system can still \emph{admit} new
transactions during coordination, but it cannot execute them.

Your question also reinforces the conclusion of Figure 10. If coordination is
frequent, then segmented \invariantconfluence{} becomes slow, even slower than
linearizable replication. Segmented \invariantconfluent{} works well only when
coordination is infrequent.

\paragraph{Reviewer 1: D13}
\begin{feedback}
  In 8.1 again, at the end, more discussion about the tradeoff of
  fault-tolerance would be interesting.
\end{feedback}
See Reviewer 4: 5, 7.

\paragraph{Reviewer 1: Typos}
\begin{feedback}
  end of first column of page 3 ``that it execute a transaction'' $\to$ ``that
  it executes a transaction''

  section 5, in example 3, ``Our start state state'' $\to$ two times state

  section 8.2 in example 12: ``and a optional'' $\to$ ``and an optional''
\end{feedback}
Fixed.

%%%%%%%%%%%%%%%%%%%%%%%%%%%%%%%%%%%%%%%%%%%%%%%%%%%%%%%%%%%%%%%%%%%%%%%%%%%%%%%
\section*{Reviewer 3}
\paragraph{Reviewer 3: Overview}
\begin{feedback}
  This manuscript is an extension of the VLDB 2019 conference version of
  ``Interactive Checks for Coordination Avoidance''. The original PVLDB paper
  is a solid piece of work that is well written and easy to follow. It was
  published in September 2018. In this manuscript, the authors elaborated more
  by adding more examples, proofs, and figures to make it easier for the reader
  to understand specific concepts and definitions. Some readers may find that
  the original paper is clear and enough without these additions. Moreover, the
  authors added two new sections, namely Section 4.3 and 7, which do not show a
  significant improvement to the original contribution. In almost two years, it
  was expected that the authors will address some of the future work mentioned
  in the original paper, such as ``In the future, we plan on exploring ways to
  help a user reason about unreachable states and ways to discover sets of
  unreachable states automatically.'' Overall, the authors did not manage to
  show that the additions were significant enough to warrant publication.
\end{feedback}
We address this feedback in detail below. We also disagree that the time
between publication and now should have any bearing on the revisions.

\paragraph{Reviewer 3: Section 2.1}
\begin{feedback}
  The authors revised a small part of this section to clarify that the merge
  operator of a distributed object does not have to satisfy any special
  mathematical properties like associativity and commutativity.
\end{feedback}
Yup. This was requested by many readers who assumed early on that the merge
operator did require some special properties. In Bailis et al.'s original
paper, for example, they assume the merge operator is associative, commutative,
and idempotent~\cite{bailis2014coordination}.

\paragraph{Reviewer 3: Section 2.3}
\begin{feedback}
  The authors added this section to prove that their definition of invariant
  confluence and the original definition from Bailis et al.\ are equivalent.
  The authors did not clarify why it is important to prove that both
  definitions are equivalent.
\end{feedback}
We believe that proving equivalent definitions are, in fact, equivalent is a
fundamental tenant of theoretical research. Without a proof of the two
definitions being equivalent, it's possible that our notion of
\invariantconfluence{} and Bailis et al.'s original notion differ.  If so, none
of our theorems or decision procedures could be applied to Bailis et al.'s
findings. We'd be talking about two completely different concepts.  Proving
they are equivalent unifies the work. Without this knowledge, proofs using the
different definitions would be mutually incompatible.

\paragraph{Reviewer 3: Section 3}
\begin{feedback}
  The authors revised a small part of the original section to be in the form of
  a proof. For example in the PVLDB paper, the authors said: ``The forward
  direction of Theorem 2 follows immediately from Theorem 1. The backward
  direction holds because any two invariant satisfying states s1 and s2 must be
  reachable (by assumption), so their join s1 t s2 is also reachable. And
  because O is (s0, T, I)-confluent, all reachable points, including s1 t s2,
  satisfy the invariant.'' And in the VLDBJ submission, the authors said:
  ``Proof. The forward direction of Theorem 3 follows immediately from Theorem
  2. The backward direction holds because any two invariant satisfying states
  s1 and s2 must be reachable (by assumption), so their join s1 ⊔ s2 is also
  reachable. And because O is (s0, T, I)- confluent, all reachable points,
  including s1 ⊔s2, satisfy the invariant. Moreover s0 is reachable, so I(s0)
  because O is (s0,T,I)-confluent.''
\end{feedback}
Yup, a minor but (we hope) useful improvement.

\paragraph{Reviewer 3: Section 4.1}
\begin{feedback}
  the authors expanded their correctness proof of the interactive decision
  procedure and added a few other minor elaborations. The original proof in the
  PVLDB paper was clear enough to the reader.
\end{feedback}
We're happy that the original proof was clear to the reviewer, but we felt it
was not clear to most people, including us.

\paragraph{Reviewer 3: Section 4.2}
\begin{feedback}
  In Section 4.2, the authors added a very brief example showing why invariant
  closure is undecidable. The reader can find original section 4.2 is clear
  enough. Instead in both versions of section 4.2, i.e., PVLDB and VLDBJ, the
  authors said: ``In the future, we plan on exploring ways to help a user
  reason about unreachable states and ways to discover sets of unreachable
  states automatically.'' After almost two years of publishing the PVLDB
  version of this submission, this manuscript would be much stronger, if the
  authors explored these ways.
\end{feedback}
Automatically discovering unreachable states is hard, really hard. As we
describe in the paper, it is easy to verify that a state is reachable, simply
give an expression that evaluates to it. Verifying that a state is unreachable
is significantly harder, as we have to create a proof showing that there does
not exist any execution in which the state is reached. Open mathematical
problems like the Collatz conjecture~\cite{mottaanalysis} show that reasoning
about the reachability or unreachability of even simple programs can be
tremendously difficult. If we developed any substantial techniques to identify
unreachable states in a general way, we feel that would constitute a strong
publication on its own.

\paragraph{Reviewer 3: Section 4.3}
\begin{feedback}
  Section 4.3 is newly added to clarify that the decision procedure relies on
  human input to guide the decision procedure and explain how to mitigate these
  errors.
\end{feedback}
Yes, and we feel this is one of the core additions to the paper. The two
biggest drawbacks of interactivity are (1) the time and labor required of a
user and (2) the possibility of introducing errors. We break down the two types
of user errors (i.e.\ labelling an unreachable state as reachable and labelling
a reachable state as unreachable) and provide solutions for both, eliminating
the possibility of an erroneously verified \invariantconfluent{} object from
violating an invariant in deployment. This addition is very important for
anyone deploying \invariantconfluent{} objects in the wild. It's also
surprising and non-obvious. We originally did not think that our decision
procedures could be made tolerant to user errors.

\paragraph{Reviewer 3: Section 5}
\begin{feedback}
  In section 5, the authors added more examples, proof, and Figure 4 to provide
  a better understanding of merge-reducible, non-merge- reducible objects.
  Still, the reader can easily understand these concepts from the original
  paper.
\end{feedback}
We strongly disagree. We believe that Section 5 was the most difficult section
to understand in our original publication, mostly because we did not have
enough space to fully explain all of our definitions, theorems, and proofs. The
statement and proof of Theorem 4 (in our original publication) is particularly
nuanced. We think that our added proofs greatly clarify this section, and our
figures illustrating the intuition behind the proofs are especially helpful.

\paragraph{Reviewer 3: Section 6}
\begin{feedback}
  In Section 6, the authors elaborated more to clarify the concepts and
  definitions in this section. Some readers may find the original paper was
  enough to understand these concepts and definitions.
\end{feedback}
We added an example to the paper to better explain co-reachable and
co-unreachable states.  We agree that this was a minor bit of clarification,
and that the section was already relatively clear. However, we also made a
number of other additions to Section 6. We explained the difference between
segmented \invariantconfluence{} and a distributed locking approach to
maintaining invariants~\cite{balegas2015putting, gotsman2016cause}. We
discussed two performance optimizations to our segmented \invariantconfluence{}
system model.  We discussed how segmented \invariantconfluence{} can leverage
our results on \mergereducibility{}. We elaborated on the fault-tolerance of
our model. We believe that, taken together, these additions are significant.

\paragraph{Reviewer 3: Section 7}
\begin{feedback}
  The authors added Section 7 to highlight a dual ``operation-based'' approach.
  This is not a significant improvement to the original contribution.
\end{feedback}
We disagree. The dual operation-based definitions of \invariantconfluence{} are
both theoretically interesting and practically important. Theoretically, as
with work on CRDTs, having dual models of the same concept can be helpful in
writing proofs and building intuition. Some properties are more naturally
expressed in the state-based model, while others are more natural in the
operation-based model. The operation-based model also has practical importance.
Many CRDTs are easier to understand as state-based objects, but are more
efficiently implemented as operation-based objects. By introducing
operation-based definitions of \invariantconfluence{}, we allow system
designers to choose whichever form of CRDT is more efficient. As we describe in
the paper, for example, the operation-based definitions let us replace a
PN-counter with a more efficient integer.

\paragraph{Reviewer 3: Reviewer 3: Other Comments}
\begin{feedback}
  Figure 5 should appear before Figure 6.
\end{feedback}
Fixed!

%%%%%%%%%%%%%%%%%%%%%%%%%%%%%%%%%%%%%%%%%%%%%%%%%%%%%%%%%%%%%%%%%%%%%%%%%%%%%%%
\section*{Reviewer 4}
\paragraph{Reviewer 4: Overview}
\begin{feedback}
  The paper is very well written, and it makes substantial improvement over the
  VLDB 2019 conference version.
\end{feedback}
Thanks! We appreciate it!

\paragraph{Reviewer 4: 1}
\begin{feedback}
  In the introduction the concept of ``invariant'' is missing some context. Where
  does this invariant come from? It looks like this invariant is a given
  invariant for the program executing on the replicated object, and it is not
  necessarily a tight invariant.  Clarifying this in advance would be useful,
  as I realized this only in the beginning of Section 2, I read this:
  ``Informally, a replicated object is invariant confluent *with respect to an
  invariant* if every replica of the object is guaranteed to satisfy the
  invariant despite the possibility of different replicas being concurrently
  modified or merged together.'' Mentioning the system model explicitly in the
  introduction also would help: ``a distributed object and accompanying
  invariant is replicated across a set of servers.''
\end{feedback}
We re-worded the introduction with your suggestion and also added a short
example of \invariantconfluence{} to the introduction.

\paragraph{Reviewer 4: 2}
\begin{feedback}
  Invariant confluence is about safety, however, could there be a liveness part
  of the problem worth investigating/discussing for replicated object
  management? Would it be possible that forgoing coordination across replicas
  could jeopardize/violate some liveness properties?
\end{feedback}
Forgoing coordination across replicas in favor of \invariantconfluence{}
actually \emph{increases} liveness, rather than jeopardizing or violating it.
Any alive replica of an \invariantconfluent{} object can service client
requests and execute transactions. This is the strongest form of liveness we
can hope to achieve. In fact, if we aim for strongly consistent replication,
then the CAP theorem tells us that total availability (the kind that
\invariantconfluence{} achieves) is impossible~\cite{brewer2012cap,
gilbert2002brewer}.

Note, however, that a segmented \invariantconfluent{} object does suffer from
liveness problems, as we discuss in Section 6.4. Specifically, transactions
that require global coordination are live only as long as every replica is
live. We discuss this more in Reviewer 4: 5 below.

\paragraph{Reviewer 4: 3}
\begin{feedback}
  Mergeable Replicated Data Types paper (OOPSLA'19) is relevant work. The paper
  does not cite MRDT understandably because it is very recent work, however
  including a discussion of this in the related work would be helpful.
\end{feedback}
Thanks for pointing out this paper. We hadn't heard of it before, and it is
definitely relevant to our work on \invariantconfluence{}. We added a
discussion of MRDTs to our related works section.

\paragraph{Reviewer 4: 4}
\begin{feedback}
  Section 4 is about invariant tightening, which has a broader applicability in
  distributed systems/programs. Unfortunately the techniques used here are not
  new. Involving the user for identifying unreachable states in not practical,
  and the discussion on tolerating user errors is weak as it does not offer a
  remedy. Would using some heuristics and abstracting state provide any help
  here?
\end{feedback}
We agree that involving the user for identifying unreachable states is not
ideal, and in many cases not practical. We think our discussion of tolerating
user errors is also not ideal, but not weak. By requiring users to provide
proofs of reachability and by deploying the invariant $I - NR$ instead of $I$,
invariant violations are impossible.
%
Heuristics and abstracting state and other techniques from invariant tightening
would probably help, but they are out of scope of the paper. As you mention,
the techniques are not new. We view it as an orthogonal but complementary line
of research.

\paragraph{Reviewer 4: 5}
\begin{feedback}
  Section 6 on segmented invariant confluence is nice. It would be nice if
  Section 6.4 could be improved. I liked the fourth point in Section 6.4, would
  it be possible to elaborate on this. The sixth point however is bad, as
  replicating every server with state-machine-replication is overkill. Would it
  be possible to use use a membership service approach (as in Vertical Paxos
  and Chain Replication configuration) to prevent a single crash to flunk
  global coordination?
\end{feedback}
We added a couple paragraphs to Example 13 with a concrete example of the
optimization described in our fourth point. We also added a forward reference
in Section 6.4 to Example 13.

Your comment on our sixth point is interesting. We hadn't thought of that
before. Coincidentally, since our initial publication, we spent some time
researching membership changes in state machine replication protocols---which
you can find here:
\url{https://mwhittaker.github.io/publications/matchmaker_paxos.html}---so we
can answer your question with quite a bit of confidence. Assume a client sends
a transaction to a server $p$. The transaction happens to not need global
coordination, so $p$ executes the transaction locally and sends back the result
to the client. This scenario makes membership changes difficult for the
following reason. Consider a different execution in which $p$ is partitioned
from the rest of the servers. The system erroneously thinks that $p$ has failed
and reconfigures to some other set of servers. However, $p$ and the client are
oblivious to this. The client can still contact $p$ and have transactions
executed, even though the system has moved to a completely new set of servers.
This can lead to situations that are impossible without membership changes. For
example, $p$ may be in a different active segment than the rest of the servers.

This problem can be mitigated by making stronger assumptions (e.g., a failure
detector, clock synchrony assumptions with leases). Alternatively, we could
require that a client contact multiple servers to execute a transaction and
then use quorum intersection to perform membership changes safely. We think
that this is a very interesting direction of future work, but unfortunately (as
with all membership change protocols) is very nuanced and very easy to get
wrong. It is out of the scope of our paper.

\begin{feedback}
  Is there plans to make Lucy implementation open source?
\end{feedback}
Yes. Lucy is available online at
\url{https://github.com/mwhittaker/enforced_invariant_confluence}. We also
added this link to Section 8 of the paper.

\paragraph{Reviewer 4: 7}
\begin{feedback}
  In the throughput evaluation of segmented invariant confluence systems, what
  would be the effects/overhead of faulttolerance? (Related to item 5 above.)
\end{feedback}
If we replicate every server with a state machine replication protocol, the
throughput of the segmented \invariantconfluent{} object certainly
decreases. However, since ever server can still independently execute
transactions, the throughput of the system would still be scalable for
workloads that don't require frequent coordination.
%
The performance overheads of more complex forms of fault tolerance (as
described in item 5 above) are out of scope for our paper.

\paragraph{Reviewer 4: 8}
\begin{feedback}
  Would it be possible to elaborate more on escrow transaction implementation.
  It was not clear to me from Example 13.
\end{feedback}
We added two more paragraphs of explanation to Example 13, including a concrete
example. Hopefully, that makes things clearer.

\bibliographystyle{abbrv}
\bibliography{references}
\end{multicols*}
\end{document}
