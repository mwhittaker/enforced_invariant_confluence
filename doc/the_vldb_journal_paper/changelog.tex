\documentclass[12pt]{article}

\usepackage[letterpaper, margin=1in]{geometry}
\usepackage{natbib}

\begin{document}
\begin{center}
  {\huge Change Log for ``Interactive Checks for Coordination Avoidance''}

  Michael Whittaker, Joseph M. Hellerstein
\end{center}

Here, we describe the additions that we made to our VLDB 2019 conference
version of ``Interactive Checks for Coordination Avoidance''.

\begin{itemize}
  \item
    In our conference paper, we present a definition of invariant confluence
    that is slightly different than the original definition from Bailis et
    al.~\cite{bailis2014coordination}, but we do not comment on the difference.
    We added a  subsection (Section 2.3) clarifying that the difference in
    definitions is superficial. We describe Bailis' original definition using
    our formalism and prove that the two definitions are equivalent.

  \item
    We expanded the correctness proof of our interactive decision procedure. A
    key step in understanding the correctness of our interactive invariant
    confluence decision procedure---namely that $I$-reachable and
    $(I-NR)$-reachable states were equal---was omitted from our conference
    paper. We added a formal proof of this fact as well as a few other minor
    elaborations.

  \item
    One of the core contributions of our paper is an interactive invariant
    confluence decision procedure. Here, ``interactive'' means that the
    decision procedure relies on human input to guide the decision procedure.
    To err is human. It's possible for a person to provide the decision
    procedure with incorrect information. We added a section (Section 4.3)
    describing how to mitigate these errors.

  \item
    We added a formal proof that criteria 1 through 4 are sufficient for
    invariant confluence. The idea behind the proof is straightforward but the
    details are tedious. To help readers see the forest through the trees, we
    also added an illustration of the key idea of the proof.

  \item
    Our conference paper presents formal definitions of distributed objects and
    invariant confluence. The definitions revolve around servers gossiping
    state to one another and merging the states together using a binary merge
    operator. This approach is called a ``state-based'' approach. We added a
    section (Section 7) outlining a dual ``operation-based'' approach in which
    transactions, rather than states, are gossiped. This duality between
    state-based and operation-based definitions is well established in the CRDT
    literature.

  \item
    We clarified that the merge operator of a distributed object does not have
    to satisfy any special mathematical properties like associativity,
    commutativity, or idempotence. Our results are more general and do not rely
    on any of these assumptions.

  \item
    We added a formal proof that invariant closure is sufficient for invariant
    confluence.

  \item
    We added a formal proof of our key result equating invariant closure and
    invariant confluence.

  \item
    In our conference paper, we asserted that checking for invariant closure is
    undecidable in general. We added a brief example showing \emph{why}
    invariant closure is undecidable, by reducing it to the Halting Problem.

  \item
    We added examples of merge-reducible and non-merge-reducible objects to
    make it easier for the reader to understand the definition of
    merge-reducibility.

  \item
    We added a formal proof that merge-reducibility is a sufficient condition
    for invariant-confluence. Similarly, we added a counterexample showing that
    merge-reducibility is not a necessary condition for invariant-confluence.

  \item
    In our conference paper, we provide an example of a distributed object that
    satisfies criteria 1 through 4 outlined in Theorem 4 but is not invariant
    closed. In the example, we asserted that criteria 1 through 4 are satisfied
    but did not elaborate on why. We added a brief explanation making it clear
    why the criteria are met.

  \item
    We added a Euler diagram illustrating the relationship between invariant
    confluence, invariant closure, merge reducibility, and criteria 1 through
    4.

  \item
    We added an example of co-reachable and co-unreachable states to help the
    reader understand the definition of co-reachability.

  \item
    We added a paragraph explaining the relationship between segmented
    invariant confluence and a distributed locking approach to maintaining
    invariants (that can be found in related works).

  \item
    We added a paragraph explaining a couple of optimizations that can be made
    to our segmented invariant confluence system model to reduce the frequency
    of global coordination.

  \item
    We added a paragraph explaining how a segmented invariant confluence
    decision procedure can leverage our merge reducibility results in addition
    to our results on invariant closure.

  \item
    We clarified the fault tolerance limitations of segmented invariant
    confluence. A naive implementation of our segmented invariant confluence
    decision procedure does not guarantee liveness, but there are
    straightforward techniques we can leverage to introduce liveness. We added
    an explanation of this to the paper.

  \item
    We added a ``Program Analysis in Database Systems'' paragraph to our
    related works section. In it, we describe other efforts to use program
    analysis to improve the performance of database systems.
\end{itemize}

\bibliographystyle{abbrv}
\bibliography{references}
\end{document}
